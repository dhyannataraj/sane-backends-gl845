\documentclass[11pt,DVIps]{report}

\usepackage{times,epsfig,changebar,html}

\setlength{\parindent}{0pt}
\setlength{\parskip}{1.5ex plus 0.5ex minus 0.5ex}
\setlength{\textwidth}{6.5in}
\setlength{\textheight}{8.5in}
\setlength{\marginparwidth}{0pt}
\setlength{\oddsidemargin}{0pt}
\setlength{\evensidemargin}{0pt}
\setlength{\marginparsep}{0pt}
\addtolength{\topmargin}{-0.75in}

\title{\huge SANE Standard Version 2.0 proposal 0.08 - rauch/beck}
\author{}
\date{Dec 8, 2002} 

\makeindex

\begin{document}

\newcommand{\filename}[1]{{\tt #1}}
\newcommand{\code}[1]{{\tt #1}}
\newcommand{\var}[1]{{\it #1}}
\newcommand{\defn}[1]{#1\index{#1}}

\begin{latexonly}
  \setcounter{changebargrey}{0}   % black change bars
\end{latexonly}

\maketitle
\tableofcontents
\listoffigures
\listoftables


\chapter{Preface}

The SANE standard is being developed by a group of free-software
developers.  The process is open to the public and comments as well as
suggestions for improvements are welcome.  Information on how to join
the SANE development process can be found in Chapter
\ref{chap:contact}.

The SANE standard is intended to streamline software development by
providing a standard application programming interface to access
raster scanner hardware.  This should reduce the number of different
driver implementations, thereby reducing the need for reimplementing
similar code.


\section{About This Document}

This document is intended for developers who are creating either an
application that requires access to raster scanner hardware and for
developers who are implementing a SANE driver.  It does not cover
specific implementations of SANE components.  Its sole purpose is to
describe and define the SANE application interface that will enable
any application on any platform to interoperate with any SANE backend
for that platform.

The remainder of this document is organized as follows.
Chapter~\ref{chap:intro} provides introductional material.
Chapter~\ref{chap:environ} presents the environment SANE is designed
for.  Chapter~\ref{chap:api} details the SANE Application Programmer
Interface.  Chapter~\ref{chap:net} specifies the network protocol that
can be used to implement the SANE API in a network transparent
fashion.  Finally, Chapter~\ref{chap:contact} gives information on how
to join the SANE development process.

\subsection{Typographic Conventions}

Changes since the last revision of this document are highlighted
like this:

\begin{changebar}
   Paragraphs that changed since the last revision of the documention
   are marked like this paragraph.
\end{changebar}

\chapter{Introduction}\label{chap:intro}

SANE is an application programming interface (API) that provides
standardized access to any raster image scanner hardware.  The
standardized interface allows to write just one driver for each
scanner device instead of one driver for each scanner and application.
The reduction in the number of required drivers provides significant
savings in development time.  More importantly, SANE raises the level
at which applications can work.  As such, it will enable applications
that were previously unheard of in the UNIX world.  While SANE is
primarily targeted at a UNIX environment, the standard has been
carefully designed to make it possible to implement the API on
virtually any hardware or operating system.

SANE is an acronym for ``Scanner Access Now Easy.''  Also, the hope is
that SANE is sane in the sense that it will allow easy implementation
of the API while accommodating all features required by today's
scanner hardware and applications.  Specifically, SANE should be broad
enough to accommodate devices such as scanners, digital still and
video cameras, as well as virtual devices like image file filters.

\section{Terminology}

An application that uses the SANE interface is called a SANE {\em
  frontend}.  A driver that implements the SANE interface is called a
SANE {\em backend}.  A {\em meta backend\/} provides some means to
manage one or more other backends.


\chapter{The SANE Environment}\label{chap:environ}

SANE is defined as a C-callable library interface.  Accessing a raster
scanner device typically consists of two phases: first, various
controls of the scanner need to be setup or queried.  In the second
phase, one or more images are acquired.

Since the device controls are widely different from device to device,
SANE provides a generic interface that makes it easy for a frontend to
give a user access to all controls without having to understand each
and every device control.  The design principle used here is to
abstract each device control into a SANE {\em option\/}.  An option is
a self-describing name/value pair.  For example, the brightness
control of a camera might be represented by an option called
\code{brightness} whose value is an integer in the range from 0 to
255.

With self-describing options, a backend need not be concerned with
{\em presentation\/} issues: the backend simply provides a list of
options that describe all the controls available in the device.
Similarly, there are benefits to the frontend: it need not be
concerned with the {\em meaning\/} of each option.  It simply provides
means to present and alter the options defined by the backend.


\section{Attaching to a SANE backend}

The process through which a SANE frontend connects to a backend is
platform dependent.  Several possibilities exist:
\begin{itemize}

\item {\bf Static linking:} A SANE backend may be linked directly into
  a frontend.  While the simplest method of attaching to a backend, it
  is somewhat limited in functionality since the available devices is
  limited to the ones for which support has been linked in when the
  frontend was built.  But even so static linking can be quite useful,
  particularly when combined with a backend that can access scanners
  via a network.  Also, it is possible to support multiple backends
  simultaneously by implementing a meta backend that manages several
  backends that have been compiled in such a manner that they export
  unique function names.  For example, a backend called \code{be}
  would normally export a function called \code{sane\_read()}.  If
  each backend would provide such a function, static linking would
  fail due to multiple conflicting definitions of the same symbol.
  This can be resolved by having backend \code{be} include a
  header file that has lines of the form:
  \begin{quote}
\begin{verbatim}
#define sane_read be_sane_read
\end{verbatim}
  \end{quote}
  With definitions of this kind, backend \code{be} will export
  function name \code{be\_sane\_read()}.  Thus, all backends will
  export unique names.  As long as a meta backend knows about these
  names, it is possible to combine several backends at link time and
  select and use them dynamically at runtime.

\item {\bf Dynamic linking:} A simpler yet more powerful way to
  support multiple backends is to exploit dynamic linking on platforms
  that support it.  In this case, a frontend is linked against a
  shared library that implements any SANE backend.  Since each
  dynamically linked backend exports the same set of global symbols
  (all starting with the prefix \code{sane\_}), the dynamic library
  that gets loaded at runtime does not necessarily have to be the same
  one as one the frontend got linked against.  In other words, it is
  possible to switch the backend by installing the appropriate backend
  dynamic library.

  More importantly, dynamic linking makes it easy to implement a meta
  backend that loads other backends {\em on demand}.  This is a
  powerful mechanism since it allows adding new backends merely by
  installing a shared library and updating a configuration file.

\item {\bf Network connection:} Arguably the ultimate way to attach to
  a scanner is by using the network to connect to a backend on a
  remote machine.  This makes it possible to scan images from any host
  in the universe, as long as there is a network connection to that
  host and provided the user is permitted to access that scanner.

\end{itemize}

\begin{figure}[htbp]
  \begin{center}
    \leavevmode
    \psfig{file=figs/hierarchy.eps,angle=270,width=\textwidth}
    \caption{Example SANE Hiearchy}
    \label{fig:hierarchy}
  \end{center}
\end{figure}

The above discussion lists just a few ways for frontends to attach to
a backend.  It is of course possible to combine these solutions to
provide an entire hierarchy of SANE backends.  Such a hierarchy is
depicted in Figure~\ref{fig:hierarchy}.  The figure shows that machine
A uses a dynamic-linking based meta backend called \code{dll} to
access the backends called \code{pnm}, \code{mustek}, and \code{net}.
The first two are real backends, whereas the last one is a meta
backend that provides network transparent access to remote scanners.
In the figure, machine B provides non-local access to its scanners
through the SANE frontend called \code{saned}.  The \code{saned} in
turn has access to the \code{hp} and \code{autolum} backends through
another instance of the \code{dll} backend.  The \code{autolum} meta
backend is used to automatically adjust the luminance (brightness) of
the image data acquired by the camera backend called \code{qcam}.

Note that a meta backend really is both a frontend and a backend at
the same time.  It is a frontend from the viewpoint of the backends
that it manages and a backend from the viewpoint of the frontends that
access it.  The name ``meta backend'' was chosen primarily because the
SANE standard describes the interface from the viewpoint of a (real)
frontend.


\section{Image Data Format}\label{sec:imageformat}\index{image data format}

Arguably the most important aspect of an image acquisition system is how
images are represented. The SANE approach is to define a simple yet powerful
representation that is sufficient for vast majority of applications and
devices. While the representation is simple, the interface has been defined
carefully to allow extending it in the future without breaking backwards
compatibility. Thus, it will be possible to accommodate future applications or
devices that were not anticipated at the time this standard was created.

A SANE image is a rectangular area.  The rectangular area is subdivided into a
number of rows and columns.  At the intersection of each row and column is a
(preferable quadratic) pixel. A pixel consists of one or more sample values.
Each sample value represents one channel (e.g., the red channel).

The SANE API transmits an image as a sequence of frames.  Each frame covers
the same rectangular area as the entire image, but may contain only a subset
of the channels in the final image.  For example, a red/green/blue image could
either be transmitted as a single frame that contains the sample values for
all three channels or it could be transmitted as a sequence of three frames:
the first frame containing the red channel, the second the green channel, and
the third the blue channel.

When transmitting an image frame by frame, the frontend needs to know what
part of the image a frame represents (and how many frames it should expect).
For that purpose, the SANE API tags every frame with a type and a format
descriptor.

\begin{changebar}
  There are two different types of frames: pixel oriented frames
  \code{\defn{SANE\_FRAME\_RAW}} and arbitrary data frames
  \code{\defn{SANE\_FRAME\_MIME}}. These types are discussed in detail in the
  following sections. The frame types used by the previous version~1 of this
  standard (\code{\defn{SANE\_FRAME\_GRAY}}, \code{\defn{SANE\_FRAME\_RGB}},
  \code{\defn{SANE\_FRAME\_RED}}, \code{\defn{SANE\_FRAME\_GREEN}}, and
  \code{\defn{SANE\_FRAME\_BLUE}}) are obsolete and superseded by
  \code{\defn{SANE\_FRAME\_RAW}}.

  \subsection{Pixel oriented frames}

  The type of pixel oriented frames is \code{\defn{SANE\_FRAME\_RAW}}. The
  frame contains one or more channels of data in a channel-interleaved format,
  that represents sample values from a property of the individual pixels that
  is subject to further description in the \code{format\_desc} member of the
  \code{SANE\_Parameters} structured type. See section~\ref{sec:sanegetparameters}
  on page~\pageref{sec:sanegetparameters} for details about the format
  descriptions.
\end{changebar}

  Each sample value has a certain bit depth.  The bit depth is fixed for the
  entire image and can be as small as one bit. Valid bit depths are 1, 8, or
  16 bits per sample. If a device's natural bit depth is something else, it is
  up to the driver to scale the sample values appropriately (e.g., a 4 bit
  sample could be scaled by a factor of four to represent a sample value of
  depth 8).

\begin{changebar}
  The complete image may consist of several different channels. The number of channels
  is defined by member \code{channels\_per\_image} of \code{SANE\_Parameters}.
  The image may be transmitted in an arbitrary number of frames which can be
  determined by watching the \code{SANE\_PFLAG\_LAST\_FRAME} flag in said type (or by
  counting the channels).  Note: This frame type replaces all frame types of
  the SANE standard version 1.
\end{changebar}

Conceptually, each pixel oriented frame is transmitted a byte at a time.  Each
byte may contain 8 sample values (for an image bit depth of 1), one full
sample value (for an image bit depth of 8), or a partial sample value (for an
image bit depth of 16 or bigger).  In the latter case, the bytes of each
sample value are transmitted in the machine's native byte order.
\begin{quote}
  \begin{center}
    {\bf Backend Implementation Note}
  \end{center}
  A network-based meta backend will have to ensure that the byte order
  in image data is adjusted appropriately if necessary.  For example,
  when the meta backend attaches to the server proxy, the proxy may
  inform the backend of the server's byte order.  The backend can then
  apply the adjustment if necessary.  In essence, this implements a
  ``receiver-makes-right'' approach.
\end{quote}

\begin{figure}[htbp]
  \begin{center}
    \leavevmode
    \psfig{file=figs/xfer.eps,width=0.5\textwidth}
    \caption{Transfer order of image data bytes}
    \label{fig:xfer}
  \end{center}
\end{figure}

The order in which the sample values in a frame are transmitted is illustrated
in Figure~\ref{fig:xfer}.  As can be seen, the values are transmitted row by
row and each row is transmitted from left-most to right-most column.  The
left-to-right, top-to-bottom transmission order applies when the image is
viewed in its normal orientation (as it would be displayed on a screen, for
example).

If a frame contains multiple channels, then the channels are transmitted in an
interleaved fashion.  Figure~\ref{fig:pixels} illustrates this for the case
where a frame contains a complete red/green/blue image with a bit-depth of 8.

\begin{figure}[htbp]
  \begin{center}
    \leavevmode
    \psfig{file=figs/image-data.eps,width=0.8\textwidth}
    \caption{Bit and byte order of image data}
    \label{fig:pixels}
  \end{center}
\end{figure}

For a bit depth of 1, each byte contains 8 sample values of a {\em single\/}
channel. In other words, a bit depth 1 frame is transmitted in a byte
interleaved fashion. The first sample of each byte is represented by the most
significant bit.

\begin{changebar}
For gray channels at a bit depth of 1 only two sample values are possible: 1
represents minimum intensity (black) and 0 represents maximum intensity
(white).  For all other channel types and bit depths a sample value of 0
represents minimum intensity and larger values represent increasing intensity.

\subsection{Arbitrary data frames}

It also is possible to transmit arbitrary (not necessaryly pixel oriented)
data. This allows transmission of compressed images like jpeg, tiff, etc.

The type of arbitrary data frames is \code{\defn{SANE\_FRAME\_MIME}}.
The frame contains arbitrary data of the MIME (see RFC 1521/1522) type that is
given in the \code{format\_desc} member of the \code{SANE\_Parameters}
structured type (see section~\ref{sec:sanegetparameters} on
page~\pageref{sec:sanegetparameters}).  As such, it is assumed to be
incomprehensible to the frontend, except for selected types the frontend is
specifically capable of handling internally.  The frontend is free to ignore
those frames, or employ any appropriate means to otherwise handle this data
(like saving them to disk or spawning an external viewer).
\end{changebar}

\chapter{The SANE Application Programmer Interface (API)}\label{chap:api}

\begin{changebar}
This Section defines version 2 of the SANE application
programmer interface (API).  Any SANE frontend must depend on the
interface defined in this section only. Converseley, any SANE backend
must implement its functionality in accordance with this
specification. The interface as documented here is declared as a C
callable interface in a file called \filename{sane/sane-2.h}. This file should
normally be included via a C pre-processor directive of the form:
\begin{verbatim}
  #include <sane/sane-2.h>
\end{verbatim}
\end{changebar}


\section{Version Control}

The SANE standard is expected to evolve over time.  Whenever a change
to the SANE standard is made that may render an existing frontend or
backend incompatible with the new standard, the major version number
must be increased.  Thus, any frontend/backend pair is compatible
provided the major version number of the SANE standard they implement
is the same.  A frontend may implement backwards compatiblity by
allowing major numbers that are smaller than the expected major number
(provided the frontend really can cope with the older version).  In
contrast, a backend always provides support for one and only one
version of the standard.  If a specific application does require that
two different versions of the same backend are accessible at the same
time, it is possible to do so by installing the two versions under
different names.

SANE version control also includes a minor version number and a build
revision.  While control of these numbers remains with the implementor
of a backend, the recommended use is as follows.  The minor version is
incremented with each official release of a backend.  The build
revision is increased with each build of a backend.

The SANE API provides the following five macros to manage version
numbers.
\begin{quote}
  \begin{description}
  \item[\code{\defn{SANE\_CURRENT\_MAJOR}}:] The value of this macro is the
    number of the SANE standard that the interface implements.

  \item[\code{\defn{SANE\_VERSION\_CODE}(\var{maj},\var{min},\var{bld})}:]
    \label{sec:saneversioncode}
    This macro can be used to build a monotonically increasing version
    code.  A SANE version code consists of the SANE standard major
    version number (\var{maj}), the minor version number \var{min},
    and the build revision of a backend (\var{bld}).  The major and
    minor version numbers must be in the range 0\ldots255 and the
    build revision must be in the range 0\ldots65535.

    Version codes are monotonic in the sense that it is possible to
    apply relational operators (e.g., equality or less-than test)
    directly on the version code rather than individually on the three
    components of the version code.

    Note that the major version number alone determines whether a
    frontend/backend pair is compatible.  The minor version and the
    build revision are used for informational and bug-fixing purposes
    only.
  
  \item[\code{\defn{SANE\_VERSION\_MAJOR}(\var{vc})}:] This macro returns the
    major version number component of the version code passed in
    argument \var{vc}.
  \item[\code{SANE\_VERSION\_MINOR(\var{vc})}:] This macro returns the
    minor version number component of the version code passed in
    argument \var{vc}.
  \item[\code{SANE\_VERSION\_BUILD(\var{vc})}:] This macro returns the
    build revision component of the version code passed in argument
    \var{vc}.
  \end{description}
\end{quote}


\section{Data Types}

\subsection{Base Types}

The SANE standard is based on just two SANE-specific base types: the
SANE byte and word.
\begin{quote}
  \code{typedef \var{some-scalar-type\/} \defn{SANE\_Byte};} \\
  \code{typedef \var{some-scalar-type\/} \defn{SANE\_Word};}
\end{quote}
\verb|SANE_Byte| must correspond to some scalar C type that is capable
of holding values in the range 0 to 255.  \verb|SANE_Word| must be
capable of holding any of the following:
\begin{itemize}
  \item the truth values \verb|SANE_FALSE| and \verb|SANE_TRUE|
  \item signed integers in the range $-2^{31}\ldots2^{31}-1$
  \item fixed point values in the range $-32768\ldots32767.9999$ with
    a resolution of $1/65536$
  \item 32 bits (for bit sets)
\end{itemize}
Note that the SANE standard does not define what C type
\verb|SANE_Byte| and \verb|SANE_Word| map to.  For example, on some
platforms, the latter may map to \verb|long int| whereas on others it
may map to \verb|int|.  A portable SANE frontend or backend must
therefore not depend on a particular mapping.

\subsection{Boolean Type}

\code{\defn{SANE\_Bool}} is used for variables that can take one of
the two truth values \code{\defn{SANE\_FALSE}} and
\code{\defn{SANE\_TRUE}}.  The former value is defined to be 0,
whereas the latter is 1.\footnote{This is different from ANSI C where
  any non-zero integer value represents logical TRUE.} The C
declarations for this type are given below.
\begin{quote}
\begin{verbatim}
#define SANE_FALSE      0
#define SANE_TRUE       1
typedef SANE_Word SANE_Bool;
\end{verbatim}
\end{quote}
Note that \verb|SANE_Bool| is simply an alias of \verb|SANE_Word|.  It
is therefore always legal to use the latter type in place of the
former.  However, for clarity, it is recommended to use
\verb|SANE_Bool| whenever a given variable or formal argument has a
fixed interpretation as a boolean object.

\subsection{Integer Type}

\code{\defn{SANE\_Int}} is used for variables that can take integer
values in the range $-2^{32}$ to $2^{31}-1$.  Its C declaration is
given below.
\begin{quote}
\begin{verbatim}
typedef SANE_Word SANE_Int;
\end{verbatim}
\end{quote}
Note that \verb|SANE_Int| is simply an alias of \verb|SANE_Word|.  It
is therefore always legal to use the latter type in place of the
former.  However, for clarity, it is recommended to use
\verb|SANE_Int| whenever a given variable or formal argument has a
fixed interpretation as an integer object.


\subsection{Fixed-point Type}

\code{\defn{SANE\_Fixed}} is used for variables that can take fixed
point values in the range $-32768$ to $32767.9999$ with a resolution
of $1/65535$.  The C declarations relating to this type are given
below.
\begin{quote}
\begin{verbatim}
#define SANE_FIXED_SCALE_SHIFT  16
typedef SANE_Word SANE_Fixed;
\end{verbatim}
\end{quote}
The macro \code{\defn{SANE\_FIXED\_SCALE\_SHIFT}} gives the location
of the fixed binary point.  This standard defines that value to be 16,
which yields a resolution of $1/65536$.

Note that \verb|SANE_Fixed| is simply an alias of \verb|SANE_Word|.
It is therefore always legal to use the latter type in place of the
former.  However, for clarity, it is recommended to use
\verb|SANE_Fixed| whenever a given variable or formal argument has a
fixed interpretation as a fixed-point object.

For convenience, SANE also defines two macros that convert fixed-point
values to and from C double floating point values.
\begin{quote}
  \begin{description}

  \item[\code{\defn{SANE\_FIX}(\var{d})}:] Returns the largest SANE
    fixed-point value that is smaller than the double value \var{d}.
    No range checking is performed.  If the value of \var{d} is out of
    range, the result is undefined.

  \item[\code{\defn{SANE\_UNFIX}(\var{w})}:] Returns the nearest
    double machine number that corresponds to fixed-point value
    \var{w}.

  \end{description}
\end{quote}
SANE does {\em not\/} require that the following two expressions hold
true (even if the values of \var{w} and \var{d} are in range):
\begin{quote}
\begin{verbatim}
SANE_UNFIX(SANE_FIX(d)) == d
SANE_FIX(SANE_UNFIX(w)) == w
\end{verbatim}
\end{quote}
In other words, conversion between fixed and double values may be
lossy.  It is therefore recommended to avoid repeated conversions
between the two representations.


\subsection{Text}

\subsubsection{Character Type}

Type \code{\defn{SANE\_Char}} represents a single text character or
symbol.  At present, this type maps directly to the underlying C
\verb|char| type (typically one byte).  The encoding for such
characters is currently fixed as ISO LATIN-1.  Future versions of this
standard may map this type to a wider type and allow multi-byte
encodings to support internationalization.  As a result of this, care
should be taken to avoid the assumption that
\verb|sizeof(SANE_Char)==sizeof(char)|.
\begin{quote}
\begin{verbatim}
typedef char SANE_Char;
\end{verbatim}
\end{quote}

\subsubsection{String Type}

Type \code{\defn{SANE\_String}} represents a text string as a sequence
of C \verb|char| values.  The end of the sequence is indicated by a
\verb|'\0'| (\defn{NUL}) character.  
\begin{quote}
\begin{verbatim}
typedef SANE_Char *SANE_String;
typedef const SANE_Char *SANE_String_Const;
\end{verbatim}
\end{quote}
The type \code{\defn{SANE\_String\_Const}} is provided by SANE to
enable declaring strings whose contents is unchangable.  Note that in
ANSI C, the declaration
\begin{quote}
\begin{verbatim}
const SANE_String str;
\end{verbatim}
\end{quote}
declares a string pointer that is constant (not a string pointer that
points to a constant value).


\subsection{Scanner Handle Type}

Access to a scanner is provided through an opaque type called
\code{\defn{SANE\_Handle}}.  The C declaration of this type is given
below.
\begin{quote}
\begin{verbatim}
typedef void *SANE_Handle;
\end{verbatim}
\end{quote}
While this type is declared to be a void pointer, an application must
not attempt to interpret the value of a \verb|SANE_Handle|.  In
particular, SANE does not require that a value of this type is a legal
pointer value.


\subsection{Status Type}

Most SANE operations return a value of type \code{\defn{SANE\_Status}}
to indicate whether the completion status of the operation.  If an
operation completes successfully, \verb|SANE_STATUS_GOOD| is returned.
In case of an error, a value is returned that indicates the nature of
the problem.  The complete list of available status codes is listed in
Table \ref{tab:status}.  It is recommended to use function
\code{sane\_strstatus()} to convert status codes into a legible
string.

\begin{table}[htbp]
  \begin{center}
    \begin{tabular}{|l|r|l|}
      \hline
      \multicolumn{1}{|c|}{\bf Symbol} & \multicolumn{1}{c|}{\bf Code} &
      \multicolumn{1}{c|}{\bf Description} \\
      \hline\hline
\code{\defn{SANE\_STATUS\_GOOD}}
        &  0 & Operation completed succesfully. \\
\code{\defn{SANE\_STATUS\_UNSUPPORTED}}
        &  1 & Operation is not supported. \\
\code{\defn{SANE\_STATUS\_CANCELLED}}
        &  2 & Operation was cancelled. \\
\code{\defn{SANE\_STATUS\_DEVICE\_BUSY}}
        &  3 & Device is busy---retry later. \\
\code{\defn{SANE\_STATUS\_INVAL}}
        &  4 & Data or argument is invalid. \\
\code{\defn{SANE\_STATUS\_EOF}}
        &  5 & No more data available (end-of-file). \\
\code{\defn{SANE\_STATUS\_JAMMED}}
        &  6 & Document feeder jammed. \\
\code{\defn{SANE\_STATUS\_NO\_DOCS}}
        &  7 & Document feeder out of documents. \\
\code{\defn{SANE\_STATUS\_COVER\_OPEN}}
        &  8 & Scanner cover is open. \\
\code{\defn{SANE\_STATUS\_IO\_ERROR}}
        &  9 & Error during device I/O. \\
\code{\defn{SANE\_STATUS\_NO\_MEM}}
        & 10 & Out of memory. \\
\code{\defn{SANE\_STATUS\_ACCESS\_DENIED}}
        & 11 & Access to resource has been denied. \\
      \hline
    \end{tabular}
    \caption{Status Codes}\label{tab:status}
  \end{center}
\end{table}


\subsection{Device Descriptor Type}
\begin{changebar}

Each SANE device is represented by a structure of type
\code{\defn{SANE\_Device}}.  The C declaration of this type is given
below.
\begin{quote}
\begin{verbatim}
typedef struct
  {
    SANE_String_Const name;
    SANE_String_Const vendor;
    SANE_String_Const model;
    SANE_String_Const type;
    SANE_String_Const email_backend_author;
    SANE_String_Const backend_website;
    SANE_String_Const device_location;
    SANE_String_Const comment;
    SANE_String_Const reserved_string;
    SANE_Int backend_version_code;
    SANE_Int backend_capablity_flags;
    SANE_Int reserved_int;
  }
SANE_Device;
\end{verbatim}
\end{quote}
\end{changebar}
\index{device-name}
The structure provides the unique name of the scanner in member
\code{name}.  It is this unique name that should be passed in a call
to \code{sane\_open()}.  The format of this name is completely up to
the backend.  The only constraints are that the name is unique among
all devices supported by the backend and that the name is a legal SANE
text string.  To simplify presentation of unique names, their length
should not be excessive.  It is {\em recommended\/} that backends keep
unique names below 32 characters in length.  However, applications
{\em must\/} be able to cope with arbitrary length unique names.

The next three members in the device structure provide additional
information on the device corresponding to the unique name.
Specifically, members \code{vendor}, \code{model}, and \code{type} are
single-line strings that give information on the vendor
(manufacturer), model, and the type of the device.  For consistency's
sake, the following strings should be used when appropriate (the lists
will be expanded as need arises):

\begin{table}[htbp]
  \begin{center}
    \leavevmode
    \hspace{\fill}
    \begin{tabular}[t]{|ll|}
      \hline
      \multicolumn{2}{|c|}{\bf \defn{Vendor Strings}} \\
      \hline\hline
      \code{AGFA}            & \code{Microtek} \\
      \code{Abaton}          & \code{Minolta} \\
      \code{Acer}            & \code{Mustek} \\
      \code{Apple}           & \code{NEC} \\
      \code{Artec}           & \code{Nikon} \\
      \code{Avision}         & \code{Plustek} \\
      \code{CANON}           & \code{Polaroid} \\
      \code{Connectix}       & \code{Ricoh} \\
      \code{Epson}           & \code{Sharp} \\
      \code{Fujitsu}         & \code{Siemens} \\
      \code{Hewlett-Packard} & \code{Tamarack} \\
      \code{IBM}             & \code{UMAX} \\
      \code{Kodak}           & \code{Noname} \\
      \code{Logitech}        & \\ 
      \hline
    \end{tabular}
    \hspace{\fill}
    \begin{tabular}[t]{|l|}
      \hline
      \multicolumn{1}{|c|}{\bf \defn{Type Strings}} \\
      \hline\hline
      \code{film scanner} \\
      \code{flatbed scanner} \\
      \code{frame grabber} \\
      \code{handheld scanner} \\
      \code{multi-function peripheral} \\
      \code{sheetfed scanner} \\
      \code{still camera} \\
      \code{video camera} \\
      \code{virtual device} \\
      \hline
    \end{tabular}
    \hspace{\fill}
    \caption{Predefined Device Information Strings}
    \label{tab:devinfo}
  \end{center}
\end{table}
Note that vendor string \code{Noname} can be used for virtual devices
that have no physical vendor associated.  Also, there are no
predefined model name strings since those are vendor specific and
therefore completely under control of the respective backends.

\begin{changebar}
The backend has to set up the string \code{email\_backend\_author}
with the name and the email address of the backend author or a contact person
in the format:
\begin{verbatim}
Firstname Lastname <name@domain.org>
\end{verbatim}

The string \code{backend\_website} should be set up by the backend
with the website or ftp address of the backend in the format:
\begin{verbatim}
http://www.domain.org/sane-hello/index.html
\end{verbatim}

The backend should fill the string \code{device\_location} with a
text that describes where a user can find this device. The text 
should be configurable by the administrator. This could e.g. look
like this:
\begin{verbatim}
building 93, 2nd plane, room 2124
\end{verbatim}

The string \code{comment} can be used to display any comment about the device
to the user. The text should be configurable by the administrator.

The string \code{reserved\_string} is unused currently but planned for
future use.

All unused strings must be set to ``'' by the backend.

With member \code{backend\_version\_code} a frontend can find
out the version of a backend it is connected to via one or more meta
backends.

The member \code{backend\_capability\_flags} contains 32 bits that
are planned to give the backend the chance to inform the frontend about
its capabilities. The meaning of the flags will be defined when there
is the need for it. The backend has to set all not defined bits
(in the moment all 32 bits) to 0.

The member \code{reserved\_int} is planned for future use, the backend has
to set the value of the integer to 0.
\end{changebar}

\subsection{Option Descriptor Type}\label{sec:odesc}

Option descriptors are at the same time the most intricate and
powerful type in the SANE standard.  Options are used to control
virtually all aspects of device operation.  Much of the power of the
SANE API stems from the fact that most device controls are completely
described by their respective option descriptor.  Thus, a frontend can
control a scanner abstractly, without requiring knowledge as to what
the purpose of any given option is.  Conversely, a scanner can
describe its controls without requiring knowledge of how the frontend
operates.  The C declaration of the
\code{\defn{SANE\_Option\_Descriptor}} type is given below.
\begin{quote}
\begin{verbatim}
typedef struct
  {
    SANE_String_Const name;
    SANE_String_Const title;
    SANE_String_Const desc;
    SANE_Value_Type type;
    SANE_Unit unit;
    SANE_Int size;
    SANE_Int cap;
    SANE_Constraint_Type constraint_type;
    union
      {
        const SANE_String_Const *string_list;
        const SANE_Word *word_list;
        const SANE_Range *range;
      }
    constraint;
  }
SANE_Option_Descriptor;
\end{verbatim}
\end{quote}

\subsubsection{Option Name}

Member \code{name} is a string that uniquely identifies the option.
The name must be unique for a given device (i.e., the option names
across different backends or devices need not be unique).  The option
name must consist of lower-case ASCII letters (\code{a}--\code{z}),
digits (\code{0}--\code{9}), or the dash character (\code{-}) only.
The first character must be a lower-case ASCII character (i.e., not a
digit or a dash).

\subsubsection{Option Title}

Member \code{title} is a single-line string that can be used by the
frontend as a title string.  This should typically be a short (one or
two-word) string that is chosen based on the function of the option.

\subsubsection{Option Description}

Member \code{desc} is a (potentially very) long string that can be
used as a help text to describe the option.  It is the responsibility
of the frontend to break the string into managable-length lines.
Newline characters in this string should be interpreted as paragraph
breaks.

\subsubsection{Option Value Type}

Member \code{type} specifies the type of the option value.  The
possible values for type \code{\defn{SANE\_Value\_Type}} are described
in Table \ref{tab:valuetype}.

\begin{table}[htbp]
  \begin{center}
    \leavevmode
    \begin{tabular}{|l|l|p{0.6\textwidth}|}
\hline
\multicolumn{1}{|c|}{\bf Symbol} &
\multicolumn{1}{c|}{\bf Code} &
\multicolumn{1}{c|}{\bf Description} \\
\hline\hline

\code{\defn{SANE\_TYPE\_BOOL}} & 0 & Option value is of type
  \verb|SANE_Bool|. \\ 

\code{\defn{SANE\_TYPE\_INT}} & 1 & Option value is of type
  \verb|SANE_Int|. \\ 

\code{\defn{SANE\_TYPE\_FIXED}}&2 & Option value is of type
  \verb|SANE_Fixed|. \\ 

\code{\defn{SANE\_TYPE\_STRING}}&3 & Option value is of type
  \verb|SANE_String|. \\ 

\code{\defn{SANE\_TYPE\_BUTTON}} & 4 & An option of this type has no value.
Instead, setting an option of this type has an option-specific
side-effect.  For example, a button-typed option could be used by a
backend to provide a means to select default values or to the tell an
automatic document feeder to advance to the next sheet of paper. \\ 

\code{\defn{SANE\_TYPE\_GROUP}} & 5 & An option of this type has no value.
This type is used to group logically related options.  A group option
is in effect up to the point where another group option is encountered
(or up to the end of the option list, if there are no other group
options).  For group options, only members \code{title} and
\code{type} are valid in the option descriptor. \\ 

  \hline
    \end{tabular}
    \caption{Option Value Types (\code{SANE\_Value\_Type})}
    \label{tab:valuetype}
  \end{center}
\end{table}

\subsubsection{Option Value Unit}

Member \code{unit} specifies what the physical unit of the option
value is.  The possible values for type \code{\defn{SANE\_U\-nit}} are
described in Table \ref{tab:units}.  Note that the specified unit is
what the SANE backend expects.  It is entirely up to a frontend as to
how these units a presented to the user.  For example, SANE expresses
all lengths in millimeters.  A frontend is generally expected to
provide appropriate conversion routines so that a user can express
quantities in a customary unit (e.g., inches or centimeters).

\begin{table}[htbp]
  \begin{center}
    \leavevmode
    \begin{tabular}{|l|l|l|}
\hline
\multicolumn{1}{|c|}{\bf Symbol} &
\multicolumn{1}{|c|}{\bf Code} &
\multicolumn{1}{|c|}{\bf Description} \\

\hline\hline

\code{\defn{SANE\_UNIT\_NONE}}   & 0 & Value is unit-less (e.g., page count).\\
\code{\defn{SANE\_UNIT\_PIXEL}}  & 1 & Value is in number of pixels. \\
\code{\defn{SANE\_UNIT\_BIT}}    & 2 & Value is in number of bits. \\
\code{\defn{SANE\_UNIT\_MM}}     & 3 & Value is in millimeters. \\
\code{\defn{SANE\_UNIT\_DPI}}    & 4 & Value is a resolution in dots/inch. \\
\code{\defn{SANE\_UNIT\_PERCENT}}& 5 & Value is a percentage. \\
\code{\defn{SANE\_UNIT\_MICROSECOND}}& 6 & Value is time in $\mu$-seconds. \\

\hline
    \end{tabular}
    \caption{Physical Units (\code{SANE\_Unit})}
    \label{tab:units}
  \end{center}
\end{table}

\subsubsection{Option Value Size}\label{sec:valuesize}

Member \code{size} specifies the size of the option value (in bytes).
This member has a slightly different interpretation depending on the
type of the option value:
\begin{quote}
  \begin{description}
  \item[\code{SANE\_TYPE\_STRING}:] The size is the maximum size of
    the string.  For the purpose of string size calcuations, the
    terminating \code{NUL} character is considered to be part of the
    string.  Note that the terminating \code{NUL} character must
    always be present in string option values.
  \item[\code{SANE\_TYPE\_INT}, \code{SANE\_TYPE\_FIXED}:] The size
    must be a positive integer multiple of the size of a
    \verb|SANE_Word|.  The option value is a vector of length
    \[ \code{size}/\code{sizeof(SANE\_Word)}. \]
  \item[\code{SANE\_TYPE\_BOOL}:] The size must be set to
    \code{sizeof(SANE\_Word)}.
  \item[\code{SANE\_TYPE\_BUTTON}, \code{SANE\_TYPE\_GROUP}:] The
    option size is ignored.
  \end{description}
\end{quote}

\subsubsection{Option Capabilities}

Member \code{cap} describes what capabilities the option posseses.
This is a bitset that is formed as the inclusive logical OR of the
capabilities described in Table \ref{tab:capabilities}.  The SANE API
provides the following to macros to test certain features of a given
capability bitset:
\begin{quote}
  \begin{description}

  \item[\code{\defn{SANE\_OPTION\_IS\_ACTIVE}(\var{cap})}:] This macro
    returns \code{SANE\_TRUE} if and only if the option with the
    capability set \var{cap} is currently active.

  \item[\code{\defn{SANE\_OPTION\_IS\_SETTABLE}(\var{cap})}:] This
    macro returns \code{SANE\_TRUE} if and only if the option with the
    capability set \var{cap} is software settable.
  \end{description}
\end{quote}

\begin{table}[htbp]
  \begin{center}
    \leavevmode
    \begin{tabular}{|l|r|p{0.59\textwidth}|}
\hline
\multicolumn{1}{|c|}{\bf Symbol} &
\multicolumn{1}{c|}{\bf Code} &
\multicolumn{1}{c|}{\bf Description} \\
\hline\hline

\code{\defn{SANE\_CAP\_SOFT\_SELECT}} & 1 & The option
  value can be set by a call to \code{sane\_con\-trol\_opt\-ion()}.\\ 

\code{\defn{SANE\_CAP\_HARD\_SELECT}} & 2 & The option value can be set by
  user-intervention (e.g., by flipping a switch).  The user-interface
  should prompt the user to execute the appropriate action to set such
  an option.  This capability is mutually exclusive with
  SANE\_CAP\_SOFT\_SELECT (either one of them can be set, but not both
  simultaneously).  \\ 

\code{\defn{SANE\_CAP\_SOFT\_DETECT}} & 4 & The option
  value can be detected by software.  If
  \code{SANE\_\-CAP\_\-SO\-FT\_SEL\-ECT} is set, this capability {\em must\/}
  be set.  If \code{SANE\_CAP\_HARD\_SELECT} is set, this capability
  may or may not be set.  If this capability is set but neither
  \code{SANE\_CAP\_SO\-FT\_SEL\-ECT} nor \code{SANE\_CAP\_HA\-RD\_SEL\-ECT}
  are, then there is no way to control the option.  That is, the
  option provides read-out of the current value only.  \\ 

\code{\defn{SANE\_CAP\_EMULATED}} & 8 & If set, this capability indicates
  that an option is not directly supported by the device and is
  instead emulated in the backend.  A sophisticated frontend may
  elect to use its own (presumably better) emulation in lieu of an emulated
  option. \\ 

\code{\defn{SANE\_CAP\_AUTOMATIC}} & 16 & If set, this capability indicates
  that the backend (or the device) is capable to picking a reasonable
  option value automatically.  For such options, it is possible to
  select automatic operation by calling \code{sane\_control\_option()}
  with an action value of \code{SANE\_ACTION\_SET\_AUTO}.  \\ 

\code{\defn{SANE\_CAP\_INACTIVE}} & 32 & If set, this capability indicates
  that the option is not currently active (e.g., because it's
  meaningful only if another option is set to some other value). \\ 

\code{\defn{SANE\_CAP\_ADVANCED}} & 64 &
  If set, this capability indicates that the option should be
  considered an ``advanced user option''. 
\begin{changebar}
  If this capability is set for an option of type
  \code{SANE\_TYPE\_GROUP}, all options belonging to the group are also
  advanced, even if they don't set the capabilty themselves. 
\end{changebar}
  A frontend typically
  displays such options in a less conspicuous way than regular options
  (e.g., a command line interface may list such options last or a
  graphical interface may make them available in a seperate ``advanced
  settings'' dialog). 
  \\

\code{\defn{SANE\_CAP\_HIDDEN}} & 128 &
\begin{changebar}
  If set, this capability indicates that the option shouldn't be
  displayed to and used by the user directly. Instead a hidden option
  is supposed to be automatically used by the frontend, like e.g. the
  ``preview'' option.
  If this capability is set for an option of type
  \code{SANE\_TYPE\_GROUP}, all options belonging to the group are also
  hidden, even if they don't set the capabilty themselves. 
  A frontend typically doesn't display such options by default but there
  should be a way to override this default behaviour.
\end{changebar}
  \\

\code{\defn{SANE\_CAP\_ALWAYS\_SETTABLE}} & 256 & 
\begin{changebar}
  If set, this capability indicates that the option may be at any time between
  \code{sane\_open()} and \code{sane\_close()}. I.e.  it's allowed to set it
  even while an image is acquired.
\end{changebar}
\\

\hline
    \end{tabular}
    \caption{Option Capabilities}
    \label{tab:capabilities}
  \end{center}
\end{table}

\subsubsection{Option Value Constraints}

It is often useful to constrain the values that an option can take.
For example, constraints can be used by a frontend to determine how to
represent a given option.  Member \code{constraint\_type} indicates
what constraint is in effect for the option.  The constrained values
that are allowed for the option are described by one of the union
members of member \code{constraint}.  The possible values of type
\code{\defn{SANE\_Constraint\_Type}} and the interpretation of the
\code{constraint} union is described in Table~\ref{tab:constraints}.

\begin{table}[htbp]
  \begin{center}
    \leavevmode
    \begin{tabular}{|l|r|p{0.5\textwidth}|}
\hline
\multicolumn{1}{|c|}{\bf Symbol} &
\multicolumn{1}{|c|}{\bf Code} &
\multicolumn{1}{|c|}{\bf Description} \\

\hline\hline

\code{\defn{SANE\_CONSTRAINT\_NONE}} & 0 & The value is unconstrained.
  The option can take any of the values possible for the option's
  type. \\

  \code{\defn{SANE\_CONSTRAINT\_RANGE}} & 1 & This constraint is
  applicable to integer and fixed-point valued options only.  It
  constrains the option value to a possibly quantized range of
  numbers.  Option descriptor member \code{constraint.range} points to
  a range of the type \code{\defn{SANE\_Range}}.  This type is illustrated
  below:
  \begin{quote}
\begin{verbatim}
typedef struct
  {
    SANE_Word min;
    SANE_Word max;
    SANE_Word quant;
  }
SANE_Range;
\end{verbatim}
  \end{quote}
  All three members in this structure are interpreted according to the
  option value type (\verb|SANE_TYPE_INT| or \verb|SANE_TYPE_FIXED|).
  Members \code{min} and \code{max} specify the minimum and maximum
  values, respectively.  If member \code{quant} is non-zero, it
  specifies the quantization value.  If $l$ is the minimum value, $u$
  the maximum value and $q$ the (non-zero) quantization of a range,
  then the legal values are $v=k\cdot q+l$ for all non-negative
  integer values of $k$ such that $v<=u$.  \\ 

\code{\defn{SANE\_CONSTRAINT\_WORD\_LIST}} & 2 & This constraint is applicable
  to integer and fixed-point valued options only.  It constrains the
  option value to a list of numeric values.  Option descriptor member
  \code{constraint.word\_list} points to a list of words that
  enumerates the legal values.  The first element in that list is an
  integer (\verb|SANE_Int|) that specifies the length of the list (not
  counting the length itself).  The remaining elements in the list are
  interpreted according to the type of the option value
  (\verb|SANE_TYPE_INT| or \verb|SANE_TYPE_FIXED|). \\ 

\code{\defn{SANE\_CONSTRAINT\_STRING\_LIST}} & 3 & This constraint is
  applicable to string-valued options only.  It constrains the option
  value to a list of strings.  The option descriptor member
  \code{con\-strai\-nt.str\-ing\_list} points to a \code{NULL} terminated
  list of strings that enumerate the legal values for the option
  value.
\\\hline
    \end{tabular}
    \caption{Option Value Constraints}
    \label{tab:constraints}
  \end{center}
\end{table}


\subsection{Internationalization}\label{sec:i18n}
\begin{changebar}
All backend texts should be written in english.
Localization (translation of backend texts) has to be done
in the frontend. To automatically prepare translation tables
(e.g. english to german) it is necessary to mark the texts
that shall be translated.

\subsubsection{How is a text marked for translation}

The keyword \code{SANE\_I18N} is used to mark a text for translation.
\code{SANE\_I18N} has to be defined as a dummy prototype:

  \begin{quote}
\begin{verbatim}
#ifndef SANE_I18N
#  define SANE_I18N(text) text
#endif
\end{verbatim}
  \end{quote}

You should not mark prototypes or variables with \code{SANE\_I18N}
because it is not possible (or very hard) to find out the
texts that are covered by prototypes (\verb|SANE_I18N(START_SCAN_TEXT)|) and
variables (\verb|SANE_I18N(option[7].name)|).

A correct marked text can look like this:\\
\verb|snprintf(buf, sizeof(buf), SANE_I18N("Start scan")|\\
or\\
\verb|#define START_SCAN_TEXT SANE_I18N("Start scan")|.\\
It also is allowed to mark texts in structs because the
prototype \code{SANE\_I18N} has no effect to the compiler.

\subsubsection{Which texts shall be marked for translation?}
All option texts that are visible for the user should be marked for
translation. This is:
\begin{itemize}
\item
member \code{title}
\item
member \code{desc}
\item
member \code{string\_list}
\end{itemize}
of \code{SANE\_Option\_Descriptor}.

It is not allowed to mark/translate member \code{name}.
Please also do not mark any error or debug messages that are
displayed by the backend.

\subsubsection{File formats and translation functions}
The recommended file formats for translation tables are the \code{po} files
and \code{mo} or \code{gmo} files.
The po file contains the original text marked with the keyword \code{msgid}
and the translated text marked with the keyword \code{msgstr}.
A \code{po} file that contains all needed \code{msgid}s can be created
e.g. by the gnu gettext tool \verb|xgettext| with the option
\verb|-k SANE_I18N|.
The translator adds the translated texts to the \code{po} files.
The gettext tool \code{msgfmt} converts the \code{po} files to the
\code{mo} or \code{gmo} files.\\

Translation is done in the frontend. A backend has nothing
to do with the translation of texts.
The frontend should use the function \code{gettext}
to translate the texts. This e.g. can look like this:\\
\verb|snprintf(buf, sizeof(buf), gettext("english text"));|

If a frontend author decides to use translation functions that
need different translation tables, then the frontend is
responsible to create/convert the translation tables.
In this case it should use the \code{po} files to create its
own translation tables from it.\\

Note that it is strongly recommended to use the gettext tools.
\end{changebar}


\section{Operations}

\subsection{\code{sane\_init}}

This function must be called before any other SANE function can be called.
The behavior of a SANE backend is undefined if this function is not called
first or if the status code returned by \code{sane\_init} is different from
\code{\defn{SANE\_STATUS\_GOOD}}.  The version code of the backend is returned
in the value pointed to by \code{version\_code}.  If that pointer is
\code{NULL}, no version code is returned.  Argument \code{authorize} is either
a pointer to a function that is invoked when the backend requires
authentication for a specific resource or \code{NULL} if the frontend does not
support authentication.
\begin{quote}\index{sane\_init}
\begin{verbatim}
SANE_Status sane_init (SANE_Int * version_code,
                       SANE_Authorization_Callback authorize);
\end{verbatim}
\end{quote}

The authorization function may be called by a backend in response to
any of the following calls:
\begin{quote}
  \code{sane\_open}, \code{sane\_control\_option}, \code{sane\_start}
\end{quote}
If a backend was initialized without authorization function, then
authorization requests that cannot be handled by the backend itself
will fail automatically and the user may be prevented from accessing
protected resources.  Backends are encouraged to implement means of
authentication that do not require user assistance.  E.g., on a
multi-user system that authenticates users through a login process a
backend could automatically lookup the apporpriate password based on
resource- and user-name.

The authentication function type has the following declaration:
\begin{quote}\index{SANE\_Authorization\_Callback}
  \index{domain}\index{username}\index{password}
\begin{verbatim}
#define SANE_MAX_USERNAME_LEN   128
#define SANE_MAX_PASSWORD_LEN   128

typedef void (*SANE_Authorization_Callback)
    (SANE_String_Const resource,
     SANE_Char username[SANE_MAX_USERNAME_LEN],
     SANE_Char password[SANE_MAX_PASSWORD_LEN]);
\end{verbatim}
\end{quote}
Three arguments are passed to the authorization function:
\code{resource} is a string specifying the name of the resource that
requires authorization.  A frontend should use this string to build a
user-prompt requesting a username and a password.  The \code{username}
and \code{password} arguments are (pointers to) an array of
\code{SANE\_MAX\_USERNAME\_LEN} and \code{SANE\_MAX\_PASSWORD\_LEN}
characters, respectively.  The authorization call should place the
entered username and password in these arrays.  The returned strings
{\em must\/} be ASCII-NUL terminated.

\subsection{\code{sane\_exit}}

This function must be called to terminate use of a backend.  The
function will first close all device handles that still might be open
(it is recommended to close device handles explicitly through a call
to \code{sane\_clo\-se()}, but backends are required to release all
resources upon a call to this function).  After this function returns,
no function other than \code{sane\_init()} may be called (regardless
of the status value returned by \code{sane\_exit()}.  Neglecting to
call this function may result in some resources not being released
properly.
\begin{quote}\index{sane\_exit}
\begin{verbatim}
void sane_exit (void);
\end{verbatim}
\end{quote}


\subsection{\code{sane\_get\_devices}}

This function can be used to query the list of devices that are
available.  If the function executes successfully, it stores a pointer
to a \code{NULL} terminated array of pointers to \verb|SANE_Device|
structures in \code{*device\_list}.  The returned list is guaranteed
to remain unchanged and valid until (a) another call to this function
is performed or (b) a call to \code{sane\_exit()} is performed.  This
function can be called repeatedly to detect when new devices become
available.  If argument \code{local\_only} is true, only local devices
are returned (devices directly attached to the machine that SANE is
running on).  If it is false, the device list includes all remote
devices that are accessible to the SANE library.
\begin{quote}\index{sane\_get\_devices}
\begin{verbatim}
SANE_Status sane_get_devices (const SANE_Device *** device_list,
                              SANE_Bool local_only);
\end{verbatim}
\end{quote}

This function may fail with \code{SANE\_STATUS\_NO\_MEM} if an
insufficient amount of memory is available.

\begin{quote}
  \begin{center}
    {\bf Backend Implementation Note}
  \end{center}
  SANE does not require that this function is called before a
  \code{sane\_open()} call is performed.  A device name may be
  specified explicitly by a user which would make it unnecessary and
  undesirable to call this function first.
\begin{changebar}
  The same information about
  a device has to be returned when \code{sane\_open} is called.
\end{changebar}
\end{quote}


\subsection{\code{sane\_open}}

This function is used to establish a connection to a particular
device.  The name of the device to be opened is passed in argument
\code{name}.  If the call completes successfully, a handle for the
device is returned in \code{*h}.
\begin{changebar}
The description of the device
is returned in \code{**device\_description}. This is the
same description that is returned in the list by \code{sane\_get\_devices}.
The returned data \code{*h} and \code{*device\_description} is guaranteed
to remain unchanged and valid until a call to \code{sane\_close()}
is performed.
\end{changebar}
As a special case, specifying a zero-length string as the device requests
opening the first available device (if there is such a device).
\begin{quote}\index{sane\_open}
\begin{verbatim}
SANE_Status sane_open (SANE_String_Const name, SANE_Handle * h,
const SANE_Device ** device_description);
\end{verbatim}
\end{quote}

This function may fail with one of the following status codes.
\begin{quote}
\begin{description}
\item[\code{SANE\_STATUS\_DEVICE\_BUSY}:] The device is currently
  busy (in use by somebody else).
\item[\code{SANE\_STATUS\_INVAL}:] The device name is not valid.
\item[\code{SANE\_STATUS\_IO\_ERROR}:] An error occured while
  communicating with the device.
\item[\code{SANE\_STATUS\_NO\_MEM}:] An insufficent amount of memory
  is available.
\item[\code{SANE\_STATUS\_ACCESS\_DENIED}:] Access to the device has
  been denied due to insufficient or invalid authentication.
\end{description}
\end{quote}


\subsection{\code{sane\_close}}

This function terminates the association between the device handle
passed in argument \code{h} and the device it represents.  If the
device is presently active, a call to \code{sane\_cancel()} is
performed first.  After this function returns, handle \code{h} must
not be used anymore.

\begin{quote}\index{sane\_close}
\begin{verbatim}
void sane_close (SANE_Handle h);
\end{verbatim}
\end{quote}

\subsection{\code{sane\_get\_option\_descriptor}}

This function is used to access option descriptors.  The function
returns the option descriptor for option number \code{n} of the device
represented by handle \code{h}.  Option number 0 is guaranteed to be a
valid option.  Its value is an integer that specifies the number of
options that are available for device handle \code{h} (the count
includes option 0).  If $n$ is not a valid option index, the function
returns \code{NULL}.  The returned option descriptor is guaranteed to
remain valid (and at the returned address) until the device is closed.

\begin{quote}\index{sane\_get\_option\_descriptor}
\begin{verbatim}
const SANE_Option_Descriptor *
    sane_get_option_descriptor (SANE_Handle h, SANE_Int n);
\end{verbatim}
\end{quote}

\subsection{\code{sane\_control\_option}}\label{sec:control}

This function is used to set or inquire the current value of option
number \code{n} of the device represented by handle \code{h}.  The
manner in which the option is controlled is specified by parameter
\code{a}.  The possible values of this parameter are described in more
detail below.  The value of the option is passed through argument
\code{v}.  It is a pointer to the memory that holds the option value.
The memory area pointed to by \code{v} must be big enough to hold the
entire option value (determined by member \code{size} in the
corresponding option descriptor).  The only exception to this rule is
that when setting the value of a string option, the string pointed to
by argument \code{v} may be shorter since the backend will stop
reading the option value upon encountering the first \code{NUL}
terminator in the string.  If argument \code{i} is not \code{NULL},
the value of \code{*i} will be set to provide details on how well the
request has been met.  The meaning of this argument is described in
more detail below.
\begin{quote}\index{sane\_control\_option}
\begin{verbatim}
SANE_Status sane_control_option (SANE_Handle h, SANE_Int n,
                                 SANE_Action a, void *v,
                                 SANE_Int * i);
\end{verbatim}
\end{quote}

The way the option is affected by a call to this function is
controlled by parameter \code{a} which is a value of type
\code{\defn{SANE\_Action}}.  The possible values and their meaning is
described in Table~\ref{tab:actions}.

\begin{table}[h]
  \begin{center}
    \leavevmode
    \begin{tabular}{|l|r|p{0.5\textwidth}|}
\hline
\multicolumn{1}{|c|}{\bf Symbol} &
\multicolumn{1}{|c|}{\bf Code} &
\multicolumn{1}{|c|}{\bf Description} \\

\hline\hline

\code{\defn{SANE\_ACTION\_GET\_VALUE}} & 0 & Get current option value. \\

\code{\defn{SANE\_ACTION\_SET\_VALUE}} & 1 & Set option value.  The
  option value passed through argument \code{v} may be modified by the
  backend if the value cannot be set exactly.  \\ 

\code{\defn{SANE\_ACTION\_SET\_AUTO}} & 2 & Turn on automatic mode.  Backend
  or device will automatically select an appropriate value.  This mode
  remains effective until overridden by an explicit set value request.
  The value of parameter \code{v} is completely ignored in this case and
  may be \code{NULL}.  \\

\hline
    \end{tabular}
    \caption{Action Values (\code{SANE\_Action})}
    \label{tab:actions}
  \end{center}
\end{table}

After setting a value via an action value of
\verb|SANE_ACTION_SET_VALUE|, additional information on how well the
request has been met is returned in \code{*i} (if \code{i} is
non-\code{NULL}).  The returned value is a bitset that may contain any
combination of the values described in Table~\ref{tab:info}.
\begin{table}[htbp]
  \begin{center}
    \leavevmode
    \begin{tabular}{|l|r|p{0.5\textwidth}|}
\hline
\multicolumn{1}{|c|}{\bf Symbol} &
\multicolumn{1}{|c|}{\bf Code} &
\multicolumn{1}{|c|}{\bf Description} \\

\hline\hline

\code{\defn{SANE\_INFO\_INEXACT}} & 1 & This value is returned when
  setting an option value resulted in a value being selected that does
  not exactly match the requested value.  For example, if a scanner
  can adjust the resolution in increments of 30dpi only, setting the
  resolution to 307dpi may result in an actual setting of 300dpi.
  When this happens, the bitset returned in \code{*i} has this member
  set.  In addition, the option value is modified to reflect the
  actual (rounded) value that was used by the backend.  Note that
  inexact values are admissible for strings as well.  A backend may
  choose to ``round'' a string to the closest matching legal string
  for a constrained string value. \\

\code{\defn{SANE\_INFO\_RELOAD\_OPTIONS}} & 2 & 
The setting of an option may affect the
\begin{changebar}
  value, the availability, or the constraint of one or more {\em
   other\/} options.
\end{changebar}
  When this happens, the SANE backend sets this
  member in \code{*i} to indicate that the application should reload
  all options.  This member may be set if and only if at least one
  option changed. \\

\code{\defn{SANE\_INFO\_RELOAD\_PARAMS}} & 4 &
The setting of an option may affect the parameter values (see
\code{sane\_get\_parameters()}).
If setting an option affects the parameter values, this member will
be set in \code{*i}.  Note that this member may be set even if the
parameters did not actually change.  However, it is guaranteed that
the parameters never change without this member being set.
\\

\code{\defn{SANE\_INFO\_INVALIDATE\_PREVIEW}} & 8 &
The setting of an option may affect the validity of the preview that 
was acquired by the frontend earlier. When the preview image would change
significantly if it was scanned again, the SANE backend
sets this member in \code{*i} to indicate that the application should inform the
user of the invalidity of the preview. Examples for such option
settings may include setting a different scan source or significantly
changing the exposure.
\\


\hline
    \end{tabular}
    \caption{Additional Information Returned When Setting an Option}
    \label{tab:info}
  \end{center}
\end{table}

This function may fail with one of the following status codes.
\begin{quote}
\begin{description}
\item[\code{SANE\_STATUS\_UNSUPPORTED}:] The operation is not
  supported for the specified handle and option number.
\item[\code{SANE\_STATUS\_INVAL}:] The option value is not valid.
\item[\code{SANE\_STATUS\_IO\_ERROR}:] An error occured while
  communicating with the device.
\item[\code{SANE\_STATUS\_NO\_MEM}:] An insufficent amount of memory
  is available.
\item[\code{SANE\_STATUS\_ACCESS\_DENIED}:] Access to the option has
  been denied due to insufficient or invalid authentication.
\end{description}
\end{quote}



\subsection{\code{sane\_get\_parameters}\label{sec:sanegetparameters}}

This function is used to obtain the current scan parameters.  The
returned parameters are guaranteed to be accurate between the time a
scan has been started (\code{sane\_start()} has been called) and the
completion of that request.  Outside of that window, the returned
values are best-effort estimates of what the parameters will be when
\code{sane\_start()} gets invoked.  Calling this function before a
scan has actually started allows, for example, to get an estimate of
how big the scanned image will be.  The parameters passed to this
function are the handle \code{h} of the device for which the
parameters should be obtained and a pointer \code{p} to a parameter
structure.  The parameter structure is described in more detail below.

\begin{quote}\index{sane\_get\_parameters}
\begin{verbatim}
SANE_Status sane_get_parameters (SANE_Handle h,
                                 SANE_Parameters * p);
\end{verbatim}
\end{quote}

The scan parameters are returned in a structure of type
\code{\defn{SANE\_Parameters}}.  The C declaration of this structure
is given below.
\begin{changebar}
\begin{quote}
\begin{verbatim}
typedef struct
  {
    SANE_Frame format;
    SANE_Int flags;
    SANE_Int lines;
    SANE_Int depth;
    SANE_Int pixels_per_line;
    SANE_Int bytes_per_line;
    SANE_Int channels_per_image;
    SANE_String format_desc;
    SANE_String proposed_filename;
    SANE_String proposed_comment;
    SANE_Int dpi_x;
    SANE_Int dpi_y;
    char reserved[32]; /* 32 bytes for future use */
  }
SANE_Parameters;
\end{verbatim}
\end{quote}
\end{changebar}

\begin{changebar}
Member \code{format} specifies the format of the next frame to be
returned.  The possible values for type \code{\defn{SANE\_Frame}} are
described in Table~\ref{tab:frameformat}.  The meaning of these
values is described in more detail in section~\ref{sec:imageformat}.
The frame types used by version~1 of this standard
(\code{\defn{SANE\_FRAME\_GRAY}}, \code{\defn{SANE\_FRAME\_RGB}},
\code{\defn{SANE\_FRAME\_RED}}, \code{\defn{SANE\_FRAME\_GREEN}}, and
\code{\defn{SANE\_FRAME\_BLUE}}) are obsolete and superseded by the 
frame type \code{\defn{SANE\_FRAME\_RAW}}.

\begin{table}[htbp]
  \begin{center}
    \leavevmode
    \begin{tabular}{|l|r|l|}
\hline
\multicolumn{1}{|c|}{\bf Symbol} &
\multicolumn{1}{|c|}{\bf Code} &
\multicolumn{1}{|c|}{\bf Description} \\

\hline\hline

\code{\defn{SANE\_FRAME\_RAW}}   & 5 & Arbitrary pixel property transmission. \\
\code{\defn{SANE\_FRAME\_MIME}}  & 6 & Data described by a mime descriptor. \\

\hline
    \end{tabular}
    \caption{Frame Format (\code{SANE\_Frame})}
    \label{tab:frameformat}
  \end{center}
\end{table}

The \code{flags} member is a 32 bit bitfield, for which up to now 4 
informational bits are defined, all unused bits have to be set to 0: 

\begin{itemize}

\item
\code{SANE\_PFLAG\_LAST\_FRAME} (bit 0, bitvalue 1) is set to 1 if and
only if the frame that is currently being acquired (or the frame that
will be acquired next if there is no current frame) is the only frame
or the last frame
of a multi frame image (e.g., the current frame is the blue component
of a red, green, blue image).

\item
\code{SANE\_PFLAG\_MORE\_IMAGES} (bit 1, bitvalue 2) is set to 1 to indicate 
further pending images. The frontend is expected to call \code{sane\_start}
again after the end of the current scan to get more images, e.g.
from an automatic document feeder.  It is permissible to set that value to 1 ``in good 
faith'', as it has to be determined at a very early time, where it might 
not be detectable, if there actually are more images to transfer. E.g. 
you will usually not know if the document feeder contains further pages 
when starting to scan the current one. Thus you are allowed to set that 
bit but later fail at \code{sane\_start()}.

\item
\code{SANE\_PFLAG\_NEW\_PAGE} (bit 2, bitvalue 4) is set to 1 to indicate 
that the current frame comes from a new physical page. This bit is of 
informational character only to help frontends to group multi-image 
scans. 

\item
\code{SANE\_PFLAG\_BACKSIDE} (bit 3, bitvalue 8) tells if the current image 
was acquired from the front (0) or backside (1) of the currently processed 
sheet. It is of informational character and allows to group and order 
multi-image transfers regardless of scanner acquisition order (front 
first/back first). 

\end{itemize}

Note, that \code{flags} is compatible to member \code{last\_frame} of 
\code{SANE\_Parameters} of SANE standard version 1 (same size
and only bit 0 (bitvalue 1) was used with same function).
\end{changebar}

Member \code{lines} specifies how many scan lines the frame is
comprised of.  If this value is -1, the number of lines is not known a
priori and the frontend should call \code{sane\_read()} until it
returns a status of \code{SANE\_STATUS\_EOF}.
\begin{changebar}
Note, that even when transferring formats that have this information 
inband, it is recommended to set that member, if available. If 
unavailable or not applicable, set to -1 as mentioned above. 
\end{changebar}

Member \code{bytes\_per\_line} specifies the number of bytes that comprise one
scan line.
\begin{changebar}
If this value is not applicable for image data of type
\code{SANE\_FRAME\_MIME}, it must be set to -1. In this case the frontend
should just call \code{sane\_read} until \code{SANE\_STATUS\_EOF} is returned.
\end{changebar}

Member \code{depth} specifies the number of bits per sample.
\begin{changebar}
Note, that only -1 (for not applicable), 1, and multiples of 8 are
allowed values. Data with other depths has to be scaled up accordingly.
Depth 1 is only allowed if the image consists of a single channel (lineart or
halftone modes).
\end{changebar}

Member \code{pixels\_per\_line} specifies the number of pixels that
comprise one scan line.
\begin{changebar}
If unavailable or not applicable, set to -1.
\end{changebar}

Assume $B$ is the number of channels in the frame, then the bit depth
$d$ (as given by member \code{depth}) and the number of pixels per
line $n$ (as given by this member \code{pixels\_per\_line}) are
related to $c$, the number of bytes per line (as given by member
\code{bytes\_per\_line}) as follows:
\begin{changebar}
\[
  c >= \left\{
  \begin{array}{ll}
    \lceil B\cdot n / 8\rceil & \mbox{if $d=1$}\\
    B\cdot n \cdot d / 8 & \mbox{if $d>1$}
  \end{array}
  \right.
\]
\end{changebar}
Note that the number of bytes per line can be larger than the minimum
value imposed by the right side of this equation.  A frontend must be
able to properly cope with such ``padded'' image formats.

\begin{changebar}
Member \code{channels\_per\_image} specifies the number of channels the
image consists of. When the image is transmitted in more than one frame
\code{channels\_per\_image} has to be the same for all frames for this image.

Member \code{format\_desc} is used to describe the details of the frame
formats.  Its meaning differs between the two formats:

\begin{itemize}
\item
\code{SANE\_FRAME\_RAW}: 
The \code{format\_desc} contains a description of the channel data and
an optional depth information separated by a colon(:).

A plane is descibed by one channel, e.g. ``\code{gray}'' or ``\code{gray:12}''.

Channel interleaved data is described by a comma separated list of channel descriptions,
for example ``\code{red,green,blue}'' or ``\code{red:8,green:8,blue:8}'',
the channel data is sent in the given order.

The depth information does {\bf not} define the size of the transmitted
channel data, it is only an information for the frontend. The channel data has
to be sent in the size defined by member \code{depth}.

Well known channels are \code{red}, \code{green}, \code{blue} and \code{gray}.
It also is allowed to use other channel descriptions, e.g. if you
use an infrared camera or scanner it could be \code{infrared} or
a wavelength description like \code{1100nm}, but be aware that a
frontend may not be able to display such channels with useful colors.

Note that an image can be sent in single planes, in one interleaved
frame that contains all channels or in several frames that contain
one or more (interleaved) channels. When an RGB image is sent it
is prefered to send the image data in one interleaved frame
that consist of red, green and blue data in this order.
The number of channels is defined in member \code{channels\_per\_image}.

\item
\code{SANE\_FRAME\_MIME}: 

\code{format\_desc} contains the MIME Content-Type: header field as described
in RFC 1521 (section 4) without the prefixing "Content-Type:". MIME Types and
subtypes should be either chosen from the RFC or from the list of
IANA-approved values. The data stream must be compliant with the corresponding
specification.

When data is transmitted with the frame type \code{SANE\_FRAME\_MIME}
all data has to be transmitted within one frame, multiple frames
are not allowed (so the flag \code{last\_frame} has to be set
when using this frame type).

A SANE backend must be able to at least optionally transmit
\code{SANE\_FRAME\_RAW} (possibly with the help of a meta backend), if the
hardware supports delivering image data at all. For data that doesn't
comprise images, it's admisable to only provide MIME frames. As a
general principle, if there are several choices in MIME types, the
format that is most widely implemented should be used.
\end{itemize}

The member \code{proposed\_filename} can be used to suggest a reasonable 
default filename or -extension in case the backend can make such a 
suggestion, like e.g. an image database. 
If no such suggestion is intended, set the field to ``''. 

In the case of raw frames, \code{proposed\_filename} is expected to hold
the basename for the image, with the extension determined by the save function
of the frontend, as the frontend can fully understand the data and is thus
able to encode it in any format it wishes.

For MIME frames \code{proposed\_filename} can contain either: 

\begin{itemize}
\item
A name with a leading dot, which is considered to be a proposed 
filename extension. This could also be gotten from the mime database, 
but for systems lacking it, this might be convenient. Or: 

\item
A complete filename, including extension. 
\end{itemize}

Note, that for frontends that are able to parse a given MIME type 
internally, it is perfectly permissible to ignore the extension 
part of the proposed filename and only make use of the basename, 
when using internal save algorithms for different formats. 

The string \code{proposed\_comment} can be used to transmit additional 
image information, that can be stored in the comment areas several file formats 
offer. It can contain any textual information the backend wishes to 
convey to the user, like date/time of exposure, enganged filters, 
etc. Set to ``'' if unused.

The members \code{dpi\_x} and \code{dpi\_y} encode the horizontal and vertical
resolution. Note, that multiple-image scans may have different resolutions of
each image. It is not permissible to change resolution between frames of the
same image. If not applicable, set to -1.

The member \code{reserved} is an array of 32 bytes (char) to keep
the size of the struct unchanged when future extensions are done.
The backend has to set the reserved bytes to 0.
\end{changebar}


\subsection{\code{sane\_start}}

This function initiates aquisition of an image from the device
represented by handle \code{h}.
\begin{quote}\index{sane\_start}
\begin{verbatim}
SANE_Status sane_start (SANE_Handle h);
\end{verbatim}
\end{quote}
This function may fail with one of the following status codes.
\begin{quote}
\begin{description}
\item[\code{SANE\_STATUS\_CANCELLED}:] The operation was cancelled through
  a call to \code{sane\_cancel}.
\item[\code{SANE\_STATUS\_DEVICE\_BUSY}:] The device is busy. The
  operation should be retried later.
\item[\code{SANE\_STATUS\_JAMMED}:] The document feeder is jammed.
\item[\code{SANE\_STATUS\_NO\_DOCS}:] The document feeder is out of
  documents.
\item[\code{SANE\_STATUS\_COVER\_OPEN}:] The scanner cover is open.
\item[\code{SANE\_STATUS\_IO\_ERROR}:] An error occurred while communicating
  with the device.
\item[\code{SANE\_STATUS\_NO\_MEM}:] An insufficent amount of memory
  is available.
\item[\code{SANE\_STATUS\_INVAL}:] The scan cannot be started with the current
  set of options. The frontend should reload the option descriptors, as if
  \code{\defn{SANE\_INFO\_RELOAD\_OPTIONS}} had been returned from a call to
  \code{sane\_control\_option()}, since the device's capabilities may have
  changed.
\end{description}
\end{quote}


\subsection{\code{sane\_read}}

This function is used to read image data from the device represented
by handle \code{h}.  Argument \code{buf} is a pointer to a memory area
that is at least \code{maxlen} bytes long.  The number of bytes
returned is stored in \code{*len}.  A backend must set this to zero
when a status other than \code{SANE\_STA\-TUS\_GOOD} is returned).
When the call succeeds, the number of bytes returned can be anywhere in
the range from 0 to \code{maxlen} bytes.
\begin{quote}\index{sane\_read}
\begin{verbatim}
SANE_Status sane_read (SANE_Handle h, SANE_Byte * buf,
                       SANE_Int maxlen, SANE_Int * len);
\end{verbatim}
\end{quote}
\begin{changebar}
For efficiency reasons, medium to large 
block sizes (in the range of a few kilobytes) should be used. 
Returning short reads is allowed to allow for small buffers 
in the backend. 
\end{changebar}
If this function is called when no data is available, one of two
things may happen, depending on the I/O mode that is in effect for
handle \code{h}.
\begin{enumerate}
\item If the device is in blocking I/O mode (the default mode), the
  call blocks until at least one data byte is available (or until some
  error occurs).

\item If the device is in non-blocking I/O mode, the call returns
  immediately with status \code{SANE\_STA\-TUS\_GOOD} and with
  \code{*len} set to zero.
\end{enumerate}
The I/O mode of handle \code{h} can be set via a call to
\code{sane\_set\_io\_mode()}.

This function may fail with one of the following status codes.
\begin{quote}
\begin{description}
\item[\code{SANE\_STATUS\_CANCELLED}:] The operation was cancelled through
  a call to \code{sane\_cancel}.
\item[\code{SANE\_STATUS\_EOF}:] No more data is available for the
  current frame.
\begin{changebar}
If \code{sane\_read} sends back any image data it
is not allowed to return with \code{SANE\_STATUS\_EOF}.
\end{changebar}
\item[\code{SANE\_STATUS\_JAMMED}:] The document feeder is jammed.
\item[\code{SANE\_STATUS\_NO\_DOCS}:] The document feeder is out of
  documents.
\item[\code{SANE\_STATUS\_COVER\_OPEN}:] The scanner cover is open.
\item[\code{SANE\_STATUS\_IO\_ERROR}:] An error occurred while communicating
  with the device.
\item[\code{SANE\_STATUS\_NO\_MEM}:] An insufficent amount of memory
  is available.
\item[\code{SANE\_STATUS\_ACCESS\_DENIED}:] Access to the device has
  been denied due to insufficient or invalid authentication.
\end{description}
\end{quote}


\subsection{\code{sane\_cancel}}

This function is used to immediately or as quickly as possible cancel
the currently pending operation of the device represented by handle
\code{h}.
\begin{quote}\index{sane\_cancel}
\begin{verbatim}
void sane_cancel (SANE_Handle h);
\end{verbatim}
\end{quote}
This function can be called at any time (as long as handle \code{h} is
a valid handle) but usually affects long-running operations only (such
as image is acquisition).  It is safe to call this function
asynchronously (e.g., from within a signal handler).  It is important
to note that completion of this operaton does {\em not\/} imply that
the currently pending operation has been cancelled.  It only
guarantees that cancellation has been {\em initiated}.  Cancellation
completes only when the cancelled call returns (typically with a
status value of \code{SANE\_STATUS\_CANCELLED}).  Since the SANE API
does not require any other operations to be re-entrant, this implies
that a frontend must {\em not\/} call any other operation until the
cancelled operation has returned.


\subsection{\code{sane\_set\_io\_mode}}

This function is used to set the I/O mode of handle \code{h}.  The I/O mode
can be either blocking or non-blocking.  If argument \code{m} is
\code{SANE\_TRUE}, the mode is set to non-blocking mode, otherwise it's set to
blocking mode.  This function can be called only after a call to
\code{sane\_start()} has been performed.
\begin{quote}\index{sane\_set\_io\_mode}
\begin{verbatim}
SANE_Status sane_set_io_mode (SANE_Handle h, SANE_Bool m);
\end{verbatim}
\end{quote}
By default, newly opened handles operate in blocking mode.  A backend
may elect not to support non-blocking I/O mode.  In such a case the
status value \code{SANE\_STATUS\_UNSUPPORTED} is returned.  Blocking
I/O must be supported by all backends, so calling this function with
argument \code{m} set to \code{SANE\_FALSE} is guaranteed to complete
successfully.

This function may fail with one of the following status codes:
\begin{quote}
\begin{description}
\item[\code{SANE\_STATUS\_INVAL}:] No image acquisition is pending.
\item[\code{SANE\_STATUS\_UNSUPPORTED}:] The backend does not support
the requested I/O mode.
\end{description}
\end{quote}


\subsection{\code{sane\_get\_select\_fd}}

This function is used to obtain a (platform-specific) file-descriptor
for handle \code{h} that is readable if and only if image data is
available (i.e., when a call to \code{sane\_read()} will return at
least one byte of data).  If the call completes successfully, the
select file-descriptor is returned in \code{*fd}.
\begin{quote}\index{sane\_get\_select\_fd}
\begin{verbatim}
SANE_Status sane_get_select_fd (SANE_Handle h, SANE_Int *fd);
\end{verbatim}
\end{quote}
This function can be called only after a call to \code{sane\_start()}
has been performed and the returned file-descriptor is guaranteed to
remain valid for the duration of the current image acquisition (i.e.,
until \code{sane\_cancel()} or \code{sane\_start()} get called again
or until \code{sane\_read()} returns with status
\code{SANE\_STA\-TUS\_EOF}).  Indeed, a backend must guarantee to
close the returned select file descriptor at the point when the next
\code{sane\_read()} call would return \code{SANE\_STA\-TUS\_EOF}.
This is necessary to ensure the application can detect when this
condition occurs without actually having to call \code{sane\_read()}.

A backend may elect not to support this operation.  In such a case,
the function returns with status code
\code{SANE\_STATUS\_UNSUPPORTED}.

Note that the only operation supported by the returned file-descriptor
is a host operating-system dependent test whether the file-descriptor
is readable (e.g., this test can be implemented using \code{select()}
or \code{poll()} under UNIX).  If any other operation is performed on
the file descriptor, the behavior of the backend becomes
unpredictable.  Once the file-descriptor signals ``readable'' status,
it will remain in that state until a call to \code{sane\_read()} is
performed.  Since many input devices are very slow, support for this
operation is strongly encouraged as it permits an application to do
other work while image acquisition is in progress.

This function may fail with one of the following status codes:
\begin{quote}
\begin{description}
\item[\code{SANE\_STATUS\_INVAL}:] No image acquisition is pending.
\item[\code{SANE\_STATUS\_UNSUPPORTED}:] The backend does not support
  this operation.
\end{description}
\end{quote}


\subsection{\code{sane\_strstatus}}

This function can be used to translate a SANE status code into a
printable string.  The returned string is a single line of text that
forms a complete sentence, but without the trailing period
(full-stop).  The function is guaranteed to never return \code{NULL}.
The returned pointer is valid at least until the next call to this
function is performed.
\begin{quote}\index{sane\_strstatus}
\begin{verbatim}
SANE_String_Const sane_strstatus (SANE_Status status);
\end{verbatim}
\end{quote}

\section{Code Flow}\index{code flow}

The code flow for the SANE API is illustrated in
Figure~\ref{fig:flow}.  Functions \code{sane\_init()} and
\code{sane\_exit()} initialize and exit the backend, respectively.
All other calls must be performed after initialization and before
exiting the backend.

\begin{figure}[htb]
  \begin{center}
    \leavevmode
    \psfig{file=figs/flow.eps,height=0.5\textheight}
    \caption{Code flow}
    \label{fig:flow}
  \end{center}
\end{figure}

Function \code{sane\_get\_devices()} can be called any time after
\code{sane\_init()} has been called.  It returns the list of the
devices that are known at the time of the call.  This list may change
over time since some devices may be turned on or off or a remote host
may boot or shutdown between different calls.  It should be noted that
this operation may be relatively slow since it requires contacting all
configured devices (some of which may be on remote hosts).  A frontend
may therefore want to provide the ability for a user to directly
select a desired device without requiring a call to this function.

Once a device has been chosen, it is opened using a call to
\code{sane\_open()}.  Multiple devices can be open at any given time.
A SANE backend must not impose artificial constraints on how many
devices can be open at any given time.

An opened device can be setup through the corresponding device handle
using functions \code{sane\_get\_opt\-ion\_desc\-riptor()} and
\code{sane\_control\_option()}.  While setting up a device, obtaining
option descriptors and setting and reading of option values can be
mixed freely.  It is typical for a frontend to read out all available
options at the beginning and then build a dialog (either graphical or
a command-line oriented option list) that allows to control the
available options.  It should be noted that the number of options is
fixed for a given handle. 
\begin{changebar}
  However, as options are set, other options may become active or inactive or
  their constraint may change. Thus, after setting an option, it may be
  necessary to re-read the descriptors.
\end{changebar}
While
setting up the device, it is also admissible to call
\code{sane\_get\_parameters()} to get an estimate of what the image
parameters will look like once image acquisition begins.

The device handle can be put in blocking or non-blocking mode by a
call to \code{sane\_set\_io\_mode()}.  Devices are required to support
blocking mode (which is the default mode), but support for
non-blocking I/O is strongly encouraged for operating systems such as
UNIX.

After the device is setup properly, image acquisition can be started
by a call to \code{sane\_start()}.  The backend calculates the exact
image parameters at this point.  So future calls to
\code{sane\_get\_parameters()} will return the exact values, rather
than estimates.  Whether the physical image acquisition starts at this
point or during the first call to \code{sane\_read()} is unspecified
by the SANE API.  If non-blocking I/O and/or a select-style interface
is desired, the frontend may attempt to call
\code{sane\_set\_io\_mode()} and/or \code{sane\_get\_select\_fd()} at
this point.  Either of these functions may fail if the backend does
not support the requested operation.

\begin{changebar}
Image data is collected by repeatedly calling \code{sane\_read()}
until this function will return an end-of-file status
(\code{SANE\_STATUS\_EOF}).  This indicates the end of the current
frame. If the frontend expects additional frames (e.g., the
individual channels of a red/green/blue image or multiple images),
it can call \code{sane\_start()} again.
If the \code{SANE\_PFLAG\_LAST\_FRAME} bit is set in \code{flags}, the
current image is complete. In this case, it should be tested, if
\code{flags} has the \code{SANE\_PFLAG\_MORE\_IMAGES} bit set.
If yes, further calls to \code{sane\_start()} can be made to acquire
more images. Please note, that as this bit has to be set at the beginning
of a the transmission of the last frame before the new image, it is possible,
that no reliable decision can be made at this time. It is thus permissible
for a backend to set this bit, and then later at the actual call to
\code{sane\_start()} return an error like \code{SANE\_STATUS\_NO\_DOCS}. 
Such a sequence is permitted to transmit multiple images from a single 
page as well as multiple pages. This behaviour should be controlled by 
backend options as required, to allow single-page scanning as well as 
ADF-batch-scanning. The frontend should always continue reading all images 
until a frame with \code{SANE\_PFLAG\_LAST\_FRAME} on
and \code{SANE\_PFLAG\_MORE\_IMAGES} off is encountered, or an error other
than \code{SANE\_STATUS\_EOF} occurs in a SANE function.
Note that \code{SANE\_STATUS\_NO\_DOCS} also is an allowed way for the backend
to indicate the end of a multiple image scan.

A frontend may choose to skip frames (e.g. because it cannot parse them), 
which is accomplished by simply calling \code{sane\_start} again, which will get 
you to the next frame, without having to read and discard the current one.

In order to prematurely stop scanning and to reset the backend state, 
\code{sane\_cancel()} can be called at any time. This call is required 
as well after normal termination of a multiple image scan as described above.
\end{changebar}

When done using the device, the handle should be closed by a call to
\code{sane\_close()}.  Finally, before exiting the application,
function \code{sane\_exit()} must be called.  It is important not to
forget to call this function since otherwise some resources (e.g.,
temporary files or locks) may remain unclaimed.

\begin{changebar}
The following C sample code implements a reference loop for acquiring 
multiple images:

\begin{verbatim}
SANE_Parameters parms;
SANE_Status     status;

do
  {
    do
      {
        /* Now start acquiring the next frame. */
        status = sane_start (handle);

        /* if that failed, we have a problem, and no more frames can be 
         * read at this time. Due to SANE_PFLAG_MORE_IMAGES still 
         * being clear, this will break out of _BOTH_ loops. 
         */
        if (status != SANE_STATUS_GOOD)
          break;

        /* Now let us see what the next frame brings. */
        status = sane_get_parameters (handle, &parms);

        /* This actually should not fail, but maybe the doc feeder 
         * jammed or something, so we break as well, if something 
         * is wrong. 
         */
        if (status != SANE_STATUS_GOOD)
          break;

        /* Now we check the announced parameters, if we can make use 
         * of the frame data. If not, we skip over to the next frame. 
         */
        if (do_i_like_that (&parms) == NO)
          continue;

        /* Set up for reading the data here. Mangle filenames, 
         * allocate memory, rewind multiframe files, ask user 
         * for confirmation, ... 
         */
        setup_for_transfer (...);

        /* Now we read in the frame data and process it. This should 
         * return SANE_STATUS_GOOD, until the frame is complete, 
         * what causes SANE_STATUS_EOF to be returned. 
         */
        while (SANE_STATUS_GOOD == (status = sane_read (...)))
          read_in_and_process_data_as_required ();

        /* If transfer was broken due to anything but EOF, break out. */
        if (status != SANE_STATUS_EOF)
          break;

        /* Now loop until we have all frames of an image. */
      }
    while (!(parms.flag & SANE_PFLAG_LAST_FRAME));

    /* O.K. - we now have a complete image. Fit it together, save it, 
     * flush buffers, transmit it, increment filenames, etc. 
     */

    /* Now check for more pending images. If we have more, redo from start. 
     * Some backends might cheat here and send us for an extra round which 
     * will fail at sane_start, as they were not able to determine if they 
     * would have more data at the start of the last frame we read. 
     */
  }
while (parms.flags & SANE_PFLAG_MORE_IMAGES);

/* No more data. Fine. Reset the backend and go back to option-control 
 * loop. 
 */

sane_cancel (handle);
\end{verbatim}

\end{changebar}


\section{Well-Known Options}\index{well-known options}

While most backend options are completely self-describing, there are
cases where a user interface might want to special-case the handling
of certain options.  For example, the scan area is typically defined
by four options that specify the top-left and bottom-right corners of
the area.  With a graphical user interface, it would be tedious to
force the user to type in these four numbers.  Instead, most such
interfaces will want to present to the user a preview (low-resolution
scan of the full scanner surface or a high(er) resolution scan of a subpart
of the scanner surface) and let the user pick the scan area by
dragging a rectangle into the desired position.  For this reason, the
SANE API specifies a small number of option names that have
well-defined meanings.

\subsection{Option Number Count}\index{option count}

Option number 0 has an empty string as its name.  The value of this
option is of type \code{SANE\_TYPE\_INT} and it specifies the total
number of options available for a given device (the count includes
option number 0).  This means that there are two ways of counting the
number of options available: a frontend can either cycle through all
option numbers starting at one until
\code{sane\_get\_option\_descriptor()} returns \code{NULL}, or a
frontend can directly read out the value of option number 0.

\subsection{Scan Resolution Options}\index{scan resolution}\index{resolution option}

Option \code{resolution} is used to select the resolution at which an
image should be acquired.
\begin{changebar}
When the backend wants to allow different
values for x- and y-resolution it has to define the options
\code{x-resolution} and \code{y-resolution}. Note that only
the option \code{resolution} {\bf or} the options
\code{x-resolution} {\bf and} \code{y-resolution} may be active at the
same time.
\end{changebar}

The type of this option is either \code{SANE\_TYPE\_INT} or
\code{SANE\_TYPE\_FIXED}.  The unit is \code{SANE\_UNIT\_DPI} (dots/inch).

These options are not mandatory, but if a backend does support them, it
must implement them in a manner consistent with the above definition.

\subsection{Preview Mode Option}\index{preview mode}

The boolean option \code{preview} is used by a frontend to inform the
backend when image acquisition should be optimized for speed, rather
than quality (``preview mode'').  When set to \code{SANE\_TRUE},
preview mode is in effect, when set to \code{SANE\_FALSE} image
acquisition should proceed in normal quality mode.  The setting of
this option \emph{must not\/} affect any other option.  That is, as
far as the other options are concerned, the preview mode is completely
side effect free.  A backend can assume that the frontend will take
care of appropriately setting the scan resolution for preview mode
(through option \code{resolution}).  A backend is free to override the
\code{resolution} value with its own choice for preview mode, but it
is advised to leave this choice to the frontend wherever possible.
\begin{changebar}
When the \code{preview} option is set the backend should transfer
the image in frame type \code{SANE\_FRAME\_RAW} if possible.
\end{changebar}

This option is not mandatory, but if a backend does support it, it
must implement it in a manner consistent with the above definition.

\subsection{Scan Area Options\label{subsec:scanarea}}\index{scan area options}

The four most important well-known options are the ones that define
the scan area.  The scan area is defined by two points (x/y coordinate
pairs) that specify the top-left and the bottom-right corners.  This
is illustrated in Figure~\ref{fig:area}.  Note that the origin of the
coordinate system is at the top-left corner of the scan surface as
seen by the sensor (which typically is a mirror image of the scan
surface seen by the user).  For this reason, the top-left corner is
the corner for which the abscissa and ordinate values are
simultaneously the {\em smallest} and the bottom-right corner is the
corner for which the abscissa and ordinate values are simulatenously
the {\em largest}.  If this coordinate system is not natural for a
given device, it is the job of the backend to perform the necessary
conversions.
\begin{figure}[tbp]
  \begin{center}
    \leavevmode
    \psfig{file=figs/area.eps,height=0.3\textheight}
    \caption{Scan area options}
    \label{fig:area}
  \end{center}
\end{figure}

The names of the four options that define the scan area are given in
the table below:
\begin{center}
\begin{tabular}{ll}
{\bf Name} & {\bf Description} \\
\code{\defn{tl-x}} & Top-left $x$ coordinate value \\
\code{\defn{tl-y}} & Top-left $y$ coordinate value \\
\code{\defn{br-x}} & Bottom-right $x$ coordinate value \\
\code{\defn{br-y}} & Bottom-right $y$ coordinate value \\
\end{tabular}
\end{center}
There are several rules that should be followed by front and backends
regarding these options:
\begin{itemize}

\item Backends must attach a unit of either pixels
  (\code{SANE\_UNIT\_PIXEL}) or millimeters (\code{SANE\_UNIT\_MM}) to
  these options.  The unit of all four options must be identical.

\item Whenever meaningful, a backend should attach a range or a
  word-list constraint to these options.

\item A frontend can determine the size of the scan surface by first
  checking that the options have range constraints associated.  If a
  range or word-list constraints exist, the frontend can take the
  minimum and maximum values of one of the x and y option
  range-constraints to determine the scan surface size.

\item A frontend must work properly with any or all of these options
  missing.
  
\begin{changebar}
\item A frontend may temporarily set the values in a way that
\code{tl-x} is larger than \code{br-x} and \code{tl-y} is larger than
\code{br-y}.
\end{changebar}

\end{itemize}

\begin{changebar}
These options are not mandatory, but if a backend does support them, it
must implement them in a manner consistent with the above definition.
\end{changebar}

\begin{changebar}
\subsection{Depth Option}\index{bit depth option}
Option \code{depth} is used to select the image depth in bits/sample
in multi bit mode (e.g. for 24 bit RGB mode this value must be 8).
The type of this option is \code{SANE\_TYPE\_INT}.
The unit is \code{SANE\_UNIT\_BIT}. For 1 bit modes
(Lineart or Halftone) this option has to be inactive.
For selection of 1 bit modes (Lineart or Halftone) the
backend should use the well-known option \code{mode}.
 
This option is not mandatory, but if a backend does support it, it
must implement it in a manner consistent with the above definition.
\end{changebar}
 
\begin{changebar}
\subsection{Scan Mode Options}\index{mode options}
The option \code{mode} defines a \code{SANE\_CONSTRAINT\_STRING\_LIST}
of type \code{SANE\_TYPE\_STRING}.
It is used to select the scanmode (e.g. Color or Gray).
Well known modes are: \code{Color}, \code{Gray}, \code{Halftone}
and \code{Lineart}. \code{Color} and \code{Gray} are multi bit
modes (8 or 16 bits/sample), \code{Halftone} and \code{Lineart}
are single bit modes. When well-known scan modes are used,
a frontend is able to automatically decide which mode is appropriate
for a specific task, e.g the mode \code{Gray}
for scanning a fax.
 
This option is not mandatory, but if a backend does support it, it
must implement it in a manner consistent with the above definition.
\end{changebar}
 
\begin{changebar}
\subsection{Scan Source Options}\index{source options}
The option \code{source} is used to select the scan source
(e.g. Automatic Document Feeder).
It defines a \code{SANE\_CONSTRAINT\_STRING\_LIST}
of type \code{SANE\_TYPE\_STRING}.
Well known sources are: \code{Flatbed}, \code{Transparancy Adapter} and
\code{Automatic Document Feeder}.
 
This option is not mandatory, but if a backend does support it, it
must implement it in a manner consistent with the above definition.
\end{changebar}

\begin{changebar}
\subsection{Threshold Option}\index{threshold option}
The option \code{threshold} is used to define the threshold
for Lineart and maybe Halftone mode. In multi bit modes
this option should be set inactive.
The type of this option is \code{SANE\_TYPE\_FIXED}.
The unit is \code{SANE\_UNIT\_PERCENT}. The value range
should be 0.0\ldots100.0 if possible.
It defines the minimum intensity to get a white point / full intensity.

The backend has to scale the values in the following way:
A value of 0.0 means all pixels get white / full intensity.  A value of 50.0
means intensities brighter than medium gray get white / full intensity.  A
value of 100.0 means all pixels get black. If the
scanner is not able to cover the full range the backend has to define a
reduced value range (e.g. 30\ldots70 percent).

This option is not mandatory, but if a backend does support it, it
must implement it in a manner consistent with the above definition.
\end{changebar}

\begin{changebar}
\subsection{Gamma Table Options}\index{gamma table options}
The \code{gamma-table} option defines a \code{SANE\_CONSTRAINT\_RANGE}
of the type \code{SANE\_TYPE\_INT} which represents the gamma correction
table for gray.  In color mode the \code{gamma-table} may be used to set
a common gamma correction for red, green and blue. The options
\code{red-gamma-table}, \code{green-gamma-table} and \code{blue-gamma-table}
are used in color mode to set a gamma correction for each color
separately. In color mode the backend is free to use only the
\code{gamma-table} option, only the \code{red-}, \code{green-} and
\code{blue-gamma-table} or all four options. When all four options
are used then the color tables should do a gamma correction with
the same input and output bit depth and the gray gamma table should
reduce (if necessary) the bit depth from the scanner internal
bit depth to the output bit depth. This should e.g. look like this:
\begin{verbatim}
red_value   = gamma-table(red-gamma-table(value))
green_value = gamma-table(green-gamma-table(value))
blue_value  = gamma-table(blue-gamma-table(value))
\end{verbatim}

The backend should not use the gamma tables to emulate other functions or options
like highlight, shadow, contrast, brightness, threshold, analog\_gamma.
These functions are common for all backends and should be added to the frontend
or a meta-backend.

It is also discouraged to emulate gamma tables in the backend. The backend
should disable (or not define) this option when the scanner does not support
gamma tables in hardware.
 
This option is not mandatory, but if a backend does support it, it
must implement it in a manner consistent with the above definition.
\end{changebar}

\begin{changebar}
\subsection{Analog Gamma}\index{analog gamma option}
The option \code{analog-gamma} is used to define the gamma value 
for an analog gamma function of the scanner in multi bit modes.
In 1 bit modes this option should be set inactive.
The type of this option is \code{SANE\_TYPE\_FIXED}.
The unit is \code{SANE\_UNIT\_NONE}. The value range
can be defined by the backend as supported. The values
have to be positive. A gamma value of 1.0 means that
the gamma correction has no effect. A value larger than
1.0 increases the brightness of the image.
In color mode there also can be options \code{analog-gamma-red},
\code{analog-gamma-green} and \code{analog-gamma-blue}.
It is not allowed to emulate an analog gamma function by
a digital gamma table. The backend has to disable (or not
define) this option when the scanner does not support an
analog (hardware) gamma function.

When analog gamma, highlight and shadow functions are available
at the same time then the backend author has to care about the order
in which the functions are implemented in the scanner hardware.
The SANE standard expects that changing the analog gamma value
has no effect on the shadow and highlight function. When the
analog gamma function is executed in the scanner hardware
before the shadow and highlight functions then the backend
should do a compensation. For this the shadow and highlight
values have to be gamma corrected with the relevant analog gamma value.

This option is not mandatory, but if a backend does support it, it
must implement it in a manner consistent with the above definition.
\end{changebar}


\begin{changebar}
\subsection{Shadow Option}\index{shadow options}
The option \code{shadow} is used to define the shadow level / black point level.
The type of this option is \code{SANE\_TYPE\_FIXED}.
The unit is \code{SANE\_UNIT\_PERCENT}. The value range
should be 0.0\ldots100.0 if possible.
It is used to define the maximum intensity level that creates an image data value
of 0 (black).

The backend has to scale the values in the following way:
A value of 0.0 means that the sensitivity range is not reduced, only the
minimum intensity produces an image data value of 0 (black).
A value of 50.0 means that that a medium intensity and everything that is darker
produces an image data value of 0 (black).
A value of 100.0 means the sensitivity range is reduced to 0, all image
data values are 0 (black).  If the scanner is not able to
cover the full range the backend has to define a reduced
value range (e.g. 30\ldots70 percent).
In color mode there can be options \code{shadow-red}, \code{shadow-green}
and \code{shadow-blue}, in this case the \code{shadow} function has to be disabled.
It is not allowed to emulate a shadow function by
a digital gamma table. The backend has to disable (or not
define) this option when the scanner does not support an
analog (hardware) shadow function.

This option is not mandatory, but if a backend does support it, it
must implement it in a manner consistent with the above definition.
\end{changebar}


\begin{changebar}
\subsection{Highlight Option}\index{hightlight options}
The option \code{highlight} is used to define the highlight level / white point level.
The type of this option is \code{SANE\_TYPE\_FIXED}.
The unit is \code{SANE\_UNIT\_PERCENT}. The value range
should be 0.0\ldots100.0 if possible.
It is used to define the minimum intensity level that creates the maximum possible
image data value (white/full intensity). 

The backend has to scale the values in the following way:
A value of 0.0 means the sensitivity range is reduced to 0, all image
data have maximum value (white / full intensity).
A value of 50.0 means that a medium intensity and everything that is brighter
produces the maximum possible image data value (white / full intensity).
A value of 100.0 means that the sensitivity range is not reduced, only the
maximum intensity produces an image data with maximum possible value (white / full intensity).
If  the scanner is not able to cover the full range the backend has to define a reduced
value range (e.g. 30\ldots70 percent).
In color mode there can be options \code{highlight-red}, \code{highlight-green}
and \code{highlight-blue}, in this case \code{highlight} has to be disabled.
It is not allowed to emulate a highlight function by
a digital gamma table. The backend has to disable (or not
define) this option when the scanner does not support an
analog (hardware) highlight function.

This option is not mandatory, but if a backend does support it, it
must implement it in a manner consistent with the above definition.
\end{changebar}


\begin{changebar}
\subsection{Lamp Options}\index{lamp-on option}\index{lamp-off option}

The options \code{lamp-on} and \code{lamp-off} are button options
(\code{SANE\_TYPE\_BUTTON}) and don't have a unit (\code{SANE\_UNIT\_NONE}).

Option \code{lamp-on} is used to turn on the lamp of the scanner while
\code{lamp-off} turns it off.

These options are not mandatory, but if a backend does support them, it
must implement them in a manner consistent with the above definition.
\end{changebar}


\begin{changebar}
\subsection{Scanner Button Options}\index{scanner button options}
Some scanners have buttons which state can be read by the scanner driver. 
It is necessary to implement a locking function for the buttons
because it is possible that several frontends try to connect to the
same backend/scanner at the same time. Imagine what could happen
when no locking would be implemented:
Five people have started a scanning application which is connected
via network to the scanner you want to use. You start a frontend,
put a paper to the scanner and press the scan-button on the scanner.
The scanner does scan three times (because three frontends asked the
button status when you pressed the button). For three people the
image is saved to the harddisk, but it is not sure that your
frontend did scan the image.

A backend that does make available the scanner-buttons has to
implement the following options:\\
\code{scanner-buttons-lock} is of type \code{SANE\_TYPE\_BOOL}, default = \code{SANE\_FALSE}\\
\code{scanner-buttons-status} is of type \code{SANE\_TYPE\_INT}, default = 0\\
\code{scanner-buttons-status-update} is of type \code{SANE\_TYPE\_BUTTON}\\
When setting these options the backend does not set \code{SANE\_INFO\_RELOAD\_OPTIONS}
or \code{SANE\_INFO\_RELOAD\_PARAMS} if not explictly defined.

A frontend has to disable the usage of the scanner-buttons by default. This is important
because other frontends will not be able to use the buttons when the button-functions are locked.
Another important thing is that some scanners do not turn off their lamp when the driver
does frequently talk to the scanner (what is done when reading the button status from the scanner).

\begin{itemize}

\item
A frontend that wants to read the button status has to lock the
button functions at first. For this it does set the option
\code{scanner-buttons-lock} to \code{SANE\_TRUE}.
While setting the value of option \code{scanner-buttons-lock} to \code{SANE\_TRUE}
the backend does check if a lockfile (e.g. ``backend''-buttons.lock) does exist.
The lockfile has to be placed in a directory where every user has read and write access to.

  \begin{itemize}
  \item
  If the lockfile does not exist then the backend creates the lockfile and writes the
  process ID (PID) of the backend to the file. Button access is allowed: the value
  of option \code{scanner-buttons-lock} is set to \code{SANE\_TRUE}
  \item
  If the file does exist and the backend PID is not the file PID then the
  backend has to check if the process with the PID stored in the lockfile
  still is running. If yes then the button access is not allowed: the value of
  option \code{scanner-buttons-lock} is set to \code{SANE\_FALSE}. If not then the lockfile
  is recreated and the PID of the backend is stored in the lockfile, button
  access is allowed: the value of option \code{scanner-buttons-lock} is set to \code{SANE\_TRUE}
  \end{itemize}

\item
The frontend does read the value of option \code{scanner-buttons-lock}. If
it is \code{SANE\_TRUE} then the frontend has access to the scanner buttons.
If it is \code{SANE\_FALSE} then access has been denied.

\item
If the button access is allowed the frontend has to do the following about
once per second (while not scanning):
  \begin{itemize}
  \item
  The frontend does set option \code{scanner-buttons-status-update}.
  The backend checks if access to the buttons is allowed by comparing
  the backend PID with the lockfile PID. If access is allowed it
  does read the button status from the scanner and stores it in
  the option \code{scanner-buttons-status}, each bit represents a button, a
  value of 0 means the button is not pressed, a value of 1 means that the button
  is pressed. When the scanner is busy the backend must not wait, it has to
  return immediately and the button state keeps unchanged.
  The backend has to implement a timeout function. When no button has been pressed
  within a predefined time (e.g. 15 minutes) then the access permission is lost.
  In this case the backend does set option \code{scanner-buttons-lock} to \code{SANE\_FALSE}
  and does set \code{SANE\_INFO\_RELOAD\_OPTIONS} to inform the frontend that it has
  lost permission to access the scanner-button functions.
  If access is not allowed it does set the \code{scanner-buttons-status} to 0.

  \item
  The frontend does read the value of option \code{scanner-buttons-status}
  \end{itemize}

\item
When the frontend does exit or it does not want to use the buttons
it does set option \code{scanner-buttons-lock} to \code{SANE\_FALSE}.
The backend does check if the backend PID and the lockfile PID are
the same. If this is true then it removes the lockfile and sets the value 
of \code{scanner-buttons-lock} to \code{SANE\_FALSE}.

\code{sane\_close()} should do the same as setting option
\code{scanner-buttons-lock} to \code{SANE\_FALSE}.

\end{itemize}
\end{changebar}

\begin{changebar}
\subsection{Batch Scan Options}\index{Batch Scan Options}

The batch scan options can be used by a frontend to indicate that more than
one image will be scanned in a batch. The backend can optimize for such scans
by e.g. avoiding calibration and not moving home the sensor array in this case.
The backend provides the options \code{batch-scan},
\code{batch-scan-next-tl-x}, \code{batch-scan-next-tl-y},
\code{batch-scan-next-br-x}, and \code{batch-scan-next-br-y}. The option
\code{batch-scan} provides a string list with the values of \code{No} if
batch scanning isn't used, \code{Start} for the first, \code{End} for the last
and \code{Loop} for any other (intermediate) image.  The batch-scan-next
options specify the coordinates of the next scan and follow the same rules as
the scan area options (see section \ref{subsec:scanarea}).

These options are not mandatory, but if a backend does support them, it must
implement them in a manner consistent with the above definition.  In this
case, all options must be implemented.
\end{changebar}




\input{net.tex}

\chapter{Contact Information}\label{chap:contact}

The SANE standard is discussed and evolved via a mailing list.
Anybody with email access to the Internet can automatically join and
leave the discussion group by sending mail to the following address.
\begin{quote}\index{mailing list}
\begin{verbatim}
sane-devel-request@mostang.com
\end{verbatim}
\end{quote}
To subscribe, send a mail with the body ``\verb|subscribe sane-devel|'' to the
above address.

A complete list of commands supported can be obtained by sending a
mail with a subject of ``\code{help}'' to the above address.  The
mailing list is archived and available through the SANE home page at
URL:
\begin{quote}
\url{http://www.mostang.com/sane/}
\end{quote}

\newpage
\documentclass[11pt,DVIps]{report}

\usepackage{times,epsfig,changebar,html}

\setlength{\parindent}{0pt}
\setlength{\parskip}{1.5ex plus 0.5ex minus 0.5ex}
\setlength{\textwidth}{6.5in}
\setlength{\textheight}{8.5in}
\setlength{\marginparwidth}{0pt}
\setlength{\oddsidemargin}{0pt}
\setlength{\evensidemargin}{0pt}
\setlength{\marginparsep}{0pt}
\addtolength{\topmargin}{-0.75in}

\title{\huge SANE Standard Version 2.0 proposal 0.08 - rauch/beck}
\author{}
\date{Dec 8, 2002} 

\makeindex

\begin{document}

\newcommand{\filename}[1]{{\tt #1}}
\newcommand{\code}[1]{{\tt #1}}
\newcommand{\var}[1]{{\it #1}}
\newcommand{\defn}[1]{#1\index{#1}}

\begin{latexonly}
  \setcounter{changebargrey}{0}   % black change bars
\end{latexonly}

\maketitle
\tableofcontents
\listoffigures
\listoftables


\chapter{Preface}

The SANE standard is being developed by a group of free-software
developers.  The process is open to the public and comments as well as
suggestions for improvements are welcome.  Information on how to join
the SANE development process can be found in Chapter
\ref{chap:contact}.

The SANE standard is intended to streamline software development by
providing a standard application programming interface to access
raster scanner hardware.  This should reduce the number of different
driver implementations, thereby reducing the need for reimplementing
similar code.


\section{About This Document}

This document is intended for developers who are creating either an
application that requires access to raster scanner hardware and for
developers who are implementing a SANE driver.  It does not cover
specific implementations of SANE components.  Its sole purpose is to
describe and define the SANE application interface that will enable
any application on any platform to interoperate with any SANE backend
for that platform.

The remainder of this document is organized as follows.
Chapter~\ref{chap:intro} provides introductional material.
Chapter~\ref{chap:environ} presents the environment SANE is designed
for.  Chapter~\ref{chap:api} details the SANE Application Programmer
Interface.  Chapter~\ref{chap:net} specifies the network protocol that
can be used to implement the SANE API in a network transparent
fashion.  Finally, Chapter~\ref{chap:contact} gives information on how
to join the SANE development process.

\subsection{Typographic Conventions}

Changes since the last revision of this document are highlighted
like this:

\begin{changebar}
   Paragraphs that changed since the last revision of the documention
   are marked like this paragraph.
\end{changebar}

\chapter{Introduction}\label{chap:intro}

SANE is an application programming interface (API) that provides
standardized access to any raster image scanner hardware.  The
standardized interface allows to write just one driver for each
scanner device instead of one driver for each scanner and application.
The reduction in the number of required drivers provides significant
savings in development time.  More importantly, SANE raises the level
at which applications can work.  As such, it will enable applications
that were previously unheard of in the UNIX world.  While SANE is
primarily targeted at a UNIX environment, the standard has been
carefully designed to make it possible to implement the API on
virtually any hardware or operating system.

SANE is an acronym for ``Scanner Access Now Easy.''  Also, the hope is
that SANE is sane in the sense that it will allow easy implementation
of the API while accommodating all features required by today's
scanner hardware and applications.  Specifically, SANE should be broad
enough to accommodate devices such as scanners, digital still and
video cameras, as well as virtual devices like image file filters.

\section{Terminology}

An application that uses the SANE interface is called a SANE {\em
  frontend}.  A driver that implements the SANE interface is called a
SANE {\em backend}.  A {\em meta backend\/} provides some means to
manage one or more other backends.


\chapter{The SANE Environment}\label{chap:environ}

SANE is defined as a C-callable library interface.  Accessing a raster
scanner device typically consists of two phases: first, various
controls of the scanner need to be setup or queried.  In the second
phase, one or more images are acquired.

Since the device controls are widely different from device to device,
SANE provides a generic interface that makes it easy for a frontend to
give a user access to all controls without having to understand each
and every device control.  The design principle used here is to
abstract each device control into a SANE {\em option\/}.  An option is
a self-describing name/value pair.  For example, the brightness
control of a camera might be represented by an option called
\code{brightness} whose value is an integer in the range from 0 to
255.

With self-describing options, a backend need not be concerned with
{\em presentation\/} issues: the backend simply provides a list of
options that describe all the controls available in the device.
Similarly, there are benefits to the frontend: it need not be
concerned with the {\em meaning\/} of each option.  It simply provides
means to present and alter the options defined by the backend.


\section{Attaching to a SANE backend}

The process through which a SANE frontend connects to a backend is
platform dependent.  Several possibilities exist:
\begin{itemize}

\item {\bf Static linking:} A SANE backend may be linked directly into
  a frontend.  While the simplest method of attaching to a backend, it
  is somewhat limited in functionality since the available devices is
  limited to the ones for which support has been linked in when the
  frontend was built.  But even so static linking can be quite useful,
  particularly when combined with a backend that can access scanners
  via a network.  Also, it is possible to support multiple backends
  simultaneously by implementing a meta backend that manages several
  backends that have been compiled in such a manner that they export
  unique function names.  For example, a backend called \code{be}
  would normally export a function called \code{sane\_read()}.  If
  each backend would provide such a function, static linking would
  fail due to multiple conflicting definitions of the same symbol.
  This can be resolved by having backend \code{be} include a
  header file that has lines of the form:
  \begin{quote}
\begin{verbatim}
#define sane_read be_sane_read
\end{verbatim}
  \end{quote}
  With definitions of this kind, backend \code{be} will export
  function name \code{be\_sane\_read()}.  Thus, all backends will
  export unique names.  As long as a meta backend knows about these
  names, it is possible to combine several backends at link time and
  select and use them dynamically at runtime.

\item {\bf Dynamic linking:} A simpler yet more powerful way to
  support multiple backends is to exploit dynamic linking on platforms
  that support it.  In this case, a frontend is linked against a
  shared library that implements any SANE backend.  Since each
  dynamically linked backend exports the same set of global symbols
  (all starting with the prefix \code{sane\_}), the dynamic library
  that gets loaded at runtime does not necessarily have to be the same
  one as one the frontend got linked against.  In other words, it is
  possible to switch the backend by installing the appropriate backend
  dynamic library.

  More importantly, dynamic linking makes it easy to implement a meta
  backend that loads other backends {\em on demand}.  This is a
  powerful mechanism since it allows adding new backends merely by
  installing a shared library and updating a configuration file.

\item {\bf Network connection:} Arguably the ultimate way to attach to
  a scanner is by using the network to connect to a backend on a
  remote machine.  This makes it possible to scan images from any host
  in the universe, as long as there is a network connection to that
  host and provided the user is permitted to access that scanner.

\end{itemize}

\begin{figure}[htbp]
  \begin{center}
    \leavevmode
    \psfig{file=figs/hierarchy.eps,angle=270,width=\textwidth}
    \caption{Example SANE Hiearchy}
    \label{fig:hierarchy}
  \end{center}
\end{figure}

The above discussion lists just a few ways for frontends to attach to
a backend.  It is of course possible to combine these solutions to
provide an entire hierarchy of SANE backends.  Such a hierarchy is
depicted in Figure~\ref{fig:hierarchy}.  The figure shows that machine
A uses a dynamic-linking based meta backend called \code{dll} to
access the backends called \code{pnm}, \code{mustek}, and \code{net}.
The first two are real backends, whereas the last one is a meta
backend that provides network transparent access to remote scanners.
In the figure, machine B provides non-local access to its scanners
through the SANE frontend called \code{saned}.  The \code{saned} in
turn has access to the \code{hp} and \code{autolum} backends through
another instance of the \code{dll} backend.  The \code{autolum} meta
backend is used to automatically adjust the luminance (brightness) of
the image data acquired by the camera backend called \code{qcam}.

Note that a meta backend really is both a frontend and a backend at
the same time.  It is a frontend from the viewpoint of the backends
that it manages and a backend from the viewpoint of the frontends that
access it.  The name ``meta backend'' was chosen primarily because the
SANE standard describes the interface from the viewpoint of a (real)
frontend.


\section{Image Data Format}\label{sec:imageformat}\index{image data format}

Arguably the most important aspect of an image acquisition system is how
images are represented. The SANE approach is to define a simple yet powerful
representation that is sufficient for vast majority of applications and
devices. While the representation is simple, the interface has been defined
carefully to allow extending it in the future without breaking backwards
compatibility. Thus, it will be possible to accommodate future applications or
devices that were not anticipated at the time this standard was created.

A SANE image is a rectangular area.  The rectangular area is subdivided into a
number of rows and columns.  At the intersection of each row and column is a
(preferable quadratic) pixel. A pixel consists of one or more sample values.
Each sample value represents one channel (e.g., the red channel).

The SANE API transmits an image as a sequence of frames.  Each frame covers
the same rectangular area as the entire image, but may contain only a subset
of the channels in the final image.  For example, a red/green/blue image could
either be transmitted as a single frame that contains the sample values for
all three channels or it could be transmitted as a sequence of three frames:
the first frame containing the red channel, the second the green channel, and
the third the blue channel.

When transmitting an image frame by frame, the frontend needs to know what
part of the image a frame represents (and how many frames it should expect).
For that purpose, the SANE API tags every frame with a type and a format
descriptor.

\begin{changebar}
  There are two different types of frames: pixel oriented frames
  \code{\defn{SANE\_FRAME\_RAW}} and arbitrary data frames
  \code{\defn{SANE\_FRAME\_MIME}}. These types are discussed in detail in the
  following sections. The frame types used by the previous version~1 of this
  standard (\code{\defn{SANE\_FRAME\_GRAY}}, \code{\defn{SANE\_FRAME\_RGB}},
  \code{\defn{SANE\_FRAME\_RED}}, \code{\defn{SANE\_FRAME\_GREEN}}, and
  \code{\defn{SANE\_FRAME\_BLUE}}) are obsolete and superseded by
  \code{\defn{SANE\_FRAME\_RAW}}.

  \subsection{Pixel oriented frames}

  The type of pixel oriented frames is \code{\defn{SANE\_FRAME\_RAW}}. The
  frame contains one or more channels of data in a channel-interleaved format,
  that represents sample values from a property of the individual pixels that
  is subject to further description in the \code{format\_desc} member of the
  \code{SANE\_Parameters} structured type. See section~\ref{sec:sanegetparameters}
  on page~\pageref{sec:sanegetparameters} for details about the format
  descriptions.
\end{changebar}

  Each sample value has a certain bit depth.  The bit depth is fixed for the
  entire image and can be as small as one bit. Valid bit depths are 1, 8, or
  16 bits per sample. If a device's natural bit depth is something else, it is
  up to the driver to scale the sample values appropriately (e.g., a 4 bit
  sample could be scaled by a factor of four to represent a sample value of
  depth 8).

\begin{changebar}
  The complete image may consist of several different channels. The number of channels
  is defined by member \code{channels\_per\_image} of \code{SANE\_Parameters}.
  The image may be transmitted in an arbitrary number of frames which can be
  determined by watching the \code{SANE\_PFLAG\_LAST\_FRAME} flag in said type (or by
  counting the channels).  Note: This frame type replaces all frame types of
  the SANE standard version 1.
\end{changebar}

Conceptually, each pixel oriented frame is transmitted a byte at a time.  Each
byte may contain 8 sample values (for an image bit depth of 1), one full
sample value (for an image bit depth of 8), or a partial sample value (for an
image bit depth of 16 or bigger).  In the latter case, the bytes of each
sample value are transmitted in the machine's native byte order.
\begin{quote}
  \begin{center}
    {\bf Backend Implementation Note}
  \end{center}
  A network-based meta backend will have to ensure that the byte order
  in image data is adjusted appropriately if necessary.  For example,
  when the meta backend attaches to the server proxy, the proxy may
  inform the backend of the server's byte order.  The backend can then
  apply the adjustment if necessary.  In essence, this implements a
  ``receiver-makes-right'' approach.
\end{quote}

\begin{figure}[htbp]
  \begin{center}
    \leavevmode
    \psfig{file=figs/xfer.eps,width=0.5\textwidth}
    \caption{Transfer order of image data bytes}
    \label{fig:xfer}
  \end{center}
\end{figure}

The order in which the sample values in a frame are transmitted is illustrated
in Figure~\ref{fig:xfer}.  As can be seen, the values are transmitted row by
row and each row is transmitted from left-most to right-most column.  The
left-to-right, top-to-bottom transmission order applies when the image is
viewed in its normal orientation (as it would be displayed on a screen, for
example).

If a frame contains multiple channels, then the channels are transmitted in an
interleaved fashion.  Figure~\ref{fig:pixels} illustrates this for the case
where a frame contains a complete red/green/blue image with a bit-depth of 8.

\begin{figure}[htbp]
  \begin{center}
    \leavevmode
    \psfig{file=figs/image-data.eps,width=0.8\textwidth}
    \caption{Bit and byte order of image data}
    \label{fig:pixels}
  \end{center}
\end{figure}

For a bit depth of 1, each byte contains 8 sample values of a {\em single\/}
channel. In other words, a bit depth 1 frame is transmitted in a byte
interleaved fashion. The first sample of each byte is represented by the most
significant bit.

\begin{changebar}
For gray channels at a bit depth of 1 only two sample values are possible: 1
represents minimum intensity (black) and 0 represents maximum intensity
(white).  For all other channel types and bit depths a sample value of 0
represents minimum intensity and larger values represent increasing intensity.

\subsection{Arbitrary data frames}

It also is possible to transmit arbitrary (not necessaryly pixel oriented)
data. This allows transmission of compressed images like jpeg, tiff, etc.

The type of arbitrary data frames is \code{\defn{SANE\_FRAME\_MIME}}.
The frame contains arbitrary data of the MIME (see RFC 1521/1522) type that is
given in the \code{format\_desc} member of the \code{SANE\_Parameters}
structured type (see section~\ref{sec:sanegetparameters} on
page~\pageref{sec:sanegetparameters}).  As such, it is assumed to be
incomprehensible to the frontend, except for selected types the frontend is
specifically capable of handling internally.  The frontend is free to ignore
those frames, or employ any appropriate means to otherwise handle this data
(like saving them to disk or spawning an external viewer).
\end{changebar}

\chapter{The SANE Application Programmer Interface (API)}\label{chap:api}

\begin{changebar}
This Section defines version 2 of the SANE application
programmer interface (API).  Any SANE frontend must depend on the
interface defined in this section only. Converseley, any SANE backend
must implement its functionality in accordance with this
specification. The interface as documented here is declared as a C
callable interface in a file called \filename{sane/sane-2.h}. This file should
normally be included via a C pre-processor directive of the form:
\begin{verbatim}
  #include <sane/sane-2.h>
\end{verbatim}
\end{changebar}


\section{Version Control}

The SANE standard is expected to evolve over time.  Whenever a change
to the SANE standard is made that may render an existing frontend or
backend incompatible with the new standard, the major version number
must be increased.  Thus, any frontend/backend pair is compatible
provided the major version number of the SANE standard they implement
is the same.  A frontend may implement backwards compatiblity by
allowing major numbers that are smaller than the expected major number
(provided the frontend really can cope with the older version).  In
contrast, a backend always provides support for one and only one
version of the standard.  If a specific application does require that
two different versions of the same backend are accessible at the same
time, it is possible to do so by installing the two versions under
different names.

SANE version control also includes a minor version number and a build
revision.  While control of these numbers remains with the implementor
of a backend, the recommended use is as follows.  The minor version is
incremented with each official release of a backend.  The build
revision is increased with each build of a backend.

The SANE API provides the following five macros to manage version
numbers.
\begin{quote}
  \begin{description}
  \item[\code{\defn{SANE\_CURRENT\_MAJOR}}:] The value of this macro is the
    number of the SANE standard that the interface implements.

  \item[\code{\defn{SANE\_VERSION\_CODE}(\var{maj},\var{min},\var{bld})}:]
    \label{sec:saneversioncode}
    This macro can be used to build a monotonically increasing version
    code.  A SANE version code consists of the SANE standard major
    version number (\var{maj}), the minor version number \var{min},
    and the build revision of a backend (\var{bld}).  The major and
    minor version numbers must be in the range 0\ldots255 and the
    build revision must be in the range 0\ldots65535.

    Version codes are monotonic in the sense that it is possible to
    apply relational operators (e.g., equality or less-than test)
    directly on the version code rather than individually on the three
    components of the version code.

    Note that the major version number alone determines whether a
    frontend/backend pair is compatible.  The minor version and the
    build revision are used for informational and bug-fixing purposes
    only.
  
  \item[\code{\defn{SANE\_VERSION\_MAJOR}(\var{vc})}:] This macro returns the
    major version number component of the version code passed in
    argument \var{vc}.
  \item[\code{SANE\_VERSION\_MINOR(\var{vc})}:] This macro returns the
    minor version number component of the version code passed in
    argument \var{vc}.
  \item[\code{SANE\_VERSION\_BUILD(\var{vc})}:] This macro returns the
    build revision component of the version code passed in argument
    \var{vc}.
  \end{description}
\end{quote}


\section{Data Types}

\subsection{Base Types}

The SANE standard is based on just two SANE-specific base types: the
SANE byte and word.
\begin{quote}
  \code{typedef \var{some-scalar-type\/} \defn{SANE\_Byte};} \\
  \code{typedef \var{some-scalar-type\/} \defn{SANE\_Word};}
\end{quote}
\verb|SANE_Byte| must correspond to some scalar C type that is capable
of holding values in the range 0 to 255.  \verb|SANE_Word| must be
capable of holding any of the following:
\begin{itemize}
  \item the truth values \verb|SANE_FALSE| and \verb|SANE_TRUE|
  \item signed integers in the range $-2^{31}\ldots2^{31}-1$
  \item fixed point values in the range $-32768\ldots32767.9999$ with
    a resolution of $1/65536$
  \item 32 bits (for bit sets)
\end{itemize}
Note that the SANE standard does not define what C type
\verb|SANE_Byte| and \verb|SANE_Word| map to.  For example, on some
platforms, the latter may map to \verb|long int| whereas on others it
may map to \verb|int|.  A portable SANE frontend or backend must
therefore not depend on a particular mapping.

\subsection{Boolean Type}

\code{\defn{SANE\_Bool}} is used for variables that can take one of
the two truth values \code{\defn{SANE\_FALSE}} and
\code{\defn{SANE\_TRUE}}.  The former value is defined to be 0,
whereas the latter is 1.\footnote{This is different from ANSI C where
  any non-zero integer value represents logical TRUE.} The C
declarations for this type are given below.
\begin{quote}
\begin{verbatim}
#define SANE_FALSE      0
#define SANE_TRUE       1
typedef SANE_Word SANE_Bool;
\end{verbatim}
\end{quote}
Note that \verb|SANE_Bool| is simply an alias of \verb|SANE_Word|.  It
is therefore always legal to use the latter type in place of the
former.  However, for clarity, it is recommended to use
\verb|SANE_Bool| whenever a given variable or formal argument has a
fixed interpretation as a boolean object.

\subsection{Integer Type}

\code{\defn{SANE\_Int}} is used for variables that can take integer
values in the range $-2^{32}$ to $2^{31}-1$.  Its C declaration is
given below.
\begin{quote}
\begin{verbatim}
typedef SANE_Word SANE_Int;
\end{verbatim}
\end{quote}
Note that \verb|SANE_Int| is simply an alias of \verb|SANE_Word|.  It
is therefore always legal to use the latter type in place of the
former.  However, for clarity, it is recommended to use
\verb|SANE_Int| whenever a given variable or formal argument has a
fixed interpretation as an integer object.


\subsection{Fixed-point Type}

\code{\defn{SANE\_Fixed}} is used for variables that can take fixed
point values in the range $-32768$ to $32767.9999$ with a resolution
of $1/65535$.  The C declarations relating to this type are given
below.
\begin{quote}
\begin{verbatim}
#define SANE_FIXED_SCALE_SHIFT  16
typedef SANE_Word SANE_Fixed;
\end{verbatim}
\end{quote}
The macro \code{\defn{SANE\_FIXED\_SCALE\_SHIFT}} gives the location
of the fixed binary point.  This standard defines that value to be 16,
which yields a resolution of $1/65536$.

Note that \verb|SANE_Fixed| is simply an alias of \verb|SANE_Word|.
It is therefore always legal to use the latter type in place of the
former.  However, for clarity, it is recommended to use
\verb|SANE_Fixed| whenever a given variable or formal argument has a
fixed interpretation as a fixed-point object.

For convenience, SANE also defines two macros that convert fixed-point
values to and from C double floating point values.
\begin{quote}
  \begin{description}

  \item[\code{\defn{SANE\_FIX}(\var{d})}:] Returns the largest SANE
    fixed-point value that is smaller than the double value \var{d}.
    No range checking is performed.  If the value of \var{d} is out of
    range, the result is undefined.

  \item[\code{\defn{SANE\_UNFIX}(\var{w})}:] Returns the nearest
    double machine number that corresponds to fixed-point value
    \var{w}.

  \end{description}
\end{quote}
SANE does {\em not\/} require that the following two expressions hold
true (even if the values of \var{w} and \var{d} are in range):
\begin{quote}
\begin{verbatim}
SANE_UNFIX(SANE_FIX(d)) == d
SANE_FIX(SANE_UNFIX(w)) == w
\end{verbatim}
\end{quote}
In other words, conversion between fixed and double values may be
lossy.  It is therefore recommended to avoid repeated conversions
between the two representations.


\subsection{Text}

\subsubsection{Character Type}

Type \code{\defn{SANE\_Char}} represents a single text character or
symbol.  At present, this type maps directly to the underlying C
\verb|char| type (typically one byte).  The encoding for such
characters is currently fixed as ISO LATIN-1.  Future versions of this
standard may map this type to a wider type and allow multi-byte
encodings to support internationalization.  As a result of this, care
should be taken to avoid the assumption that
\verb|sizeof(SANE_Char)==sizeof(char)|.
\begin{quote}
\begin{verbatim}
typedef char SANE_Char;
\end{verbatim}
\end{quote}

\subsubsection{String Type}

Type \code{\defn{SANE\_String}} represents a text string as a sequence
of C \verb|char| values.  The end of the sequence is indicated by a
\verb|'\0'| (\defn{NUL}) character.  
\begin{quote}
\begin{verbatim}
typedef SANE_Char *SANE_String;
typedef const SANE_Char *SANE_String_Const;
\end{verbatim}
\end{quote}
The type \code{\defn{SANE\_String\_Const}} is provided by SANE to
enable declaring strings whose contents is unchangable.  Note that in
ANSI C, the declaration
\begin{quote}
\begin{verbatim}
const SANE_String str;
\end{verbatim}
\end{quote}
declares a string pointer that is constant (not a string pointer that
points to a constant value).


\subsection{Scanner Handle Type}

Access to a scanner is provided through an opaque type called
\code{\defn{SANE\_Handle}}.  The C declaration of this type is given
below.
\begin{quote}
\begin{verbatim}
typedef void *SANE_Handle;
\end{verbatim}
\end{quote}
While this type is declared to be a void pointer, an application must
not attempt to interpret the value of a \verb|SANE_Handle|.  In
particular, SANE does not require that a value of this type is a legal
pointer value.


\subsection{Status Type}

Most SANE operations return a value of type \code{\defn{SANE\_Status}}
to indicate whether the completion status of the operation.  If an
operation completes successfully, \verb|SANE_STATUS_GOOD| is returned.
In case of an error, a value is returned that indicates the nature of
the problem.  The complete list of available status codes is listed in
Table \ref{tab:status}.  It is recommended to use function
\code{sane\_strstatus()} to convert status codes into a legible
string.

\begin{table}[htbp]
  \begin{center}
    \begin{tabular}{|l|r|l|}
      \hline
      \multicolumn{1}{|c|}{\bf Symbol} & \multicolumn{1}{c|}{\bf Code} &
      \multicolumn{1}{c|}{\bf Description} \\
      \hline\hline
\code{\defn{SANE\_STATUS\_GOOD}}
        &  0 & Operation completed succesfully. \\
\code{\defn{SANE\_STATUS\_UNSUPPORTED}}
        &  1 & Operation is not supported. \\
\code{\defn{SANE\_STATUS\_CANCELLED}}
        &  2 & Operation was cancelled. \\
\code{\defn{SANE\_STATUS\_DEVICE\_BUSY}}
        &  3 & Device is busy---retry later. \\
\code{\defn{SANE\_STATUS\_INVAL}}
        &  4 & Data or argument is invalid. \\
\code{\defn{SANE\_STATUS\_EOF}}
        &  5 & No more data available (end-of-file). \\
\code{\defn{SANE\_STATUS\_JAMMED}}
        &  6 & Document feeder jammed. \\
\code{\defn{SANE\_STATUS\_NO\_DOCS}}
        &  7 & Document feeder out of documents. \\
\code{\defn{SANE\_STATUS\_COVER\_OPEN}}
        &  8 & Scanner cover is open. \\
\code{\defn{SANE\_STATUS\_IO\_ERROR}}
        &  9 & Error during device I/O. \\
\code{\defn{SANE\_STATUS\_NO\_MEM}}
        & 10 & Out of memory. \\
\code{\defn{SANE\_STATUS\_ACCESS\_DENIED}}
        & 11 & Access to resource has been denied. \\
      \hline
    \end{tabular}
    \caption{Status Codes}\label{tab:status}
  \end{center}
\end{table}


\subsection{Device Descriptor Type}
\begin{changebar}

Each SANE device is represented by a structure of type
\code{\defn{SANE\_Device}}.  The C declaration of this type is given
below.
\begin{quote}
\begin{verbatim}
typedef struct
  {
    SANE_String_Const name;
    SANE_String_Const vendor;
    SANE_String_Const model;
    SANE_String_Const type;
    SANE_String_Const email_backend_author;
    SANE_String_Const backend_website;
    SANE_String_Const device_location;
    SANE_String_Const comment;
    SANE_String_Const reserved_string;
    SANE_Int backend_version_code;
    SANE_Int backend_capablity_flags;
    SANE_Int reserved_int;
  }
SANE_Device;
\end{verbatim}
\end{quote}
\end{changebar}
\index{device-name}
The structure provides the unique name of the scanner in member
\code{name}.  It is this unique name that should be passed in a call
to \code{sane\_open()}.  The format of this name is completely up to
the backend.  The only constraints are that the name is unique among
all devices supported by the backend and that the name is a legal SANE
text string.  To simplify presentation of unique names, their length
should not be excessive.  It is {\em recommended\/} that backends keep
unique names below 32 characters in length.  However, applications
{\em must\/} be able to cope with arbitrary length unique names.

The next three members in the device structure provide additional
information on the device corresponding to the unique name.
Specifically, members \code{vendor}, \code{model}, and \code{type} are
single-line strings that give information on the vendor
(manufacturer), model, and the type of the device.  For consistency's
sake, the following strings should be used when appropriate (the lists
will be expanded as need arises):

\begin{table}[htbp]
  \begin{center}
    \leavevmode
    \hspace{\fill}
    \begin{tabular}[t]{|ll|}
      \hline
      \multicolumn{2}{|c|}{\bf \defn{Vendor Strings}} \\
      \hline\hline
      \code{AGFA}            & \code{Microtek} \\
      \code{Abaton}          & \code{Minolta} \\
      \code{Acer}            & \code{Mustek} \\
      \code{Apple}           & \code{NEC} \\
      \code{Artec}           & \code{Nikon} \\
      \code{Avision}         & \code{Plustek} \\
      \code{CANON}           & \code{Polaroid} \\
      \code{Connectix}       & \code{Ricoh} \\
      \code{Epson}           & \code{Sharp} \\
      \code{Fujitsu}         & \code{Siemens} \\
      \code{Hewlett-Packard} & \code{Tamarack} \\
      \code{IBM}             & \code{UMAX} \\
      \code{Kodak}           & \code{Noname} \\
      \code{Logitech}        & \\ 
      \hline
    \end{tabular}
    \hspace{\fill}
    \begin{tabular}[t]{|l|}
      \hline
      \multicolumn{1}{|c|}{\bf \defn{Type Strings}} \\
      \hline\hline
      \code{film scanner} \\
      \code{flatbed scanner} \\
      \code{frame grabber} \\
      \code{handheld scanner} \\
      \code{multi-function peripheral} \\
      \code{sheetfed scanner} \\
      \code{still camera} \\
      \code{video camera} \\
      \code{virtual device} \\
      \hline
    \end{tabular}
    \hspace{\fill}
    \caption{Predefined Device Information Strings}
    \label{tab:devinfo}
  \end{center}
\end{table}
Note that vendor string \code{Noname} can be used for virtual devices
that have no physical vendor associated.  Also, there are no
predefined model name strings since those are vendor specific and
therefore completely under control of the respective backends.

\begin{changebar}
The backend has to set up the string \code{email\_backend\_author}
with the name and the email address of the backend author or a contact person
in the format:
\begin{verbatim}
Firstname Lastname <name@domain.org>
\end{verbatim}

The string \code{backend\_website} should be set up by the backend
with the website or ftp address of the backend in the format:
\begin{verbatim}
http://www.domain.org/sane-hello/index.html
\end{verbatim}

The backend should fill the string \code{device\_location} with a
text that describes where a user can find this device. The text 
should be configurable by the administrator. This could e.g. look
like this:
\begin{verbatim}
building 93, 2nd plane, room 2124
\end{verbatim}

The string \code{comment} can be used to display any comment about the device
to the user. The text should be configurable by the administrator.

The string \code{reserved\_string} is unused currently but planned for
future use.

All unused strings must be set to ``'' by the backend.

With member \code{backend\_version\_code} a frontend can find
out the version of a backend it is connected to via one or more meta
backends.

The member \code{backend\_capability\_flags} contains 32 bits that
are planned to give the backend the chance to inform the frontend about
its capabilities. The meaning of the flags will be defined when there
is the need for it. The backend has to set all not defined bits
(in the moment all 32 bits) to 0.

The member \code{reserved\_int} is planned for future use, the backend has
to set the value of the integer to 0.
\end{changebar}

\subsection{Option Descriptor Type}\label{sec:odesc}

Option descriptors are at the same time the most intricate and
powerful type in the SANE standard.  Options are used to control
virtually all aspects of device operation.  Much of the power of the
SANE API stems from the fact that most device controls are completely
described by their respective option descriptor.  Thus, a frontend can
control a scanner abstractly, without requiring knowledge as to what
the purpose of any given option is.  Conversely, a scanner can
describe its controls without requiring knowledge of how the frontend
operates.  The C declaration of the
\code{\defn{SANE\_Option\_Descriptor}} type is given below.
\begin{quote}
\begin{verbatim}
typedef struct
  {
    SANE_String_Const name;
    SANE_String_Const title;
    SANE_String_Const desc;
    SANE_Value_Type type;
    SANE_Unit unit;
    SANE_Int size;
    SANE_Int cap;
    SANE_Constraint_Type constraint_type;
    union
      {
        const SANE_String_Const *string_list;
        const SANE_Word *word_list;
        const SANE_Range *range;
      }
    constraint;
  }
SANE_Option_Descriptor;
\end{verbatim}
\end{quote}

\subsubsection{Option Name}

Member \code{name} is a string that uniquely identifies the option.
The name must be unique for a given device (i.e., the option names
across different backends or devices need not be unique).  The option
name must consist of lower-case ASCII letters (\code{a}--\code{z}),
digits (\code{0}--\code{9}), or the dash character (\code{-}) only.
The first character must be a lower-case ASCII character (i.e., not a
digit or a dash).

\subsubsection{Option Title}

Member \code{title} is a single-line string that can be used by the
frontend as a title string.  This should typically be a short (one or
two-word) string that is chosen based on the function of the option.

\subsubsection{Option Description}

Member \code{desc} is a (potentially very) long string that can be
used as a help text to describe the option.  It is the responsibility
of the frontend to break the string into managable-length lines.
Newline characters in this string should be interpreted as paragraph
breaks.

\subsubsection{Option Value Type}

Member \code{type} specifies the type of the option value.  The
possible values for type \code{\defn{SANE\_Value\_Type}} are described
in Table \ref{tab:valuetype}.

\begin{table}[htbp]
  \begin{center}
    \leavevmode
    \begin{tabular}{|l|l|p{0.6\textwidth}|}
\hline
\multicolumn{1}{|c|}{\bf Symbol} &
\multicolumn{1}{c|}{\bf Code} &
\multicolumn{1}{c|}{\bf Description} \\
\hline\hline

\code{\defn{SANE\_TYPE\_BOOL}} & 0 & Option value is of type
  \verb|SANE_Bool|. \\ 

\code{\defn{SANE\_TYPE\_INT}} & 1 & Option value is of type
  \verb|SANE_Int|. \\ 

\code{\defn{SANE\_TYPE\_FIXED}}&2 & Option value is of type
  \verb|SANE_Fixed|. \\ 

\code{\defn{SANE\_TYPE\_STRING}}&3 & Option value is of type
  \verb|SANE_String|. \\ 

\code{\defn{SANE\_TYPE\_BUTTON}} & 4 & An option of this type has no value.
Instead, setting an option of this type has an option-specific
side-effect.  For example, a button-typed option could be used by a
backend to provide a means to select default values or to the tell an
automatic document feeder to advance to the next sheet of paper. \\ 

\code{\defn{SANE\_TYPE\_GROUP}} & 5 & An option of this type has no value.
This type is used to group logically related options.  A group option
is in effect up to the point where another group option is encountered
(or up to the end of the option list, if there are no other group
options).  For group options, only members \code{title} and
\code{type} are valid in the option descriptor. \\ 

  \hline
    \end{tabular}
    \caption{Option Value Types (\code{SANE\_Value\_Type})}
    \label{tab:valuetype}
  \end{center}
\end{table}

\subsubsection{Option Value Unit}

Member \code{unit} specifies what the physical unit of the option
value is.  The possible values for type \code{\defn{SANE\_U\-nit}} are
described in Table \ref{tab:units}.  Note that the specified unit is
what the SANE backend expects.  It is entirely up to a frontend as to
how these units a presented to the user.  For example, SANE expresses
all lengths in millimeters.  A frontend is generally expected to
provide appropriate conversion routines so that a user can express
quantities in a customary unit (e.g., inches or centimeters).

\begin{table}[htbp]
  \begin{center}
    \leavevmode
    \begin{tabular}{|l|l|l|}
\hline
\multicolumn{1}{|c|}{\bf Symbol} &
\multicolumn{1}{|c|}{\bf Code} &
\multicolumn{1}{|c|}{\bf Description} \\

\hline\hline

\code{\defn{SANE\_UNIT\_NONE}}   & 0 & Value is unit-less (e.g., page count).\\
\code{\defn{SANE\_UNIT\_PIXEL}}  & 1 & Value is in number of pixels. \\
\code{\defn{SANE\_UNIT\_BIT}}    & 2 & Value is in number of bits. \\
\code{\defn{SANE\_UNIT\_MM}}     & 3 & Value is in millimeters. \\
\code{\defn{SANE\_UNIT\_DPI}}    & 4 & Value is a resolution in dots/inch. \\
\code{\defn{SANE\_UNIT\_PERCENT}}& 5 & Value is a percentage. \\
\code{\defn{SANE\_UNIT\_MICROSECOND}}& 6 & Value is time in $\mu$-seconds. \\

\hline
    \end{tabular}
    \caption{Physical Units (\code{SANE\_Unit})}
    \label{tab:units}
  \end{center}
\end{table}

\subsubsection{Option Value Size}\label{sec:valuesize}

Member \code{size} specifies the size of the option value (in bytes).
This member has a slightly different interpretation depending on the
type of the option value:
\begin{quote}
  \begin{description}
  \item[\code{SANE\_TYPE\_STRING}:] The size is the maximum size of
    the string.  For the purpose of string size calcuations, the
    terminating \code{NUL} character is considered to be part of the
    string.  Note that the terminating \code{NUL} character must
    always be present in string option values.
  \item[\code{SANE\_TYPE\_INT}, \code{SANE\_TYPE\_FIXED}:] The size
    must be a positive integer multiple of the size of a
    \verb|SANE_Word|.  The option value is a vector of length
    \[ \code{size}/\code{sizeof(SANE\_Word)}. \]
  \item[\code{SANE\_TYPE\_BOOL}:] The size must be set to
    \code{sizeof(SANE\_Word)}.
  \item[\code{SANE\_TYPE\_BUTTON}, \code{SANE\_TYPE\_GROUP}:] The
    option size is ignored.
  \end{description}
\end{quote}

\subsubsection{Option Capabilities}

Member \code{cap} describes what capabilities the option posseses.
This is a bitset that is formed as the inclusive logical OR of the
capabilities described in Table \ref{tab:capabilities}.  The SANE API
provides the following to macros to test certain features of a given
capability bitset:
\begin{quote}
  \begin{description}

  \item[\code{\defn{SANE\_OPTION\_IS\_ACTIVE}(\var{cap})}:] This macro
    returns \code{SANE\_TRUE} if and only if the option with the
    capability set \var{cap} is currently active.

  \item[\code{\defn{SANE\_OPTION\_IS\_SETTABLE}(\var{cap})}:] This
    macro returns \code{SANE\_TRUE} if and only if the option with the
    capability set \var{cap} is software settable.
  \end{description}
\end{quote}

\begin{table}[htbp]
  \begin{center}
    \leavevmode
    \begin{tabular}{|l|r|p{0.59\textwidth}|}
\hline
\multicolumn{1}{|c|}{\bf Symbol} &
\multicolumn{1}{c|}{\bf Code} &
\multicolumn{1}{c|}{\bf Description} \\
\hline\hline

\code{\defn{SANE\_CAP\_SOFT\_SELECT}} & 1 & The option
  value can be set by a call to \code{sane\_con\-trol\_opt\-ion()}.\\ 

\code{\defn{SANE\_CAP\_HARD\_SELECT}} & 2 & The option value can be set by
  user-intervention (e.g., by flipping a switch).  The user-interface
  should prompt the user to execute the appropriate action to set such
  an option.  This capability is mutually exclusive with
  SANE\_CAP\_SOFT\_SELECT (either one of them can be set, but not both
  simultaneously).  \\ 

\code{\defn{SANE\_CAP\_SOFT\_DETECT}} & 4 & The option
  value can be detected by software.  If
  \code{SANE\_\-CAP\_\-SO\-FT\_SEL\-ECT} is set, this capability {\em must\/}
  be set.  If \code{SANE\_CAP\_HARD\_SELECT} is set, this capability
  may or may not be set.  If this capability is set but neither
  \code{SANE\_CAP\_SO\-FT\_SEL\-ECT} nor \code{SANE\_CAP\_HA\-RD\_SEL\-ECT}
  are, then there is no way to control the option.  That is, the
  option provides read-out of the current value only.  \\ 

\code{\defn{SANE\_CAP\_EMULATED}} & 8 & If set, this capability indicates
  that an option is not directly supported by the device and is
  instead emulated in the backend.  A sophisticated frontend may
  elect to use its own (presumably better) emulation in lieu of an emulated
  option. \\ 

\code{\defn{SANE\_CAP\_AUTOMATIC}} & 16 & If set, this capability indicates
  that the backend (or the device) is capable to picking a reasonable
  option value automatically.  For such options, it is possible to
  select automatic operation by calling \code{sane\_control\_option()}
  with an action value of \code{SANE\_ACTION\_SET\_AUTO}.  \\ 

\code{\defn{SANE\_CAP\_INACTIVE}} & 32 & If set, this capability indicates
  that the option is not currently active (e.g., because it's
  meaningful only if another option is set to some other value). \\ 

\code{\defn{SANE\_CAP\_ADVANCED}} & 64 &
  If set, this capability indicates that the option should be
  considered an ``advanced user option''. 
\begin{changebar}
  If this capability is set for an option of type
  \code{SANE\_TYPE\_GROUP}, all options belonging to the group are also
  advanced, even if they don't set the capabilty themselves. 
\end{changebar}
  A frontend typically
  displays such options in a less conspicuous way than regular options
  (e.g., a command line interface may list such options last or a
  graphical interface may make them available in a seperate ``advanced
  settings'' dialog). 
  \\

\code{\defn{SANE\_CAP\_HIDDEN}} & 128 &
\begin{changebar}
  If set, this capability indicates that the option shouldn't be
  displayed to and used by the user directly. Instead a hidden option
  is supposed to be automatically used by the frontend, like e.g. the
  ``preview'' option.
  If this capability is set for an option of type
  \code{SANE\_TYPE\_GROUP}, all options belonging to the group are also
  hidden, even if they don't set the capabilty themselves. 
  A frontend typically doesn't display such options by default but there
  should be a way to override this default behaviour.
\end{changebar}
  \\

\code{\defn{SANE\_CAP\_ALWAYS\_SETTABLE}} & 256 & 
\begin{changebar}
  If set, this capability indicates that the option may be at any time between
  \code{sane\_open()} and \code{sane\_close()}. I.e.  it's allowed to set it
  even while an image is acquired.
\end{changebar}
\\

\hline
    \end{tabular}
    \caption{Option Capabilities}
    \label{tab:capabilities}
  \end{center}
\end{table}

\subsubsection{Option Value Constraints}

It is often useful to constrain the values that an option can take.
For example, constraints can be used by a frontend to determine how to
represent a given option.  Member \code{constraint\_type} indicates
what constraint is in effect for the option.  The constrained values
that are allowed for the option are described by one of the union
members of member \code{constraint}.  The possible values of type
\code{\defn{SANE\_Constraint\_Type}} and the interpretation of the
\code{constraint} union is described in Table~\ref{tab:constraints}.

\begin{table}[htbp]
  \begin{center}
    \leavevmode
    \begin{tabular}{|l|r|p{0.5\textwidth}|}
\hline
\multicolumn{1}{|c|}{\bf Symbol} &
\multicolumn{1}{|c|}{\bf Code} &
\multicolumn{1}{|c|}{\bf Description} \\

\hline\hline

\code{\defn{SANE\_CONSTRAINT\_NONE}} & 0 & The value is unconstrained.
  The option can take any of the values possible for the option's
  type. \\

  \code{\defn{SANE\_CONSTRAINT\_RANGE}} & 1 & This constraint is
  applicable to integer and fixed-point valued options only.  It
  constrains the option value to a possibly quantized range of
  numbers.  Option descriptor member \code{constraint.range} points to
  a range of the type \code{\defn{SANE\_Range}}.  This type is illustrated
  below:
  \begin{quote}
\begin{verbatim}
typedef struct
  {
    SANE_Word min;
    SANE_Word max;
    SANE_Word quant;
  }
SANE_Range;
\end{verbatim}
  \end{quote}
  All three members in this structure are interpreted according to the
  option value type (\verb|SANE_TYPE_INT| or \verb|SANE_TYPE_FIXED|).
  Members \code{min} and \code{max} specify the minimum and maximum
  values, respectively.  If member \code{quant} is non-zero, it
  specifies the quantization value.  If $l$ is the minimum value, $u$
  the maximum value and $q$ the (non-zero) quantization of a range,
  then the legal values are $v=k\cdot q+l$ for all non-negative
  integer values of $k$ such that $v<=u$.  \\ 

\code{\defn{SANE\_CONSTRAINT\_WORD\_LIST}} & 2 & This constraint is applicable
  to integer and fixed-point valued options only.  It constrains the
  option value to a list of numeric values.  Option descriptor member
  \code{constraint.word\_list} points to a list of words that
  enumerates the legal values.  The first element in that list is an
  integer (\verb|SANE_Int|) that specifies the length of the list (not
  counting the length itself).  The remaining elements in the list are
  interpreted according to the type of the option value
  (\verb|SANE_TYPE_INT| or \verb|SANE_TYPE_FIXED|). \\ 

\code{\defn{SANE\_CONSTRAINT\_STRING\_LIST}} & 3 & This constraint is
  applicable to string-valued options only.  It constrains the option
  value to a list of strings.  The option descriptor member
  \code{con\-strai\-nt.str\-ing\_list} points to a \code{NULL} terminated
  list of strings that enumerate the legal values for the option
  value.
\\\hline
    \end{tabular}
    \caption{Option Value Constraints}
    \label{tab:constraints}
  \end{center}
\end{table}


\subsection{Internationalization}\label{sec:i18n}
\begin{changebar}
All backend texts should be written in english.
Localization (translation of backend texts) has to be done
in the frontend. To automatically prepare translation tables
(e.g. english to german) it is necessary to mark the texts
that shall be translated.

\subsubsection{How is a text marked for translation}

The keyword \code{SANE\_I18N} is used to mark a text for translation.
\code{SANE\_I18N} has to be defined as a dummy prototype:

  \begin{quote}
\begin{verbatim}
#ifndef SANE_I18N
#  define SANE_I18N(text) text
#endif
\end{verbatim}
  \end{quote}

You should not mark prototypes or variables with \code{SANE\_I18N}
because it is not possible (or very hard) to find out the
texts that are covered by prototypes (\verb|SANE_I18N(START_SCAN_TEXT)|) and
variables (\verb|SANE_I18N(option[7].name)|).

A correct marked text can look like this:\\
\verb|snprintf(buf, sizeof(buf), SANE_I18N("Start scan")|\\
or\\
\verb|#define START_SCAN_TEXT SANE_I18N("Start scan")|.\\
It also is allowed to mark texts in structs because the
prototype \code{SANE\_I18N} has no effect to the compiler.

\subsubsection{Which texts shall be marked for translation?}
All option texts that are visible for the user should be marked for
translation. This is:
\begin{itemize}
\item
member \code{title}
\item
member \code{desc}
\item
member \code{string\_list}
\end{itemize}
of \code{SANE\_Option\_Descriptor}.

It is not allowed to mark/translate member \code{name}.
Please also do not mark any error or debug messages that are
displayed by the backend.

\subsubsection{File formats and translation functions}
The recommended file formats for translation tables are the \code{po} files
and \code{mo} or \code{gmo} files.
The po file contains the original text marked with the keyword \code{msgid}
and the translated text marked with the keyword \code{msgstr}.
A \code{po} file that contains all needed \code{msgid}s can be created
e.g. by the gnu gettext tool \verb|xgettext| with the option
\verb|-k SANE_I18N|.
The translator adds the translated texts to the \code{po} files.
The gettext tool \code{msgfmt} converts the \code{po} files to the
\code{mo} or \code{gmo} files.\\

Translation is done in the frontend. A backend has nothing
to do with the translation of texts.
The frontend should use the function \code{gettext}
to translate the texts. This e.g. can look like this:\\
\verb|snprintf(buf, sizeof(buf), gettext("english text"));|

If a frontend author decides to use translation functions that
need different translation tables, then the frontend is
responsible to create/convert the translation tables.
In this case it should use the \code{po} files to create its
own translation tables from it.\\

Note that it is strongly recommended to use the gettext tools.
\end{changebar}


\section{Operations}

\subsection{\code{sane\_init}}

This function must be called before any other SANE function can be called.
The behavior of a SANE backend is undefined if this function is not called
first or if the status code returned by \code{sane\_init} is different from
\code{\defn{SANE\_STATUS\_GOOD}}.  The version code of the backend is returned
in the value pointed to by \code{version\_code}.  If that pointer is
\code{NULL}, no version code is returned.  Argument \code{authorize} is either
a pointer to a function that is invoked when the backend requires
authentication for a specific resource or \code{NULL} if the frontend does not
support authentication.
\begin{quote}\index{sane\_init}
\begin{verbatim}
SANE_Status sane_init (SANE_Int * version_code,
                       SANE_Authorization_Callback authorize);
\end{verbatim}
\end{quote}

The authorization function may be called by a backend in response to
any of the following calls:
\begin{quote}
  \code{sane\_open}, \code{sane\_control\_option}, \code{sane\_start}
\end{quote}
If a backend was initialized without authorization function, then
authorization requests that cannot be handled by the backend itself
will fail automatically and the user may be prevented from accessing
protected resources.  Backends are encouraged to implement means of
authentication that do not require user assistance.  E.g., on a
multi-user system that authenticates users through a login process a
backend could automatically lookup the apporpriate password based on
resource- and user-name.

The authentication function type has the following declaration:
\begin{quote}\index{SANE\_Authorization\_Callback}
  \index{domain}\index{username}\index{password}
\begin{verbatim}
#define SANE_MAX_USERNAME_LEN   128
#define SANE_MAX_PASSWORD_LEN   128

typedef void (*SANE_Authorization_Callback)
    (SANE_String_Const resource,
     SANE_Char username[SANE_MAX_USERNAME_LEN],
     SANE_Char password[SANE_MAX_PASSWORD_LEN]);
\end{verbatim}
\end{quote}
Three arguments are passed to the authorization function:
\code{resource} is a string specifying the name of the resource that
requires authorization.  A frontend should use this string to build a
user-prompt requesting a username and a password.  The \code{username}
and \code{password} arguments are (pointers to) an array of
\code{SANE\_MAX\_USERNAME\_LEN} and \code{SANE\_MAX\_PASSWORD\_LEN}
characters, respectively.  The authorization call should place the
entered username and password in these arrays.  The returned strings
{\em must\/} be ASCII-NUL terminated.

\subsection{\code{sane\_exit}}

This function must be called to terminate use of a backend.  The
function will first close all device handles that still might be open
(it is recommended to close device handles explicitly through a call
to \code{sane\_clo\-se()}, but backends are required to release all
resources upon a call to this function).  After this function returns,
no function other than \code{sane\_init()} may be called (regardless
of the status value returned by \code{sane\_exit()}.  Neglecting to
call this function may result in some resources not being released
properly.
\begin{quote}\index{sane\_exit}
\begin{verbatim}
void sane_exit (void);
\end{verbatim}
\end{quote}


\subsection{\code{sane\_get\_devices}}

This function can be used to query the list of devices that are
available.  If the function executes successfully, it stores a pointer
to a \code{NULL} terminated array of pointers to \verb|SANE_Device|
structures in \code{*device\_list}.  The returned list is guaranteed
to remain unchanged and valid until (a) another call to this function
is performed or (b) a call to \code{sane\_exit()} is performed.  This
function can be called repeatedly to detect when new devices become
available.  If argument \code{local\_only} is true, only local devices
are returned (devices directly attached to the machine that SANE is
running on).  If it is false, the device list includes all remote
devices that are accessible to the SANE library.
\begin{quote}\index{sane\_get\_devices}
\begin{verbatim}
SANE_Status sane_get_devices (const SANE_Device *** device_list,
                              SANE_Bool local_only);
\end{verbatim}
\end{quote}

This function may fail with \code{SANE\_STATUS\_NO\_MEM} if an
insufficient amount of memory is available.

\begin{quote}
  \begin{center}
    {\bf Backend Implementation Note}
  \end{center}
  SANE does not require that this function is called before a
  \code{sane\_open()} call is performed.  A device name may be
  specified explicitly by a user which would make it unnecessary and
  undesirable to call this function first.
\begin{changebar}
  The same information about
  a device has to be returned when \code{sane\_open} is called.
\end{changebar}
\end{quote}


\subsection{\code{sane\_open}}

This function is used to establish a connection to a particular
device.  The name of the device to be opened is passed in argument
\code{name}.  If the call completes successfully, a handle for the
device is returned in \code{*h}.
\begin{changebar}
The description of the device
is returned in \code{**device\_description}. This is the
same description that is returned in the list by \code{sane\_get\_devices}.
The returned data \code{*h} and \code{*device\_description} is guaranteed
to remain unchanged and valid until a call to \code{sane\_close()}
is performed.
\end{changebar}
As a special case, specifying a zero-length string as the device requests
opening the first available device (if there is such a device).
\begin{quote}\index{sane\_open}
\begin{verbatim}
SANE_Status sane_open (SANE_String_Const name, SANE_Handle * h,
const SANE_Device ** device_description);
\end{verbatim}
\end{quote}

This function may fail with one of the following status codes.
\begin{quote}
\begin{description}
\item[\code{SANE\_STATUS\_DEVICE\_BUSY}:] The device is currently
  busy (in use by somebody else).
\item[\code{SANE\_STATUS\_INVAL}:] The device name is not valid.
\item[\code{SANE\_STATUS\_IO\_ERROR}:] An error occured while
  communicating with the device.
\item[\code{SANE\_STATUS\_NO\_MEM}:] An insufficent amount of memory
  is available.
\item[\code{SANE\_STATUS\_ACCESS\_DENIED}:] Access to the device has
  been denied due to insufficient or invalid authentication.
\end{description}
\end{quote}


\subsection{\code{sane\_close}}

This function terminates the association between the device handle
passed in argument \code{h} and the device it represents.  If the
device is presently active, a call to \code{sane\_cancel()} is
performed first.  After this function returns, handle \code{h} must
not be used anymore.

\begin{quote}\index{sane\_close}
\begin{verbatim}
void sane_close (SANE_Handle h);
\end{verbatim}
\end{quote}

\subsection{\code{sane\_get\_option\_descriptor}}

This function is used to access option descriptors.  The function
returns the option descriptor for option number \code{n} of the device
represented by handle \code{h}.  Option number 0 is guaranteed to be a
valid option.  Its value is an integer that specifies the number of
options that are available for device handle \code{h} (the count
includes option 0).  If $n$ is not a valid option index, the function
returns \code{NULL}.  The returned option descriptor is guaranteed to
remain valid (and at the returned address) until the device is closed.

\begin{quote}\index{sane\_get\_option\_descriptor}
\begin{verbatim}
const SANE_Option_Descriptor *
    sane_get_option_descriptor (SANE_Handle h, SANE_Int n);
\end{verbatim}
\end{quote}

\subsection{\code{sane\_control\_option}}\label{sec:control}

This function is used to set or inquire the current value of option
number \code{n} of the device represented by handle \code{h}.  The
manner in which the option is controlled is specified by parameter
\code{a}.  The possible values of this parameter are described in more
detail below.  The value of the option is passed through argument
\code{v}.  It is a pointer to the memory that holds the option value.
The memory area pointed to by \code{v} must be big enough to hold the
entire option value (determined by member \code{size} in the
corresponding option descriptor).  The only exception to this rule is
that when setting the value of a string option, the string pointed to
by argument \code{v} may be shorter since the backend will stop
reading the option value upon encountering the first \code{NUL}
terminator in the string.  If argument \code{i} is not \code{NULL},
the value of \code{*i} will be set to provide details on how well the
request has been met.  The meaning of this argument is described in
more detail below.
\begin{quote}\index{sane\_control\_option}
\begin{verbatim}
SANE_Status sane_control_option (SANE_Handle h, SANE_Int n,
                                 SANE_Action a, void *v,
                                 SANE_Int * i);
\end{verbatim}
\end{quote}

The way the option is affected by a call to this function is
controlled by parameter \code{a} which is a value of type
\code{\defn{SANE\_Action}}.  The possible values and their meaning is
described in Table~\ref{tab:actions}.

\begin{table}[h]
  \begin{center}
    \leavevmode
    \begin{tabular}{|l|r|p{0.5\textwidth}|}
\hline
\multicolumn{1}{|c|}{\bf Symbol} &
\multicolumn{1}{|c|}{\bf Code} &
\multicolumn{1}{|c|}{\bf Description} \\

\hline\hline

\code{\defn{SANE\_ACTION\_GET\_VALUE}} & 0 & Get current option value. \\

\code{\defn{SANE\_ACTION\_SET\_VALUE}} & 1 & Set option value.  The
  option value passed through argument \code{v} may be modified by the
  backend if the value cannot be set exactly.  \\ 

\code{\defn{SANE\_ACTION\_SET\_AUTO}} & 2 & Turn on automatic mode.  Backend
  or device will automatically select an appropriate value.  This mode
  remains effective until overridden by an explicit set value request.
  The value of parameter \code{v} is completely ignored in this case and
  may be \code{NULL}.  \\

\hline
    \end{tabular}
    \caption{Action Values (\code{SANE\_Action})}
    \label{tab:actions}
  \end{center}
\end{table}

After setting a value via an action value of
\verb|SANE_ACTION_SET_VALUE|, additional information on how well the
request has been met is returned in \code{*i} (if \code{i} is
non-\code{NULL}).  The returned value is a bitset that may contain any
combination of the values described in Table~\ref{tab:info}.
\begin{table}[htbp]
  \begin{center}
    \leavevmode
    \begin{tabular}{|l|r|p{0.5\textwidth}|}
\hline
\multicolumn{1}{|c|}{\bf Symbol} &
\multicolumn{1}{|c|}{\bf Code} &
\multicolumn{1}{|c|}{\bf Description} \\

\hline\hline

\code{\defn{SANE\_INFO\_INEXACT}} & 1 & This value is returned when
  setting an option value resulted in a value being selected that does
  not exactly match the requested value.  For example, if a scanner
  can adjust the resolution in increments of 30dpi only, setting the
  resolution to 307dpi may result in an actual setting of 300dpi.
  When this happens, the bitset returned in \code{*i} has this member
  set.  In addition, the option value is modified to reflect the
  actual (rounded) value that was used by the backend.  Note that
  inexact values are admissible for strings as well.  A backend may
  choose to ``round'' a string to the closest matching legal string
  for a constrained string value. \\

\code{\defn{SANE\_INFO\_RELOAD\_OPTIONS}} & 2 & 
The setting of an option may affect the
\begin{changebar}
  value, the availability, or the constraint of one or more {\em
   other\/} options.
\end{changebar}
  When this happens, the SANE backend sets this
  member in \code{*i} to indicate that the application should reload
  all options.  This member may be set if and only if at least one
  option changed. \\

\code{\defn{SANE\_INFO\_RELOAD\_PARAMS}} & 4 &
The setting of an option may affect the parameter values (see
\code{sane\_get\_parameters()}).
If setting an option affects the parameter values, this member will
be set in \code{*i}.  Note that this member may be set even if the
parameters did not actually change.  However, it is guaranteed that
the parameters never change without this member being set.
\\

\code{\defn{SANE\_INFO\_INVALIDATE\_PREVIEW}} & 8 &
The setting of an option may affect the validity of the preview that 
was acquired by the frontend earlier. When the preview image would change
significantly if it was scanned again, the SANE backend
sets this member in \code{*i} to indicate that the application should inform the
user of the invalidity of the preview. Examples for such option
settings may include setting a different scan source or significantly
changing the exposure.
\\


\hline
    \end{tabular}
    \caption{Additional Information Returned When Setting an Option}
    \label{tab:info}
  \end{center}
\end{table}

This function may fail with one of the following status codes.
\begin{quote}
\begin{description}
\item[\code{SANE\_STATUS\_UNSUPPORTED}:] The operation is not
  supported for the specified handle and option number.
\item[\code{SANE\_STATUS\_INVAL}:] The option value is not valid.
\item[\code{SANE\_STATUS\_IO\_ERROR}:] An error occured while
  communicating with the device.
\item[\code{SANE\_STATUS\_NO\_MEM}:] An insufficent amount of memory
  is available.
\item[\code{SANE\_STATUS\_ACCESS\_DENIED}:] Access to the option has
  been denied due to insufficient or invalid authentication.
\end{description}
\end{quote}



\subsection{\code{sane\_get\_parameters}\label{sec:sanegetparameters}}

This function is used to obtain the current scan parameters.  The
returned parameters are guaranteed to be accurate between the time a
scan has been started (\code{sane\_start()} has been called) and the
completion of that request.  Outside of that window, the returned
values are best-effort estimates of what the parameters will be when
\code{sane\_start()} gets invoked.  Calling this function before a
scan has actually started allows, for example, to get an estimate of
how big the scanned image will be.  The parameters passed to this
function are the handle \code{h} of the device for which the
parameters should be obtained and a pointer \code{p} to a parameter
structure.  The parameter structure is described in more detail below.

\begin{quote}\index{sane\_get\_parameters}
\begin{verbatim}
SANE_Status sane_get_parameters (SANE_Handle h,
                                 SANE_Parameters * p);
\end{verbatim}
\end{quote}

The scan parameters are returned in a structure of type
\code{\defn{SANE\_Parameters}}.  The C declaration of this structure
is given below.
\begin{changebar}
\begin{quote}
\begin{verbatim}
typedef struct
  {
    SANE_Frame format;
    SANE_Int flags;
    SANE_Int lines;
    SANE_Int depth;
    SANE_Int pixels_per_line;
    SANE_Int bytes_per_line;
    SANE_Int channels_per_image;
    SANE_String format_desc;
    SANE_String proposed_filename;
    SANE_String proposed_comment;
    SANE_Int dpi_x;
    SANE_Int dpi_y;
    char reserved[32]; /* 32 bytes for future use */
  }
SANE_Parameters;
\end{verbatim}
\end{quote}
\end{changebar}

\begin{changebar}
Member \code{format} specifies the format of the next frame to be
returned.  The possible values for type \code{\defn{SANE\_Frame}} are
described in Table~\ref{tab:frameformat}.  The meaning of these
values is described in more detail in section~\ref{sec:imageformat}.
The frame types used by version~1 of this standard
(\code{\defn{SANE\_FRAME\_GRAY}}, \code{\defn{SANE\_FRAME\_RGB}},
\code{\defn{SANE\_FRAME\_RED}}, \code{\defn{SANE\_FRAME\_GREEN}}, and
\code{\defn{SANE\_FRAME\_BLUE}}) are obsolete and superseded by the 
frame type \code{\defn{SANE\_FRAME\_RAW}}.

\begin{table}[htbp]
  \begin{center}
    \leavevmode
    \begin{tabular}{|l|r|l|}
\hline
\multicolumn{1}{|c|}{\bf Symbol} &
\multicolumn{1}{|c|}{\bf Code} &
\multicolumn{1}{|c|}{\bf Description} \\

\hline\hline

\code{\defn{SANE\_FRAME\_RAW}}   & 5 & Arbitrary pixel property transmission. \\
\code{\defn{SANE\_FRAME\_MIME}}  & 6 & Data described by a mime descriptor. \\

\hline
    \end{tabular}
    \caption{Frame Format (\code{SANE\_Frame})}
    \label{tab:frameformat}
  \end{center}
\end{table}

The \code{flags} member is a 32 bit bitfield, for which up to now 4 
informational bits are defined, all unused bits have to be set to 0: 

\begin{itemize}

\item
\code{SANE\_PFLAG\_LAST\_FRAME} (bit 0, bitvalue 1) is set to 1 if and
only if the frame that is currently being acquired (or the frame that
will be acquired next if there is no current frame) is the only frame
or the last frame
of a multi frame image (e.g., the current frame is the blue component
of a red, green, blue image).

\item
\code{SANE\_PFLAG\_MORE\_IMAGES} (bit 1, bitvalue 2) is set to 1 to indicate 
further pending images. The frontend is expected to call \code{sane\_start}
again after the end of the current scan to get more images, e.g.
from an automatic document feeder.  It is permissible to set that value to 1 ``in good 
faith'', as it has to be determined at a very early time, where it might 
not be detectable, if there actually are more images to transfer. E.g. 
you will usually not know if the document feeder contains further pages 
when starting to scan the current one. Thus you are allowed to set that 
bit but later fail at \code{sane\_start()}.

\item
\code{SANE\_PFLAG\_NEW\_PAGE} (bit 2, bitvalue 4) is set to 1 to indicate 
that the current frame comes from a new physical page. This bit is of 
informational character only to help frontends to group multi-image 
scans. 

\item
\code{SANE\_PFLAG\_BACKSIDE} (bit 3, bitvalue 8) tells if the current image 
was acquired from the front (0) or backside (1) of the currently processed 
sheet. It is of informational character and allows to group and order 
multi-image transfers regardless of scanner acquisition order (front 
first/back first). 

\end{itemize}

Note, that \code{flags} is compatible to member \code{last\_frame} of 
\code{SANE\_Parameters} of SANE standard version 1 (same size
and only bit 0 (bitvalue 1) was used with same function).
\end{changebar}

Member \code{lines} specifies how many scan lines the frame is
comprised of.  If this value is -1, the number of lines is not known a
priori and the frontend should call \code{sane\_read()} until it
returns a status of \code{SANE\_STATUS\_EOF}.
\begin{changebar}
Note, that even when transferring formats that have this information 
inband, it is recommended to set that member, if available. If 
unavailable or not applicable, set to -1 as mentioned above. 
\end{changebar}

Member \code{bytes\_per\_line} specifies the number of bytes that comprise one
scan line.
\begin{changebar}
If this value is not applicable for image data of type
\code{SANE\_FRAME\_MIME}, it must be set to -1. In this case the frontend
should just call \code{sane\_read} until \code{SANE\_STATUS\_EOF} is returned.
\end{changebar}

Member \code{depth} specifies the number of bits per sample.
\begin{changebar}
Note, that only -1 (for not applicable), 1, and multiples of 8 are
allowed values. Data with other depths has to be scaled up accordingly.
Depth 1 is only allowed if the image consists of a single channel (lineart or
halftone modes).
\end{changebar}

Member \code{pixels\_per\_line} specifies the number of pixels that
comprise one scan line.
\begin{changebar}
If unavailable or not applicable, set to -1.
\end{changebar}

Assume $B$ is the number of channels in the frame, then the bit depth
$d$ (as given by member \code{depth}) and the number of pixels per
line $n$ (as given by this member \code{pixels\_per\_line}) are
related to $c$, the number of bytes per line (as given by member
\code{bytes\_per\_line}) as follows:
\begin{changebar}
\[
  c >= \left\{
  \begin{array}{ll}
    \lceil B\cdot n / 8\rceil & \mbox{if $d=1$}\\
    B\cdot n \cdot d / 8 & \mbox{if $d>1$}
  \end{array}
  \right.
\]
\end{changebar}
Note that the number of bytes per line can be larger than the minimum
value imposed by the right side of this equation.  A frontend must be
able to properly cope with such ``padded'' image formats.

\begin{changebar}
Member \code{channels\_per\_image} specifies the number of channels the
image consists of. When the image is transmitted in more than one frame
\code{channels\_per\_image} has to be the same for all frames for this image.

Member \code{format\_desc} is used to describe the details of the frame
formats.  Its meaning differs between the two formats:

\begin{itemize}
\item
\code{SANE\_FRAME\_RAW}: 
The \code{format\_desc} contains a description of the channel data and
an optional depth information separated by a colon(:).

A plane is descibed by one channel, e.g. ``\code{gray}'' or ``\code{gray:12}''.

Channel interleaved data is described by a comma separated list of channel descriptions,
for example ``\code{red,green,blue}'' or ``\code{red:8,green:8,blue:8}'',
the channel data is sent in the given order.

The depth information does {\bf not} define the size of the transmitted
channel data, it is only an information for the frontend. The channel data has
to be sent in the size defined by member \code{depth}.

Well known channels are \code{red}, \code{green}, \code{blue} and \code{gray}.
It also is allowed to use other channel descriptions, e.g. if you
use an infrared camera or scanner it could be \code{infrared} or
a wavelength description like \code{1100nm}, but be aware that a
frontend may not be able to display such channels with useful colors.

Note that an image can be sent in single planes, in one interleaved
frame that contains all channels or in several frames that contain
one or more (interleaved) channels. When an RGB image is sent it
is prefered to send the image data in one interleaved frame
that consist of red, green and blue data in this order.
The number of channels is defined in member \code{channels\_per\_image}.

\item
\code{SANE\_FRAME\_MIME}: 

\code{format\_desc} contains the MIME Content-Type: header field as described
in RFC 1521 (section 4) without the prefixing "Content-Type:". MIME Types and
subtypes should be either chosen from the RFC or from the list of
IANA-approved values. The data stream must be compliant with the corresponding
specification.

When data is transmitted with the frame type \code{SANE\_FRAME\_MIME}
all data has to be transmitted within one frame, multiple frames
are not allowed (so the flag \code{last\_frame} has to be set
when using this frame type).

A SANE backend must be able to at least optionally transmit
\code{SANE\_FRAME\_RAW} (possibly with the help of a meta backend), if the
hardware supports delivering image data at all. For data that doesn't
comprise images, it's admisable to only provide MIME frames. As a
general principle, if there are several choices in MIME types, the
format that is most widely implemented should be used.
\end{itemize}

The member \code{proposed\_filename} can be used to suggest a reasonable 
default filename or -extension in case the backend can make such a 
suggestion, like e.g. an image database. 
If no such suggestion is intended, set the field to ``''. 

In the case of raw frames, \code{proposed\_filename} is expected to hold
the basename for the image, with the extension determined by the save function
of the frontend, as the frontend can fully understand the data and is thus
able to encode it in any format it wishes.

For MIME frames \code{proposed\_filename} can contain either: 

\begin{itemize}
\item
A name with a leading dot, which is considered to be a proposed 
filename extension. This could also be gotten from the mime database, 
but for systems lacking it, this might be convenient. Or: 

\item
A complete filename, including extension. 
\end{itemize}

Note, that for frontends that are able to parse a given MIME type 
internally, it is perfectly permissible to ignore the extension 
part of the proposed filename and only make use of the basename, 
when using internal save algorithms for different formats. 

The string \code{proposed\_comment} can be used to transmit additional 
image information, that can be stored in the comment areas several file formats 
offer. It can contain any textual information the backend wishes to 
convey to the user, like date/time of exposure, enganged filters, 
etc. Set to ``'' if unused.

The members \code{dpi\_x} and \code{dpi\_y} encode the horizontal and vertical
resolution. Note, that multiple-image scans may have different resolutions of
each image. It is not permissible to change resolution between frames of the
same image. If not applicable, set to -1.

The member \code{reserved} is an array of 32 bytes (char) to keep
the size of the struct unchanged when future extensions are done.
The backend has to set the reserved bytes to 0.
\end{changebar}


\subsection{\code{sane\_start}}

This function initiates aquisition of an image from the device
represented by handle \code{h}.
\begin{quote}\index{sane\_start}
\begin{verbatim}
SANE_Status sane_start (SANE_Handle h);
\end{verbatim}
\end{quote}
This function may fail with one of the following status codes.
\begin{quote}
\begin{description}
\item[\code{SANE\_STATUS\_CANCELLED}:] The operation was cancelled through
  a call to \code{sane\_cancel}.
\item[\code{SANE\_STATUS\_DEVICE\_BUSY}:] The device is busy. The
  operation should be retried later.
\item[\code{SANE\_STATUS\_JAMMED}:] The document feeder is jammed.
\item[\code{SANE\_STATUS\_NO\_DOCS}:] The document feeder is out of
  documents.
\item[\code{SANE\_STATUS\_COVER\_OPEN}:] The scanner cover is open.
\item[\code{SANE\_STATUS\_IO\_ERROR}:] An error occurred while communicating
  with the device.
\item[\code{SANE\_STATUS\_NO\_MEM}:] An insufficent amount of memory
  is available.
\item[\code{SANE\_STATUS\_INVAL}:] The scan cannot be started with the current
  set of options. The frontend should reload the option descriptors, as if
  \code{\defn{SANE\_INFO\_RELOAD\_OPTIONS}} had been returned from a call to
  \code{sane\_control\_option()}, since the device's capabilities may have
  changed.
\end{description}
\end{quote}


\subsection{\code{sane\_read}}

This function is used to read image data from the device represented
by handle \code{h}.  Argument \code{buf} is a pointer to a memory area
that is at least \code{maxlen} bytes long.  The number of bytes
returned is stored in \code{*len}.  A backend must set this to zero
when a status other than \code{SANE\_STA\-TUS\_GOOD} is returned).
When the call succeeds, the number of bytes returned can be anywhere in
the range from 0 to \code{maxlen} bytes.
\begin{quote}\index{sane\_read}
\begin{verbatim}
SANE_Status sane_read (SANE_Handle h, SANE_Byte * buf,
                       SANE_Int maxlen, SANE_Int * len);
\end{verbatim}
\end{quote}
\begin{changebar}
For efficiency reasons, medium to large 
block sizes (in the range of a few kilobytes) should be used. 
Returning short reads is allowed to allow for small buffers 
in the backend. 
\end{changebar}
If this function is called when no data is available, one of two
things may happen, depending on the I/O mode that is in effect for
handle \code{h}.
\begin{enumerate}
\item If the device is in blocking I/O mode (the default mode), the
  call blocks until at least one data byte is available (or until some
  error occurs).

\item If the device is in non-blocking I/O mode, the call returns
  immediately with status \code{SANE\_STA\-TUS\_GOOD} and with
  \code{*len} set to zero.
\end{enumerate}
The I/O mode of handle \code{h} can be set via a call to
\code{sane\_set\_io\_mode()}.

This function may fail with one of the following status codes.
\begin{quote}
\begin{description}
\item[\code{SANE\_STATUS\_CANCELLED}:] The operation was cancelled through
  a call to \code{sane\_cancel}.
\item[\code{SANE\_STATUS\_EOF}:] No more data is available for the
  current frame.
\begin{changebar}
If \code{sane\_read} sends back any image data it
is not allowed to return with \code{SANE\_STATUS\_EOF}.
\end{changebar}
\item[\code{SANE\_STATUS\_JAMMED}:] The document feeder is jammed.
\item[\code{SANE\_STATUS\_NO\_DOCS}:] The document feeder is out of
  documents.
\item[\code{SANE\_STATUS\_COVER\_OPEN}:] The scanner cover is open.
\item[\code{SANE\_STATUS\_IO\_ERROR}:] An error occurred while communicating
  with the device.
\item[\code{SANE\_STATUS\_NO\_MEM}:] An insufficent amount of memory
  is available.
\item[\code{SANE\_STATUS\_ACCESS\_DENIED}:] Access to the device has
  been denied due to insufficient or invalid authentication.
\end{description}
\end{quote}


\subsection{\code{sane\_cancel}}

This function is used to immediately or as quickly as possible cancel
the currently pending operation of the device represented by handle
\code{h}.
\begin{quote}\index{sane\_cancel}
\begin{verbatim}
void sane_cancel (SANE_Handle h);
\end{verbatim}
\end{quote}
This function can be called at any time (as long as handle \code{h} is
a valid handle) but usually affects long-running operations only (such
as image is acquisition).  It is safe to call this function
asynchronously (e.g., from within a signal handler).  It is important
to note that completion of this operaton does {\em not\/} imply that
the currently pending operation has been cancelled.  It only
guarantees that cancellation has been {\em initiated}.  Cancellation
completes only when the cancelled call returns (typically with a
status value of \code{SANE\_STATUS\_CANCELLED}).  Since the SANE API
does not require any other operations to be re-entrant, this implies
that a frontend must {\em not\/} call any other operation until the
cancelled operation has returned.


\subsection{\code{sane\_set\_io\_mode}}

This function is used to set the I/O mode of handle \code{h}.  The I/O mode
can be either blocking or non-blocking.  If argument \code{m} is
\code{SANE\_TRUE}, the mode is set to non-blocking mode, otherwise it's set to
blocking mode.  This function can be called only after a call to
\code{sane\_start()} has been performed.
\begin{quote}\index{sane\_set\_io\_mode}
\begin{verbatim}
SANE_Status sane_set_io_mode (SANE_Handle h, SANE_Bool m);
\end{verbatim}
\end{quote}
By default, newly opened handles operate in blocking mode.  A backend
may elect not to support non-blocking I/O mode.  In such a case the
status value \code{SANE\_STATUS\_UNSUPPORTED} is returned.  Blocking
I/O must be supported by all backends, so calling this function with
argument \code{m} set to \code{SANE\_FALSE} is guaranteed to complete
successfully.

This function may fail with one of the following status codes:
\begin{quote}
\begin{description}
\item[\code{SANE\_STATUS\_INVAL}:] No image acquisition is pending.
\item[\code{SANE\_STATUS\_UNSUPPORTED}:] The backend does not support
the requested I/O mode.
\end{description}
\end{quote}


\subsection{\code{sane\_get\_select\_fd}}

This function is used to obtain a (platform-specific) file-descriptor
for handle \code{h} that is readable if and only if image data is
available (i.e., when a call to \code{sane\_read()} will return at
least one byte of data).  If the call completes successfully, the
select file-descriptor is returned in \code{*fd}.
\begin{quote}\index{sane\_get\_select\_fd}
\begin{verbatim}
SANE_Status sane_get_select_fd (SANE_Handle h, SANE_Int *fd);
\end{verbatim}
\end{quote}
This function can be called only after a call to \code{sane\_start()}
has been performed and the returned file-descriptor is guaranteed to
remain valid for the duration of the current image acquisition (i.e.,
until \code{sane\_cancel()} or \code{sane\_start()} get called again
or until \code{sane\_read()} returns with status
\code{SANE\_STA\-TUS\_EOF}).  Indeed, a backend must guarantee to
close the returned select file descriptor at the point when the next
\code{sane\_read()} call would return \code{SANE\_STA\-TUS\_EOF}.
This is necessary to ensure the application can detect when this
condition occurs without actually having to call \code{sane\_read()}.

A backend may elect not to support this operation.  In such a case,
the function returns with status code
\code{SANE\_STATUS\_UNSUPPORTED}.

Note that the only operation supported by the returned file-descriptor
is a host operating-system dependent test whether the file-descriptor
is readable (e.g., this test can be implemented using \code{select()}
or \code{poll()} under UNIX).  If any other operation is performed on
the file descriptor, the behavior of the backend becomes
unpredictable.  Once the file-descriptor signals ``readable'' status,
it will remain in that state until a call to \code{sane\_read()} is
performed.  Since many input devices are very slow, support for this
operation is strongly encouraged as it permits an application to do
other work while image acquisition is in progress.

This function may fail with one of the following status codes:
\begin{quote}
\begin{description}
\item[\code{SANE\_STATUS\_INVAL}:] No image acquisition is pending.
\item[\code{SANE\_STATUS\_UNSUPPORTED}:] The backend does not support
  this operation.
\end{description}
\end{quote}


\subsection{\code{sane\_strstatus}}

This function can be used to translate a SANE status code into a
printable string.  The returned string is a single line of text that
forms a complete sentence, but without the trailing period
(full-stop).  The function is guaranteed to never return \code{NULL}.
The returned pointer is valid at least until the next call to this
function is performed.
\begin{quote}\index{sane\_strstatus}
\begin{verbatim}
SANE_String_Const sane_strstatus (SANE_Status status);
\end{verbatim}
\end{quote}

\section{Code Flow}\index{code flow}

The code flow for the SANE API is illustrated in
Figure~\ref{fig:flow}.  Functions \code{sane\_init()} and
\code{sane\_exit()} initialize and exit the backend, respectively.
All other calls must be performed after initialization and before
exiting the backend.

\begin{figure}[htb]
  \begin{center}
    \leavevmode
    \psfig{file=figs/flow.eps,height=0.5\textheight}
    \caption{Code flow}
    \label{fig:flow}
  \end{center}
\end{figure}

Function \code{sane\_get\_devices()} can be called any time after
\code{sane\_init()} has been called.  It returns the list of the
devices that are known at the time of the call.  This list may change
over time since some devices may be turned on or off or a remote host
may boot or shutdown between different calls.  It should be noted that
this operation may be relatively slow since it requires contacting all
configured devices (some of which may be on remote hosts).  A frontend
may therefore want to provide the ability for a user to directly
select a desired device without requiring a call to this function.

Once a device has been chosen, it is opened using a call to
\code{sane\_open()}.  Multiple devices can be open at any given time.
A SANE backend must not impose artificial constraints on how many
devices can be open at any given time.

An opened device can be setup through the corresponding device handle
using functions \code{sane\_get\_opt\-ion\_desc\-riptor()} and
\code{sane\_control\_option()}.  While setting up a device, obtaining
option descriptors and setting and reading of option values can be
mixed freely.  It is typical for a frontend to read out all available
options at the beginning and then build a dialog (either graphical or
a command-line oriented option list) that allows to control the
available options.  It should be noted that the number of options is
fixed for a given handle. 
\begin{changebar}
  However, as options are set, other options may become active or inactive or
  their constraint may change. Thus, after setting an option, it may be
  necessary to re-read the descriptors.
\end{changebar}
While
setting up the device, it is also admissible to call
\code{sane\_get\_parameters()} to get an estimate of what the image
parameters will look like once image acquisition begins.

The device handle can be put in blocking or non-blocking mode by a
call to \code{sane\_set\_io\_mode()}.  Devices are required to support
blocking mode (which is the default mode), but support for
non-blocking I/O is strongly encouraged for operating systems such as
UNIX.

After the device is setup properly, image acquisition can be started
by a call to \code{sane\_start()}.  The backend calculates the exact
image parameters at this point.  So future calls to
\code{sane\_get\_parameters()} will return the exact values, rather
than estimates.  Whether the physical image acquisition starts at this
point or during the first call to \code{sane\_read()} is unspecified
by the SANE API.  If non-blocking I/O and/or a select-style interface
is desired, the frontend may attempt to call
\code{sane\_set\_io\_mode()} and/or \code{sane\_get\_select\_fd()} at
this point.  Either of these functions may fail if the backend does
not support the requested operation.

\begin{changebar}
Image data is collected by repeatedly calling \code{sane\_read()}
until this function will return an end-of-file status
(\code{SANE\_STATUS\_EOF}).  This indicates the end of the current
frame. If the frontend expects additional frames (e.g., the
individual channels of a red/green/blue image or multiple images),
it can call \code{sane\_start()} again.
If the \code{SANE\_PFLAG\_LAST\_FRAME} bit is set in \code{flags}, the
current image is complete. In this case, it should be tested, if
\code{flags} has the \code{SANE\_PFLAG\_MORE\_IMAGES} bit set.
If yes, further calls to \code{sane\_start()} can be made to acquire
more images. Please note, that as this bit has to be set at the beginning
of a the transmission of the last frame before the new image, it is possible,
that no reliable decision can be made at this time. It is thus permissible
for a backend to set this bit, and then later at the actual call to
\code{sane\_start()} return an error like \code{SANE\_STATUS\_NO\_DOCS}. 
Such a sequence is permitted to transmit multiple images from a single 
page as well as multiple pages. This behaviour should be controlled by 
backend options as required, to allow single-page scanning as well as 
ADF-batch-scanning. The frontend should always continue reading all images 
until a frame with \code{SANE\_PFLAG\_LAST\_FRAME} on
and \code{SANE\_PFLAG\_MORE\_IMAGES} off is encountered, or an error other
than \code{SANE\_STATUS\_EOF} occurs in a SANE function.
Note that \code{SANE\_STATUS\_NO\_DOCS} also is an allowed way for the backend
to indicate the end of a multiple image scan.

A frontend may choose to skip frames (e.g. because it cannot parse them), 
which is accomplished by simply calling \code{sane\_start} again, which will get 
you to the next frame, without having to read and discard the current one.

In order to prematurely stop scanning and to reset the backend state, 
\code{sane\_cancel()} can be called at any time. This call is required 
as well after normal termination of a multiple image scan as described above.
\end{changebar}

When done using the device, the handle should be closed by a call to
\code{sane\_close()}.  Finally, before exiting the application,
function \code{sane\_exit()} must be called.  It is important not to
forget to call this function since otherwise some resources (e.g.,
temporary files or locks) may remain unclaimed.

\begin{changebar}
The following C sample code implements a reference loop for acquiring 
multiple images:

\begin{verbatim}
SANE_Parameters parms;
SANE_Status     status;

do
  {
    do
      {
        /* Now start acquiring the next frame. */
        status = sane_start (handle);

        /* if that failed, we have a problem, and no more frames can be 
         * read at this time. Due to SANE_PFLAG_MORE_IMAGES still 
         * being clear, this will break out of _BOTH_ loops. 
         */
        if (status != SANE_STATUS_GOOD)
          break;

        /* Now let us see what the next frame brings. */
        status = sane_get_parameters (handle, &parms);

        /* This actually should not fail, but maybe the doc feeder 
         * jammed or something, so we break as well, if something 
         * is wrong. 
         */
        if (status != SANE_STATUS_GOOD)
          break;

        /* Now we check the announced parameters, if we can make use 
         * of the frame data. If not, we skip over to the next frame. 
         */
        if (do_i_like_that (&parms) == NO)
          continue;

        /* Set up for reading the data here. Mangle filenames, 
         * allocate memory, rewind multiframe files, ask user 
         * for confirmation, ... 
         */
        setup_for_transfer (...);

        /* Now we read in the frame data and process it. This should 
         * return SANE_STATUS_GOOD, until the frame is complete, 
         * what causes SANE_STATUS_EOF to be returned. 
         */
        while (SANE_STATUS_GOOD == (status = sane_read (...)))
          read_in_and_process_data_as_required ();

        /* If transfer was broken due to anything but EOF, break out. */
        if (status != SANE_STATUS_EOF)
          break;

        /* Now loop until we have all frames of an image. */
      }
    while (!(parms.flag & SANE_PFLAG_LAST_FRAME));

    /* O.K. - we now have a complete image. Fit it together, save it, 
     * flush buffers, transmit it, increment filenames, etc. 
     */

    /* Now check for more pending images. If we have more, redo from start. 
     * Some backends might cheat here and send us for an extra round which 
     * will fail at sane_start, as they were not able to determine if they 
     * would have more data at the start of the last frame we read. 
     */
  }
while (parms.flags & SANE_PFLAG_MORE_IMAGES);

/* No more data. Fine. Reset the backend and go back to option-control 
 * loop. 
 */

sane_cancel (handle);
\end{verbatim}

\end{changebar}


\section{Well-Known Options}\index{well-known options}

While most backend options are completely self-describing, there are
cases where a user interface might want to special-case the handling
of certain options.  For example, the scan area is typically defined
by four options that specify the top-left and bottom-right corners of
the area.  With a graphical user interface, it would be tedious to
force the user to type in these four numbers.  Instead, most such
interfaces will want to present to the user a preview (low-resolution
scan of the full scanner surface or a high(er) resolution scan of a subpart
of the scanner surface) and let the user pick the scan area by
dragging a rectangle into the desired position.  For this reason, the
SANE API specifies a small number of option names that have
well-defined meanings.

\subsection{Option Number Count}\index{option count}

Option number 0 has an empty string as its name.  The value of this
option is of type \code{SANE\_TYPE\_INT} and it specifies the total
number of options available for a given device (the count includes
option number 0).  This means that there are two ways of counting the
number of options available: a frontend can either cycle through all
option numbers starting at one until
\code{sane\_get\_option\_descriptor()} returns \code{NULL}, or a
frontend can directly read out the value of option number 0.

\subsection{Scan Resolution Options}\index{scan resolution}\index{resolution option}

Option \code{resolution} is used to select the resolution at which an
image should be acquired.
\begin{changebar}
When the backend wants to allow different
values for x- and y-resolution it has to define the options
\code{x-resolution} and \code{y-resolution}. Note that only
the option \code{resolution} {\bf or} the options
\code{x-resolution} {\bf and} \code{y-resolution} may be active at the
same time.
\end{changebar}

The type of this option is either \code{SANE\_TYPE\_INT} or
\code{SANE\_TYPE\_FIXED}.  The unit is \code{SANE\_UNIT\_DPI} (dots/inch).

These options are not mandatory, but if a backend does support them, it
must implement them in a manner consistent with the above definition.

\subsection{Preview Mode Option}\index{preview mode}

The boolean option \code{preview} is used by a frontend to inform the
backend when image acquisition should be optimized for speed, rather
than quality (``preview mode'').  When set to \code{SANE\_TRUE},
preview mode is in effect, when set to \code{SANE\_FALSE} image
acquisition should proceed in normal quality mode.  The setting of
this option \emph{must not\/} affect any other option.  That is, as
far as the other options are concerned, the preview mode is completely
side effect free.  A backend can assume that the frontend will take
care of appropriately setting the scan resolution for preview mode
(through option \code{resolution}).  A backend is free to override the
\code{resolution} value with its own choice for preview mode, but it
is advised to leave this choice to the frontend wherever possible.
\begin{changebar}
When the \code{preview} option is set the backend should transfer
the image in frame type \code{SANE\_FRAME\_RAW} if possible.
\end{changebar}

This option is not mandatory, but if a backend does support it, it
must implement it in a manner consistent with the above definition.

\subsection{Scan Area Options\label{subsec:scanarea}}\index{scan area options}

The four most important well-known options are the ones that define
the scan area.  The scan area is defined by two points (x/y coordinate
pairs) that specify the top-left and the bottom-right corners.  This
is illustrated in Figure~\ref{fig:area}.  Note that the origin of the
coordinate system is at the top-left corner of the scan surface as
seen by the sensor (which typically is a mirror image of the scan
surface seen by the user).  For this reason, the top-left corner is
the corner for which the abscissa and ordinate values are
simultaneously the {\em smallest} and the bottom-right corner is the
corner for which the abscissa and ordinate values are simulatenously
the {\em largest}.  If this coordinate system is not natural for a
given device, it is the job of the backend to perform the necessary
conversions.
\begin{figure}[tbp]
  \begin{center}
    \leavevmode
    \psfig{file=figs/area.eps,height=0.3\textheight}
    \caption{Scan area options}
    \label{fig:area}
  \end{center}
\end{figure}

The names of the four options that define the scan area are given in
the table below:
\begin{center}
\begin{tabular}{ll}
{\bf Name} & {\bf Description} \\
\code{\defn{tl-x}} & Top-left $x$ coordinate value \\
\code{\defn{tl-y}} & Top-left $y$ coordinate value \\
\code{\defn{br-x}} & Bottom-right $x$ coordinate value \\
\code{\defn{br-y}} & Bottom-right $y$ coordinate value \\
\end{tabular}
\end{center}
There are several rules that should be followed by front and backends
regarding these options:
\begin{itemize}

\item Backends must attach a unit of either pixels
  (\code{SANE\_UNIT\_PIXEL}) or millimeters (\code{SANE\_UNIT\_MM}) to
  these options.  The unit of all four options must be identical.

\item Whenever meaningful, a backend should attach a range or a
  word-list constraint to these options.

\item A frontend can determine the size of the scan surface by first
  checking that the options have range constraints associated.  If a
  range or word-list constraints exist, the frontend can take the
  minimum and maximum values of one of the x and y option
  range-constraints to determine the scan surface size.

\item A frontend must work properly with any or all of these options
  missing.
  
\begin{changebar}
\item A frontend may temporarily set the values in a way that
\code{tl-x} is larger than \code{br-x} and \code{tl-y} is larger than
\code{br-y}.
\end{changebar}

\end{itemize}

\begin{changebar}
These options are not mandatory, but if a backend does support them, it
must implement them in a manner consistent with the above definition.
\end{changebar}

\begin{changebar}
\subsection{Depth Option}\index{bit depth option}
Option \code{depth} is used to select the image depth in bits/sample
in multi bit mode (e.g. for 24 bit RGB mode this value must be 8).
The type of this option is \code{SANE\_TYPE\_INT}.
The unit is \code{SANE\_UNIT\_BIT}. For 1 bit modes
(Lineart or Halftone) this option has to be inactive.
For selection of 1 bit modes (Lineart or Halftone) the
backend should use the well-known option \code{mode}.
 
This option is not mandatory, but if a backend does support it, it
must implement it in a manner consistent with the above definition.
\end{changebar}
 
\begin{changebar}
\subsection{Scan Mode Options}\index{mode options}
The option \code{mode} defines a \code{SANE\_CONSTRAINT\_STRING\_LIST}
of type \code{SANE\_TYPE\_STRING}.
It is used to select the scanmode (e.g. Color or Gray).
Well known modes are: \code{Color}, \code{Gray}, \code{Halftone}
and \code{Lineart}. \code{Color} and \code{Gray} are multi bit
modes (8 or 16 bits/sample), \code{Halftone} and \code{Lineart}
are single bit modes. When well-known scan modes are used,
a frontend is able to automatically decide which mode is appropriate
for a specific task, e.g the mode \code{Gray}
for scanning a fax.
 
This option is not mandatory, but if a backend does support it, it
must implement it in a manner consistent with the above definition.
\end{changebar}
 
\begin{changebar}
\subsection{Scan Source Options}\index{source options}
The option \code{source} is used to select the scan source
(e.g. Automatic Document Feeder).
It defines a \code{SANE\_CONSTRAINT\_STRING\_LIST}
of type \code{SANE\_TYPE\_STRING}.
Well known sources are: \code{Flatbed}, \code{Transparancy Adapter} and
\code{Automatic Document Feeder}.
 
This option is not mandatory, but if a backend does support it, it
must implement it in a manner consistent with the above definition.
\end{changebar}

\begin{changebar}
\subsection{Threshold Option}\index{threshold option}
The option \code{threshold} is used to define the threshold
for Lineart and maybe Halftone mode. In multi bit modes
this option should be set inactive.
The type of this option is \code{SANE\_TYPE\_FIXED}.
The unit is \code{SANE\_UNIT\_PERCENT}. The value range
should be 0.0\ldots100.0 if possible.
It defines the minimum intensity to get a white point / full intensity.

The backend has to scale the values in the following way:
A value of 0.0 means all pixels get white / full intensity.  A value of 50.0
means intensities brighter than medium gray get white / full intensity.  A
value of 100.0 means all pixels get black. If the
scanner is not able to cover the full range the backend has to define a
reduced value range (e.g. 30\ldots70 percent).

This option is not mandatory, but if a backend does support it, it
must implement it in a manner consistent with the above definition.
\end{changebar}

\begin{changebar}
\subsection{Gamma Table Options}\index{gamma table options}
The \code{gamma-table} option defines a \code{SANE\_CONSTRAINT\_RANGE}
of the type \code{SANE\_TYPE\_INT} which represents the gamma correction
table for gray.  In color mode the \code{gamma-table} may be used to set
a common gamma correction for red, green and blue. The options
\code{red-gamma-table}, \code{green-gamma-table} and \code{blue-gamma-table}
are used in color mode to set a gamma correction for each color
separately. In color mode the backend is free to use only the
\code{gamma-table} option, only the \code{red-}, \code{green-} and
\code{blue-gamma-table} or all four options. When all four options
are used then the color tables should do a gamma correction with
the same input and output bit depth and the gray gamma table should
reduce (if necessary) the bit depth from the scanner internal
bit depth to the output bit depth. This should e.g. look like this:
\begin{verbatim}
red_value   = gamma-table(red-gamma-table(value))
green_value = gamma-table(green-gamma-table(value))
blue_value  = gamma-table(blue-gamma-table(value))
\end{verbatim}

The backend should not use the gamma tables to emulate other functions or options
like highlight, shadow, contrast, brightness, threshold, analog\_gamma.
These functions are common for all backends and should be added to the frontend
or a meta-backend.

It is also discouraged to emulate gamma tables in the backend. The backend
should disable (or not define) this option when the scanner does not support
gamma tables in hardware.
 
This option is not mandatory, but if a backend does support it, it
must implement it in a manner consistent with the above definition.
\end{changebar}

\begin{changebar}
\subsection{Analog Gamma}\index{analog gamma option}
The option \code{analog-gamma} is used to define the gamma value 
for an analog gamma function of the scanner in multi bit modes.
In 1 bit modes this option should be set inactive.
The type of this option is \code{SANE\_TYPE\_FIXED}.
The unit is \code{SANE\_UNIT\_NONE}. The value range
can be defined by the backend as supported. The values
have to be positive. A gamma value of 1.0 means that
the gamma correction has no effect. A value larger than
1.0 increases the brightness of the image.
In color mode there also can be options \code{analog-gamma-red},
\code{analog-gamma-green} and \code{analog-gamma-blue}.
It is not allowed to emulate an analog gamma function by
a digital gamma table. The backend has to disable (or not
define) this option when the scanner does not support an
analog (hardware) gamma function.

When analog gamma, highlight and shadow functions are available
at the same time then the backend author has to care about the order
in which the functions are implemented in the scanner hardware.
The SANE standard expects that changing the analog gamma value
has no effect on the shadow and highlight function. When the
analog gamma function is executed in the scanner hardware
before the shadow and highlight functions then the backend
should do a compensation. For this the shadow and highlight
values have to be gamma corrected with the relevant analog gamma value.

This option is not mandatory, but if a backend does support it, it
must implement it in a manner consistent with the above definition.
\end{changebar}


\begin{changebar}
\subsection{Shadow Option}\index{shadow options}
The option \code{shadow} is used to define the shadow level / black point level.
The type of this option is \code{SANE\_TYPE\_FIXED}.
The unit is \code{SANE\_UNIT\_PERCENT}. The value range
should be 0.0\ldots100.0 if possible.
It is used to define the maximum intensity level that creates an image data value
of 0 (black).

The backend has to scale the values in the following way:
A value of 0.0 means that the sensitivity range is not reduced, only the
minimum intensity produces an image data value of 0 (black).
A value of 50.0 means that that a medium intensity and everything that is darker
produces an image data value of 0 (black).
A value of 100.0 means the sensitivity range is reduced to 0, all image
data values are 0 (black).  If the scanner is not able to
cover the full range the backend has to define a reduced
value range (e.g. 30\ldots70 percent).
In color mode there can be options \code{shadow-red}, \code{shadow-green}
and \code{shadow-blue}, in this case the \code{shadow} function has to be disabled.
It is not allowed to emulate a shadow function by
a digital gamma table. The backend has to disable (or not
define) this option when the scanner does not support an
analog (hardware) shadow function.

This option is not mandatory, but if a backend does support it, it
must implement it in a manner consistent with the above definition.
\end{changebar}


\begin{changebar}
\subsection{Highlight Option}\index{hightlight options}
The option \code{highlight} is used to define the highlight level / white point level.
The type of this option is \code{SANE\_TYPE\_FIXED}.
The unit is \code{SANE\_UNIT\_PERCENT}. The value range
should be 0.0\ldots100.0 if possible.
It is used to define the minimum intensity level that creates the maximum possible
image data value (white/full intensity). 

The backend has to scale the values in the following way:
A value of 0.0 means the sensitivity range is reduced to 0, all image
data have maximum value (white / full intensity).
A value of 50.0 means that a medium intensity and everything that is brighter
produces the maximum possible image data value (white / full intensity).
A value of 100.0 means that the sensitivity range is not reduced, only the
maximum intensity produces an image data with maximum possible value (white / full intensity).
If  the scanner is not able to cover the full range the backend has to define a reduced
value range (e.g. 30\ldots70 percent).
In color mode there can be options \code{highlight-red}, \code{highlight-green}
and \code{highlight-blue}, in this case \code{highlight} has to be disabled.
It is not allowed to emulate a highlight function by
a digital gamma table. The backend has to disable (or not
define) this option when the scanner does not support an
analog (hardware) highlight function.

This option is not mandatory, but if a backend does support it, it
must implement it in a manner consistent with the above definition.
\end{changebar}


\begin{changebar}
\subsection{Lamp Options}\index{lamp-on option}\index{lamp-off option}

The options \code{lamp-on} and \code{lamp-off} are button options
(\code{SANE\_TYPE\_BUTTON}) and don't have a unit (\code{SANE\_UNIT\_NONE}).

Option \code{lamp-on} is used to turn on the lamp of the scanner while
\code{lamp-off} turns it off.

These options are not mandatory, but if a backend does support them, it
must implement them in a manner consistent with the above definition.
\end{changebar}


\begin{changebar}
\subsection{Scanner Button Options}\index{scanner button options}
Some scanners have buttons which state can be read by the scanner driver. 
It is necessary to implement a locking function for the buttons
because it is possible that several frontends try to connect to the
same backend/scanner at the same time. Imagine what could happen
when no locking would be implemented:
Five people have started a scanning application which is connected
via network to the scanner you want to use. You start a frontend,
put a paper to the scanner and press the scan-button on the scanner.
The scanner does scan three times (because three frontends asked the
button status when you pressed the button). For three people the
image is saved to the harddisk, but it is not sure that your
frontend did scan the image.

A backend that does make available the scanner-buttons has to
implement the following options:\\
\code{scanner-buttons-lock} is of type \code{SANE\_TYPE\_BOOL}, default = \code{SANE\_FALSE}\\
\code{scanner-buttons-status} is of type \code{SANE\_TYPE\_INT}, default = 0\\
\code{scanner-buttons-status-update} is of type \code{SANE\_TYPE\_BUTTON}\\
When setting these options the backend does not set \code{SANE\_INFO\_RELOAD\_OPTIONS}
or \code{SANE\_INFO\_RELOAD\_PARAMS} if not explictly defined.

A frontend has to disable the usage of the scanner-buttons by default. This is important
because other frontends will not be able to use the buttons when the button-functions are locked.
Another important thing is that some scanners do not turn off their lamp when the driver
does frequently talk to the scanner (what is done when reading the button status from the scanner).

\begin{itemize}

\item
A frontend that wants to read the button status has to lock the
button functions at first. For this it does set the option
\code{scanner-buttons-lock} to \code{SANE\_TRUE}.
While setting the value of option \code{scanner-buttons-lock} to \code{SANE\_TRUE}
the backend does check if a lockfile (e.g. ``backend''-buttons.lock) does exist.
The lockfile has to be placed in a directory where every user has read and write access to.

  \begin{itemize}
  \item
  If the lockfile does not exist then the backend creates the lockfile and writes the
  process ID (PID) of the backend to the file. Button access is allowed: the value
  of option \code{scanner-buttons-lock} is set to \code{SANE\_TRUE}
  \item
  If the file does exist and the backend PID is not the file PID then the
  backend has to check if the process with the PID stored in the lockfile
  still is running. If yes then the button access is not allowed: the value of
  option \code{scanner-buttons-lock} is set to \code{SANE\_FALSE}. If not then the lockfile
  is recreated and the PID of the backend is stored in the lockfile, button
  access is allowed: the value of option \code{scanner-buttons-lock} is set to \code{SANE\_TRUE}
  \end{itemize}

\item
The frontend does read the value of option \code{scanner-buttons-lock}. If
it is \code{SANE\_TRUE} then the frontend has access to the scanner buttons.
If it is \code{SANE\_FALSE} then access has been denied.

\item
If the button access is allowed the frontend has to do the following about
once per second (while not scanning):
  \begin{itemize}
  \item
  The frontend does set option \code{scanner-buttons-status-update}.
  The backend checks if access to the buttons is allowed by comparing
  the backend PID with the lockfile PID. If access is allowed it
  does read the button status from the scanner and stores it in
  the option \code{scanner-buttons-status}, each bit represents a button, a
  value of 0 means the button is not pressed, a value of 1 means that the button
  is pressed. When the scanner is busy the backend must not wait, it has to
  return immediately and the button state keeps unchanged.
  The backend has to implement a timeout function. When no button has been pressed
  within a predefined time (e.g. 15 minutes) then the access permission is lost.
  In this case the backend does set option \code{scanner-buttons-lock} to \code{SANE\_FALSE}
  and does set \code{SANE\_INFO\_RELOAD\_OPTIONS} to inform the frontend that it has
  lost permission to access the scanner-button functions.
  If access is not allowed it does set the \code{scanner-buttons-status} to 0.

  \item
  The frontend does read the value of option \code{scanner-buttons-status}
  \end{itemize}

\item
When the frontend does exit or it does not want to use the buttons
it does set option \code{scanner-buttons-lock} to \code{SANE\_FALSE}.
The backend does check if the backend PID and the lockfile PID are
the same. If this is true then it removes the lockfile and sets the value 
of \code{scanner-buttons-lock} to \code{SANE\_FALSE}.

\code{sane\_close()} should do the same as setting option
\code{scanner-buttons-lock} to \code{SANE\_FALSE}.

\end{itemize}
\end{changebar}

\begin{changebar}
\subsection{Batch Scan Options}\index{Batch Scan Options}

The batch scan options can be used by a frontend to indicate that more than
one image will be scanned in a batch. The backend can optimize for such scans
by e.g. avoiding calibration and not moving home the sensor array in this case.
The backend provides the options \code{batch-scan},
\code{batch-scan-next-tl-x}, \code{batch-scan-next-tl-y},
\code{batch-scan-next-br-x}, and \code{batch-scan-next-br-y}. The option
\code{batch-scan} provides a string list with the values of \code{No} if
batch scanning isn't used, \code{Start} for the first, \code{End} for the last
and \code{Loop} for any other (intermediate) image.  The batch-scan-next
options specify the coordinates of the next scan and follow the same rules as
the scan area options (see section \ref{subsec:scanarea}).

These options are not mandatory, but if a backend does support them, it must
implement them in a manner consistent with the above definition.  In this
case, all options must be implemented.
\end{changebar}




\input{net.tex}

\chapter{Contact Information}\label{chap:contact}

The SANE standard is discussed and evolved via a mailing list.
Anybody with email access to the Internet can automatically join and
leave the discussion group by sending mail to the following address.
\begin{quote}\index{mailing list}
\begin{verbatim}
sane-devel-request@mostang.com
\end{verbatim}
\end{quote}
To subscribe, send a mail with the body ``\verb|subscribe sane-devel|'' to the
above address.

A complete list of commands supported can be obtained by sending a
mail with a subject of ``\code{help}'' to the above address.  The
mailing list is archived and available through the SANE home page at
URL:
\begin{quote}
\url{http://www.mostang.com/sane/}
\end{quote}

\newpage
\documentclass[11pt,DVIps]{report}

\usepackage{times,epsfig,changebar,html}

\setlength{\parindent}{0pt}
\setlength{\parskip}{1.5ex plus 0.5ex minus 0.5ex}
\setlength{\textwidth}{6.5in}
\setlength{\textheight}{8.5in}
\setlength{\marginparwidth}{0pt}
\setlength{\oddsidemargin}{0pt}
\setlength{\evensidemargin}{0pt}
\setlength{\marginparsep}{0pt}
\addtolength{\topmargin}{-0.75in}

\title{\huge SANE Standard Version 2.0 proposal 0.08 - rauch/beck}
\author{}
\date{Dec 8, 2002} 

\makeindex

\begin{document}

\newcommand{\filename}[1]{{\tt #1}}
\newcommand{\code}[1]{{\tt #1}}
\newcommand{\var}[1]{{\it #1}}
\newcommand{\defn}[1]{#1\index{#1}}

\begin{latexonly}
  \setcounter{changebargrey}{0}   % black change bars
\end{latexonly}

\maketitle
\tableofcontents
\listoffigures
\listoftables


\chapter{Preface}

The SANE standard is being developed by a group of free-software
developers.  The process is open to the public and comments as well as
suggestions for improvements are welcome.  Information on how to join
the SANE development process can be found in Chapter
\ref{chap:contact}.

The SANE standard is intended to streamline software development by
providing a standard application programming interface to access
raster scanner hardware.  This should reduce the number of different
driver implementations, thereby reducing the need for reimplementing
similar code.


\section{About This Document}

This document is intended for developers who are creating either an
application that requires access to raster scanner hardware and for
developers who are implementing a SANE driver.  It does not cover
specific implementations of SANE components.  Its sole purpose is to
describe and define the SANE application interface that will enable
any application on any platform to interoperate with any SANE backend
for that platform.

The remainder of this document is organized as follows.
Chapter~\ref{chap:intro} provides introductional material.
Chapter~\ref{chap:environ} presents the environment SANE is designed
for.  Chapter~\ref{chap:api} details the SANE Application Programmer
Interface.  Chapter~\ref{chap:net} specifies the network protocol that
can be used to implement the SANE API in a network transparent
fashion.  Finally, Chapter~\ref{chap:contact} gives information on how
to join the SANE development process.

\subsection{Typographic Conventions}

Changes since the last revision of this document are highlighted
like this:

\begin{changebar}
   Paragraphs that changed since the last revision of the documention
   are marked like this paragraph.
\end{changebar}

\chapter{Introduction}\label{chap:intro}

SANE is an application programming interface (API) that provides
standardized access to any raster image scanner hardware.  The
standardized interface allows to write just one driver for each
scanner device instead of one driver for each scanner and application.
The reduction in the number of required drivers provides significant
savings in development time.  More importantly, SANE raises the level
at which applications can work.  As such, it will enable applications
that were previously unheard of in the UNIX world.  While SANE is
primarily targeted at a UNIX environment, the standard has been
carefully designed to make it possible to implement the API on
virtually any hardware or operating system.

SANE is an acronym for ``Scanner Access Now Easy.''  Also, the hope is
that SANE is sane in the sense that it will allow easy implementation
of the API while accommodating all features required by today's
scanner hardware and applications.  Specifically, SANE should be broad
enough to accommodate devices such as scanners, digital still and
video cameras, as well as virtual devices like image file filters.

\section{Terminology}

An application that uses the SANE interface is called a SANE {\em
  frontend}.  A driver that implements the SANE interface is called a
SANE {\em backend}.  A {\em meta backend\/} provides some means to
manage one or more other backends.


\chapter{The SANE Environment}\label{chap:environ}

SANE is defined as a C-callable library interface.  Accessing a raster
scanner device typically consists of two phases: first, various
controls of the scanner need to be setup or queried.  In the second
phase, one or more images are acquired.

Since the device controls are widely different from device to device,
SANE provides a generic interface that makes it easy for a frontend to
give a user access to all controls without having to understand each
and every device control.  The design principle used here is to
abstract each device control into a SANE {\em option\/}.  An option is
a self-describing name/value pair.  For example, the brightness
control of a camera might be represented by an option called
\code{brightness} whose value is an integer in the range from 0 to
255.

With self-describing options, a backend need not be concerned with
{\em presentation\/} issues: the backend simply provides a list of
options that describe all the controls available in the device.
Similarly, there are benefits to the frontend: it need not be
concerned with the {\em meaning\/} of each option.  It simply provides
means to present and alter the options defined by the backend.


\section{Attaching to a SANE backend}

The process through which a SANE frontend connects to a backend is
platform dependent.  Several possibilities exist:
\begin{itemize}

\item {\bf Static linking:} A SANE backend may be linked directly into
  a frontend.  While the simplest method of attaching to a backend, it
  is somewhat limited in functionality since the available devices is
  limited to the ones for which support has been linked in when the
  frontend was built.  But even so static linking can be quite useful,
  particularly when combined with a backend that can access scanners
  via a network.  Also, it is possible to support multiple backends
  simultaneously by implementing a meta backend that manages several
  backends that have been compiled in such a manner that they export
  unique function names.  For example, a backend called \code{be}
  would normally export a function called \code{sane\_read()}.  If
  each backend would provide such a function, static linking would
  fail due to multiple conflicting definitions of the same symbol.
  This can be resolved by having backend \code{be} include a
  header file that has lines of the form:
  \begin{quote}
\begin{verbatim}
#define sane_read be_sane_read
\end{verbatim}
  \end{quote}
  With definitions of this kind, backend \code{be} will export
  function name \code{be\_sane\_read()}.  Thus, all backends will
  export unique names.  As long as a meta backend knows about these
  names, it is possible to combine several backends at link time and
  select and use them dynamically at runtime.

\item {\bf Dynamic linking:} A simpler yet more powerful way to
  support multiple backends is to exploit dynamic linking on platforms
  that support it.  In this case, a frontend is linked against a
  shared library that implements any SANE backend.  Since each
  dynamically linked backend exports the same set of global symbols
  (all starting with the prefix \code{sane\_}), the dynamic library
  that gets loaded at runtime does not necessarily have to be the same
  one as one the frontend got linked against.  In other words, it is
  possible to switch the backend by installing the appropriate backend
  dynamic library.

  More importantly, dynamic linking makes it easy to implement a meta
  backend that loads other backends {\em on demand}.  This is a
  powerful mechanism since it allows adding new backends merely by
  installing a shared library and updating a configuration file.

\item {\bf Network connection:} Arguably the ultimate way to attach to
  a scanner is by using the network to connect to a backend on a
  remote machine.  This makes it possible to scan images from any host
  in the universe, as long as there is a network connection to that
  host and provided the user is permitted to access that scanner.

\end{itemize}

\begin{figure}[htbp]
  \begin{center}
    \leavevmode
    \psfig{file=figs/hierarchy.eps,angle=270,width=\textwidth}
    \caption{Example SANE Hiearchy}
    \label{fig:hierarchy}
  \end{center}
\end{figure}

The above discussion lists just a few ways for frontends to attach to
a backend.  It is of course possible to combine these solutions to
provide an entire hierarchy of SANE backends.  Such a hierarchy is
depicted in Figure~\ref{fig:hierarchy}.  The figure shows that machine
A uses a dynamic-linking based meta backend called \code{dll} to
access the backends called \code{pnm}, \code{mustek}, and \code{net}.
The first two are real backends, whereas the last one is a meta
backend that provides network transparent access to remote scanners.
In the figure, machine B provides non-local access to its scanners
through the SANE frontend called \code{saned}.  The \code{saned} in
turn has access to the \code{hp} and \code{autolum} backends through
another instance of the \code{dll} backend.  The \code{autolum} meta
backend is used to automatically adjust the luminance (brightness) of
the image data acquired by the camera backend called \code{qcam}.

Note that a meta backend really is both a frontend and a backend at
the same time.  It is a frontend from the viewpoint of the backends
that it manages and a backend from the viewpoint of the frontends that
access it.  The name ``meta backend'' was chosen primarily because the
SANE standard describes the interface from the viewpoint of a (real)
frontend.


\section{Image Data Format}\label{sec:imageformat}\index{image data format}

Arguably the most important aspect of an image acquisition system is how
images are represented. The SANE approach is to define a simple yet powerful
representation that is sufficient for vast majority of applications and
devices. While the representation is simple, the interface has been defined
carefully to allow extending it in the future without breaking backwards
compatibility. Thus, it will be possible to accommodate future applications or
devices that were not anticipated at the time this standard was created.

A SANE image is a rectangular area.  The rectangular area is subdivided into a
number of rows and columns.  At the intersection of each row and column is a
(preferable quadratic) pixel. A pixel consists of one or more sample values.
Each sample value represents one channel (e.g., the red channel).

The SANE API transmits an image as a sequence of frames.  Each frame covers
the same rectangular area as the entire image, but may contain only a subset
of the channels in the final image.  For example, a red/green/blue image could
either be transmitted as a single frame that contains the sample values for
all three channels or it could be transmitted as a sequence of three frames:
the first frame containing the red channel, the second the green channel, and
the third the blue channel.

When transmitting an image frame by frame, the frontend needs to know what
part of the image a frame represents (and how many frames it should expect).
For that purpose, the SANE API tags every frame with a type and a format
descriptor.

\begin{changebar}
  There are two different types of frames: pixel oriented frames
  \code{\defn{SANE\_FRAME\_RAW}} and arbitrary data frames
  \code{\defn{SANE\_FRAME\_MIME}}. These types are discussed in detail in the
  following sections. The frame types used by the previous version~1 of this
  standard (\code{\defn{SANE\_FRAME\_GRAY}}, \code{\defn{SANE\_FRAME\_RGB}},
  \code{\defn{SANE\_FRAME\_RED}}, \code{\defn{SANE\_FRAME\_GREEN}}, and
  \code{\defn{SANE\_FRAME\_BLUE}}) are obsolete and superseded by
  \code{\defn{SANE\_FRAME\_RAW}}.

  \subsection{Pixel oriented frames}

  The type of pixel oriented frames is \code{\defn{SANE\_FRAME\_RAW}}. The
  frame contains one or more channels of data in a channel-interleaved format,
  that represents sample values from a property of the individual pixels that
  is subject to further description in the \code{format\_desc} member of the
  \code{SANE\_Parameters} structured type. See section~\ref{sec:sanegetparameters}
  on page~\pageref{sec:sanegetparameters} for details about the format
  descriptions.
\end{changebar}

  Each sample value has a certain bit depth.  The bit depth is fixed for the
  entire image and can be as small as one bit. Valid bit depths are 1, 8, or
  16 bits per sample. If a device's natural bit depth is something else, it is
  up to the driver to scale the sample values appropriately (e.g., a 4 bit
  sample could be scaled by a factor of four to represent a sample value of
  depth 8).

\begin{changebar}
  The complete image may consist of several different channels. The number of channels
  is defined by member \code{channels\_per\_image} of \code{SANE\_Parameters}.
  The image may be transmitted in an arbitrary number of frames which can be
  determined by watching the \code{SANE\_PFLAG\_LAST\_FRAME} flag in said type (or by
  counting the channels).  Note: This frame type replaces all frame types of
  the SANE standard version 1.
\end{changebar}

Conceptually, each pixel oriented frame is transmitted a byte at a time.  Each
byte may contain 8 sample values (for an image bit depth of 1), one full
sample value (for an image bit depth of 8), or a partial sample value (for an
image bit depth of 16 or bigger).  In the latter case, the bytes of each
sample value are transmitted in the machine's native byte order.
\begin{quote}
  \begin{center}
    {\bf Backend Implementation Note}
  \end{center}
  A network-based meta backend will have to ensure that the byte order
  in image data is adjusted appropriately if necessary.  For example,
  when the meta backend attaches to the server proxy, the proxy may
  inform the backend of the server's byte order.  The backend can then
  apply the adjustment if necessary.  In essence, this implements a
  ``receiver-makes-right'' approach.
\end{quote}

\begin{figure}[htbp]
  \begin{center}
    \leavevmode
    \psfig{file=figs/xfer.eps,width=0.5\textwidth}
    \caption{Transfer order of image data bytes}
    \label{fig:xfer}
  \end{center}
\end{figure}

The order in which the sample values in a frame are transmitted is illustrated
in Figure~\ref{fig:xfer}.  As can be seen, the values are transmitted row by
row and each row is transmitted from left-most to right-most column.  The
left-to-right, top-to-bottom transmission order applies when the image is
viewed in its normal orientation (as it would be displayed on a screen, for
example).

If a frame contains multiple channels, then the channels are transmitted in an
interleaved fashion.  Figure~\ref{fig:pixels} illustrates this for the case
where a frame contains a complete red/green/blue image with a bit-depth of 8.

\begin{figure}[htbp]
  \begin{center}
    \leavevmode
    \psfig{file=figs/image-data.eps,width=0.8\textwidth}
    \caption{Bit and byte order of image data}
    \label{fig:pixels}
  \end{center}
\end{figure}

For a bit depth of 1, each byte contains 8 sample values of a {\em single\/}
channel. In other words, a bit depth 1 frame is transmitted in a byte
interleaved fashion. The first sample of each byte is represented by the most
significant bit.

\begin{changebar}
For gray channels at a bit depth of 1 only two sample values are possible: 1
represents minimum intensity (black) and 0 represents maximum intensity
(white).  For all other channel types and bit depths a sample value of 0
represents minimum intensity and larger values represent increasing intensity.

\subsection{Arbitrary data frames}

It also is possible to transmit arbitrary (not necessaryly pixel oriented)
data. This allows transmission of compressed images like jpeg, tiff, etc.

The type of arbitrary data frames is \code{\defn{SANE\_FRAME\_MIME}}.
The frame contains arbitrary data of the MIME (see RFC 1521/1522) type that is
given in the \code{format\_desc} member of the \code{SANE\_Parameters}
structured type (see section~\ref{sec:sanegetparameters} on
page~\pageref{sec:sanegetparameters}).  As such, it is assumed to be
incomprehensible to the frontend, except for selected types the frontend is
specifically capable of handling internally.  The frontend is free to ignore
those frames, or employ any appropriate means to otherwise handle this data
(like saving them to disk or spawning an external viewer).
\end{changebar}

\chapter{The SANE Application Programmer Interface (API)}\label{chap:api}

\begin{changebar}
This Section defines version 2 of the SANE application
programmer interface (API).  Any SANE frontend must depend on the
interface defined in this section only. Converseley, any SANE backend
must implement its functionality in accordance with this
specification. The interface as documented here is declared as a C
callable interface in a file called \filename{sane/sane-2.h}. This file should
normally be included via a C pre-processor directive of the form:
\begin{verbatim}
  #include <sane/sane-2.h>
\end{verbatim}
\end{changebar}


\section{Version Control}

The SANE standard is expected to evolve over time.  Whenever a change
to the SANE standard is made that may render an existing frontend or
backend incompatible with the new standard, the major version number
must be increased.  Thus, any frontend/backend pair is compatible
provided the major version number of the SANE standard they implement
is the same.  A frontend may implement backwards compatiblity by
allowing major numbers that are smaller than the expected major number
(provided the frontend really can cope with the older version).  In
contrast, a backend always provides support for one and only one
version of the standard.  If a specific application does require that
two different versions of the same backend are accessible at the same
time, it is possible to do so by installing the two versions under
different names.

SANE version control also includes a minor version number and a build
revision.  While control of these numbers remains with the implementor
of a backend, the recommended use is as follows.  The minor version is
incremented with each official release of a backend.  The build
revision is increased with each build of a backend.

The SANE API provides the following five macros to manage version
numbers.
\begin{quote}
  \begin{description}
  \item[\code{\defn{SANE\_CURRENT\_MAJOR}}:] The value of this macro is the
    number of the SANE standard that the interface implements.

  \item[\code{\defn{SANE\_VERSION\_CODE}(\var{maj},\var{min},\var{bld})}:]
    \label{sec:saneversioncode}
    This macro can be used to build a monotonically increasing version
    code.  A SANE version code consists of the SANE standard major
    version number (\var{maj}), the minor version number \var{min},
    and the build revision of a backend (\var{bld}).  The major and
    minor version numbers must be in the range 0\ldots255 and the
    build revision must be in the range 0\ldots65535.

    Version codes are monotonic in the sense that it is possible to
    apply relational operators (e.g., equality or less-than test)
    directly on the version code rather than individually on the three
    components of the version code.

    Note that the major version number alone determines whether a
    frontend/backend pair is compatible.  The minor version and the
    build revision are used for informational and bug-fixing purposes
    only.
  
  \item[\code{\defn{SANE\_VERSION\_MAJOR}(\var{vc})}:] This macro returns the
    major version number component of the version code passed in
    argument \var{vc}.
  \item[\code{SANE\_VERSION\_MINOR(\var{vc})}:] This macro returns the
    minor version number component of the version code passed in
    argument \var{vc}.
  \item[\code{SANE\_VERSION\_BUILD(\var{vc})}:] This macro returns the
    build revision component of the version code passed in argument
    \var{vc}.
  \end{description}
\end{quote}


\section{Data Types}

\subsection{Base Types}

The SANE standard is based on just two SANE-specific base types: the
SANE byte and word.
\begin{quote}
  \code{typedef \var{some-scalar-type\/} \defn{SANE\_Byte};} \\
  \code{typedef \var{some-scalar-type\/} \defn{SANE\_Word};}
\end{quote}
\verb|SANE_Byte| must correspond to some scalar C type that is capable
of holding values in the range 0 to 255.  \verb|SANE_Word| must be
capable of holding any of the following:
\begin{itemize}
  \item the truth values \verb|SANE_FALSE| and \verb|SANE_TRUE|
  \item signed integers in the range $-2^{31}\ldots2^{31}-1$
  \item fixed point values in the range $-32768\ldots32767.9999$ with
    a resolution of $1/65536$
  \item 32 bits (for bit sets)
\end{itemize}
Note that the SANE standard does not define what C type
\verb|SANE_Byte| and \verb|SANE_Word| map to.  For example, on some
platforms, the latter may map to \verb|long int| whereas on others it
may map to \verb|int|.  A portable SANE frontend or backend must
therefore not depend on a particular mapping.

\subsection{Boolean Type}

\code{\defn{SANE\_Bool}} is used for variables that can take one of
the two truth values \code{\defn{SANE\_FALSE}} and
\code{\defn{SANE\_TRUE}}.  The former value is defined to be 0,
whereas the latter is 1.\footnote{This is different from ANSI C where
  any non-zero integer value represents logical TRUE.} The C
declarations for this type are given below.
\begin{quote}
\begin{verbatim}
#define SANE_FALSE      0
#define SANE_TRUE       1
typedef SANE_Word SANE_Bool;
\end{verbatim}
\end{quote}
Note that \verb|SANE_Bool| is simply an alias of \verb|SANE_Word|.  It
is therefore always legal to use the latter type in place of the
former.  However, for clarity, it is recommended to use
\verb|SANE_Bool| whenever a given variable or formal argument has a
fixed interpretation as a boolean object.

\subsection{Integer Type}

\code{\defn{SANE\_Int}} is used for variables that can take integer
values in the range $-2^{32}$ to $2^{31}-1$.  Its C declaration is
given below.
\begin{quote}
\begin{verbatim}
typedef SANE_Word SANE_Int;
\end{verbatim}
\end{quote}
Note that \verb|SANE_Int| is simply an alias of \verb|SANE_Word|.  It
is therefore always legal to use the latter type in place of the
former.  However, for clarity, it is recommended to use
\verb|SANE_Int| whenever a given variable or formal argument has a
fixed interpretation as an integer object.


\subsection{Fixed-point Type}

\code{\defn{SANE\_Fixed}} is used for variables that can take fixed
point values in the range $-32768$ to $32767.9999$ with a resolution
of $1/65535$.  The C declarations relating to this type are given
below.
\begin{quote}
\begin{verbatim}
#define SANE_FIXED_SCALE_SHIFT  16
typedef SANE_Word SANE_Fixed;
\end{verbatim}
\end{quote}
The macro \code{\defn{SANE\_FIXED\_SCALE\_SHIFT}} gives the location
of the fixed binary point.  This standard defines that value to be 16,
which yields a resolution of $1/65536$.

Note that \verb|SANE_Fixed| is simply an alias of \verb|SANE_Word|.
It is therefore always legal to use the latter type in place of the
former.  However, for clarity, it is recommended to use
\verb|SANE_Fixed| whenever a given variable or formal argument has a
fixed interpretation as a fixed-point object.

For convenience, SANE also defines two macros that convert fixed-point
values to and from C double floating point values.
\begin{quote}
  \begin{description}

  \item[\code{\defn{SANE\_FIX}(\var{d})}:] Returns the largest SANE
    fixed-point value that is smaller than the double value \var{d}.
    No range checking is performed.  If the value of \var{d} is out of
    range, the result is undefined.

  \item[\code{\defn{SANE\_UNFIX}(\var{w})}:] Returns the nearest
    double machine number that corresponds to fixed-point value
    \var{w}.

  \end{description}
\end{quote}
SANE does {\em not\/} require that the following two expressions hold
true (even if the values of \var{w} and \var{d} are in range):
\begin{quote}
\begin{verbatim}
SANE_UNFIX(SANE_FIX(d)) == d
SANE_FIX(SANE_UNFIX(w)) == w
\end{verbatim}
\end{quote}
In other words, conversion between fixed and double values may be
lossy.  It is therefore recommended to avoid repeated conversions
between the two representations.


\subsection{Text}

\subsubsection{Character Type}

Type \code{\defn{SANE\_Char}} represents a single text character or
symbol.  At present, this type maps directly to the underlying C
\verb|char| type (typically one byte).  The encoding for such
characters is currently fixed as ISO LATIN-1.  Future versions of this
standard may map this type to a wider type and allow multi-byte
encodings to support internationalization.  As a result of this, care
should be taken to avoid the assumption that
\verb|sizeof(SANE_Char)==sizeof(char)|.
\begin{quote}
\begin{verbatim}
typedef char SANE_Char;
\end{verbatim}
\end{quote}

\subsubsection{String Type}

Type \code{\defn{SANE\_String}} represents a text string as a sequence
of C \verb|char| values.  The end of the sequence is indicated by a
\verb|'\0'| (\defn{NUL}) character.  
\begin{quote}
\begin{verbatim}
typedef SANE_Char *SANE_String;
typedef const SANE_Char *SANE_String_Const;
\end{verbatim}
\end{quote}
The type \code{\defn{SANE\_String\_Const}} is provided by SANE to
enable declaring strings whose contents is unchangable.  Note that in
ANSI C, the declaration
\begin{quote}
\begin{verbatim}
const SANE_String str;
\end{verbatim}
\end{quote}
declares a string pointer that is constant (not a string pointer that
points to a constant value).


\subsection{Scanner Handle Type}

Access to a scanner is provided through an opaque type called
\code{\defn{SANE\_Handle}}.  The C declaration of this type is given
below.
\begin{quote}
\begin{verbatim}
typedef void *SANE_Handle;
\end{verbatim}
\end{quote}
While this type is declared to be a void pointer, an application must
not attempt to interpret the value of a \verb|SANE_Handle|.  In
particular, SANE does not require that a value of this type is a legal
pointer value.


\subsection{Status Type}

Most SANE operations return a value of type \code{\defn{SANE\_Status}}
to indicate whether the completion status of the operation.  If an
operation completes successfully, \verb|SANE_STATUS_GOOD| is returned.
In case of an error, a value is returned that indicates the nature of
the problem.  The complete list of available status codes is listed in
Table \ref{tab:status}.  It is recommended to use function
\code{sane\_strstatus()} to convert status codes into a legible
string.

\begin{table}[htbp]
  \begin{center}
    \begin{tabular}{|l|r|l|}
      \hline
      \multicolumn{1}{|c|}{\bf Symbol} & \multicolumn{1}{c|}{\bf Code} &
      \multicolumn{1}{c|}{\bf Description} \\
      \hline\hline
\code{\defn{SANE\_STATUS\_GOOD}}
        &  0 & Operation completed succesfully. \\
\code{\defn{SANE\_STATUS\_UNSUPPORTED}}
        &  1 & Operation is not supported. \\
\code{\defn{SANE\_STATUS\_CANCELLED}}
        &  2 & Operation was cancelled. \\
\code{\defn{SANE\_STATUS\_DEVICE\_BUSY}}
        &  3 & Device is busy---retry later. \\
\code{\defn{SANE\_STATUS\_INVAL}}
        &  4 & Data or argument is invalid. \\
\code{\defn{SANE\_STATUS\_EOF}}
        &  5 & No more data available (end-of-file). \\
\code{\defn{SANE\_STATUS\_JAMMED}}
        &  6 & Document feeder jammed. \\
\code{\defn{SANE\_STATUS\_NO\_DOCS}}
        &  7 & Document feeder out of documents. \\
\code{\defn{SANE\_STATUS\_COVER\_OPEN}}
        &  8 & Scanner cover is open. \\
\code{\defn{SANE\_STATUS\_IO\_ERROR}}
        &  9 & Error during device I/O. \\
\code{\defn{SANE\_STATUS\_NO\_MEM}}
        & 10 & Out of memory. \\
\code{\defn{SANE\_STATUS\_ACCESS\_DENIED}}
        & 11 & Access to resource has been denied. \\
      \hline
    \end{tabular}
    \caption{Status Codes}\label{tab:status}
  \end{center}
\end{table}


\subsection{Device Descriptor Type}
\begin{changebar}

Each SANE device is represented by a structure of type
\code{\defn{SANE\_Device}}.  The C declaration of this type is given
below.
\begin{quote}
\begin{verbatim}
typedef struct
  {
    SANE_String_Const name;
    SANE_String_Const vendor;
    SANE_String_Const model;
    SANE_String_Const type;
    SANE_String_Const email_backend_author;
    SANE_String_Const backend_website;
    SANE_String_Const device_location;
    SANE_String_Const comment;
    SANE_String_Const reserved_string;
    SANE_Int backend_version_code;
    SANE_Int backend_capablity_flags;
    SANE_Int reserved_int;
  }
SANE_Device;
\end{verbatim}
\end{quote}
\end{changebar}
\index{device-name}
The structure provides the unique name of the scanner in member
\code{name}.  It is this unique name that should be passed in a call
to \code{sane\_open()}.  The format of this name is completely up to
the backend.  The only constraints are that the name is unique among
all devices supported by the backend and that the name is a legal SANE
text string.  To simplify presentation of unique names, their length
should not be excessive.  It is {\em recommended\/} that backends keep
unique names below 32 characters in length.  However, applications
{\em must\/} be able to cope with arbitrary length unique names.

The next three members in the device structure provide additional
information on the device corresponding to the unique name.
Specifically, members \code{vendor}, \code{model}, and \code{type} are
single-line strings that give information on the vendor
(manufacturer), model, and the type of the device.  For consistency's
sake, the following strings should be used when appropriate (the lists
will be expanded as need arises):

\begin{table}[htbp]
  \begin{center}
    \leavevmode
    \hspace{\fill}
    \begin{tabular}[t]{|ll|}
      \hline
      \multicolumn{2}{|c|}{\bf \defn{Vendor Strings}} \\
      \hline\hline
      \code{AGFA}            & \code{Microtek} \\
      \code{Abaton}          & \code{Minolta} \\
      \code{Acer}            & \code{Mustek} \\
      \code{Apple}           & \code{NEC} \\
      \code{Artec}           & \code{Nikon} \\
      \code{Avision}         & \code{Plustek} \\
      \code{CANON}           & \code{Polaroid} \\
      \code{Connectix}       & \code{Ricoh} \\
      \code{Epson}           & \code{Sharp} \\
      \code{Fujitsu}         & \code{Siemens} \\
      \code{Hewlett-Packard} & \code{Tamarack} \\
      \code{IBM}             & \code{UMAX} \\
      \code{Kodak}           & \code{Noname} \\
      \code{Logitech}        & \\ 
      \hline
    \end{tabular}
    \hspace{\fill}
    \begin{tabular}[t]{|l|}
      \hline
      \multicolumn{1}{|c|}{\bf \defn{Type Strings}} \\
      \hline\hline
      \code{film scanner} \\
      \code{flatbed scanner} \\
      \code{frame grabber} \\
      \code{handheld scanner} \\
      \code{multi-function peripheral} \\
      \code{sheetfed scanner} \\
      \code{still camera} \\
      \code{video camera} \\
      \code{virtual device} \\
      \hline
    \end{tabular}
    \hspace{\fill}
    \caption{Predefined Device Information Strings}
    \label{tab:devinfo}
  \end{center}
\end{table}
Note that vendor string \code{Noname} can be used for virtual devices
that have no physical vendor associated.  Also, there are no
predefined model name strings since those are vendor specific and
therefore completely under control of the respective backends.

\begin{changebar}
The backend has to set up the string \code{email\_backend\_author}
with the name and the email address of the backend author or a contact person
in the format:
\begin{verbatim}
Firstname Lastname <name@domain.org>
\end{verbatim}

The string \code{backend\_website} should be set up by the backend
with the website or ftp address of the backend in the format:
\begin{verbatim}
http://www.domain.org/sane-hello/index.html
\end{verbatim}

The backend should fill the string \code{device\_location} with a
text that describes where a user can find this device. The text 
should be configurable by the administrator. This could e.g. look
like this:
\begin{verbatim}
building 93, 2nd plane, room 2124
\end{verbatim}

The string \code{comment} can be used to display any comment about the device
to the user. The text should be configurable by the administrator.

The string \code{reserved\_string} is unused currently but planned for
future use.

All unused strings must be set to ``'' by the backend.

With member \code{backend\_version\_code} a frontend can find
out the version of a backend it is connected to via one or more meta
backends.

The member \code{backend\_capability\_flags} contains 32 bits that
are planned to give the backend the chance to inform the frontend about
its capabilities. The meaning of the flags will be defined when there
is the need for it. The backend has to set all not defined bits
(in the moment all 32 bits) to 0.

The member \code{reserved\_int} is planned for future use, the backend has
to set the value of the integer to 0.
\end{changebar}

\subsection{Option Descriptor Type}\label{sec:odesc}

Option descriptors are at the same time the most intricate and
powerful type in the SANE standard.  Options are used to control
virtually all aspects of device operation.  Much of the power of the
SANE API stems from the fact that most device controls are completely
described by their respective option descriptor.  Thus, a frontend can
control a scanner abstractly, without requiring knowledge as to what
the purpose of any given option is.  Conversely, a scanner can
describe its controls without requiring knowledge of how the frontend
operates.  The C declaration of the
\code{\defn{SANE\_Option\_Descriptor}} type is given below.
\begin{quote}
\begin{verbatim}
typedef struct
  {
    SANE_String_Const name;
    SANE_String_Const title;
    SANE_String_Const desc;
    SANE_Value_Type type;
    SANE_Unit unit;
    SANE_Int size;
    SANE_Int cap;
    SANE_Constraint_Type constraint_type;
    union
      {
        const SANE_String_Const *string_list;
        const SANE_Word *word_list;
        const SANE_Range *range;
      }
    constraint;
  }
SANE_Option_Descriptor;
\end{verbatim}
\end{quote}

\subsubsection{Option Name}

Member \code{name} is a string that uniquely identifies the option.
The name must be unique for a given device (i.e., the option names
across different backends or devices need not be unique).  The option
name must consist of lower-case ASCII letters (\code{a}--\code{z}),
digits (\code{0}--\code{9}), or the dash character (\code{-}) only.
The first character must be a lower-case ASCII character (i.e., not a
digit or a dash).

\subsubsection{Option Title}

Member \code{title} is a single-line string that can be used by the
frontend as a title string.  This should typically be a short (one or
two-word) string that is chosen based on the function of the option.

\subsubsection{Option Description}

Member \code{desc} is a (potentially very) long string that can be
used as a help text to describe the option.  It is the responsibility
of the frontend to break the string into managable-length lines.
Newline characters in this string should be interpreted as paragraph
breaks.

\subsubsection{Option Value Type}

Member \code{type} specifies the type of the option value.  The
possible values for type \code{\defn{SANE\_Value\_Type}} are described
in Table \ref{tab:valuetype}.

\begin{table}[htbp]
  \begin{center}
    \leavevmode
    \begin{tabular}{|l|l|p{0.6\textwidth}|}
\hline
\multicolumn{1}{|c|}{\bf Symbol} &
\multicolumn{1}{c|}{\bf Code} &
\multicolumn{1}{c|}{\bf Description} \\
\hline\hline

\code{\defn{SANE\_TYPE\_BOOL}} & 0 & Option value is of type
  \verb|SANE_Bool|. \\ 

\code{\defn{SANE\_TYPE\_INT}} & 1 & Option value is of type
  \verb|SANE_Int|. \\ 

\code{\defn{SANE\_TYPE\_FIXED}}&2 & Option value is of type
  \verb|SANE_Fixed|. \\ 

\code{\defn{SANE\_TYPE\_STRING}}&3 & Option value is of type
  \verb|SANE_String|. \\ 

\code{\defn{SANE\_TYPE\_BUTTON}} & 4 & An option of this type has no value.
Instead, setting an option of this type has an option-specific
side-effect.  For example, a button-typed option could be used by a
backend to provide a means to select default values or to the tell an
automatic document feeder to advance to the next sheet of paper. \\ 

\code{\defn{SANE\_TYPE\_GROUP}} & 5 & An option of this type has no value.
This type is used to group logically related options.  A group option
is in effect up to the point where another group option is encountered
(or up to the end of the option list, if there are no other group
options).  For group options, only members \code{title} and
\code{type} are valid in the option descriptor. \\ 

  \hline
    \end{tabular}
    \caption{Option Value Types (\code{SANE\_Value\_Type})}
    \label{tab:valuetype}
  \end{center}
\end{table}

\subsubsection{Option Value Unit}

Member \code{unit} specifies what the physical unit of the option
value is.  The possible values for type \code{\defn{SANE\_U\-nit}} are
described in Table \ref{tab:units}.  Note that the specified unit is
what the SANE backend expects.  It is entirely up to a frontend as to
how these units a presented to the user.  For example, SANE expresses
all lengths in millimeters.  A frontend is generally expected to
provide appropriate conversion routines so that a user can express
quantities in a customary unit (e.g., inches or centimeters).

\begin{table}[htbp]
  \begin{center}
    \leavevmode
    \begin{tabular}{|l|l|l|}
\hline
\multicolumn{1}{|c|}{\bf Symbol} &
\multicolumn{1}{|c|}{\bf Code} &
\multicolumn{1}{|c|}{\bf Description} \\

\hline\hline

\code{\defn{SANE\_UNIT\_NONE}}   & 0 & Value is unit-less (e.g., page count).\\
\code{\defn{SANE\_UNIT\_PIXEL}}  & 1 & Value is in number of pixels. \\
\code{\defn{SANE\_UNIT\_BIT}}    & 2 & Value is in number of bits. \\
\code{\defn{SANE\_UNIT\_MM}}     & 3 & Value is in millimeters. \\
\code{\defn{SANE\_UNIT\_DPI}}    & 4 & Value is a resolution in dots/inch. \\
\code{\defn{SANE\_UNIT\_PERCENT}}& 5 & Value is a percentage. \\
\code{\defn{SANE\_UNIT\_MICROSECOND}}& 6 & Value is time in $\mu$-seconds. \\

\hline
    \end{tabular}
    \caption{Physical Units (\code{SANE\_Unit})}
    \label{tab:units}
  \end{center}
\end{table}

\subsubsection{Option Value Size}\label{sec:valuesize}

Member \code{size} specifies the size of the option value (in bytes).
This member has a slightly different interpretation depending on the
type of the option value:
\begin{quote}
  \begin{description}
  \item[\code{SANE\_TYPE\_STRING}:] The size is the maximum size of
    the string.  For the purpose of string size calcuations, the
    terminating \code{NUL} character is considered to be part of the
    string.  Note that the terminating \code{NUL} character must
    always be present in string option values.
  \item[\code{SANE\_TYPE\_INT}, \code{SANE\_TYPE\_FIXED}:] The size
    must be a positive integer multiple of the size of a
    \verb|SANE_Word|.  The option value is a vector of length
    \[ \code{size}/\code{sizeof(SANE\_Word)}. \]
  \item[\code{SANE\_TYPE\_BOOL}:] The size must be set to
    \code{sizeof(SANE\_Word)}.
  \item[\code{SANE\_TYPE\_BUTTON}, \code{SANE\_TYPE\_GROUP}:] The
    option size is ignored.
  \end{description}
\end{quote}

\subsubsection{Option Capabilities}

Member \code{cap} describes what capabilities the option posseses.
This is a bitset that is formed as the inclusive logical OR of the
capabilities described in Table \ref{tab:capabilities}.  The SANE API
provides the following to macros to test certain features of a given
capability bitset:
\begin{quote}
  \begin{description}

  \item[\code{\defn{SANE\_OPTION\_IS\_ACTIVE}(\var{cap})}:] This macro
    returns \code{SANE\_TRUE} if and only if the option with the
    capability set \var{cap} is currently active.

  \item[\code{\defn{SANE\_OPTION\_IS\_SETTABLE}(\var{cap})}:] This
    macro returns \code{SANE\_TRUE} if and only if the option with the
    capability set \var{cap} is software settable.
  \end{description}
\end{quote}

\begin{table}[htbp]
  \begin{center}
    \leavevmode
    \begin{tabular}{|l|r|p{0.59\textwidth}|}
\hline
\multicolumn{1}{|c|}{\bf Symbol} &
\multicolumn{1}{c|}{\bf Code} &
\multicolumn{1}{c|}{\bf Description} \\
\hline\hline

\code{\defn{SANE\_CAP\_SOFT\_SELECT}} & 1 & The option
  value can be set by a call to \code{sane\_con\-trol\_opt\-ion()}.\\ 

\code{\defn{SANE\_CAP\_HARD\_SELECT}} & 2 & The option value can be set by
  user-intervention (e.g., by flipping a switch).  The user-interface
  should prompt the user to execute the appropriate action to set such
  an option.  This capability is mutually exclusive with
  SANE\_CAP\_SOFT\_SELECT (either one of them can be set, but not both
  simultaneously).  \\ 

\code{\defn{SANE\_CAP\_SOFT\_DETECT}} & 4 & The option
  value can be detected by software.  If
  \code{SANE\_\-CAP\_\-SO\-FT\_SEL\-ECT} is set, this capability {\em must\/}
  be set.  If \code{SANE\_CAP\_HARD\_SELECT} is set, this capability
  may or may not be set.  If this capability is set but neither
  \code{SANE\_CAP\_SO\-FT\_SEL\-ECT} nor \code{SANE\_CAP\_HA\-RD\_SEL\-ECT}
  are, then there is no way to control the option.  That is, the
  option provides read-out of the current value only.  \\ 

\code{\defn{SANE\_CAP\_EMULATED}} & 8 & If set, this capability indicates
  that an option is not directly supported by the device and is
  instead emulated in the backend.  A sophisticated frontend may
  elect to use its own (presumably better) emulation in lieu of an emulated
  option. \\ 

\code{\defn{SANE\_CAP\_AUTOMATIC}} & 16 & If set, this capability indicates
  that the backend (or the device) is capable to picking a reasonable
  option value automatically.  For such options, it is possible to
  select automatic operation by calling \code{sane\_control\_option()}
  with an action value of \code{SANE\_ACTION\_SET\_AUTO}.  \\ 

\code{\defn{SANE\_CAP\_INACTIVE}} & 32 & If set, this capability indicates
  that the option is not currently active (e.g., because it's
  meaningful only if another option is set to some other value). \\ 

\code{\defn{SANE\_CAP\_ADVANCED}} & 64 &
  If set, this capability indicates that the option should be
  considered an ``advanced user option''. 
\begin{changebar}
  If this capability is set for an option of type
  \code{SANE\_TYPE\_GROUP}, all options belonging to the group are also
  advanced, even if they don't set the capabilty themselves. 
\end{changebar}
  A frontend typically
  displays such options in a less conspicuous way than regular options
  (e.g., a command line interface may list such options last or a
  graphical interface may make them available in a seperate ``advanced
  settings'' dialog). 
  \\

\code{\defn{SANE\_CAP\_HIDDEN}} & 128 &
\begin{changebar}
  If set, this capability indicates that the option shouldn't be
  displayed to and used by the user directly. Instead a hidden option
  is supposed to be automatically used by the frontend, like e.g. the
  ``preview'' option.
  If this capability is set for an option of type
  \code{SANE\_TYPE\_GROUP}, all options belonging to the group are also
  hidden, even if they don't set the capabilty themselves. 
  A frontend typically doesn't display such options by default but there
  should be a way to override this default behaviour.
\end{changebar}
  \\

\code{\defn{SANE\_CAP\_ALWAYS\_SETTABLE}} & 256 & 
\begin{changebar}
  If set, this capability indicates that the option may be at any time between
  \code{sane\_open()} and \code{sane\_close()}. I.e.  it's allowed to set it
  even while an image is acquired.
\end{changebar}
\\

\hline
    \end{tabular}
    \caption{Option Capabilities}
    \label{tab:capabilities}
  \end{center}
\end{table}

\subsubsection{Option Value Constraints}

It is often useful to constrain the values that an option can take.
For example, constraints can be used by a frontend to determine how to
represent a given option.  Member \code{constraint\_type} indicates
what constraint is in effect for the option.  The constrained values
that are allowed for the option are described by one of the union
members of member \code{constraint}.  The possible values of type
\code{\defn{SANE\_Constraint\_Type}} and the interpretation of the
\code{constraint} union is described in Table~\ref{tab:constraints}.

\begin{table}[htbp]
  \begin{center}
    \leavevmode
    \begin{tabular}{|l|r|p{0.5\textwidth}|}
\hline
\multicolumn{1}{|c|}{\bf Symbol} &
\multicolumn{1}{|c|}{\bf Code} &
\multicolumn{1}{|c|}{\bf Description} \\

\hline\hline

\code{\defn{SANE\_CONSTRAINT\_NONE}} & 0 & The value is unconstrained.
  The option can take any of the values possible for the option's
  type. \\

  \code{\defn{SANE\_CONSTRAINT\_RANGE}} & 1 & This constraint is
  applicable to integer and fixed-point valued options only.  It
  constrains the option value to a possibly quantized range of
  numbers.  Option descriptor member \code{constraint.range} points to
  a range of the type \code{\defn{SANE\_Range}}.  This type is illustrated
  below:
  \begin{quote}
\begin{verbatim}
typedef struct
  {
    SANE_Word min;
    SANE_Word max;
    SANE_Word quant;
  }
SANE_Range;
\end{verbatim}
  \end{quote}
  All three members in this structure are interpreted according to the
  option value type (\verb|SANE_TYPE_INT| or \verb|SANE_TYPE_FIXED|).
  Members \code{min} and \code{max} specify the minimum and maximum
  values, respectively.  If member \code{quant} is non-zero, it
  specifies the quantization value.  If $l$ is the minimum value, $u$
  the maximum value and $q$ the (non-zero) quantization of a range,
  then the legal values are $v=k\cdot q+l$ for all non-negative
  integer values of $k$ such that $v<=u$.  \\ 

\code{\defn{SANE\_CONSTRAINT\_WORD\_LIST}} & 2 & This constraint is applicable
  to integer and fixed-point valued options only.  It constrains the
  option value to a list of numeric values.  Option descriptor member
  \code{constraint.word\_list} points to a list of words that
  enumerates the legal values.  The first element in that list is an
  integer (\verb|SANE_Int|) that specifies the length of the list (not
  counting the length itself).  The remaining elements in the list are
  interpreted according to the type of the option value
  (\verb|SANE_TYPE_INT| or \verb|SANE_TYPE_FIXED|). \\ 

\code{\defn{SANE\_CONSTRAINT\_STRING\_LIST}} & 3 & This constraint is
  applicable to string-valued options only.  It constrains the option
  value to a list of strings.  The option descriptor member
  \code{con\-strai\-nt.str\-ing\_list} points to a \code{NULL} terminated
  list of strings that enumerate the legal values for the option
  value.
\\\hline
    \end{tabular}
    \caption{Option Value Constraints}
    \label{tab:constraints}
  \end{center}
\end{table}


\subsection{Internationalization}\label{sec:i18n}
\begin{changebar}
All backend texts should be written in english.
Localization (translation of backend texts) has to be done
in the frontend. To automatically prepare translation tables
(e.g. english to german) it is necessary to mark the texts
that shall be translated.

\subsubsection{How is a text marked for translation}

The keyword \code{SANE\_I18N} is used to mark a text for translation.
\code{SANE\_I18N} has to be defined as a dummy prototype:

  \begin{quote}
\begin{verbatim}
#ifndef SANE_I18N
#  define SANE_I18N(text) text
#endif
\end{verbatim}
  \end{quote}

You should not mark prototypes or variables with \code{SANE\_I18N}
because it is not possible (or very hard) to find out the
texts that are covered by prototypes (\verb|SANE_I18N(START_SCAN_TEXT)|) and
variables (\verb|SANE_I18N(option[7].name)|).

A correct marked text can look like this:\\
\verb|snprintf(buf, sizeof(buf), SANE_I18N("Start scan")|\\
or\\
\verb|#define START_SCAN_TEXT SANE_I18N("Start scan")|.\\
It also is allowed to mark texts in structs because the
prototype \code{SANE\_I18N} has no effect to the compiler.

\subsubsection{Which texts shall be marked for translation?}
All option texts that are visible for the user should be marked for
translation. This is:
\begin{itemize}
\item
member \code{title}
\item
member \code{desc}
\item
member \code{string\_list}
\end{itemize}
of \code{SANE\_Option\_Descriptor}.

It is not allowed to mark/translate member \code{name}.
Please also do not mark any error or debug messages that are
displayed by the backend.

\subsubsection{File formats and translation functions}
The recommended file formats for translation tables are the \code{po} files
and \code{mo} or \code{gmo} files.
The po file contains the original text marked with the keyword \code{msgid}
and the translated text marked with the keyword \code{msgstr}.
A \code{po} file that contains all needed \code{msgid}s can be created
e.g. by the gnu gettext tool \verb|xgettext| with the option
\verb|-k SANE_I18N|.
The translator adds the translated texts to the \code{po} files.
The gettext tool \code{msgfmt} converts the \code{po} files to the
\code{mo} or \code{gmo} files.\\

Translation is done in the frontend. A backend has nothing
to do with the translation of texts.
The frontend should use the function \code{gettext}
to translate the texts. This e.g. can look like this:\\
\verb|snprintf(buf, sizeof(buf), gettext("english text"));|

If a frontend author decides to use translation functions that
need different translation tables, then the frontend is
responsible to create/convert the translation tables.
In this case it should use the \code{po} files to create its
own translation tables from it.\\

Note that it is strongly recommended to use the gettext tools.
\end{changebar}


\section{Operations}

\subsection{\code{sane\_init}}

This function must be called before any other SANE function can be called.
The behavior of a SANE backend is undefined if this function is not called
first or if the status code returned by \code{sane\_init} is different from
\code{\defn{SANE\_STATUS\_GOOD}}.  The version code of the backend is returned
in the value pointed to by \code{version\_code}.  If that pointer is
\code{NULL}, no version code is returned.  Argument \code{authorize} is either
a pointer to a function that is invoked when the backend requires
authentication for a specific resource or \code{NULL} if the frontend does not
support authentication.
\begin{quote}\index{sane\_init}
\begin{verbatim}
SANE_Status sane_init (SANE_Int * version_code,
                       SANE_Authorization_Callback authorize);
\end{verbatim}
\end{quote}

The authorization function may be called by a backend in response to
any of the following calls:
\begin{quote}
  \code{sane\_open}, \code{sane\_control\_option}, \code{sane\_start}
\end{quote}
If a backend was initialized without authorization function, then
authorization requests that cannot be handled by the backend itself
will fail automatically and the user may be prevented from accessing
protected resources.  Backends are encouraged to implement means of
authentication that do not require user assistance.  E.g., on a
multi-user system that authenticates users through a login process a
backend could automatically lookup the apporpriate password based on
resource- and user-name.

The authentication function type has the following declaration:
\begin{quote}\index{SANE\_Authorization\_Callback}
  \index{domain}\index{username}\index{password}
\begin{verbatim}
#define SANE_MAX_USERNAME_LEN   128
#define SANE_MAX_PASSWORD_LEN   128

typedef void (*SANE_Authorization_Callback)
    (SANE_String_Const resource,
     SANE_Char username[SANE_MAX_USERNAME_LEN],
     SANE_Char password[SANE_MAX_PASSWORD_LEN]);
\end{verbatim}
\end{quote}
Three arguments are passed to the authorization function:
\code{resource} is a string specifying the name of the resource that
requires authorization.  A frontend should use this string to build a
user-prompt requesting a username and a password.  The \code{username}
and \code{password} arguments are (pointers to) an array of
\code{SANE\_MAX\_USERNAME\_LEN} and \code{SANE\_MAX\_PASSWORD\_LEN}
characters, respectively.  The authorization call should place the
entered username and password in these arrays.  The returned strings
{\em must\/} be ASCII-NUL terminated.

\subsection{\code{sane\_exit}}

This function must be called to terminate use of a backend.  The
function will first close all device handles that still might be open
(it is recommended to close device handles explicitly through a call
to \code{sane\_clo\-se()}, but backends are required to release all
resources upon a call to this function).  After this function returns,
no function other than \code{sane\_init()} may be called (regardless
of the status value returned by \code{sane\_exit()}.  Neglecting to
call this function may result in some resources not being released
properly.
\begin{quote}\index{sane\_exit}
\begin{verbatim}
void sane_exit (void);
\end{verbatim}
\end{quote}


\subsection{\code{sane\_get\_devices}}

This function can be used to query the list of devices that are
available.  If the function executes successfully, it stores a pointer
to a \code{NULL} terminated array of pointers to \verb|SANE_Device|
structures in \code{*device\_list}.  The returned list is guaranteed
to remain unchanged and valid until (a) another call to this function
is performed or (b) a call to \code{sane\_exit()} is performed.  This
function can be called repeatedly to detect when new devices become
available.  If argument \code{local\_only} is true, only local devices
are returned (devices directly attached to the machine that SANE is
running on).  If it is false, the device list includes all remote
devices that are accessible to the SANE library.
\begin{quote}\index{sane\_get\_devices}
\begin{verbatim}
SANE_Status sane_get_devices (const SANE_Device *** device_list,
                              SANE_Bool local_only);
\end{verbatim}
\end{quote}

This function may fail with \code{SANE\_STATUS\_NO\_MEM} if an
insufficient amount of memory is available.

\begin{quote}
  \begin{center}
    {\bf Backend Implementation Note}
  \end{center}
  SANE does not require that this function is called before a
  \code{sane\_open()} call is performed.  A device name may be
  specified explicitly by a user which would make it unnecessary and
  undesirable to call this function first.
\begin{changebar}
  The same information about
  a device has to be returned when \code{sane\_open} is called.
\end{changebar}
\end{quote}


\subsection{\code{sane\_open}}

This function is used to establish a connection to a particular
device.  The name of the device to be opened is passed in argument
\code{name}.  If the call completes successfully, a handle for the
device is returned in \code{*h}.
\begin{changebar}
The description of the device
is returned in \code{**device\_description}. This is the
same description that is returned in the list by \code{sane\_get\_devices}.
The returned data \code{*h} and \code{*device\_description} is guaranteed
to remain unchanged and valid until a call to \code{sane\_close()}
is performed.
\end{changebar}
As a special case, specifying a zero-length string as the device requests
opening the first available device (if there is such a device).
\begin{quote}\index{sane\_open}
\begin{verbatim}
SANE_Status sane_open (SANE_String_Const name, SANE_Handle * h,
const SANE_Device ** device_description);
\end{verbatim}
\end{quote}

This function may fail with one of the following status codes.
\begin{quote}
\begin{description}
\item[\code{SANE\_STATUS\_DEVICE\_BUSY}:] The device is currently
  busy (in use by somebody else).
\item[\code{SANE\_STATUS\_INVAL}:] The device name is not valid.
\item[\code{SANE\_STATUS\_IO\_ERROR}:] An error occured while
  communicating with the device.
\item[\code{SANE\_STATUS\_NO\_MEM}:] An insufficent amount of memory
  is available.
\item[\code{SANE\_STATUS\_ACCESS\_DENIED}:] Access to the device has
  been denied due to insufficient or invalid authentication.
\end{description}
\end{quote}


\subsection{\code{sane\_close}}

This function terminates the association between the device handle
passed in argument \code{h} and the device it represents.  If the
device is presently active, a call to \code{sane\_cancel()} is
performed first.  After this function returns, handle \code{h} must
not be used anymore.

\begin{quote}\index{sane\_close}
\begin{verbatim}
void sane_close (SANE_Handle h);
\end{verbatim}
\end{quote}

\subsection{\code{sane\_get\_option\_descriptor}}

This function is used to access option descriptors.  The function
returns the option descriptor for option number \code{n} of the device
represented by handle \code{h}.  Option number 0 is guaranteed to be a
valid option.  Its value is an integer that specifies the number of
options that are available for device handle \code{h} (the count
includes option 0).  If $n$ is not a valid option index, the function
returns \code{NULL}.  The returned option descriptor is guaranteed to
remain valid (and at the returned address) until the device is closed.

\begin{quote}\index{sane\_get\_option\_descriptor}
\begin{verbatim}
const SANE_Option_Descriptor *
    sane_get_option_descriptor (SANE_Handle h, SANE_Int n);
\end{verbatim}
\end{quote}

\subsection{\code{sane\_control\_option}}\label{sec:control}

This function is used to set or inquire the current value of option
number \code{n} of the device represented by handle \code{h}.  The
manner in which the option is controlled is specified by parameter
\code{a}.  The possible values of this parameter are described in more
detail below.  The value of the option is passed through argument
\code{v}.  It is a pointer to the memory that holds the option value.
The memory area pointed to by \code{v} must be big enough to hold the
entire option value (determined by member \code{size} in the
corresponding option descriptor).  The only exception to this rule is
that when setting the value of a string option, the string pointed to
by argument \code{v} may be shorter since the backend will stop
reading the option value upon encountering the first \code{NUL}
terminator in the string.  If argument \code{i} is not \code{NULL},
the value of \code{*i} will be set to provide details on how well the
request has been met.  The meaning of this argument is described in
more detail below.
\begin{quote}\index{sane\_control\_option}
\begin{verbatim}
SANE_Status sane_control_option (SANE_Handle h, SANE_Int n,
                                 SANE_Action a, void *v,
                                 SANE_Int * i);
\end{verbatim}
\end{quote}

The way the option is affected by a call to this function is
controlled by parameter \code{a} which is a value of type
\code{\defn{SANE\_Action}}.  The possible values and their meaning is
described in Table~\ref{tab:actions}.

\begin{table}[h]
  \begin{center}
    \leavevmode
    \begin{tabular}{|l|r|p{0.5\textwidth}|}
\hline
\multicolumn{1}{|c|}{\bf Symbol} &
\multicolumn{1}{|c|}{\bf Code} &
\multicolumn{1}{|c|}{\bf Description} \\

\hline\hline

\code{\defn{SANE\_ACTION\_GET\_VALUE}} & 0 & Get current option value. \\

\code{\defn{SANE\_ACTION\_SET\_VALUE}} & 1 & Set option value.  The
  option value passed through argument \code{v} may be modified by the
  backend if the value cannot be set exactly.  \\ 

\code{\defn{SANE\_ACTION\_SET\_AUTO}} & 2 & Turn on automatic mode.  Backend
  or device will automatically select an appropriate value.  This mode
  remains effective until overridden by an explicit set value request.
  The value of parameter \code{v} is completely ignored in this case and
  may be \code{NULL}.  \\

\hline
    \end{tabular}
    \caption{Action Values (\code{SANE\_Action})}
    \label{tab:actions}
  \end{center}
\end{table}

After setting a value via an action value of
\verb|SANE_ACTION_SET_VALUE|, additional information on how well the
request has been met is returned in \code{*i} (if \code{i} is
non-\code{NULL}).  The returned value is a bitset that may contain any
combination of the values described in Table~\ref{tab:info}.
\begin{table}[htbp]
  \begin{center}
    \leavevmode
    \begin{tabular}{|l|r|p{0.5\textwidth}|}
\hline
\multicolumn{1}{|c|}{\bf Symbol} &
\multicolumn{1}{|c|}{\bf Code} &
\multicolumn{1}{|c|}{\bf Description} \\

\hline\hline

\code{\defn{SANE\_INFO\_INEXACT}} & 1 & This value is returned when
  setting an option value resulted in a value being selected that does
  not exactly match the requested value.  For example, if a scanner
  can adjust the resolution in increments of 30dpi only, setting the
  resolution to 307dpi may result in an actual setting of 300dpi.
  When this happens, the bitset returned in \code{*i} has this member
  set.  In addition, the option value is modified to reflect the
  actual (rounded) value that was used by the backend.  Note that
  inexact values are admissible for strings as well.  A backend may
  choose to ``round'' a string to the closest matching legal string
  for a constrained string value. \\

\code{\defn{SANE\_INFO\_RELOAD\_OPTIONS}} & 2 & 
The setting of an option may affect the
\begin{changebar}
  value, the availability, or the constraint of one or more {\em
   other\/} options.
\end{changebar}
  When this happens, the SANE backend sets this
  member in \code{*i} to indicate that the application should reload
  all options.  This member may be set if and only if at least one
  option changed. \\

\code{\defn{SANE\_INFO\_RELOAD\_PARAMS}} & 4 &
The setting of an option may affect the parameter values (see
\code{sane\_get\_parameters()}).
If setting an option affects the parameter values, this member will
be set in \code{*i}.  Note that this member may be set even if the
parameters did not actually change.  However, it is guaranteed that
the parameters never change without this member being set.
\\

\code{\defn{SANE\_INFO\_INVALIDATE\_PREVIEW}} & 8 &
The setting of an option may affect the validity of the preview that 
was acquired by the frontend earlier. When the preview image would change
significantly if it was scanned again, the SANE backend
sets this member in \code{*i} to indicate that the application should inform the
user of the invalidity of the preview. Examples for such option
settings may include setting a different scan source or significantly
changing the exposure.
\\


\hline
    \end{tabular}
    \caption{Additional Information Returned When Setting an Option}
    \label{tab:info}
  \end{center}
\end{table}

This function may fail with one of the following status codes.
\begin{quote}
\begin{description}
\item[\code{SANE\_STATUS\_UNSUPPORTED}:] The operation is not
  supported for the specified handle and option number.
\item[\code{SANE\_STATUS\_INVAL}:] The option value is not valid.
\item[\code{SANE\_STATUS\_IO\_ERROR}:] An error occured while
  communicating with the device.
\item[\code{SANE\_STATUS\_NO\_MEM}:] An insufficent amount of memory
  is available.
\item[\code{SANE\_STATUS\_ACCESS\_DENIED}:] Access to the option has
  been denied due to insufficient or invalid authentication.
\end{description}
\end{quote}



\subsection{\code{sane\_get\_parameters}\label{sec:sanegetparameters}}

This function is used to obtain the current scan parameters.  The
returned parameters are guaranteed to be accurate between the time a
scan has been started (\code{sane\_start()} has been called) and the
completion of that request.  Outside of that window, the returned
values are best-effort estimates of what the parameters will be when
\code{sane\_start()} gets invoked.  Calling this function before a
scan has actually started allows, for example, to get an estimate of
how big the scanned image will be.  The parameters passed to this
function are the handle \code{h} of the device for which the
parameters should be obtained and a pointer \code{p} to a parameter
structure.  The parameter structure is described in more detail below.

\begin{quote}\index{sane\_get\_parameters}
\begin{verbatim}
SANE_Status sane_get_parameters (SANE_Handle h,
                                 SANE_Parameters * p);
\end{verbatim}
\end{quote}

The scan parameters are returned in a structure of type
\code{\defn{SANE\_Parameters}}.  The C declaration of this structure
is given below.
\begin{changebar}
\begin{quote}
\begin{verbatim}
typedef struct
  {
    SANE_Frame format;
    SANE_Int flags;
    SANE_Int lines;
    SANE_Int depth;
    SANE_Int pixels_per_line;
    SANE_Int bytes_per_line;
    SANE_Int channels_per_image;
    SANE_String format_desc;
    SANE_String proposed_filename;
    SANE_String proposed_comment;
    SANE_Int dpi_x;
    SANE_Int dpi_y;
    char reserved[32]; /* 32 bytes for future use */
  }
SANE_Parameters;
\end{verbatim}
\end{quote}
\end{changebar}

\begin{changebar}
Member \code{format} specifies the format of the next frame to be
returned.  The possible values for type \code{\defn{SANE\_Frame}} are
described in Table~\ref{tab:frameformat}.  The meaning of these
values is described in more detail in section~\ref{sec:imageformat}.
The frame types used by version~1 of this standard
(\code{\defn{SANE\_FRAME\_GRAY}}, \code{\defn{SANE\_FRAME\_RGB}},
\code{\defn{SANE\_FRAME\_RED}}, \code{\defn{SANE\_FRAME\_GREEN}}, and
\code{\defn{SANE\_FRAME\_BLUE}}) are obsolete and superseded by the 
frame type \code{\defn{SANE\_FRAME\_RAW}}.

\begin{table}[htbp]
  \begin{center}
    \leavevmode
    \begin{tabular}{|l|r|l|}
\hline
\multicolumn{1}{|c|}{\bf Symbol} &
\multicolumn{1}{|c|}{\bf Code} &
\multicolumn{1}{|c|}{\bf Description} \\

\hline\hline

\code{\defn{SANE\_FRAME\_RAW}}   & 5 & Arbitrary pixel property transmission. \\
\code{\defn{SANE\_FRAME\_MIME}}  & 6 & Data described by a mime descriptor. \\

\hline
    \end{tabular}
    \caption{Frame Format (\code{SANE\_Frame})}
    \label{tab:frameformat}
  \end{center}
\end{table}

The \code{flags} member is a 32 bit bitfield, for which up to now 4 
informational bits are defined, all unused bits have to be set to 0: 

\begin{itemize}

\item
\code{SANE\_PFLAG\_LAST\_FRAME} (bit 0, bitvalue 1) is set to 1 if and
only if the frame that is currently being acquired (or the frame that
will be acquired next if there is no current frame) is the only frame
or the last frame
of a multi frame image (e.g., the current frame is the blue component
of a red, green, blue image).

\item
\code{SANE\_PFLAG\_MORE\_IMAGES} (bit 1, bitvalue 2) is set to 1 to indicate 
further pending images. The frontend is expected to call \code{sane\_start}
again after the end of the current scan to get more images, e.g.
from an automatic document feeder.  It is permissible to set that value to 1 ``in good 
faith'', as it has to be determined at a very early time, where it might 
not be detectable, if there actually are more images to transfer. E.g. 
you will usually not know if the document feeder contains further pages 
when starting to scan the current one. Thus you are allowed to set that 
bit but later fail at \code{sane\_start()}.

\item
\code{SANE\_PFLAG\_NEW\_PAGE} (bit 2, bitvalue 4) is set to 1 to indicate 
that the current frame comes from a new physical page. This bit is of 
informational character only to help frontends to group multi-image 
scans. 

\item
\code{SANE\_PFLAG\_BACKSIDE} (bit 3, bitvalue 8) tells if the current image 
was acquired from the front (0) or backside (1) of the currently processed 
sheet. It is of informational character and allows to group and order 
multi-image transfers regardless of scanner acquisition order (front 
first/back first). 

\end{itemize}

Note, that \code{flags} is compatible to member \code{last\_frame} of 
\code{SANE\_Parameters} of SANE standard version 1 (same size
and only bit 0 (bitvalue 1) was used with same function).
\end{changebar}

Member \code{lines} specifies how many scan lines the frame is
comprised of.  If this value is -1, the number of lines is not known a
priori and the frontend should call \code{sane\_read()} until it
returns a status of \code{SANE\_STATUS\_EOF}.
\begin{changebar}
Note, that even when transferring formats that have this information 
inband, it is recommended to set that member, if available. If 
unavailable or not applicable, set to -1 as mentioned above. 
\end{changebar}

Member \code{bytes\_per\_line} specifies the number of bytes that comprise one
scan line.
\begin{changebar}
If this value is not applicable for image data of type
\code{SANE\_FRAME\_MIME}, it must be set to -1. In this case the frontend
should just call \code{sane\_read} until \code{SANE\_STATUS\_EOF} is returned.
\end{changebar}

Member \code{depth} specifies the number of bits per sample.
\begin{changebar}
Note, that only -1 (for not applicable), 1, and multiples of 8 are
allowed values. Data with other depths has to be scaled up accordingly.
Depth 1 is only allowed if the image consists of a single channel (lineart or
halftone modes).
\end{changebar}

Member \code{pixels\_per\_line} specifies the number of pixels that
comprise one scan line.
\begin{changebar}
If unavailable or not applicable, set to -1.
\end{changebar}

Assume $B$ is the number of channels in the frame, then the bit depth
$d$ (as given by member \code{depth}) and the number of pixels per
line $n$ (as given by this member \code{pixels\_per\_line}) are
related to $c$, the number of bytes per line (as given by member
\code{bytes\_per\_line}) as follows:
\begin{changebar}
\[
  c >= \left\{
  \begin{array}{ll}
    \lceil B\cdot n / 8\rceil & \mbox{if $d=1$}\\
    B\cdot n \cdot d / 8 & \mbox{if $d>1$}
  \end{array}
  \right.
\]
\end{changebar}
Note that the number of bytes per line can be larger than the minimum
value imposed by the right side of this equation.  A frontend must be
able to properly cope with such ``padded'' image formats.

\begin{changebar}
Member \code{channels\_per\_image} specifies the number of channels the
image consists of. When the image is transmitted in more than one frame
\code{channels\_per\_image} has to be the same for all frames for this image.

Member \code{format\_desc} is used to describe the details of the frame
formats.  Its meaning differs between the two formats:

\begin{itemize}
\item
\code{SANE\_FRAME\_RAW}: 
The \code{format\_desc} contains a description of the channel data and
an optional depth information separated by a colon(:).

A plane is descibed by one channel, e.g. ``\code{gray}'' or ``\code{gray:12}''.

Channel interleaved data is described by a comma separated list of channel descriptions,
for example ``\code{red,green,blue}'' or ``\code{red:8,green:8,blue:8}'',
the channel data is sent in the given order.

The depth information does {\bf not} define the size of the transmitted
channel data, it is only an information for the frontend. The channel data has
to be sent in the size defined by member \code{depth}.

Well known channels are \code{red}, \code{green}, \code{blue} and \code{gray}.
It also is allowed to use other channel descriptions, e.g. if you
use an infrared camera or scanner it could be \code{infrared} or
a wavelength description like \code{1100nm}, but be aware that a
frontend may not be able to display such channels with useful colors.

Note that an image can be sent in single planes, in one interleaved
frame that contains all channels or in several frames that contain
one or more (interleaved) channels. When an RGB image is sent it
is prefered to send the image data in one interleaved frame
that consist of red, green and blue data in this order.
The number of channels is defined in member \code{channels\_per\_image}.

\item
\code{SANE\_FRAME\_MIME}: 

\code{format\_desc} contains the MIME Content-Type: header field as described
in RFC 1521 (section 4) without the prefixing "Content-Type:". MIME Types and
subtypes should be either chosen from the RFC or from the list of
IANA-approved values. The data stream must be compliant with the corresponding
specification.

When data is transmitted with the frame type \code{SANE\_FRAME\_MIME}
all data has to be transmitted within one frame, multiple frames
are not allowed (so the flag \code{last\_frame} has to be set
when using this frame type).

A SANE backend must be able to at least optionally transmit
\code{SANE\_FRAME\_RAW} (possibly with the help of a meta backend), if the
hardware supports delivering image data at all. For data that doesn't
comprise images, it's admisable to only provide MIME frames. As a
general principle, if there are several choices in MIME types, the
format that is most widely implemented should be used.
\end{itemize}

The member \code{proposed\_filename} can be used to suggest a reasonable 
default filename or -extension in case the backend can make such a 
suggestion, like e.g. an image database. 
If no such suggestion is intended, set the field to ``''. 

In the case of raw frames, \code{proposed\_filename} is expected to hold
the basename for the image, with the extension determined by the save function
of the frontend, as the frontend can fully understand the data and is thus
able to encode it in any format it wishes.

For MIME frames \code{proposed\_filename} can contain either: 

\begin{itemize}
\item
A name with a leading dot, which is considered to be a proposed 
filename extension. This could also be gotten from the mime database, 
but for systems lacking it, this might be convenient. Or: 

\item
A complete filename, including extension. 
\end{itemize}

Note, that for frontends that are able to parse a given MIME type 
internally, it is perfectly permissible to ignore the extension 
part of the proposed filename and only make use of the basename, 
when using internal save algorithms for different formats. 

The string \code{proposed\_comment} can be used to transmit additional 
image information, that can be stored in the comment areas several file formats 
offer. It can contain any textual information the backend wishes to 
convey to the user, like date/time of exposure, enganged filters, 
etc. Set to ``'' if unused.

The members \code{dpi\_x} and \code{dpi\_y} encode the horizontal and vertical
resolution. Note, that multiple-image scans may have different resolutions of
each image. It is not permissible to change resolution between frames of the
same image. If not applicable, set to -1.

The member \code{reserved} is an array of 32 bytes (char) to keep
the size of the struct unchanged when future extensions are done.
The backend has to set the reserved bytes to 0.
\end{changebar}


\subsection{\code{sane\_start}}

This function initiates aquisition of an image from the device
represented by handle \code{h}.
\begin{quote}\index{sane\_start}
\begin{verbatim}
SANE_Status sane_start (SANE_Handle h);
\end{verbatim}
\end{quote}
This function may fail with one of the following status codes.
\begin{quote}
\begin{description}
\item[\code{SANE\_STATUS\_CANCELLED}:] The operation was cancelled through
  a call to \code{sane\_cancel}.
\item[\code{SANE\_STATUS\_DEVICE\_BUSY}:] The device is busy. The
  operation should be retried later.
\item[\code{SANE\_STATUS\_JAMMED}:] The document feeder is jammed.
\item[\code{SANE\_STATUS\_NO\_DOCS}:] The document feeder is out of
  documents.
\item[\code{SANE\_STATUS\_COVER\_OPEN}:] The scanner cover is open.
\item[\code{SANE\_STATUS\_IO\_ERROR}:] An error occurred while communicating
  with the device.
\item[\code{SANE\_STATUS\_NO\_MEM}:] An insufficent amount of memory
  is available.
\item[\code{SANE\_STATUS\_INVAL}:] The scan cannot be started with the current
  set of options. The frontend should reload the option descriptors, as if
  \code{\defn{SANE\_INFO\_RELOAD\_OPTIONS}} had been returned from a call to
  \code{sane\_control\_option()}, since the device's capabilities may have
  changed.
\end{description}
\end{quote}


\subsection{\code{sane\_read}}

This function is used to read image data from the device represented
by handle \code{h}.  Argument \code{buf} is a pointer to a memory area
that is at least \code{maxlen} bytes long.  The number of bytes
returned is stored in \code{*len}.  A backend must set this to zero
when a status other than \code{SANE\_STA\-TUS\_GOOD} is returned).
When the call succeeds, the number of bytes returned can be anywhere in
the range from 0 to \code{maxlen} bytes.
\begin{quote}\index{sane\_read}
\begin{verbatim}
SANE_Status sane_read (SANE_Handle h, SANE_Byte * buf,
                       SANE_Int maxlen, SANE_Int * len);
\end{verbatim}
\end{quote}
\begin{changebar}
For efficiency reasons, medium to large 
block sizes (in the range of a few kilobytes) should be used. 
Returning short reads is allowed to allow for small buffers 
in the backend. 
\end{changebar}
If this function is called when no data is available, one of two
things may happen, depending on the I/O mode that is in effect for
handle \code{h}.
\begin{enumerate}
\item If the device is in blocking I/O mode (the default mode), the
  call blocks until at least one data byte is available (or until some
  error occurs).

\item If the device is in non-blocking I/O mode, the call returns
  immediately with status \code{SANE\_STA\-TUS\_GOOD} and with
  \code{*len} set to zero.
\end{enumerate}
The I/O mode of handle \code{h} can be set via a call to
\code{sane\_set\_io\_mode()}.

This function may fail with one of the following status codes.
\begin{quote}
\begin{description}
\item[\code{SANE\_STATUS\_CANCELLED}:] The operation was cancelled through
  a call to \code{sane\_cancel}.
\item[\code{SANE\_STATUS\_EOF}:] No more data is available for the
  current frame.
\begin{changebar}
If \code{sane\_read} sends back any image data it
is not allowed to return with \code{SANE\_STATUS\_EOF}.
\end{changebar}
\item[\code{SANE\_STATUS\_JAMMED}:] The document feeder is jammed.
\item[\code{SANE\_STATUS\_NO\_DOCS}:] The document feeder is out of
  documents.
\item[\code{SANE\_STATUS\_COVER\_OPEN}:] The scanner cover is open.
\item[\code{SANE\_STATUS\_IO\_ERROR}:] An error occurred while communicating
  with the device.
\item[\code{SANE\_STATUS\_NO\_MEM}:] An insufficent amount of memory
  is available.
\item[\code{SANE\_STATUS\_ACCESS\_DENIED}:] Access to the device has
  been denied due to insufficient or invalid authentication.
\end{description}
\end{quote}


\subsection{\code{sane\_cancel}}

This function is used to immediately or as quickly as possible cancel
the currently pending operation of the device represented by handle
\code{h}.
\begin{quote}\index{sane\_cancel}
\begin{verbatim}
void sane_cancel (SANE_Handle h);
\end{verbatim}
\end{quote}
This function can be called at any time (as long as handle \code{h} is
a valid handle) but usually affects long-running operations only (such
as image is acquisition).  It is safe to call this function
asynchronously (e.g., from within a signal handler).  It is important
to note that completion of this operaton does {\em not\/} imply that
the currently pending operation has been cancelled.  It only
guarantees that cancellation has been {\em initiated}.  Cancellation
completes only when the cancelled call returns (typically with a
status value of \code{SANE\_STATUS\_CANCELLED}).  Since the SANE API
does not require any other operations to be re-entrant, this implies
that a frontend must {\em not\/} call any other operation until the
cancelled operation has returned.


\subsection{\code{sane\_set\_io\_mode}}

This function is used to set the I/O mode of handle \code{h}.  The I/O mode
can be either blocking or non-blocking.  If argument \code{m} is
\code{SANE\_TRUE}, the mode is set to non-blocking mode, otherwise it's set to
blocking mode.  This function can be called only after a call to
\code{sane\_start()} has been performed.
\begin{quote}\index{sane\_set\_io\_mode}
\begin{verbatim}
SANE_Status sane_set_io_mode (SANE_Handle h, SANE_Bool m);
\end{verbatim}
\end{quote}
By default, newly opened handles operate in blocking mode.  A backend
may elect not to support non-blocking I/O mode.  In such a case the
status value \code{SANE\_STATUS\_UNSUPPORTED} is returned.  Blocking
I/O must be supported by all backends, so calling this function with
argument \code{m} set to \code{SANE\_FALSE} is guaranteed to complete
successfully.

This function may fail with one of the following status codes:
\begin{quote}
\begin{description}
\item[\code{SANE\_STATUS\_INVAL}:] No image acquisition is pending.
\item[\code{SANE\_STATUS\_UNSUPPORTED}:] The backend does not support
the requested I/O mode.
\end{description}
\end{quote}


\subsection{\code{sane\_get\_select\_fd}}

This function is used to obtain a (platform-specific) file-descriptor
for handle \code{h} that is readable if and only if image data is
available (i.e., when a call to \code{sane\_read()} will return at
least one byte of data).  If the call completes successfully, the
select file-descriptor is returned in \code{*fd}.
\begin{quote}\index{sane\_get\_select\_fd}
\begin{verbatim}
SANE_Status sane_get_select_fd (SANE_Handle h, SANE_Int *fd);
\end{verbatim}
\end{quote}
This function can be called only after a call to \code{sane\_start()}
has been performed and the returned file-descriptor is guaranteed to
remain valid for the duration of the current image acquisition (i.e.,
until \code{sane\_cancel()} or \code{sane\_start()} get called again
or until \code{sane\_read()} returns with status
\code{SANE\_STA\-TUS\_EOF}).  Indeed, a backend must guarantee to
close the returned select file descriptor at the point when the next
\code{sane\_read()} call would return \code{SANE\_STA\-TUS\_EOF}.
This is necessary to ensure the application can detect when this
condition occurs without actually having to call \code{sane\_read()}.

A backend may elect not to support this operation.  In such a case,
the function returns with status code
\code{SANE\_STATUS\_UNSUPPORTED}.

Note that the only operation supported by the returned file-descriptor
is a host operating-system dependent test whether the file-descriptor
is readable (e.g., this test can be implemented using \code{select()}
or \code{poll()} under UNIX).  If any other operation is performed on
the file descriptor, the behavior of the backend becomes
unpredictable.  Once the file-descriptor signals ``readable'' status,
it will remain in that state until a call to \code{sane\_read()} is
performed.  Since many input devices are very slow, support for this
operation is strongly encouraged as it permits an application to do
other work while image acquisition is in progress.

This function may fail with one of the following status codes:
\begin{quote}
\begin{description}
\item[\code{SANE\_STATUS\_INVAL}:] No image acquisition is pending.
\item[\code{SANE\_STATUS\_UNSUPPORTED}:] The backend does not support
  this operation.
\end{description}
\end{quote}


\subsection{\code{sane\_strstatus}}

This function can be used to translate a SANE status code into a
printable string.  The returned string is a single line of text that
forms a complete sentence, but without the trailing period
(full-stop).  The function is guaranteed to never return \code{NULL}.
The returned pointer is valid at least until the next call to this
function is performed.
\begin{quote}\index{sane\_strstatus}
\begin{verbatim}
SANE_String_Const sane_strstatus (SANE_Status status);
\end{verbatim}
\end{quote}

\section{Code Flow}\index{code flow}

The code flow for the SANE API is illustrated in
Figure~\ref{fig:flow}.  Functions \code{sane\_init()} and
\code{sane\_exit()} initialize and exit the backend, respectively.
All other calls must be performed after initialization and before
exiting the backend.

\begin{figure}[htb]
  \begin{center}
    \leavevmode
    \psfig{file=figs/flow.eps,height=0.5\textheight}
    \caption{Code flow}
    \label{fig:flow}
  \end{center}
\end{figure}

Function \code{sane\_get\_devices()} can be called any time after
\code{sane\_init()} has been called.  It returns the list of the
devices that are known at the time of the call.  This list may change
over time since some devices may be turned on or off or a remote host
may boot or shutdown between different calls.  It should be noted that
this operation may be relatively slow since it requires contacting all
configured devices (some of which may be on remote hosts).  A frontend
may therefore want to provide the ability for a user to directly
select a desired device without requiring a call to this function.

Once a device has been chosen, it is opened using a call to
\code{sane\_open()}.  Multiple devices can be open at any given time.
A SANE backend must not impose artificial constraints on how many
devices can be open at any given time.

An opened device can be setup through the corresponding device handle
using functions \code{sane\_get\_opt\-ion\_desc\-riptor()} and
\code{sane\_control\_option()}.  While setting up a device, obtaining
option descriptors and setting and reading of option values can be
mixed freely.  It is typical for a frontend to read out all available
options at the beginning and then build a dialog (either graphical or
a command-line oriented option list) that allows to control the
available options.  It should be noted that the number of options is
fixed for a given handle. 
\begin{changebar}
  However, as options are set, other options may become active or inactive or
  their constraint may change. Thus, after setting an option, it may be
  necessary to re-read the descriptors.
\end{changebar}
While
setting up the device, it is also admissible to call
\code{sane\_get\_parameters()} to get an estimate of what the image
parameters will look like once image acquisition begins.

The device handle can be put in blocking or non-blocking mode by a
call to \code{sane\_set\_io\_mode()}.  Devices are required to support
blocking mode (which is the default mode), but support for
non-blocking I/O is strongly encouraged for operating systems such as
UNIX.

After the device is setup properly, image acquisition can be started
by a call to \code{sane\_start()}.  The backend calculates the exact
image parameters at this point.  So future calls to
\code{sane\_get\_parameters()} will return the exact values, rather
than estimates.  Whether the physical image acquisition starts at this
point or during the first call to \code{sane\_read()} is unspecified
by the SANE API.  If non-blocking I/O and/or a select-style interface
is desired, the frontend may attempt to call
\code{sane\_set\_io\_mode()} and/or \code{sane\_get\_select\_fd()} at
this point.  Either of these functions may fail if the backend does
not support the requested operation.

\begin{changebar}
Image data is collected by repeatedly calling \code{sane\_read()}
until this function will return an end-of-file status
(\code{SANE\_STATUS\_EOF}).  This indicates the end of the current
frame. If the frontend expects additional frames (e.g., the
individual channels of a red/green/blue image or multiple images),
it can call \code{sane\_start()} again.
If the \code{SANE\_PFLAG\_LAST\_FRAME} bit is set in \code{flags}, the
current image is complete. In this case, it should be tested, if
\code{flags} has the \code{SANE\_PFLAG\_MORE\_IMAGES} bit set.
If yes, further calls to \code{sane\_start()} can be made to acquire
more images. Please note, that as this bit has to be set at the beginning
of a the transmission of the last frame before the new image, it is possible,
that no reliable decision can be made at this time. It is thus permissible
for a backend to set this bit, and then later at the actual call to
\code{sane\_start()} return an error like \code{SANE\_STATUS\_NO\_DOCS}. 
Such a sequence is permitted to transmit multiple images from a single 
page as well as multiple pages. This behaviour should be controlled by 
backend options as required, to allow single-page scanning as well as 
ADF-batch-scanning. The frontend should always continue reading all images 
until a frame with \code{SANE\_PFLAG\_LAST\_FRAME} on
and \code{SANE\_PFLAG\_MORE\_IMAGES} off is encountered, or an error other
than \code{SANE\_STATUS\_EOF} occurs in a SANE function.
Note that \code{SANE\_STATUS\_NO\_DOCS} also is an allowed way for the backend
to indicate the end of a multiple image scan.

A frontend may choose to skip frames (e.g. because it cannot parse them), 
which is accomplished by simply calling \code{sane\_start} again, which will get 
you to the next frame, without having to read and discard the current one.

In order to prematurely stop scanning and to reset the backend state, 
\code{sane\_cancel()} can be called at any time. This call is required 
as well after normal termination of a multiple image scan as described above.
\end{changebar}

When done using the device, the handle should be closed by a call to
\code{sane\_close()}.  Finally, before exiting the application,
function \code{sane\_exit()} must be called.  It is important not to
forget to call this function since otherwise some resources (e.g.,
temporary files or locks) may remain unclaimed.

\begin{changebar}
The following C sample code implements a reference loop for acquiring 
multiple images:

\begin{verbatim}
SANE_Parameters parms;
SANE_Status     status;

do
  {
    do
      {
        /* Now start acquiring the next frame. */
        status = sane_start (handle);

        /* if that failed, we have a problem, and no more frames can be 
         * read at this time. Due to SANE_PFLAG_MORE_IMAGES still 
         * being clear, this will break out of _BOTH_ loops. 
         */
        if (status != SANE_STATUS_GOOD)
          break;

        /* Now let us see what the next frame brings. */
        status = sane_get_parameters (handle, &parms);

        /* This actually should not fail, but maybe the doc feeder 
         * jammed or something, so we break as well, if something 
         * is wrong. 
         */
        if (status != SANE_STATUS_GOOD)
          break;

        /* Now we check the announced parameters, if we can make use 
         * of the frame data. If not, we skip over to the next frame. 
         */
        if (do_i_like_that (&parms) == NO)
          continue;

        /* Set up for reading the data here. Mangle filenames, 
         * allocate memory, rewind multiframe files, ask user 
         * for confirmation, ... 
         */
        setup_for_transfer (...);

        /* Now we read in the frame data and process it. This should 
         * return SANE_STATUS_GOOD, until the frame is complete, 
         * what causes SANE_STATUS_EOF to be returned. 
         */
        while (SANE_STATUS_GOOD == (status = sane_read (...)))
          read_in_and_process_data_as_required ();

        /* If transfer was broken due to anything but EOF, break out. */
        if (status != SANE_STATUS_EOF)
          break;

        /* Now loop until we have all frames of an image. */
      }
    while (!(parms.flag & SANE_PFLAG_LAST_FRAME));

    /* O.K. - we now have a complete image. Fit it together, save it, 
     * flush buffers, transmit it, increment filenames, etc. 
     */

    /* Now check for more pending images. If we have more, redo from start. 
     * Some backends might cheat here and send us for an extra round which 
     * will fail at sane_start, as they were not able to determine if they 
     * would have more data at the start of the last frame we read. 
     */
  }
while (parms.flags & SANE_PFLAG_MORE_IMAGES);

/* No more data. Fine. Reset the backend and go back to option-control 
 * loop. 
 */

sane_cancel (handle);
\end{verbatim}

\end{changebar}


\section{Well-Known Options}\index{well-known options}

While most backend options are completely self-describing, there are
cases where a user interface might want to special-case the handling
of certain options.  For example, the scan area is typically defined
by four options that specify the top-left and bottom-right corners of
the area.  With a graphical user interface, it would be tedious to
force the user to type in these four numbers.  Instead, most such
interfaces will want to present to the user a preview (low-resolution
scan of the full scanner surface or a high(er) resolution scan of a subpart
of the scanner surface) and let the user pick the scan area by
dragging a rectangle into the desired position.  For this reason, the
SANE API specifies a small number of option names that have
well-defined meanings.

\subsection{Option Number Count}\index{option count}

Option number 0 has an empty string as its name.  The value of this
option is of type \code{SANE\_TYPE\_INT} and it specifies the total
number of options available for a given device (the count includes
option number 0).  This means that there are two ways of counting the
number of options available: a frontend can either cycle through all
option numbers starting at one until
\code{sane\_get\_option\_descriptor()} returns \code{NULL}, or a
frontend can directly read out the value of option number 0.

\subsection{Scan Resolution Options}\index{scan resolution}\index{resolution option}

Option \code{resolution} is used to select the resolution at which an
image should be acquired.
\begin{changebar}
When the backend wants to allow different
values for x- and y-resolution it has to define the options
\code{x-resolution} and \code{y-resolution}. Note that only
the option \code{resolution} {\bf or} the options
\code{x-resolution} {\bf and} \code{y-resolution} may be active at the
same time.
\end{changebar}

The type of this option is either \code{SANE\_TYPE\_INT} or
\code{SANE\_TYPE\_FIXED}.  The unit is \code{SANE\_UNIT\_DPI} (dots/inch).

These options are not mandatory, but if a backend does support them, it
must implement them in a manner consistent with the above definition.

\subsection{Preview Mode Option}\index{preview mode}

The boolean option \code{preview} is used by a frontend to inform the
backend when image acquisition should be optimized for speed, rather
than quality (``preview mode'').  When set to \code{SANE\_TRUE},
preview mode is in effect, when set to \code{SANE\_FALSE} image
acquisition should proceed in normal quality mode.  The setting of
this option \emph{must not\/} affect any other option.  That is, as
far as the other options are concerned, the preview mode is completely
side effect free.  A backend can assume that the frontend will take
care of appropriately setting the scan resolution for preview mode
(through option \code{resolution}).  A backend is free to override the
\code{resolution} value with its own choice for preview mode, but it
is advised to leave this choice to the frontend wherever possible.
\begin{changebar}
When the \code{preview} option is set the backend should transfer
the image in frame type \code{SANE\_FRAME\_RAW} if possible.
\end{changebar}

This option is not mandatory, but if a backend does support it, it
must implement it in a manner consistent with the above definition.

\subsection{Scan Area Options\label{subsec:scanarea}}\index{scan area options}

The four most important well-known options are the ones that define
the scan area.  The scan area is defined by two points (x/y coordinate
pairs) that specify the top-left and the bottom-right corners.  This
is illustrated in Figure~\ref{fig:area}.  Note that the origin of the
coordinate system is at the top-left corner of the scan surface as
seen by the sensor (which typically is a mirror image of the scan
surface seen by the user).  For this reason, the top-left corner is
the corner for which the abscissa and ordinate values are
simultaneously the {\em smallest} and the bottom-right corner is the
corner for which the abscissa and ordinate values are simulatenously
the {\em largest}.  If this coordinate system is not natural for a
given device, it is the job of the backend to perform the necessary
conversions.
\begin{figure}[tbp]
  \begin{center}
    \leavevmode
    \psfig{file=figs/area.eps,height=0.3\textheight}
    \caption{Scan area options}
    \label{fig:area}
  \end{center}
\end{figure}

The names of the four options that define the scan area are given in
the table below:
\begin{center}
\begin{tabular}{ll}
{\bf Name} & {\bf Description} \\
\code{\defn{tl-x}} & Top-left $x$ coordinate value \\
\code{\defn{tl-y}} & Top-left $y$ coordinate value \\
\code{\defn{br-x}} & Bottom-right $x$ coordinate value \\
\code{\defn{br-y}} & Bottom-right $y$ coordinate value \\
\end{tabular}
\end{center}
There are several rules that should be followed by front and backends
regarding these options:
\begin{itemize}

\item Backends must attach a unit of either pixels
  (\code{SANE\_UNIT\_PIXEL}) or millimeters (\code{SANE\_UNIT\_MM}) to
  these options.  The unit of all four options must be identical.

\item Whenever meaningful, a backend should attach a range or a
  word-list constraint to these options.

\item A frontend can determine the size of the scan surface by first
  checking that the options have range constraints associated.  If a
  range or word-list constraints exist, the frontend can take the
  minimum and maximum values of one of the x and y option
  range-constraints to determine the scan surface size.

\item A frontend must work properly with any or all of these options
  missing.
  
\begin{changebar}
\item A frontend may temporarily set the values in a way that
\code{tl-x} is larger than \code{br-x} and \code{tl-y} is larger than
\code{br-y}.
\end{changebar}

\end{itemize}

\begin{changebar}
These options are not mandatory, but if a backend does support them, it
must implement them in a manner consistent with the above definition.
\end{changebar}

\begin{changebar}
\subsection{Depth Option}\index{bit depth option}
Option \code{depth} is used to select the image depth in bits/sample
in multi bit mode (e.g. for 24 bit RGB mode this value must be 8).
The type of this option is \code{SANE\_TYPE\_INT}.
The unit is \code{SANE\_UNIT\_BIT}. For 1 bit modes
(Lineart or Halftone) this option has to be inactive.
For selection of 1 bit modes (Lineart or Halftone) the
backend should use the well-known option \code{mode}.
 
This option is not mandatory, but if a backend does support it, it
must implement it in a manner consistent with the above definition.
\end{changebar}
 
\begin{changebar}
\subsection{Scan Mode Options}\index{mode options}
The option \code{mode} defines a \code{SANE\_CONSTRAINT\_STRING\_LIST}
of type \code{SANE\_TYPE\_STRING}.
It is used to select the scanmode (e.g. Color or Gray).
Well known modes are: \code{Color}, \code{Gray}, \code{Halftone}
and \code{Lineart}. \code{Color} and \code{Gray} are multi bit
modes (8 or 16 bits/sample), \code{Halftone} and \code{Lineart}
are single bit modes. When well-known scan modes are used,
a frontend is able to automatically decide which mode is appropriate
for a specific task, e.g the mode \code{Gray}
for scanning a fax.
 
This option is not mandatory, but if a backend does support it, it
must implement it in a manner consistent with the above definition.
\end{changebar}
 
\begin{changebar}
\subsection{Scan Source Options}\index{source options}
The option \code{source} is used to select the scan source
(e.g. Automatic Document Feeder).
It defines a \code{SANE\_CONSTRAINT\_STRING\_LIST}
of type \code{SANE\_TYPE\_STRING}.
Well known sources are: \code{Flatbed}, \code{Transparancy Adapter} and
\code{Automatic Document Feeder}.
 
This option is not mandatory, but if a backend does support it, it
must implement it in a manner consistent with the above definition.
\end{changebar}

\begin{changebar}
\subsection{Threshold Option}\index{threshold option}
The option \code{threshold} is used to define the threshold
for Lineart and maybe Halftone mode. In multi bit modes
this option should be set inactive.
The type of this option is \code{SANE\_TYPE\_FIXED}.
The unit is \code{SANE\_UNIT\_PERCENT}. The value range
should be 0.0\ldots100.0 if possible.
It defines the minimum intensity to get a white point / full intensity.

The backend has to scale the values in the following way:
A value of 0.0 means all pixels get white / full intensity.  A value of 50.0
means intensities brighter than medium gray get white / full intensity.  A
value of 100.0 means all pixels get black. If the
scanner is not able to cover the full range the backend has to define a
reduced value range (e.g. 30\ldots70 percent).

This option is not mandatory, but if a backend does support it, it
must implement it in a manner consistent with the above definition.
\end{changebar}

\begin{changebar}
\subsection{Gamma Table Options}\index{gamma table options}
The \code{gamma-table} option defines a \code{SANE\_CONSTRAINT\_RANGE}
of the type \code{SANE\_TYPE\_INT} which represents the gamma correction
table for gray.  In color mode the \code{gamma-table} may be used to set
a common gamma correction for red, green and blue. The options
\code{red-gamma-table}, \code{green-gamma-table} and \code{blue-gamma-table}
are used in color mode to set a gamma correction for each color
separately. In color mode the backend is free to use only the
\code{gamma-table} option, only the \code{red-}, \code{green-} and
\code{blue-gamma-table} or all four options. When all four options
are used then the color tables should do a gamma correction with
the same input and output bit depth and the gray gamma table should
reduce (if necessary) the bit depth from the scanner internal
bit depth to the output bit depth. This should e.g. look like this:
\begin{verbatim}
red_value   = gamma-table(red-gamma-table(value))
green_value = gamma-table(green-gamma-table(value))
blue_value  = gamma-table(blue-gamma-table(value))
\end{verbatim}

The backend should not use the gamma tables to emulate other functions or options
like highlight, shadow, contrast, brightness, threshold, analog\_gamma.
These functions are common for all backends and should be added to the frontend
or a meta-backend.

It is also discouraged to emulate gamma tables in the backend. The backend
should disable (or not define) this option when the scanner does not support
gamma tables in hardware.
 
This option is not mandatory, but if a backend does support it, it
must implement it in a manner consistent with the above definition.
\end{changebar}

\begin{changebar}
\subsection{Analog Gamma}\index{analog gamma option}
The option \code{analog-gamma} is used to define the gamma value 
for an analog gamma function of the scanner in multi bit modes.
In 1 bit modes this option should be set inactive.
The type of this option is \code{SANE\_TYPE\_FIXED}.
The unit is \code{SANE\_UNIT\_NONE}. The value range
can be defined by the backend as supported. The values
have to be positive. A gamma value of 1.0 means that
the gamma correction has no effect. A value larger than
1.0 increases the brightness of the image.
In color mode there also can be options \code{analog-gamma-red},
\code{analog-gamma-green} and \code{analog-gamma-blue}.
It is not allowed to emulate an analog gamma function by
a digital gamma table. The backend has to disable (or not
define) this option when the scanner does not support an
analog (hardware) gamma function.

When analog gamma, highlight and shadow functions are available
at the same time then the backend author has to care about the order
in which the functions are implemented in the scanner hardware.
The SANE standard expects that changing the analog gamma value
has no effect on the shadow and highlight function. When the
analog gamma function is executed in the scanner hardware
before the shadow and highlight functions then the backend
should do a compensation. For this the shadow and highlight
values have to be gamma corrected with the relevant analog gamma value.

This option is not mandatory, but if a backend does support it, it
must implement it in a manner consistent with the above definition.
\end{changebar}


\begin{changebar}
\subsection{Shadow Option}\index{shadow options}
The option \code{shadow} is used to define the shadow level / black point level.
The type of this option is \code{SANE\_TYPE\_FIXED}.
The unit is \code{SANE\_UNIT\_PERCENT}. The value range
should be 0.0\ldots100.0 if possible.
It is used to define the maximum intensity level that creates an image data value
of 0 (black).

The backend has to scale the values in the following way:
A value of 0.0 means that the sensitivity range is not reduced, only the
minimum intensity produces an image data value of 0 (black).
A value of 50.0 means that that a medium intensity and everything that is darker
produces an image data value of 0 (black).
A value of 100.0 means the sensitivity range is reduced to 0, all image
data values are 0 (black).  If the scanner is not able to
cover the full range the backend has to define a reduced
value range (e.g. 30\ldots70 percent).
In color mode there can be options \code{shadow-red}, \code{shadow-green}
and \code{shadow-blue}, in this case the \code{shadow} function has to be disabled.
It is not allowed to emulate a shadow function by
a digital gamma table. The backend has to disable (or not
define) this option when the scanner does not support an
analog (hardware) shadow function.

This option is not mandatory, but if a backend does support it, it
must implement it in a manner consistent with the above definition.
\end{changebar}


\begin{changebar}
\subsection{Highlight Option}\index{hightlight options}
The option \code{highlight} is used to define the highlight level / white point level.
The type of this option is \code{SANE\_TYPE\_FIXED}.
The unit is \code{SANE\_UNIT\_PERCENT}. The value range
should be 0.0\ldots100.0 if possible.
It is used to define the minimum intensity level that creates the maximum possible
image data value (white/full intensity). 

The backend has to scale the values in the following way:
A value of 0.0 means the sensitivity range is reduced to 0, all image
data have maximum value (white / full intensity).
A value of 50.0 means that a medium intensity and everything that is brighter
produces the maximum possible image data value (white / full intensity).
A value of 100.0 means that the sensitivity range is not reduced, only the
maximum intensity produces an image data with maximum possible value (white / full intensity).
If  the scanner is not able to cover the full range the backend has to define a reduced
value range (e.g. 30\ldots70 percent).
In color mode there can be options \code{highlight-red}, \code{highlight-green}
and \code{highlight-blue}, in this case \code{highlight} has to be disabled.
It is not allowed to emulate a highlight function by
a digital gamma table. The backend has to disable (or not
define) this option when the scanner does not support an
analog (hardware) highlight function.

This option is not mandatory, but if a backend does support it, it
must implement it in a manner consistent with the above definition.
\end{changebar}


\begin{changebar}
\subsection{Lamp Options}\index{lamp-on option}\index{lamp-off option}

The options \code{lamp-on} and \code{lamp-off} are button options
(\code{SANE\_TYPE\_BUTTON}) and don't have a unit (\code{SANE\_UNIT\_NONE}).

Option \code{lamp-on} is used to turn on the lamp of the scanner while
\code{lamp-off} turns it off.

These options are not mandatory, but if a backend does support them, it
must implement them in a manner consistent with the above definition.
\end{changebar}


\begin{changebar}
\subsection{Scanner Button Options}\index{scanner button options}
Some scanners have buttons which state can be read by the scanner driver. 
It is necessary to implement a locking function for the buttons
because it is possible that several frontends try to connect to the
same backend/scanner at the same time. Imagine what could happen
when no locking would be implemented:
Five people have started a scanning application which is connected
via network to the scanner you want to use. You start a frontend,
put a paper to the scanner and press the scan-button on the scanner.
The scanner does scan three times (because three frontends asked the
button status when you pressed the button). For three people the
image is saved to the harddisk, but it is not sure that your
frontend did scan the image.

A backend that does make available the scanner-buttons has to
implement the following options:\\
\code{scanner-buttons-lock} is of type \code{SANE\_TYPE\_BOOL}, default = \code{SANE\_FALSE}\\
\code{scanner-buttons-status} is of type \code{SANE\_TYPE\_INT}, default = 0\\
\code{scanner-buttons-status-update} is of type \code{SANE\_TYPE\_BUTTON}\\
When setting these options the backend does not set \code{SANE\_INFO\_RELOAD\_OPTIONS}
or \code{SANE\_INFO\_RELOAD\_PARAMS} if not explictly defined.

A frontend has to disable the usage of the scanner-buttons by default. This is important
because other frontends will not be able to use the buttons when the button-functions are locked.
Another important thing is that some scanners do not turn off their lamp when the driver
does frequently talk to the scanner (what is done when reading the button status from the scanner).

\begin{itemize}

\item
A frontend that wants to read the button status has to lock the
button functions at first. For this it does set the option
\code{scanner-buttons-lock} to \code{SANE\_TRUE}.
While setting the value of option \code{scanner-buttons-lock} to \code{SANE\_TRUE}
the backend does check if a lockfile (e.g. ``backend''-buttons.lock) does exist.
The lockfile has to be placed in a directory where every user has read and write access to.

  \begin{itemize}
  \item
  If the lockfile does not exist then the backend creates the lockfile and writes the
  process ID (PID) of the backend to the file. Button access is allowed: the value
  of option \code{scanner-buttons-lock} is set to \code{SANE\_TRUE}
  \item
  If the file does exist and the backend PID is not the file PID then the
  backend has to check if the process with the PID stored in the lockfile
  still is running. If yes then the button access is not allowed: the value of
  option \code{scanner-buttons-lock} is set to \code{SANE\_FALSE}. If not then the lockfile
  is recreated and the PID of the backend is stored in the lockfile, button
  access is allowed: the value of option \code{scanner-buttons-lock} is set to \code{SANE\_TRUE}
  \end{itemize}

\item
The frontend does read the value of option \code{scanner-buttons-lock}. If
it is \code{SANE\_TRUE} then the frontend has access to the scanner buttons.
If it is \code{SANE\_FALSE} then access has been denied.

\item
If the button access is allowed the frontend has to do the following about
once per second (while not scanning):
  \begin{itemize}
  \item
  The frontend does set option \code{scanner-buttons-status-update}.
  The backend checks if access to the buttons is allowed by comparing
  the backend PID with the lockfile PID. If access is allowed it
  does read the button status from the scanner and stores it in
  the option \code{scanner-buttons-status}, each bit represents a button, a
  value of 0 means the button is not pressed, a value of 1 means that the button
  is pressed. When the scanner is busy the backend must not wait, it has to
  return immediately and the button state keeps unchanged.
  The backend has to implement a timeout function. When no button has been pressed
  within a predefined time (e.g. 15 minutes) then the access permission is lost.
  In this case the backend does set option \code{scanner-buttons-lock} to \code{SANE\_FALSE}
  and does set \code{SANE\_INFO\_RELOAD\_OPTIONS} to inform the frontend that it has
  lost permission to access the scanner-button functions.
  If access is not allowed it does set the \code{scanner-buttons-status} to 0.

  \item
  The frontend does read the value of option \code{scanner-buttons-status}
  \end{itemize}

\item
When the frontend does exit or it does not want to use the buttons
it does set option \code{scanner-buttons-lock} to \code{SANE\_FALSE}.
The backend does check if the backend PID and the lockfile PID are
the same. If this is true then it removes the lockfile and sets the value 
of \code{scanner-buttons-lock} to \code{SANE\_FALSE}.

\code{sane\_close()} should do the same as setting option
\code{scanner-buttons-lock} to \code{SANE\_FALSE}.

\end{itemize}
\end{changebar}

\begin{changebar}
\subsection{Batch Scan Options}\index{Batch Scan Options}

The batch scan options can be used by a frontend to indicate that more than
one image will be scanned in a batch. The backend can optimize for such scans
by e.g. avoiding calibration and not moving home the sensor array in this case.
The backend provides the options \code{batch-scan},
\code{batch-scan-next-tl-x}, \code{batch-scan-next-tl-y},
\code{batch-scan-next-br-x}, and \code{batch-scan-next-br-y}. The option
\code{batch-scan} provides a string list with the values of \code{No} if
batch scanning isn't used, \code{Start} for the first, \code{End} for the last
and \code{Loop} for any other (intermediate) image.  The batch-scan-next
options specify the coordinates of the next scan and follow the same rules as
the scan area options (see section \ref{subsec:scanarea}).

These options are not mandatory, but if a backend does support them, it must
implement them in a manner consistent with the above definition.  In this
case, all options must be implemented.
\end{changebar}




\input{net.tex}

\chapter{Contact Information}\label{chap:contact}

The SANE standard is discussed and evolved via a mailing list.
Anybody with email access to the Internet can automatically join and
leave the discussion group by sending mail to the following address.
\begin{quote}\index{mailing list}
\begin{verbatim}
sane-devel-request@mostang.com
\end{verbatim}
\end{quote}
To subscribe, send a mail with the body ``\verb|subscribe sane-devel|'' to the
above address.

A complete list of commands supported can be obtained by sending a
mail with a subject of ``\code{help}'' to the above address.  The
mailing list is archived and available through the SANE home page at
URL:
\begin{quote}
\url{http://www.mostang.com/sane/}
\end{quote}

\newpage
\documentclass[11pt,DVIps]{report}

\usepackage{times,epsfig,changebar,html}

\setlength{\parindent}{0pt}
\setlength{\parskip}{1.5ex plus 0.5ex minus 0.5ex}
\setlength{\textwidth}{6.5in}
\setlength{\textheight}{8.5in}
\setlength{\marginparwidth}{0pt}
\setlength{\oddsidemargin}{0pt}
\setlength{\evensidemargin}{0pt}
\setlength{\marginparsep}{0pt}
\addtolength{\topmargin}{-0.75in}

\title{\huge SANE Standard Version 2.0 proposal 0.08 - rauch/beck}
\author{}
\date{Dec 8, 2002} 

\makeindex

\begin{document}

\newcommand{\filename}[1]{{\tt #1}}
\newcommand{\code}[1]{{\tt #1}}
\newcommand{\var}[1]{{\it #1}}
\newcommand{\defn}[1]{#1\index{#1}}

\begin{latexonly}
  \setcounter{changebargrey}{0}   % black change bars
\end{latexonly}

\maketitle
\tableofcontents
\listoffigures
\listoftables


\chapter{Preface}

The SANE standard is being developed by a group of free-software
developers.  The process is open to the public and comments as well as
suggestions for improvements are welcome.  Information on how to join
the SANE development process can be found in Chapter
\ref{chap:contact}.

The SANE standard is intended to streamline software development by
providing a standard application programming interface to access
raster scanner hardware.  This should reduce the number of different
driver implementations, thereby reducing the need for reimplementing
similar code.


\section{About This Document}

This document is intended for developers who are creating either an
application that requires access to raster scanner hardware and for
developers who are implementing a SANE driver.  It does not cover
specific implementations of SANE components.  Its sole purpose is to
describe and define the SANE application interface that will enable
any application on any platform to interoperate with any SANE backend
for that platform.

The remainder of this document is organized as follows.
Chapter~\ref{chap:intro} provides introductional material.
Chapter~\ref{chap:environ} presents the environment SANE is designed
for.  Chapter~\ref{chap:api} details the SANE Application Programmer
Interface.  Chapter~\ref{chap:net} specifies the network protocol that
can be used to implement the SANE API in a network transparent
fashion.  Finally, Chapter~\ref{chap:contact} gives information on how
to join the SANE development process.

\subsection{Typographic Conventions}

Changes since the last revision of this document are highlighted
like this:

\begin{changebar}
   Paragraphs that changed since the last revision of the documention
   are marked like this paragraph.
\end{changebar}

\chapter{Introduction}\label{chap:intro}

SANE is an application programming interface (API) that provides
standardized access to any raster image scanner hardware.  The
standardized interface allows to write just one driver for each
scanner device instead of one driver for each scanner and application.
The reduction in the number of required drivers provides significant
savings in development time.  More importantly, SANE raises the level
at which applications can work.  As such, it will enable applications
that were previously unheard of in the UNIX world.  While SANE is
primarily targeted at a UNIX environment, the standard has been
carefully designed to make it possible to implement the API on
virtually any hardware or operating system.

SANE is an acronym for ``Scanner Access Now Easy.''  Also, the hope is
that SANE is sane in the sense that it will allow easy implementation
of the API while accommodating all features required by today's
scanner hardware and applications.  Specifically, SANE should be broad
enough to accommodate devices such as scanners, digital still and
video cameras, as well as virtual devices like image file filters.

\section{Terminology}

An application that uses the SANE interface is called a SANE {\em
  frontend}.  A driver that implements the SANE interface is called a
SANE {\em backend}.  A {\em meta backend\/} provides some means to
manage one or more other backends.


\chapter{The SANE Environment}\label{chap:environ}

SANE is defined as a C-callable library interface.  Accessing a raster
scanner device typically consists of two phases: first, various
controls of the scanner need to be setup or queried.  In the second
phase, one or more images are acquired.

Since the device controls are widely different from device to device,
SANE provides a generic interface that makes it easy for a frontend to
give a user access to all controls without having to understand each
and every device control.  The design principle used here is to
abstract each device control into a SANE {\em option\/}.  An option is
a self-describing name/value pair.  For example, the brightness
control of a camera might be represented by an option called
\code{brightness} whose value is an integer in the range from 0 to
255.

With self-describing options, a backend need not be concerned with
{\em presentation\/} issues: the backend simply provides a list of
options that describe all the controls available in the device.
Similarly, there are benefits to the frontend: it need not be
concerned with the {\em meaning\/} of each option.  It simply provides
means to present and alter the options defined by the backend.


\section{Attaching to a SANE backend}

The process through which a SANE frontend connects to a backend is
platform dependent.  Several possibilities exist:
\begin{itemize}

\item {\bf Static linking:} A SANE backend may be linked directly into
  a frontend.  While the simplest method of attaching to a backend, it
  is somewhat limited in functionality since the available devices is
  limited to the ones for which support has been linked in when the
  frontend was built.  But even so static linking can be quite useful,
  particularly when combined with a backend that can access scanners
  via a network.  Also, it is possible to support multiple backends
  simultaneously by implementing a meta backend that manages several
  backends that have been compiled in such a manner that they export
  unique function names.  For example, a backend called \code{be}
  would normally export a function called \code{sane\_read()}.  If
  each backend would provide such a function, static linking would
  fail due to multiple conflicting definitions of the same symbol.
  This can be resolved by having backend \code{be} include a
  header file that has lines of the form:
  \begin{quote}
\begin{verbatim}
#define sane_read be_sane_read
\end{verbatim}
  \end{quote}
  With definitions of this kind, backend \code{be} will export
  function name \code{be\_sane\_read()}.  Thus, all backends will
  export unique names.  As long as a meta backend knows about these
  names, it is possible to combine several backends at link time and
  select and use them dynamically at runtime.

\item {\bf Dynamic linking:} A simpler yet more powerful way to
  support multiple backends is to exploit dynamic linking on platforms
  that support it.  In this case, a frontend is linked against a
  shared library that implements any SANE backend.  Since each
  dynamically linked backend exports the same set of global symbols
  (all starting with the prefix \code{sane\_}), the dynamic library
  that gets loaded at runtime does not necessarily have to be the same
  one as one the frontend got linked against.  In other words, it is
  possible to switch the backend by installing the appropriate backend
  dynamic library.

  More importantly, dynamic linking makes it easy to implement a meta
  backend that loads other backends {\em on demand}.  This is a
  powerful mechanism since it allows adding new backends merely by
  installing a shared library and updating a configuration file.

\item {\bf Network connection:} Arguably the ultimate way to attach to
  a scanner is by using the network to connect to a backend on a
  remote machine.  This makes it possible to scan images from any host
  in the universe, as long as there is a network connection to that
  host and provided the user is permitted to access that scanner.

\end{itemize}

\begin{figure}[htbp]
  \begin{center}
    \leavevmode
    \psfig{file=figs/hierarchy.eps,angle=270,width=\textwidth}
    \caption{Example SANE Hiearchy}
    \label{fig:hierarchy}
  \end{center}
\end{figure}

The above discussion lists just a few ways for frontends to attach to
a backend.  It is of course possible to combine these solutions to
provide an entire hierarchy of SANE backends.  Such a hierarchy is
depicted in Figure~\ref{fig:hierarchy}.  The figure shows that machine
A uses a dynamic-linking based meta backend called \code{dll} to
access the backends called \code{pnm}, \code{mustek}, and \code{net}.
The first two are real backends, whereas the last one is a meta
backend that provides network transparent access to remote scanners.
In the figure, machine B provides non-local access to its scanners
through the SANE frontend called \code{saned}.  The \code{saned} in
turn has access to the \code{hp} and \code{autolum} backends through
another instance of the \code{dll} backend.  The \code{autolum} meta
backend is used to automatically adjust the luminance (brightness) of
the image data acquired by the camera backend called \code{qcam}.

Note that a meta backend really is both a frontend and a backend at
the same time.  It is a frontend from the viewpoint of the backends
that it manages and a backend from the viewpoint of the frontends that
access it.  The name ``meta backend'' was chosen primarily because the
SANE standard describes the interface from the viewpoint of a (real)
frontend.


\section{Image Data Format}\label{sec:imageformat}\index{image data format}

Arguably the most important aspect of an image acquisition system is how
images are represented. The SANE approach is to define a simple yet powerful
representation that is sufficient for vast majority of applications and
devices. While the representation is simple, the interface has been defined
carefully to allow extending it in the future without breaking backwards
compatibility. Thus, it will be possible to accommodate future applications or
devices that were not anticipated at the time this standard was created.

A SANE image is a rectangular area.  The rectangular area is subdivided into a
number of rows and columns.  At the intersection of each row and column is a
(preferable quadratic) pixel. A pixel consists of one or more sample values.
Each sample value represents one channel (e.g., the red channel).

The SANE API transmits an image as a sequence of frames.  Each frame covers
the same rectangular area as the entire image, but may contain only a subset
of the channels in the final image.  For example, a red/green/blue image could
either be transmitted as a single frame that contains the sample values for
all three channels or it could be transmitted as a sequence of three frames:
the first frame containing the red channel, the second the green channel, and
the third the blue channel.

When transmitting an image frame by frame, the frontend needs to know what
part of the image a frame represents (and how many frames it should expect).
For that purpose, the SANE API tags every frame with a type and a format
descriptor.

\begin{changebar}
  There are two different types of frames: pixel oriented frames
  \code{\defn{SANE\_FRAME\_RAW}} and arbitrary data frames
  \code{\defn{SANE\_FRAME\_MIME}}. These types are discussed in detail in the
  following sections. The frame types used by the previous version~1 of this
  standard (\code{\defn{SANE\_FRAME\_GRAY}}, \code{\defn{SANE\_FRAME\_RGB}},
  \code{\defn{SANE\_FRAME\_RED}}, \code{\defn{SANE\_FRAME\_GREEN}}, and
  \code{\defn{SANE\_FRAME\_BLUE}}) are obsolete and superseded by
  \code{\defn{SANE\_FRAME\_RAW}}.

  \subsection{Pixel oriented frames}

  The type of pixel oriented frames is \code{\defn{SANE\_FRAME\_RAW}}. The
  frame contains one or more channels of data in a channel-interleaved format,
  that represents sample values from a property of the individual pixels that
  is subject to further description in the \code{format\_desc} member of the
  \code{SANE\_Parameters} structured type. See section~\ref{sec:sanegetparameters}
  on page~\pageref{sec:sanegetparameters} for details about the format
  descriptions.
\end{changebar}

  Each sample value has a certain bit depth.  The bit depth is fixed for the
  entire image and can be as small as one bit. Valid bit depths are 1, 8, or
  16 bits per sample. If a device's natural bit depth is something else, it is
  up to the driver to scale the sample values appropriately (e.g., a 4 bit
  sample could be scaled by a factor of four to represent a sample value of
  depth 8).

\begin{changebar}
  The complete image may consist of several different channels. The number of channels
  is defined by member \code{channels\_per\_image} of \code{SANE\_Parameters}.
  The image may be transmitted in an arbitrary number of frames which can be
  determined by watching the \code{SANE\_PFLAG\_LAST\_FRAME} flag in said type (or by
  counting the channels).  Note: This frame type replaces all frame types of
  the SANE standard version 1.
\end{changebar}

Conceptually, each pixel oriented frame is transmitted a byte at a time.  Each
byte may contain 8 sample values (for an image bit depth of 1), one full
sample value (for an image bit depth of 8), or a partial sample value (for an
image bit depth of 16 or bigger).  In the latter case, the bytes of each
sample value are transmitted in the machine's native byte order.
\begin{quote}
  \begin{center}
    {\bf Backend Implementation Note}
  \end{center}
  A network-based meta backend will have to ensure that the byte order
  in image data is adjusted appropriately if necessary.  For example,
  when the meta backend attaches to the server proxy, the proxy may
  inform the backend of the server's byte order.  The backend can then
  apply the adjustment if necessary.  In essence, this implements a
  ``receiver-makes-right'' approach.
\end{quote}

\begin{figure}[htbp]
  \begin{center}
    \leavevmode
    \psfig{file=figs/xfer.eps,width=0.5\textwidth}
    \caption{Transfer order of image data bytes}
    \label{fig:xfer}
  \end{center}
\end{figure}

The order in which the sample values in a frame are transmitted is illustrated
in Figure~\ref{fig:xfer}.  As can be seen, the values are transmitted row by
row and each row is transmitted from left-most to right-most column.  The
left-to-right, top-to-bottom transmission order applies when the image is
viewed in its normal orientation (as it would be displayed on a screen, for
example).

If a frame contains multiple channels, then the channels are transmitted in an
interleaved fashion.  Figure~\ref{fig:pixels} illustrates this for the case
where a frame contains a complete red/green/blue image with a bit-depth of 8.

\begin{figure}[htbp]
  \begin{center}
    \leavevmode
    \psfig{file=figs/image-data.eps,width=0.8\textwidth}
    \caption{Bit and byte order of image data}
    \label{fig:pixels}
  \end{center}
\end{figure}

For a bit depth of 1, each byte contains 8 sample values of a {\em single\/}
channel. In other words, a bit depth 1 frame is transmitted in a byte
interleaved fashion. The first sample of each byte is represented by the most
significant bit.

\begin{changebar}
For gray channels at a bit depth of 1 only two sample values are possible: 1
represents minimum intensity (black) and 0 represents maximum intensity
(white).  For all other channel types and bit depths a sample value of 0
represents minimum intensity and larger values represent increasing intensity.

\subsection{Arbitrary data frames}

It also is possible to transmit arbitrary (not necessaryly pixel oriented)
data. This allows transmission of compressed images like jpeg, tiff, etc.

The type of arbitrary data frames is \code{\defn{SANE\_FRAME\_MIME}}.
The frame contains arbitrary data of the MIME (see RFC 1521/1522) type that is
given in the \code{format\_desc} member of the \code{SANE\_Parameters}
structured type (see section~\ref{sec:sanegetparameters} on
page~\pageref{sec:sanegetparameters}).  As such, it is assumed to be
incomprehensible to the frontend, except for selected types the frontend is
specifically capable of handling internally.  The frontend is free to ignore
those frames, or employ any appropriate means to otherwise handle this data
(like saving them to disk or spawning an external viewer).
\end{changebar}

\chapter{The SANE Application Programmer Interface (API)}\label{chap:api}

\begin{changebar}
This Section defines version 2 of the SANE application
programmer interface (API).  Any SANE frontend must depend on the
interface defined in this section only. Converseley, any SANE backend
must implement its functionality in accordance with this
specification. The interface as documented here is declared as a C
callable interface in a file called \filename{sane/sane-2.h}. This file should
normally be included via a C pre-processor directive of the form:
\begin{verbatim}
  #include <sane/sane-2.h>
\end{verbatim}
\end{changebar}


\section{Version Control}

The SANE standard is expected to evolve over time.  Whenever a change
to the SANE standard is made that may render an existing frontend or
backend incompatible with the new standard, the major version number
must be increased.  Thus, any frontend/backend pair is compatible
provided the major version number of the SANE standard they implement
is the same.  A frontend may implement backwards compatiblity by
allowing major numbers that are smaller than the expected major number
(provided the frontend really can cope with the older version).  In
contrast, a backend always provides support for one and only one
version of the standard.  If a specific application does require that
two different versions of the same backend are accessible at the same
time, it is possible to do so by installing the two versions under
different names.

SANE version control also includes a minor version number and a build
revision.  While control of these numbers remains with the implementor
of a backend, the recommended use is as follows.  The minor version is
incremented with each official release of a backend.  The build
revision is increased with each build of a backend.

The SANE API provides the following five macros to manage version
numbers.
\begin{quote}
  \begin{description}
  \item[\code{\defn{SANE\_CURRENT\_MAJOR}}:] The value of this macro is the
    number of the SANE standard that the interface implements.

  \item[\code{\defn{SANE\_VERSION\_CODE}(\var{maj},\var{min},\var{bld})}:]
    \label{sec:saneversioncode}
    This macro can be used to build a monotonically increasing version
    code.  A SANE version code consists of the SANE standard major
    version number (\var{maj}), the minor version number \var{min},
    and the build revision of a backend (\var{bld}).  The major and
    minor version numbers must be in the range 0\ldots255 and the
    build revision must be in the range 0\ldots65535.

    Version codes are monotonic in the sense that it is possible to
    apply relational operators (e.g., equality or less-than test)
    directly on the version code rather than individually on the three
    components of the version code.

    Note that the major version number alone determines whether a
    frontend/backend pair is compatible.  The minor version and the
    build revision are used for informational and bug-fixing purposes
    only.
  
  \item[\code{\defn{SANE\_VERSION\_MAJOR}(\var{vc})}:] This macro returns the
    major version number component of the version code passed in
    argument \var{vc}.
  \item[\code{SANE\_VERSION\_MINOR(\var{vc})}:] This macro returns the
    minor version number component of the version code passed in
    argument \var{vc}.
  \item[\code{SANE\_VERSION\_BUILD(\var{vc})}:] This macro returns the
    build revision component of the version code passed in argument
    \var{vc}.
  \end{description}
\end{quote}


\section{Data Types}

\subsection{Base Types}

The SANE standard is based on just two SANE-specific base types: the
SANE byte and word.
\begin{quote}
  \code{typedef \var{some-scalar-type\/} \defn{SANE\_Byte};} \\
  \code{typedef \var{some-scalar-type\/} \defn{SANE\_Word};}
\end{quote}
\verb|SANE_Byte| must correspond to some scalar C type that is capable
of holding values in the range 0 to 255.  \verb|SANE_Word| must be
capable of holding any of the following:
\begin{itemize}
  \item the truth values \verb|SANE_FALSE| and \verb|SANE_TRUE|
  \item signed integers in the range $-2^{31}\ldots2^{31}-1$
  \item fixed point values in the range $-32768\ldots32767.9999$ with
    a resolution of $1/65536$
  \item 32 bits (for bit sets)
\end{itemize}
Note that the SANE standard does not define what C type
\verb|SANE_Byte| and \verb|SANE_Word| map to.  For example, on some
platforms, the latter may map to \verb|long int| whereas on others it
may map to \verb|int|.  A portable SANE frontend or backend must
therefore not depend on a particular mapping.

\subsection{Boolean Type}

\code{\defn{SANE\_Bool}} is used for variables that can take one of
the two truth values \code{\defn{SANE\_FALSE}} and
\code{\defn{SANE\_TRUE}}.  The former value is defined to be 0,
whereas the latter is 1.\footnote{This is different from ANSI C where
  any non-zero integer value represents logical TRUE.} The C
declarations for this type are given below.
\begin{quote}
\begin{verbatim}
#define SANE_FALSE      0
#define SANE_TRUE       1
typedef SANE_Word SANE_Bool;
\end{verbatim}
\end{quote}
Note that \verb|SANE_Bool| is simply an alias of \verb|SANE_Word|.  It
is therefore always legal to use the latter type in place of the
former.  However, for clarity, it is recommended to use
\verb|SANE_Bool| whenever a given variable or formal argument has a
fixed interpretation as a boolean object.

\subsection{Integer Type}

\code{\defn{SANE\_Int}} is used for variables that can take integer
values in the range $-2^{32}$ to $2^{31}-1$.  Its C declaration is
given below.
\begin{quote}
\begin{verbatim}
typedef SANE_Word SANE_Int;
\end{verbatim}
\end{quote}
Note that \verb|SANE_Int| is simply an alias of \verb|SANE_Word|.  It
is therefore always legal to use the latter type in place of the
former.  However, for clarity, it is recommended to use
\verb|SANE_Int| whenever a given variable or formal argument has a
fixed interpretation as an integer object.


\subsection{Fixed-point Type}

\code{\defn{SANE\_Fixed}} is used for variables that can take fixed
point values in the range $-32768$ to $32767.9999$ with a resolution
of $1/65535$.  The C declarations relating to this type are given
below.
\begin{quote}
\begin{verbatim}
#define SANE_FIXED_SCALE_SHIFT  16
typedef SANE_Word SANE_Fixed;
\end{verbatim}
\end{quote}
The macro \code{\defn{SANE\_FIXED\_SCALE\_SHIFT}} gives the location
of the fixed binary point.  This standard defines that value to be 16,
which yields a resolution of $1/65536$.

Note that \verb|SANE_Fixed| is simply an alias of \verb|SANE_Word|.
It is therefore always legal to use the latter type in place of the
former.  However, for clarity, it is recommended to use
\verb|SANE_Fixed| whenever a given variable or formal argument has a
fixed interpretation as a fixed-point object.

For convenience, SANE also defines two macros that convert fixed-point
values to and from C double floating point values.
\begin{quote}
  \begin{description}

  \item[\code{\defn{SANE\_FIX}(\var{d})}:] Returns the largest SANE
    fixed-point value that is smaller than the double value \var{d}.
    No range checking is performed.  If the value of \var{d} is out of
    range, the result is undefined.

  \item[\code{\defn{SANE\_UNFIX}(\var{w})}:] Returns the nearest
    double machine number that corresponds to fixed-point value
    \var{w}.

  \end{description}
\end{quote}
SANE does {\em not\/} require that the following two expressions hold
true (even if the values of \var{w} and \var{d} are in range):
\begin{quote}
\begin{verbatim}
SANE_UNFIX(SANE_FIX(d)) == d
SANE_FIX(SANE_UNFIX(w)) == w
\end{verbatim}
\end{quote}
In other words, conversion between fixed and double values may be
lossy.  It is therefore recommended to avoid repeated conversions
between the two representations.


\subsection{Text}

\subsubsection{Character Type}

Type \code{\defn{SANE\_Char}} represents a single text character or
symbol.  At present, this type maps directly to the underlying C
\verb|char| type (typically one byte).  The encoding for such
characters is currently fixed as ISO LATIN-1.  Future versions of this
standard may map this type to a wider type and allow multi-byte
encodings to support internationalization.  As a result of this, care
should be taken to avoid the assumption that
\verb|sizeof(SANE_Char)==sizeof(char)|.
\begin{quote}
\begin{verbatim}
typedef char SANE_Char;
\end{verbatim}
\end{quote}

\subsubsection{String Type}

Type \code{\defn{SANE\_String}} represents a text string as a sequence
of C \verb|char| values.  The end of the sequence is indicated by a
\verb|'\0'| (\defn{NUL}) character.  
\begin{quote}
\begin{verbatim}
typedef SANE_Char *SANE_String;
typedef const SANE_Char *SANE_String_Const;
\end{verbatim}
\end{quote}
The type \code{\defn{SANE\_String\_Const}} is provided by SANE to
enable declaring strings whose contents is unchangable.  Note that in
ANSI C, the declaration
\begin{quote}
\begin{verbatim}
const SANE_String str;
\end{verbatim}
\end{quote}
declares a string pointer that is constant (not a string pointer that
points to a constant value).


\subsection{Scanner Handle Type}

Access to a scanner is provided through an opaque type called
\code{\defn{SANE\_Handle}}.  The C declaration of this type is given
below.
\begin{quote}
\begin{verbatim}
typedef void *SANE_Handle;
\end{verbatim}
\end{quote}
While this type is declared to be a void pointer, an application must
not attempt to interpret the value of a \verb|SANE_Handle|.  In
particular, SANE does not require that a value of this type is a legal
pointer value.


\subsection{Status Type}

Most SANE operations return a value of type \code{\defn{SANE\_Status}}
to indicate whether the completion status of the operation.  If an
operation completes successfully, \verb|SANE_STATUS_GOOD| is returned.
In case of an error, a value is returned that indicates the nature of
the problem.  The complete list of available status codes is listed in
Table \ref{tab:status}.  It is recommended to use function
\code{sane\_strstatus()} to convert status codes into a legible
string.

\begin{table}[htbp]
  \begin{center}
    \begin{tabular}{|l|r|l|}
      \hline
      \multicolumn{1}{|c|}{\bf Symbol} & \multicolumn{1}{c|}{\bf Code} &
      \multicolumn{1}{c|}{\bf Description} \\
      \hline\hline
\code{\defn{SANE\_STATUS\_GOOD}}
        &  0 & Operation completed succesfully. \\
\code{\defn{SANE\_STATUS\_UNSUPPORTED}}
        &  1 & Operation is not supported. \\
\code{\defn{SANE\_STATUS\_CANCELLED}}
        &  2 & Operation was cancelled. \\
\code{\defn{SANE\_STATUS\_DEVICE\_BUSY}}
        &  3 & Device is busy---retry later. \\
\code{\defn{SANE\_STATUS\_INVAL}}
        &  4 & Data or argument is invalid. \\
\code{\defn{SANE\_STATUS\_EOF}}
        &  5 & No more data available (end-of-file). \\
\code{\defn{SANE\_STATUS\_JAMMED}}
        &  6 & Document feeder jammed. \\
\code{\defn{SANE\_STATUS\_NO\_DOCS}}
        &  7 & Document feeder out of documents. \\
\code{\defn{SANE\_STATUS\_COVER\_OPEN}}
        &  8 & Scanner cover is open. \\
\code{\defn{SANE\_STATUS\_IO\_ERROR}}
        &  9 & Error during device I/O. \\
\code{\defn{SANE\_STATUS\_NO\_MEM}}
        & 10 & Out of memory. \\
\code{\defn{SANE\_STATUS\_ACCESS\_DENIED}}
        & 11 & Access to resource has been denied. \\
      \hline
    \end{tabular}
    \caption{Status Codes}\label{tab:status}
  \end{center}
\end{table}


\subsection{Device Descriptor Type}
\begin{changebar}

Each SANE device is represented by a structure of type
\code{\defn{SANE\_Device}}.  The C declaration of this type is given
below.
\begin{quote}
\begin{verbatim}
typedef struct
  {
    SANE_String_Const name;
    SANE_String_Const vendor;
    SANE_String_Const model;
    SANE_String_Const type;
    SANE_String_Const email_backend_author;
    SANE_String_Const backend_website;
    SANE_String_Const device_location;
    SANE_String_Const comment;
    SANE_String_Const reserved_string;
    SANE_Int backend_version_code;
    SANE_Int backend_capablity_flags;
    SANE_Int reserved_int;
  }
SANE_Device;
\end{verbatim}
\end{quote}
\end{changebar}
\index{device-name}
The structure provides the unique name of the scanner in member
\code{name}.  It is this unique name that should be passed in a call
to \code{sane\_open()}.  The format of this name is completely up to
the backend.  The only constraints are that the name is unique among
all devices supported by the backend and that the name is a legal SANE
text string.  To simplify presentation of unique names, their length
should not be excessive.  It is {\em recommended\/} that backends keep
unique names below 32 characters in length.  However, applications
{\em must\/} be able to cope with arbitrary length unique names.

The next three members in the device structure provide additional
information on the device corresponding to the unique name.
Specifically, members \code{vendor}, \code{model}, and \code{type} are
single-line strings that give information on the vendor
(manufacturer), model, and the type of the device.  For consistency's
sake, the following strings should be used when appropriate (the lists
will be expanded as need arises):

\begin{table}[htbp]
  \begin{center}
    \leavevmode
    \hspace{\fill}
    \begin{tabular}[t]{|ll|}
      \hline
      \multicolumn{2}{|c|}{\bf \defn{Vendor Strings}} \\
      \hline\hline
      \code{AGFA}            & \code{Microtek} \\
      \code{Abaton}          & \code{Minolta} \\
      \code{Acer}            & \code{Mustek} \\
      \code{Apple}           & \code{NEC} \\
      \code{Artec}           & \code{Nikon} \\
      \code{Avision}         & \code{Plustek} \\
      \code{CANON}           & \code{Polaroid} \\
      \code{Connectix}       & \code{Ricoh} \\
      \code{Epson}           & \code{Sharp} \\
      \code{Fujitsu}         & \code{Siemens} \\
      \code{Hewlett-Packard} & \code{Tamarack} \\
      \code{IBM}             & \code{UMAX} \\
      \code{Kodak}           & \code{Noname} \\
      \code{Logitech}        & \\ 
      \hline
    \end{tabular}
    \hspace{\fill}
    \begin{tabular}[t]{|l|}
      \hline
      \multicolumn{1}{|c|}{\bf \defn{Type Strings}} \\
      \hline\hline
      \code{film scanner} \\
      \code{flatbed scanner} \\
      \code{frame grabber} \\
      \code{handheld scanner} \\
      \code{multi-function peripheral} \\
      \code{sheetfed scanner} \\
      \code{still camera} \\
      \code{video camera} \\
      \code{virtual device} \\
      \hline
    \end{tabular}
    \hspace{\fill}
    \caption{Predefined Device Information Strings}
    \label{tab:devinfo}
  \end{center}
\end{table}
Note that vendor string \code{Noname} can be used for virtual devices
that have no physical vendor associated.  Also, there are no
predefined model name strings since those are vendor specific and
therefore completely under control of the respective backends.

\begin{changebar}
The backend has to set up the string \code{email\_backend\_author}
with the name and the email address of the backend author or a contact person
in the format:
\begin{verbatim}
Firstname Lastname <name@domain.org>
\end{verbatim}

The string \code{backend\_website} should be set up by the backend
with the website or ftp address of the backend in the format:
\begin{verbatim}
http://www.domain.org/sane-hello/index.html
\end{verbatim}

The backend should fill the string \code{device\_location} with a
text that describes where a user can find this device. The text 
should be configurable by the administrator. This could e.g. look
like this:
\begin{verbatim}
building 93, 2nd plane, room 2124
\end{verbatim}

The string \code{comment} can be used to display any comment about the device
to the user. The text should be configurable by the administrator.

The string \code{reserved\_string} is unused currently but planned for
future use.

All unused strings must be set to ``'' by the backend.

With member \code{backend\_version\_code} a frontend can find
out the version of a backend it is connected to via one or more meta
backends.

The member \code{backend\_capability\_flags} contains 32 bits that
are planned to give the backend the chance to inform the frontend about
its capabilities. The meaning of the flags will be defined when there
is the need for it. The backend has to set all not defined bits
(in the moment all 32 bits) to 0.

The member \code{reserved\_int} is planned for future use, the backend has
to set the value of the integer to 0.
\end{changebar}

\subsection{Option Descriptor Type}\label{sec:odesc}

Option descriptors are at the same time the most intricate and
powerful type in the SANE standard.  Options are used to control
virtually all aspects of device operation.  Much of the power of the
SANE API stems from the fact that most device controls are completely
described by their respective option descriptor.  Thus, a frontend can
control a scanner abstractly, without requiring knowledge as to what
the purpose of any given option is.  Conversely, a scanner can
describe its controls without requiring knowledge of how the frontend
operates.  The C declaration of the
\code{\defn{SANE\_Option\_Descriptor}} type is given below.
\begin{quote}
\begin{verbatim}
typedef struct
  {
    SANE_String_Const name;
    SANE_String_Const title;
    SANE_String_Const desc;
    SANE_Value_Type type;
    SANE_Unit unit;
    SANE_Int size;
    SANE_Int cap;
    SANE_Constraint_Type constraint_type;
    union
      {
        const SANE_String_Const *string_list;
        const SANE_Word *word_list;
        const SANE_Range *range;
      }
    constraint;
  }
SANE_Option_Descriptor;
\end{verbatim}
\end{quote}

\subsubsection{Option Name}

Member \code{name} is a string that uniquely identifies the option.
The name must be unique for a given device (i.e., the option names
across different backends or devices need not be unique).  The option
name must consist of lower-case ASCII letters (\code{a}--\code{z}),
digits (\code{0}--\code{9}), or the dash character (\code{-}) only.
The first character must be a lower-case ASCII character (i.e., not a
digit or a dash).

\subsubsection{Option Title}

Member \code{title} is a single-line string that can be used by the
frontend as a title string.  This should typically be a short (one or
two-word) string that is chosen based on the function of the option.

\subsubsection{Option Description}

Member \code{desc} is a (potentially very) long string that can be
used as a help text to describe the option.  It is the responsibility
of the frontend to break the string into managable-length lines.
Newline characters in this string should be interpreted as paragraph
breaks.

\subsubsection{Option Value Type}

Member \code{type} specifies the type of the option value.  The
possible values for type \code{\defn{SANE\_Value\_Type}} are described
in Table \ref{tab:valuetype}.

\begin{table}[htbp]
  \begin{center}
    \leavevmode
    \begin{tabular}{|l|l|p{0.6\textwidth}|}
\hline
\multicolumn{1}{|c|}{\bf Symbol} &
\multicolumn{1}{c|}{\bf Code} &
\multicolumn{1}{c|}{\bf Description} \\
\hline\hline

\code{\defn{SANE\_TYPE\_BOOL}} & 0 & Option value is of type
  \verb|SANE_Bool|. \\ 

\code{\defn{SANE\_TYPE\_INT}} & 1 & Option value is of type
  \verb|SANE_Int|. \\ 

\code{\defn{SANE\_TYPE\_FIXED}}&2 & Option value is of type
  \verb|SANE_Fixed|. \\ 

\code{\defn{SANE\_TYPE\_STRING}}&3 & Option value is of type
  \verb|SANE_String|. \\ 

\code{\defn{SANE\_TYPE\_BUTTON}} & 4 & An option of this type has no value.
Instead, setting an option of this type has an option-specific
side-effect.  For example, a button-typed option could be used by a
backend to provide a means to select default values or to the tell an
automatic document feeder to advance to the next sheet of paper. \\ 

\code{\defn{SANE\_TYPE\_GROUP}} & 5 & An option of this type has no value.
This type is used to group logically related options.  A group option
is in effect up to the point where another group option is encountered
(or up to the end of the option list, if there are no other group
options).  For group options, only members \code{title} and
\code{type} are valid in the option descriptor. \\ 

  \hline
    \end{tabular}
    \caption{Option Value Types (\code{SANE\_Value\_Type})}
    \label{tab:valuetype}
  \end{center}
\end{table}

\subsubsection{Option Value Unit}

Member \code{unit} specifies what the physical unit of the option
value is.  The possible values for type \code{\defn{SANE\_U\-nit}} are
described in Table \ref{tab:units}.  Note that the specified unit is
what the SANE backend expects.  It is entirely up to a frontend as to
how these units a presented to the user.  For example, SANE expresses
all lengths in millimeters.  A frontend is generally expected to
provide appropriate conversion routines so that a user can express
quantities in a customary unit (e.g., inches or centimeters).

\begin{table}[htbp]
  \begin{center}
    \leavevmode
    \begin{tabular}{|l|l|l|}
\hline
\multicolumn{1}{|c|}{\bf Symbol} &
\multicolumn{1}{|c|}{\bf Code} &
\multicolumn{1}{|c|}{\bf Description} \\

\hline\hline

\code{\defn{SANE\_UNIT\_NONE}}   & 0 & Value is unit-less (e.g., page count).\\
\code{\defn{SANE\_UNIT\_PIXEL}}  & 1 & Value is in number of pixels. \\
\code{\defn{SANE\_UNIT\_BIT}}    & 2 & Value is in number of bits. \\
\code{\defn{SANE\_UNIT\_MM}}     & 3 & Value is in millimeters. \\
\code{\defn{SANE\_UNIT\_DPI}}    & 4 & Value is a resolution in dots/inch. \\
\code{\defn{SANE\_UNIT\_PERCENT}}& 5 & Value is a percentage. \\
\code{\defn{SANE\_UNIT\_MICROSECOND}}& 6 & Value is time in $\mu$-seconds. \\

\hline
    \end{tabular}
    \caption{Physical Units (\code{SANE\_Unit})}
    \label{tab:units}
  \end{center}
\end{table}

\subsubsection{Option Value Size}\label{sec:valuesize}

Member \code{size} specifies the size of the option value (in bytes).
This member has a slightly different interpretation depending on the
type of the option value:
\begin{quote}
  \begin{description}
  \item[\code{SANE\_TYPE\_STRING}:] The size is the maximum size of
    the string.  For the purpose of string size calcuations, the
    terminating \code{NUL} character is considered to be part of the
    string.  Note that the terminating \code{NUL} character must
    always be present in string option values.
  \item[\code{SANE\_TYPE\_INT}, \code{SANE\_TYPE\_FIXED}:] The size
    must be a positive integer multiple of the size of a
    \verb|SANE_Word|.  The option value is a vector of length
    \[ \code{size}/\code{sizeof(SANE\_Word)}. \]
  \item[\code{SANE\_TYPE\_BOOL}:] The size must be set to
    \code{sizeof(SANE\_Word)}.
  \item[\code{SANE\_TYPE\_BUTTON}, \code{SANE\_TYPE\_GROUP}:] The
    option size is ignored.
  \end{description}
\end{quote}

\subsubsection{Option Capabilities}

Member \code{cap} describes what capabilities the option posseses.
This is a bitset that is formed as the inclusive logical OR of the
capabilities described in Table \ref{tab:capabilities}.  The SANE API
provides the following to macros to test certain features of a given
capability bitset:
\begin{quote}
  \begin{description}

  \item[\code{\defn{SANE\_OPTION\_IS\_ACTIVE}(\var{cap})}:] This macro
    returns \code{SANE\_TRUE} if and only if the option with the
    capability set \var{cap} is currently active.

  \item[\code{\defn{SANE\_OPTION\_IS\_SETTABLE}(\var{cap})}:] This
    macro returns \code{SANE\_TRUE} if and only if the option with the
    capability set \var{cap} is software settable.
  \end{description}
\end{quote}

\begin{table}[htbp]
  \begin{center}
    \leavevmode
    \begin{tabular}{|l|r|p{0.59\textwidth}|}
\hline
\multicolumn{1}{|c|}{\bf Symbol} &
\multicolumn{1}{c|}{\bf Code} &
\multicolumn{1}{c|}{\bf Description} \\
\hline\hline

\code{\defn{SANE\_CAP\_SOFT\_SELECT}} & 1 & The option
  value can be set by a call to \code{sane\_con\-trol\_opt\-ion()}.\\ 

\code{\defn{SANE\_CAP\_HARD\_SELECT}} & 2 & The option value can be set by
  user-intervention (e.g., by flipping a switch).  The user-interface
  should prompt the user to execute the appropriate action to set such
  an option.  This capability is mutually exclusive with
  SANE\_CAP\_SOFT\_SELECT (either one of them can be set, but not both
  simultaneously).  \\ 

\code{\defn{SANE\_CAP\_SOFT\_DETECT}} & 4 & The option
  value can be detected by software.  If
  \code{SANE\_\-CAP\_\-SO\-FT\_SEL\-ECT} is set, this capability {\em must\/}
  be set.  If \code{SANE\_CAP\_HARD\_SELECT} is set, this capability
  may or may not be set.  If this capability is set but neither
  \code{SANE\_CAP\_SO\-FT\_SEL\-ECT} nor \code{SANE\_CAP\_HA\-RD\_SEL\-ECT}
  are, then there is no way to control the option.  That is, the
  option provides read-out of the current value only.  \\ 

\code{\defn{SANE\_CAP\_EMULATED}} & 8 & If set, this capability indicates
  that an option is not directly supported by the device and is
  instead emulated in the backend.  A sophisticated frontend may
  elect to use its own (presumably better) emulation in lieu of an emulated
  option. \\ 

\code{\defn{SANE\_CAP\_AUTOMATIC}} & 16 & If set, this capability indicates
  that the backend (or the device) is capable to picking a reasonable
  option value automatically.  For such options, it is possible to
  select automatic operation by calling \code{sane\_control\_option()}
  with an action value of \code{SANE\_ACTION\_SET\_AUTO}.  \\ 

\code{\defn{SANE\_CAP\_INACTIVE}} & 32 & If set, this capability indicates
  that the option is not currently active (e.g., because it's
  meaningful only if another option is set to some other value). \\ 

\code{\defn{SANE\_CAP\_ADVANCED}} & 64 &
  If set, this capability indicates that the option should be
  considered an ``advanced user option''. 
\begin{changebar}
  If this capability is set for an option of type
  \code{SANE\_TYPE\_GROUP}, all options belonging to the group are also
  advanced, even if they don't set the capabilty themselves. 
\end{changebar}
  A frontend typically
  displays such options in a less conspicuous way than regular options
  (e.g., a command line interface may list such options last or a
  graphical interface may make them available in a seperate ``advanced
  settings'' dialog). 
  \\

\code{\defn{SANE\_CAP\_HIDDEN}} & 128 &
\begin{changebar}
  If set, this capability indicates that the option shouldn't be
  displayed to and used by the user directly. Instead a hidden option
  is supposed to be automatically used by the frontend, like e.g. the
  ``preview'' option.
  If this capability is set for an option of type
  \code{SANE\_TYPE\_GROUP}, all options belonging to the group are also
  hidden, even if they don't set the capabilty themselves. 
  A frontend typically doesn't display such options by default but there
  should be a way to override this default behaviour.
\end{changebar}
  \\

\code{\defn{SANE\_CAP\_ALWAYS\_SETTABLE}} & 256 & 
\begin{changebar}
  If set, this capability indicates that the option may be at any time between
  \code{sane\_open()} and \code{sane\_close()}. I.e.  it's allowed to set it
  even while an image is acquired.
\end{changebar}
\\

\hline
    \end{tabular}
    \caption{Option Capabilities}
    \label{tab:capabilities}
  \end{center}
\end{table}

\subsubsection{Option Value Constraints}

It is often useful to constrain the values that an option can take.
For example, constraints can be used by a frontend to determine how to
represent a given option.  Member \code{constraint\_type} indicates
what constraint is in effect for the option.  The constrained values
that are allowed for the option are described by one of the union
members of member \code{constraint}.  The possible values of type
\code{\defn{SANE\_Constraint\_Type}} and the interpretation of the
\code{constraint} union is described in Table~\ref{tab:constraints}.

\begin{table}[htbp]
  \begin{center}
    \leavevmode
    \begin{tabular}{|l|r|p{0.5\textwidth}|}
\hline
\multicolumn{1}{|c|}{\bf Symbol} &
\multicolumn{1}{|c|}{\bf Code} &
\multicolumn{1}{|c|}{\bf Description} \\

\hline\hline

\code{\defn{SANE\_CONSTRAINT\_NONE}} & 0 & The value is unconstrained.
  The option can take any of the values possible for the option's
  type. \\

  \code{\defn{SANE\_CONSTRAINT\_RANGE}} & 1 & This constraint is
  applicable to integer and fixed-point valued options only.  It
  constrains the option value to a possibly quantized range of
  numbers.  Option descriptor member \code{constraint.range} points to
  a range of the type \code{\defn{SANE\_Range}}.  This type is illustrated
  below:
  \begin{quote}
\begin{verbatim}
typedef struct
  {
    SANE_Word min;
    SANE_Word max;
    SANE_Word quant;
  }
SANE_Range;
\end{verbatim}
  \end{quote}
  All three members in this structure are interpreted according to the
  option value type (\verb|SANE_TYPE_INT| or \verb|SANE_TYPE_FIXED|).
  Members \code{min} and \code{max} specify the minimum and maximum
  values, respectively.  If member \code{quant} is non-zero, it
  specifies the quantization value.  If $l$ is the minimum value, $u$
  the maximum value and $q$ the (non-zero) quantization of a range,
  then the legal values are $v=k\cdot q+l$ for all non-negative
  integer values of $k$ such that $v<=u$.  \\ 

\code{\defn{SANE\_CONSTRAINT\_WORD\_LIST}} & 2 & This constraint is applicable
  to integer and fixed-point valued options only.  It constrains the
  option value to a list of numeric values.  Option descriptor member
  \code{constraint.word\_list} points to a list of words that
  enumerates the legal values.  The first element in that list is an
  integer (\verb|SANE_Int|) that specifies the length of the list (not
  counting the length itself).  The remaining elements in the list are
  interpreted according to the type of the option value
  (\verb|SANE_TYPE_INT| or \verb|SANE_TYPE_FIXED|). \\ 

\code{\defn{SANE\_CONSTRAINT\_STRING\_LIST}} & 3 & This constraint is
  applicable to string-valued options only.  It constrains the option
  value to a list of strings.  The option descriptor member
  \code{con\-strai\-nt.str\-ing\_list} points to a \code{NULL} terminated
  list of strings that enumerate the legal values for the option
  value.
\\\hline
    \end{tabular}
    \caption{Option Value Constraints}
    \label{tab:constraints}
  \end{center}
\end{table}


\subsection{Internationalization}\label{sec:i18n}
\begin{changebar}
All backend texts should be written in english.
Localization (translation of backend texts) has to be done
in the frontend. To automatically prepare translation tables
(e.g. english to german) it is necessary to mark the texts
that shall be translated.

\subsubsection{How is a text marked for translation}

The keyword \code{SANE\_I18N} is used to mark a text for translation.
\code{SANE\_I18N} has to be defined as a dummy prototype:

  \begin{quote}
\begin{verbatim}
#ifndef SANE_I18N
#  define SANE_I18N(text) text
#endif
\end{verbatim}
  \end{quote}

You should not mark prototypes or variables with \code{SANE\_I18N}
because it is not possible (or very hard) to find out the
texts that are covered by prototypes (\verb|SANE_I18N(START_SCAN_TEXT)|) and
variables (\verb|SANE_I18N(option[7].name)|).

A correct marked text can look like this:\\
\verb|snprintf(buf, sizeof(buf), SANE_I18N("Start scan")|\\
or\\
\verb|#define START_SCAN_TEXT SANE_I18N("Start scan")|.\\
It also is allowed to mark texts in structs because the
prototype \code{SANE\_I18N} has no effect to the compiler.

\subsubsection{Which texts shall be marked for translation?}
All option texts that are visible for the user should be marked for
translation. This is:
\begin{itemize}
\item
member \code{title}
\item
member \code{desc}
\item
member \code{string\_list}
\end{itemize}
of \code{SANE\_Option\_Descriptor}.

It is not allowed to mark/translate member \code{name}.
Please also do not mark any error or debug messages that are
displayed by the backend.

\subsubsection{File formats and translation functions}
The recommended file formats for translation tables are the \code{po} files
and \code{mo} or \code{gmo} files.
The po file contains the original text marked with the keyword \code{msgid}
and the translated text marked with the keyword \code{msgstr}.
A \code{po} file that contains all needed \code{msgid}s can be created
e.g. by the gnu gettext tool \verb|xgettext| with the option
\verb|-k SANE_I18N|.
The translator adds the translated texts to the \code{po} files.
The gettext tool \code{msgfmt} converts the \code{po} files to the
\code{mo} or \code{gmo} files.\\

Translation is done in the frontend. A backend has nothing
to do with the translation of texts.
The frontend should use the function \code{gettext}
to translate the texts. This e.g. can look like this:\\
\verb|snprintf(buf, sizeof(buf), gettext("english text"));|

If a frontend author decides to use translation functions that
need different translation tables, then the frontend is
responsible to create/convert the translation tables.
In this case it should use the \code{po} files to create its
own translation tables from it.\\

Note that it is strongly recommended to use the gettext tools.
\end{changebar}


\section{Operations}

\subsection{\code{sane\_init}}

This function must be called before any other SANE function can be called.
The behavior of a SANE backend is undefined if this function is not called
first or if the status code returned by \code{sane\_init} is different from
\code{\defn{SANE\_STATUS\_GOOD}}.  The version code of the backend is returned
in the value pointed to by \code{version\_code}.  If that pointer is
\code{NULL}, no version code is returned.  Argument \code{authorize} is either
a pointer to a function that is invoked when the backend requires
authentication for a specific resource or \code{NULL} if the frontend does not
support authentication.
\begin{quote}\index{sane\_init}
\begin{verbatim}
SANE_Status sane_init (SANE_Int * version_code,
                       SANE_Authorization_Callback authorize);
\end{verbatim}
\end{quote}

The authorization function may be called by a backend in response to
any of the following calls:
\begin{quote}
  \code{sane\_open}, \code{sane\_control\_option}, \code{sane\_start}
\end{quote}
If a backend was initialized without authorization function, then
authorization requests that cannot be handled by the backend itself
will fail automatically and the user may be prevented from accessing
protected resources.  Backends are encouraged to implement means of
authentication that do not require user assistance.  E.g., on a
multi-user system that authenticates users through a login process a
backend could automatically lookup the apporpriate password based on
resource- and user-name.

The authentication function type has the following declaration:
\begin{quote}\index{SANE\_Authorization\_Callback}
  \index{domain}\index{username}\index{password}
\begin{verbatim}
#define SANE_MAX_USERNAME_LEN   128
#define SANE_MAX_PASSWORD_LEN   128

typedef void (*SANE_Authorization_Callback)
    (SANE_String_Const resource,
     SANE_Char username[SANE_MAX_USERNAME_LEN],
     SANE_Char password[SANE_MAX_PASSWORD_LEN]);
\end{verbatim}
\end{quote}
Three arguments are passed to the authorization function:
\code{resource} is a string specifying the name of the resource that
requires authorization.  A frontend should use this string to build a
user-prompt requesting a username and a password.  The \code{username}
and \code{password} arguments are (pointers to) an array of
\code{SANE\_MAX\_USERNAME\_LEN} and \code{SANE\_MAX\_PASSWORD\_LEN}
characters, respectively.  The authorization call should place the
entered username and password in these arrays.  The returned strings
{\em must\/} be ASCII-NUL terminated.

\subsection{\code{sane\_exit}}

This function must be called to terminate use of a backend.  The
function will first close all device handles that still might be open
(it is recommended to close device handles explicitly through a call
to \code{sane\_clo\-se()}, but backends are required to release all
resources upon a call to this function).  After this function returns,
no function other than \code{sane\_init()} may be called (regardless
of the status value returned by \code{sane\_exit()}.  Neglecting to
call this function may result in some resources not being released
properly.
\begin{quote}\index{sane\_exit}
\begin{verbatim}
void sane_exit (void);
\end{verbatim}
\end{quote}


\subsection{\code{sane\_get\_devices}}

This function can be used to query the list of devices that are
available.  If the function executes successfully, it stores a pointer
to a \code{NULL} terminated array of pointers to \verb|SANE_Device|
structures in \code{*device\_list}.  The returned list is guaranteed
to remain unchanged and valid until (a) another call to this function
is performed or (b) a call to \code{sane\_exit()} is performed.  This
function can be called repeatedly to detect when new devices become
available.  If argument \code{local\_only} is true, only local devices
are returned (devices directly attached to the machine that SANE is
running on).  If it is false, the device list includes all remote
devices that are accessible to the SANE library.
\begin{quote}\index{sane\_get\_devices}
\begin{verbatim}
SANE_Status sane_get_devices (const SANE_Device *** device_list,
                              SANE_Bool local_only);
\end{verbatim}
\end{quote}

This function may fail with \code{SANE\_STATUS\_NO\_MEM} if an
insufficient amount of memory is available.

\begin{quote}
  \begin{center}
    {\bf Backend Implementation Note}
  \end{center}
  SANE does not require that this function is called before a
  \code{sane\_open()} call is performed.  A device name may be
  specified explicitly by a user which would make it unnecessary and
  undesirable to call this function first.
\begin{changebar}
  The same information about
  a device has to be returned when \code{sane\_open} is called.
\end{changebar}
\end{quote}


\subsection{\code{sane\_open}}

This function is used to establish a connection to a particular
device.  The name of the device to be opened is passed in argument
\code{name}.  If the call completes successfully, a handle for the
device is returned in \code{*h}.
\begin{changebar}
The description of the device
is returned in \code{**device\_description}. This is the
same description that is returned in the list by \code{sane\_get\_devices}.
The returned data \code{*h} and \code{*device\_description} is guaranteed
to remain unchanged and valid until a call to \code{sane\_close()}
is performed.
\end{changebar}
As a special case, specifying a zero-length string as the device requests
opening the first available device (if there is such a device).
\begin{quote}\index{sane\_open}
\begin{verbatim}
SANE_Status sane_open (SANE_String_Const name, SANE_Handle * h,
const SANE_Device ** device_description);
\end{verbatim}
\end{quote}

This function may fail with one of the following status codes.
\begin{quote}
\begin{description}
\item[\code{SANE\_STATUS\_DEVICE\_BUSY}:] The device is currently
  busy (in use by somebody else).
\item[\code{SANE\_STATUS\_INVAL}:] The device name is not valid.
\item[\code{SANE\_STATUS\_IO\_ERROR}:] An error occured while
  communicating with the device.
\item[\code{SANE\_STATUS\_NO\_MEM}:] An insufficent amount of memory
  is available.
\item[\code{SANE\_STATUS\_ACCESS\_DENIED}:] Access to the device has
  been denied due to insufficient or invalid authentication.
\end{description}
\end{quote}


\subsection{\code{sane\_close}}

This function terminates the association between the device handle
passed in argument \code{h} and the device it represents.  If the
device is presently active, a call to \code{sane\_cancel()} is
performed first.  After this function returns, handle \code{h} must
not be used anymore.

\begin{quote}\index{sane\_close}
\begin{verbatim}
void sane_close (SANE_Handle h);
\end{verbatim}
\end{quote}

\subsection{\code{sane\_get\_option\_descriptor}}

This function is used to access option descriptors.  The function
returns the option descriptor for option number \code{n} of the device
represented by handle \code{h}.  Option number 0 is guaranteed to be a
valid option.  Its value is an integer that specifies the number of
options that are available for device handle \code{h} (the count
includes option 0).  If $n$ is not a valid option index, the function
returns \code{NULL}.  The returned option descriptor is guaranteed to
remain valid (and at the returned address) until the device is closed.

\begin{quote}\index{sane\_get\_option\_descriptor}
\begin{verbatim}
const SANE_Option_Descriptor *
    sane_get_option_descriptor (SANE_Handle h, SANE_Int n);
\end{verbatim}
\end{quote}

\subsection{\code{sane\_control\_option}}\label{sec:control}

This function is used to set or inquire the current value of option
number \code{n} of the device represented by handle \code{h}.  The
manner in which the option is controlled is specified by parameter
\code{a}.  The possible values of this parameter are described in more
detail below.  The value of the option is passed through argument
\code{v}.  It is a pointer to the memory that holds the option value.
The memory area pointed to by \code{v} must be big enough to hold the
entire option value (determined by member \code{size} in the
corresponding option descriptor).  The only exception to this rule is
that when setting the value of a string option, the string pointed to
by argument \code{v} may be shorter since the backend will stop
reading the option value upon encountering the first \code{NUL}
terminator in the string.  If argument \code{i} is not \code{NULL},
the value of \code{*i} will be set to provide details on how well the
request has been met.  The meaning of this argument is described in
more detail below.
\begin{quote}\index{sane\_control\_option}
\begin{verbatim}
SANE_Status sane_control_option (SANE_Handle h, SANE_Int n,
                                 SANE_Action a, void *v,
                                 SANE_Int * i);
\end{verbatim}
\end{quote}

The way the option is affected by a call to this function is
controlled by parameter \code{a} which is a value of type
\code{\defn{SANE\_Action}}.  The possible values and their meaning is
described in Table~\ref{tab:actions}.

\begin{table}[h]
  \begin{center}
    \leavevmode
    \begin{tabular}{|l|r|p{0.5\textwidth}|}
\hline
\multicolumn{1}{|c|}{\bf Symbol} &
\multicolumn{1}{|c|}{\bf Code} &
\multicolumn{1}{|c|}{\bf Description} \\

\hline\hline

\code{\defn{SANE\_ACTION\_GET\_VALUE}} & 0 & Get current option value. \\

\code{\defn{SANE\_ACTION\_SET\_VALUE}} & 1 & Set option value.  The
  option value passed through argument \code{v} may be modified by the
  backend if the value cannot be set exactly.  \\ 

\code{\defn{SANE\_ACTION\_SET\_AUTO}} & 2 & Turn on automatic mode.  Backend
  or device will automatically select an appropriate value.  This mode
  remains effective until overridden by an explicit set value request.
  The value of parameter \code{v} is completely ignored in this case and
  may be \code{NULL}.  \\

\hline
    \end{tabular}
    \caption{Action Values (\code{SANE\_Action})}
    \label{tab:actions}
  \end{center}
\end{table}

After setting a value via an action value of
\verb|SANE_ACTION_SET_VALUE|, additional information on how well the
request has been met is returned in \code{*i} (if \code{i} is
non-\code{NULL}).  The returned value is a bitset that may contain any
combination of the values described in Table~\ref{tab:info}.
\begin{table}[htbp]
  \begin{center}
    \leavevmode
    \begin{tabular}{|l|r|p{0.5\textwidth}|}
\hline
\multicolumn{1}{|c|}{\bf Symbol} &
\multicolumn{1}{|c|}{\bf Code} &
\multicolumn{1}{|c|}{\bf Description} \\

\hline\hline

\code{\defn{SANE\_INFO\_INEXACT}} & 1 & This value is returned when
  setting an option value resulted in a value being selected that does
  not exactly match the requested value.  For example, if a scanner
  can adjust the resolution in increments of 30dpi only, setting the
  resolution to 307dpi may result in an actual setting of 300dpi.
  When this happens, the bitset returned in \code{*i} has this member
  set.  In addition, the option value is modified to reflect the
  actual (rounded) value that was used by the backend.  Note that
  inexact values are admissible for strings as well.  A backend may
  choose to ``round'' a string to the closest matching legal string
  for a constrained string value. \\

\code{\defn{SANE\_INFO\_RELOAD\_OPTIONS}} & 2 & 
The setting of an option may affect the
\begin{changebar}
  value, the availability, or the constraint of one or more {\em
   other\/} options.
\end{changebar}
  When this happens, the SANE backend sets this
  member in \code{*i} to indicate that the application should reload
  all options.  This member may be set if and only if at least one
  option changed. \\

\code{\defn{SANE\_INFO\_RELOAD\_PARAMS}} & 4 &
The setting of an option may affect the parameter values (see
\code{sane\_get\_parameters()}).
If setting an option affects the parameter values, this member will
be set in \code{*i}.  Note that this member may be set even if the
parameters did not actually change.  However, it is guaranteed that
the parameters never change without this member being set.
\\

\code{\defn{SANE\_INFO\_INVALIDATE\_PREVIEW}} & 8 &
The setting of an option may affect the validity of the preview that 
was acquired by the frontend earlier. When the preview image would change
significantly if it was scanned again, the SANE backend
sets this member in \code{*i} to indicate that the application should inform the
user of the invalidity of the preview. Examples for such option
settings may include setting a different scan source or significantly
changing the exposure.
\\


\hline
    \end{tabular}
    \caption{Additional Information Returned When Setting an Option}
    \label{tab:info}
  \end{center}
\end{table}

This function may fail with one of the following status codes.
\begin{quote}
\begin{description}
\item[\code{SANE\_STATUS\_UNSUPPORTED}:] The operation is not
  supported for the specified handle and option number.
\item[\code{SANE\_STATUS\_INVAL}:] The option value is not valid.
\item[\code{SANE\_STATUS\_IO\_ERROR}:] An error occured while
  communicating with the device.
\item[\code{SANE\_STATUS\_NO\_MEM}:] An insufficent amount of memory
  is available.
\item[\code{SANE\_STATUS\_ACCESS\_DENIED}:] Access to the option has
  been denied due to insufficient or invalid authentication.
\end{description}
\end{quote}



\subsection{\code{sane\_get\_parameters}\label{sec:sanegetparameters}}

This function is used to obtain the current scan parameters.  The
returned parameters are guaranteed to be accurate between the time a
scan has been started (\code{sane\_start()} has been called) and the
completion of that request.  Outside of that window, the returned
values are best-effort estimates of what the parameters will be when
\code{sane\_start()} gets invoked.  Calling this function before a
scan has actually started allows, for example, to get an estimate of
how big the scanned image will be.  The parameters passed to this
function are the handle \code{h} of the device for which the
parameters should be obtained and a pointer \code{p} to a parameter
structure.  The parameter structure is described in more detail below.

\begin{quote}\index{sane\_get\_parameters}
\begin{verbatim}
SANE_Status sane_get_parameters (SANE_Handle h,
                                 SANE_Parameters * p);
\end{verbatim}
\end{quote}

The scan parameters are returned in a structure of type
\code{\defn{SANE\_Parameters}}.  The C declaration of this structure
is given below.
\begin{changebar}
\begin{quote}
\begin{verbatim}
typedef struct
  {
    SANE_Frame format;
    SANE_Int flags;
    SANE_Int lines;
    SANE_Int depth;
    SANE_Int pixels_per_line;
    SANE_Int bytes_per_line;
    SANE_Int channels_per_image;
    SANE_String format_desc;
    SANE_String proposed_filename;
    SANE_String proposed_comment;
    SANE_Int dpi_x;
    SANE_Int dpi_y;
    char reserved[32]; /* 32 bytes for future use */
  }
SANE_Parameters;
\end{verbatim}
\end{quote}
\end{changebar}

\begin{changebar}
Member \code{format} specifies the format of the next frame to be
returned.  The possible values for type \code{\defn{SANE\_Frame}} are
described in Table~\ref{tab:frameformat}.  The meaning of these
values is described in more detail in section~\ref{sec:imageformat}.
The frame types used by version~1 of this standard
(\code{\defn{SANE\_FRAME\_GRAY}}, \code{\defn{SANE\_FRAME\_RGB}},
\code{\defn{SANE\_FRAME\_RED}}, \code{\defn{SANE\_FRAME\_GREEN}}, and
\code{\defn{SANE\_FRAME\_BLUE}}) are obsolete and superseded by the 
frame type \code{\defn{SANE\_FRAME\_RAW}}.

\begin{table}[htbp]
  \begin{center}
    \leavevmode
    \begin{tabular}{|l|r|l|}
\hline
\multicolumn{1}{|c|}{\bf Symbol} &
\multicolumn{1}{|c|}{\bf Code} &
\multicolumn{1}{|c|}{\bf Description} \\

\hline\hline

\code{\defn{SANE\_FRAME\_RAW}}   & 5 & Arbitrary pixel property transmission. \\
\code{\defn{SANE\_FRAME\_MIME}}  & 6 & Data described by a mime descriptor. \\

\hline
    \end{tabular}
    \caption{Frame Format (\code{SANE\_Frame})}
    \label{tab:frameformat}
  \end{center}
\end{table}

The \code{flags} member is a 32 bit bitfield, for which up to now 4 
informational bits are defined, all unused bits have to be set to 0: 

\begin{itemize}

\item
\code{SANE\_PFLAG\_LAST\_FRAME} (bit 0, bitvalue 1) is set to 1 if and
only if the frame that is currently being acquired (or the frame that
will be acquired next if there is no current frame) is the only frame
or the last frame
of a multi frame image (e.g., the current frame is the blue component
of a red, green, blue image).

\item
\code{SANE\_PFLAG\_MORE\_IMAGES} (bit 1, bitvalue 2) is set to 1 to indicate 
further pending images. The frontend is expected to call \code{sane\_start}
again after the end of the current scan to get more images, e.g.
from an automatic document feeder.  It is permissible to set that value to 1 ``in good 
faith'', as it has to be determined at a very early time, where it might 
not be detectable, if there actually are more images to transfer. E.g. 
you will usually not know if the document feeder contains further pages 
when starting to scan the current one. Thus you are allowed to set that 
bit but later fail at \code{sane\_start()}.

\item
\code{SANE\_PFLAG\_NEW\_PAGE} (bit 2, bitvalue 4) is set to 1 to indicate 
that the current frame comes from a new physical page. This bit is of 
informational character only to help frontends to group multi-image 
scans. 

\item
\code{SANE\_PFLAG\_BACKSIDE} (bit 3, bitvalue 8) tells if the current image 
was acquired from the front (0) or backside (1) of the currently processed 
sheet. It is of informational character and allows to group and order 
multi-image transfers regardless of scanner acquisition order (front 
first/back first). 

\end{itemize}

Note, that \code{flags} is compatible to member \code{last\_frame} of 
\code{SANE\_Parameters} of SANE standard version 1 (same size
and only bit 0 (bitvalue 1) was used with same function).
\end{changebar}

Member \code{lines} specifies how many scan lines the frame is
comprised of.  If this value is -1, the number of lines is not known a
priori and the frontend should call \code{sane\_read()} until it
returns a status of \code{SANE\_STATUS\_EOF}.
\begin{changebar}
Note, that even when transferring formats that have this information 
inband, it is recommended to set that member, if available. If 
unavailable or not applicable, set to -1 as mentioned above. 
\end{changebar}

Member \code{bytes\_per\_line} specifies the number of bytes that comprise one
scan line.
\begin{changebar}
If this value is not applicable for image data of type
\code{SANE\_FRAME\_MIME}, it must be set to -1. In this case the frontend
should just call \code{sane\_read} until \code{SANE\_STATUS\_EOF} is returned.
\end{changebar}

Member \code{depth} specifies the number of bits per sample.
\begin{changebar}
Note, that only -1 (for not applicable), 1, and multiples of 8 are
allowed values. Data with other depths has to be scaled up accordingly.
Depth 1 is only allowed if the image consists of a single channel (lineart or
halftone modes).
\end{changebar}

Member \code{pixels\_per\_line} specifies the number of pixels that
comprise one scan line.
\begin{changebar}
If unavailable or not applicable, set to -1.
\end{changebar}

Assume $B$ is the number of channels in the frame, then the bit depth
$d$ (as given by member \code{depth}) and the number of pixels per
line $n$ (as given by this member \code{pixels\_per\_line}) are
related to $c$, the number of bytes per line (as given by member
\code{bytes\_per\_line}) as follows:
\begin{changebar}
\[
  c >= \left\{
  \begin{array}{ll}
    \lceil B\cdot n / 8\rceil & \mbox{if $d=1$}\\
    B\cdot n \cdot d / 8 & \mbox{if $d>1$}
  \end{array}
  \right.
\]
\end{changebar}
Note that the number of bytes per line can be larger than the minimum
value imposed by the right side of this equation.  A frontend must be
able to properly cope with such ``padded'' image formats.

\begin{changebar}
Member \code{channels\_per\_image} specifies the number of channels the
image consists of. When the image is transmitted in more than one frame
\code{channels\_per\_image} has to be the same for all frames for this image.

Member \code{format\_desc} is used to describe the details of the frame
formats.  Its meaning differs between the two formats:

\begin{itemize}
\item
\code{SANE\_FRAME\_RAW}: 
The \code{format\_desc} contains a description of the channel data and
an optional depth information separated by a colon(:).

A plane is descibed by one channel, e.g. ``\code{gray}'' or ``\code{gray:12}''.

Channel interleaved data is described by a comma separated list of channel descriptions,
for example ``\code{red,green,blue}'' or ``\code{red:8,green:8,blue:8}'',
the channel data is sent in the given order.

The depth information does {\bf not} define the size of the transmitted
channel data, it is only an information for the frontend. The channel data has
to be sent in the size defined by member \code{depth}.

Well known channels are \code{red}, \code{green}, \code{blue} and \code{gray}.
It also is allowed to use other channel descriptions, e.g. if you
use an infrared camera or scanner it could be \code{infrared} or
a wavelength description like \code{1100nm}, but be aware that a
frontend may not be able to display such channels with useful colors.

Note that an image can be sent in single planes, in one interleaved
frame that contains all channels or in several frames that contain
one or more (interleaved) channels. When an RGB image is sent it
is prefered to send the image data in one interleaved frame
that consist of red, green and blue data in this order.
The number of channels is defined in member \code{channels\_per\_image}.

\item
\code{SANE\_FRAME\_MIME}: 

\code{format\_desc} contains the MIME Content-Type: header field as described
in RFC 1521 (section 4) without the prefixing "Content-Type:". MIME Types and
subtypes should be either chosen from the RFC or from the list of
IANA-approved values. The data stream must be compliant with the corresponding
specification.

When data is transmitted with the frame type \code{SANE\_FRAME\_MIME}
all data has to be transmitted within one frame, multiple frames
are not allowed (so the flag \code{last\_frame} has to be set
when using this frame type).

A SANE backend must be able to at least optionally transmit
\code{SANE\_FRAME\_RAW} (possibly with the help of a meta backend), if the
hardware supports delivering image data at all. For data that doesn't
comprise images, it's admisable to only provide MIME frames. As a
general principle, if there are several choices in MIME types, the
format that is most widely implemented should be used.
\end{itemize}

The member \code{proposed\_filename} can be used to suggest a reasonable 
default filename or -extension in case the backend can make such a 
suggestion, like e.g. an image database. 
If no such suggestion is intended, set the field to ``''. 

In the case of raw frames, \code{proposed\_filename} is expected to hold
the basename for the image, with the extension determined by the save function
of the frontend, as the frontend can fully understand the data and is thus
able to encode it in any format it wishes.

For MIME frames \code{proposed\_filename} can contain either: 

\begin{itemize}
\item
A name with a leading dot, which is considered to be a proposed 
filename extension. This could also be gotten from the mime database, 
but for systems lacking it, this might be convenient. Or: 

\item
A complete filename, including extension. 
\end{itemize}

Note, that for frontends that are able to parse a given MIME type 
internally, it is perfectly permissible to ignore the extension 
part of the proposed filename and only make use of the basename, 
when using internal save algorithms for different formats. 

The string \code{proposed\_comment} can be used to transmit additional 
image information, that can be stored in the comment areas several file formats 
offer. It can contain any textual information the backend wishes to 
convey to the user, like date/time of exposure, enganged filters, 
etc. Set to ``'' if unused.

The members \code{dpi\_x} and \code{dpi\_y} encode the horizontal and vertical
resolution. Note, that multiple-image scans may have different resolutions of
each image. It is not permissible to change resolution between frames of the
same image. If not applicable, set to -1.

The member \code{reserved} is an array of 32 bytes (char) to keep
the size of the struct unchanged when future extensions are done.
The backend has to set the reserved bytes to 0.
\end{changebar}


\subsection{\code{sane\_start}}

This function initiates aquisition of an image from the device
represented by handle \code{h}.
\begin{quote}\index{sane\_start}
\begin{verbatim}
SANE_Status sane_start (SANE_Handle h);
\end{verbatim}
\end{quote}
This function may fail with one of the following status codes.
\begin{quote}
\begin{description}
\item[\code{SANE\_STATUS\_CANCELLED}:] The operation was cancelled through
  a call to \code{sane\_cancel}.
\item[\code{SANE\_STATUS\_DEVICE\_BUSY}:] The device is busy. The
  operation should be retried later.
\item[\code{SANE\_STATUS\_JAMMED}:] The document feeder is jammed.
\item[\code{SANE\_STATUS\_NO\_DOCS}:] The document feeder is out of
  documents.
\item[\code{SANE\_STATUS\_COVER\_OPEN}:] The scanner cover is open.
\item[\code{SANE\_STATUS\_IO\_ERROR}:] An error occurred while communicating
  with the device.
\item[\code{SANE\_STATUS\_NO\_MEM}:] An insufficent amount of memory
  is available.
\item[\code{SANE\_STATUS\_INVAL}:] The scan cannot be started with the current
  set of options. The frontend should reload the option descriptors, as if
  \code{\defn{SANE\_INFO\_RELOAD\_OPTIONS}} had been returned from a call to
  \code{sane\_control\_option()}, since the device's capabilities may have
  changed.
\end{description}
\end{quote}


\subsection{\code{sane\_read}}

This function is used to read image data from the device represented
by handle \code{h}.  Argument \code{buf} is a pointer to a memory area
that is at least \code{maxlen} bytes long.  The number of bytes
returned is stored in \code{*len}.  A backend must set this to zero
when a status other than \code{SANE\_STA\-TUS\_GOOD} is returned).
When the call succeeds, the number of bytes returned can be anywhere in
the range from 0 to \code{maxlen} bytes.
\begin{quote}\index{sane\_read}
\begin{verbatim}
SANE_Status sane_read (SANE_Handle h, SANE_Byte * buf,
                       SANE_Int maxlen, SANE_Int * len);
\end{verbatim}
\end{quote}
\begin{changebar}
For efficiency reasons, medium to large 
block sizes (in the range of a few kilobytes) should be used. 
Returning short reads is allowed to allow for small buffers 
in the backend. 
\end{changebar}
If this function is called when no data is available, one of two
things may happen, depending on the I/O mode that is in effect for
handle \code{h}.
\begin{enumerate}
\item If the device is in blocking I/O mode (the default mode), the
  call blocks until at least one data byte is available (or until some
  error occurs).

\item If the device is in non-blocking I/O mode, the call returns
  immediately with status \code{SANE\_STA\-TUS\_GOOD} and with
  \code{*len} set to zero.
\end{enumerate}
The I/O mode of handle \code{h} can be set via a call to
\code{sane\_set\_io\_mode()}.

This function may fail with one of the following status codes.
\begin{quote}
\begin{description}
\item[\code{SANE\_STATUS\_CANCELLED}:] The operation was cancelled through
  a call to \code{sane\_cancel}.
\item[\code{SANE\_STATUS\_EOF}:] No more data is available for the
  current frame.
\begin{changebar}
If \code{sane\_read} sends back any image data it
is not allowed to return with \code{SANE\_STATUS\_EOF}.
\end{changebar}
\item[\code{SANE\_STATUS\_JAMMED}:] The document feeder is jammed.
\item[\code{SANE\_STATUS\_NO\_DOCS}:] The document feeder is out of
  documents.
\item[\code{SANE\_STATUS\_COVER\_OPEN}:] The scanner cover is open.
\item[\code{SANE\_STATUS\_IO\_ERROR}:] An error occurred while communicating
  with the device.
\item[\code{SANE\_STATUS\_NO\_MEM}:] An insufficent amount of memory
  is available.
\item[\code{SANE\_STATUS\_ACCESS\_DENIED}:] Access to the device has
  been denied due to insufficient or invalid authentication.
\end{description}
\end{quote}


\subsection{\code{sane\_cancel}}

This function is used to immediately or as quickly as possible cancel
the currently pending operation of the device represented by handle
\code{h}.
\begin{quote}\index{sane\_cancel}
\begin{verbatim}
void sane_cancel (SANE_Handle h);
\end{verbatim}
\end{quote}
This function can be called at any time (as long as handle \code{h} is
a valid handle) but usually affects long-running operations only (such
as image is acquisition).  It is safe to call this function
asynchronously (e.g., from within a signal handler).  It is important
to note that completion of this operaton does {\em not\/} imply that
the currently pending operation has been cancelled.  It only
guarantees that cancellation has been {\em initiated}.  Cancellation
completes only when the cancelled call returns (typically with a
status value of \code{SANE\_STATUS\_CANCELLED}).  Since the SANE API
does not require any other operations to be re-entrant, this implies
that a frontend must {\em not\/} call any other operation until the
cancelled operation has returned.


\subsection{\code{sane\_set\_io\_mode}}

This function is used to set the I/O mode of handle \code{h}.  The I/O mode
can be either blocking or non-blocking.  If argument \code{m} is
\code{SANE\_TRUE}, the mode is set to non-blocking mode, otherwise it's set to
blocking mode.  This function can be called only after a call to
\code{sane\_start()} has been performed.
\begin{quote}\index{sane\_set\_io\_mode}
\begin{verbatim}
SANE_Status sane_set_io_mode (SANE_Handle h, SANE_Bool m);
\end{verbatim}
\end{quote}
By default, newly opened handles operate in blocking mode.  A backend
may elect not to support non-blocking I/O mode.  In such a case the
status value \code{SANE\_STATUS\_UNSUPPORTED} is returned.  Blocking
I/O must be supported by all backends, so calling this function with
argument \code{m} set to \code{SANE\_FALSE} is guaranteed to complete
successfully.

This function may fail with one of the following status codes:
\begin{quote}
\begin{description}
\item[\code{SANE\_STATUS\_INVAL}:] No image acquisition is pending.
\item[\code{SANE\_STATUS\_UNSUPPORTED}:] The backend does not support
the requested I/O mode.
\end{description}
\end{quote}


\subsection{\code{sane\_get\_select\_fd}}

This function is used to obtain a (platform-specific) file-descriptor
for handle \code{h} that is readable if and only if image data is
available (i.e., when a call to \code{sane\_read()} will return at
least one byte of data).  If the call completes successfully, the
select file-descriptor is returned in \code{*fd}.
\begin{quote}\index{sane\_get\_select\_fd}
\begin{verbatim}
SANE_Status sane_get_select_fd (SANE_Handle h, SANE_Int *fd);
\end{verbatim}
\end{quote}
This function can be called only after a call to \code{sane\_start()}
has been performed and the returned file-descriptor is guaranteed to
remain valid for the duration of the current image acquisition (i.e.,
until \code{sane\_cancel()} or \code{sane\_start()} get called again
or until \code{sane\_read()} returns with status
\code{SANE\_STA\-TUS\_EOF}).  Indeed, a backend must guarantee to
close the returned select file descriptor at the point when the next
\code{sane\_read()} call would return \code{SANE\_STA\-TUS\_EOF}.
This is necessary to ensure the application can detect when this
condition occurs without actually having to call \code{sane\_read()}.

A backend may elect not to support this operation.  In such a case,
the function returns with status code
\code{SANE\_STATUS\_UNSUPPORTED}.

Note that the only operation supported by the returned file-descriptor
is a host operating-system dependent test whether the file-descriptor
is readable (e.g., this test can be implemented using \code{select()}
or \code{poll()} under UNIX).  If any other operation is performed on
the file descriptor, the behavior of the backend becomes
unpredictable.  Once the file-descriptor signals ``readable'' status,
it will remain in that state until a call to \code{sane\_read()} is
performed.  Since many input devices are very slow, support for this
operation is strongly encouraged as it permits an application to do
other work while image acquisition is in progress.

This function may fail with one of the following status codes:
\begin{quote}
\begin{description}
\item[\code{SANE\_STATUS\_INVAL}:] No image acquisition is pending.
\item[\code{SANE\_STATUS\_UNSUPPORTED}:] The backend does not support
  this operation.
\end{description}
\end{quote}


\subsection{\code{sane\_strstatus}}

This function can be used to translate a SANE status code into a
printable string.  The returned string is a single line of text that
forms a complete sentence, but without the trailing period
(full-stop).  The function is guaranteed to never return \code{NULL}.
The returned pointer is valid at least until the next call to this
function is performed.
\begin{quote}\index{sane\_strstatus}
\begin{verbatim}
SANE_String_Const sane_strstatus (SANE_Status status);
\end{verbatim}
\end{quote}

\section{Code Flow}\index{code flow}

The code flow for the SANE API is illustrated in
Figure~\ref{fig:flow}.  Functions \code{sane\_init()} and
\code{sane\_exit()} initialize and exit the backend, respectively.
All other calls must be performed after initialization and before
exiting the backend.

\begin{figure}[htb]
  \begin{center}
    \leavevmode
    \psfig{file=figs/flow.eps,height=0.5\textheight}
    \caption{Code flow}
    \label{fig:flow}
  \end{center}
\end{figure}

Function \code{sane\_get\_devices()} can be called any time after
\code{sane\_init()} has been called.  It returns the list of the
devices that are known at the time of the call.  This list may change
over time since some devices may be turned on or off or a remote host
may boot or shutdown between different calls.  It should be noted that
this operation may be relatively slow since it requires contacting all
configured devices (some of which may be on remote hosts).  A frontend
may therefore want to provide the ability for a user to directly
select a desired device without requiring a call to this function.

Once a device has been chosen, it is opened using a call to
\code{sane\_open()}.  Multiple devices can be open at any given time.
A SANE backend must not impose artificial constraints on how many
devices can be open at any given time.

An opened device can be setup through the corresponding device handle
using functions \code{sane\_get\_opt\-ion\_desc\-riptor()} and
\code{sane\_control\_option()}.  While setting up a device, obtaining
option descriptors and setting and reading of option values can be
mixed freely.  It is typical for a frontend to read out all available
options at the beginning and then build a dialog (either graphical or
a command-line oriented option list) that allows to control the
available options.  It should be noted that the number of options is
fixed for a given handle. 
\begin{changebar}
  However, as options are set, other options may become active or inactive or
  their constraint may change. Thus, after setting an option, it may be
  necessary to re-read the descriptors.
\end{changebar}
While
setting up the device, it is also admissible to call
\code{sane\_get\_parameters()} to get an estimate of what the image
parameters will look like once image acquisition begins.

The device handle can be put in blocking or non-blocking mode by a
call to \code{sane\_set\_io\_mode()}.  Devices are required to support
blocking mode (which is the default mode), but support for
non-blocking I/O is strongly encouraged for operating systems such as
UNIX.

After the device is setup properly, image acquisition can be started
by a call to \code{sane\_start()}.  The backend calculates the exact
image parameters at this point.  So future calls to
\code{sane\_get\_parameters()} will return the exact values, rather
than estimates.  Whether the physical image acquisition starts at this
point or during the first call to \code{sane\_read()} is unspecified
by the SANE API.  If non-blocking I/O and/or a select-style interface
is desired, the frontend may attempt to call
\code{sane\_set\_io\_mode()} and/or \code{sane\_get\_select\_fd()} at
this point.  Either of these functions may fail if the backend does
not support the requested operation.

\begin{changebar}
Image data is collected by repeatedly calling \code{sane\_read()}
until this function will return an end-of-file status
(\code{SANE\_STATUS\_EOF}).  This indicates the end of the current
frame. If the frontend expects additional frames (e.g., the
individual channels of a red/green/blue image or multiple images),
it can call \code{sane\_start()} again.
If the \code{SANE\_PFLAG\_LAST\_FRAME} bit is set in \code{flags}, the
current image is complete. In this case, it should be tested, if
\code{flags} has the \code{SANE\_PFLAG\_MORE\_IMAGES} bit set.
If yes, further calls to \code{sane\_start()} can be made to acquire
more images. Please note, that as this bit has to be set at the beginning
of a the transmission of the last frame before the new image, it is possible,
that no reliable decision can be made at this time. It is thus permissible
for a backend to set this bit, and then later at the actual call to
\code{sane\_start()} return an error like \code{SANE\_STATUS\_NO\_DOCS}. 
Such a sequence is permitted to transmit multiple images from a single 
page as well as multiple pages. This behaviour should be controlled by 
backend options as required, to allow single-page scanning as well as 
ADF-batch-scanning. The frontend should always continue reading all images 
until a frame with \code{SANE\_PFLAG\_LAST\_FRAME} on
and \code{SANE\_PFLAG\_MORE\_IMAGES} off is encountered, or an error other
than \code{SANE\_STATUS\_EOF} occurs in a SANE function.
Note that \code{SANE\_STATUS\_NO\_DOCS} also is an allowed way for the backend
to indicate the end of a multiple image scan.

A frontend may choose to skip frames (e.g. because it cannot parse them), 
which is accomplished by simply calling \code{sane\_start} again, which will get 
you to the next frame, without having to read and discard the current one.

In order to prematurely stop scanning and to reset the backend state, 
\code{sane\_cancel()} can be called at any time. This call is required 
as well after normal termination of a multiple image scan as described above.
\end{changebar}

When done using the device, the handle should be closed by a call to
\code{sane\_close()}.  Finally, before exiting the application,
function \code{sane\_exit()} must be called.  It is important not to
forget to call this function since otherwise some resources (e.g.,
temporary files or locks) may remain unclaimed.

\begin{changebar}
The following C sample code implements a reference loop for acquiring 
multiple images:

\begin{verbatim}
SANE_Parameters parms;
SANE_Status     status;

do
  {
    do
      {
        /* Now start acquiring the next frame. */
        status = sane_start (handle);

        /* if that failed, we have a problem, and no more frames can be 
         * read at this time. Due to SANE_PFLAG_MORE_IMAGES still 
         * being clear, this will break out of _BOTH_ loops. 
         */
        if (status != SANE_STATUS_GOOD)
          break;

        /* Now let us see what the next frame brings. */
        status = sane_get_parameters (handle, &parms);

        /* This actually should not fail, but maybe the doc feeder 
         * jammed or something, so we break as well, if something 
         * is wrong. 
         */
        if (status != SANE_STATUS_GOOD)
          break;

        /* Now we check the announced parameters, if we can make use 
         * of the frame data. If not, we skip over to the next frame. 
         */
        if (do_i_like_that (&parms) == NO)
          continue;

        /* Set up for reading the data here. Mangle filenames, 
         * allocate memory, rewind multiframe files, ask user 
         * for confirmation, ... 
         */
        setup_for_transfer (...);

        /* Now we read in the frame data and process it. This should 
         * return SANE_STATUS_GOOD, until the frame is complete, 
         * what causes SANE_STATUS_EOF to be returned. 
         */
        while (SANE_STATUS_GOOD == (status = sane_read (...)))
          read_in_and_process_data_as_required ();

        /* If transfer was broken due to anything but EOF, break out. */
        if (status != SANE_STATUS_EOF)
          break;

        /* Now loop until we have all frames of an image. */
      }
    while (!(parms.flag & SANE_PFLAG_LAST_FRAME));

    /* O.K. - we now have a complete image. Fit it together, save it, 
     * flush buffers, transmit it, increment filenames, etc. 
     */

    /* Now check for more pending images. If we have more, redo from start. 
     * Some backends might cheat here and send us for an extra round which 
     * will fail at sane_start, as they were not able to determine if they 
     * would have more data at the start of the last frame we read. 
     */
  }
while (parms.flags & SANE_PFLAG_MORE_IMAGES);

/* No more data. Fine. Reset the backend and go back to option-control 
 * loop. 
 */

sane_cancel (handle);
\end{verbatim}

\end{changebar}


\section{Well-Known Options}\index{well-known options}

While most backend options are completely self-describing, there are
cases where a user interface might want to special-case the handling
of certain options.  For example, the scan area is typically defined
by four options that specify the top-left and bottom-right corners of
the area.  With a graphical user interface, it would be tedious to
force the user to type in these four numbers.  Instead, most such
interfaces will want to present to the user a preview (low-resolution
scan of the full scanner surface or a high(er) resolution scan of a subpart
of the scanner surface) and let the user pick the scan area by
dragging a rectangle into the desired position.  For this reason, the
SANE API specifies a small number of option names that have
well-defined meanings.

\subsection{Option Number Count}\index{option count}

Option number 0 has an empty string as its name.  The value of this
option is of type \code{SANE\_TYPE\_INT} and it specifies the total
number of options available for a given device (the count includes
option number 0).  This means that there are two ways of counting the
number of options available: a frontend can either cycle through all
option numbers starting at one until
\code{sane\_get\_option\_descriptor()} returns \code{NULL}, or a
frontend can directly read out the value of option number 0.

\subsection{Scan Resolution Options}\index{scan resolution}\index{resolution option}

Option \code{resolution} is used to select the resolution at which an
image should be acquired.
\begin{changebar}
When the backend wants to allow different
values for x- and y-resolution it has to define the options
\code{x-resolution} and \code{y-resolution}. Note that only
the option \code{resolution} {\bf or} the options
\code{x-resolution} {\bf and} \code{y-resolution} may be active at the
same time.
\end{changebar}

The type of this option is either \code{SANE\_TYPE\_INT} or
\code{SANE\_TYPE\_FIXED}.  The unit is \code{SANE\_UNIT\_DPI} (dots/inch).

These options are not mandatory, but if a backend does support them, it
must implement them in a manner consistent with the above definition.

\subsection{Preview Mode Option}\index{preview mode}

The boolean option \code{preview} is used by a frontend to inform the
backend when image acquisition should be optimized for speed, rather
than quality (``preview mode'').  When set to \code{SANE\_TRUE},
preview mode is in effect, when set to \code{SANE\_FALSE} image
acquisition should proceed in normal quality mode.  The setting of
this option \emph{must not\/} affect any other option.  That is, as
far as the other options are concerned, the preview mode is completely
side effect free.  A backend can assume that the frontend will take
care of appropriately setting the scan resolution for preview mode
(through option \code{resolution}).  A backend is free to override the
\code{resolution} value with its own choice for preview mode, but it
is advised to leave this choice to the frontend wherever possible.
\begin{changebar}
When the \code{preview} option is set the backend should transfer
the image in frame type \code{SANE\_FRAME\_RAW} if possible.
\end{changebar}

This option is not mandatory, but if a backend does support it, it
must implement it in a manner consistent with the above definition.

\subsection{Scan Area Options\label{subsec:scanarea}}\index{scan area options}

The four most important well-known options are the ones that define
the scan area.  The scan area is defined by two points (x/y coordinate
pairs) that specify the top-left and the bottom-right corners.  This
is illustrated in Figure~\ref{fig:area}.  Note that the origin of the
coordinate system is at the top-left corner of the scan surface as
seen by the sensor (which typically is a mirror image of the scan
surface seen by the user).  For this reason, the top-left corner is
the corner for which the abscissa and ordinate values are
simultaneously the {\em smallest} and the bottom-right corner is the
corner for which the abscissa and ordinate values are simulatenously
the {\em largest}.  If this coordinate system is not natural for a
given device, it is the job of the backend to perform the necessary
conversions.
\begin{figure}[tbp]
  \begin{center}
    \leavevmode
    \psfig{file=figs/area.eps,height=0.3\textheight}
    \caption{Scan area options}
    \label{fig:area}
  \end{center}
\end{figure}

The names of the four options that define the scan area are given in
the table below:
\begin{center}
\begin{tabular}{ll}
{\bf Name} & {\bf Description} \\
\code{\defn{tl-x}} & Top-left $x$ coordinate value \\
\code{\defn{tl-y}} & Top-left $y$ coordinate value \\
\code{\defn{br-x}} & Bottom-right $x$ coordinate value \\
\code{\defn{br-y}} & Bottom-right $y$ coordinate value \\
\end{tabular}
\end{center}
There are several rules that should be followed by front and backends
regarding these options:
\begin{itemize}

\item Backends must attach a unit of either pixels
  (\code{SANE\_UNIT\_PIXEL}) or millimeters (\code{SANE\_UNIT\_MM}) to
  these options.  The unit of all four options must be identical.

\item Whenever meaningful, a backend should attach a range or a
  word-list constraint to these options.

\item A frontend can determine the size of the scan surface by first
  checking that the options have range constraints associated.  If a
  range or word-list constraints exist, the frontend can take the
  minimum and maximum values of one of the x and y option
  range-constraints to determine the scan surface size.

\item A frontend must work properly with any or all of these options
  missing.
  
\begin{changebar}
\item A frontend may temporarily set the values in a way that
\code{tl-x} is larger than \code{br-x} and \code{tl-y} is larger than
\code{br-y}.
\end{changebar}

\end{itemize}

\begin{changebar}
These options are not mandatory, but if a backend does support them, it
must implement them in a manner consistent with the above definition.
\end{changebar}

\begin{changebar}
\subsection{Depth Option}\index{bit depth option}
Option \code{depth} is used to select the image depth in bits/sample
in multi bit mode (e.g. for 24 bit RGB mode this value must be 8).
The type of this option is \code{SANE\_TYPE\_INT}.
The unit is \code{SANE\_UNIT\_BIT}. For 1 bit modes
(Lineart or Halftone) this option has to be inactive.
For selection of 1 bit modes (Lineart or Halftone) the
backend should use the well-known option \code{mode}.
 
This option is not mandatory, but if a backend does support it, it
must implement it in a manner consistent with the above definition.
\end{changebar}
 
\begin{changebar}
\subsection{Scan Mode Options}\index{mode options}
The option \code{mode} defines a \code{SANE\_CONSTRAINT\_STRING\_LIST}
of type \code{SANE\_TYPE\_STRING}.
It is used to select the scanmode (e.g. Color or Gray).
Well known modes are: \code{Color}, \code{Gray}, \code{Halftone}
and \code{Lineart}. \code{Color} and \code{Gray} are multi bit
modes (8 or 16 bits/sample), \code{Halftone} and \code{Lineart}
are single bit modes. When well-known scan modes are used,
a frontend is able to automatically decide which mode is appropriate
for a specific task, e.g the mode \code{Gray}
for scanning a fax.
 
This option is not mandatory, but if a backend does support it, it
must implement it in a manner consistent with the above definition.
\end{changebar}
 
\begin{changebar}
\subsection{Scan Source Options}\index{source options}
The option \code{source} is used to select the scan source
(e.g. Automatic Document Feeder).
It defines a \code{SANE\_CONSTRAINT\_STRING\_LIST}
of type \code{SANE\_TYPE\_STRING}.
Well known sources are: \code{Flatbed}, \code{Transparancy Adapter} and
\code{Automatic Document Feeder}.
 
This option is not mandatory, but if a backend does support it, it
must implement it in a manner consistent with the above definition.
\end{changebar}

\begin{changebar}
\subsection{Threshold Option}\index{threshold option}
The option \code{threshold} is used to define the threshold
for Lineart and maybe Halftone mode. In multi bit modes
this option should be set inactive.
The type of this option is \code{SANE\_TYPE\_FIXED}.
The unit is \code{SANE\_UNIT\_PERCENT}. The value range
should be 0.0\ldots100.0 if possible.
It defines the minimum intensity to get a white point / full intensity.

The backend has to scale the values in the following way:
A value of 0.0 means all pixels get white / full intensity.  A value of 50.0
means intensities brighter than medium gray get white / full intensity.  A
value of 100.0 means all pixels get black. If the
scanner is not able to cover the full range the backend has to define a
reduced value range (e.g. 30\ldots70 percent).

This option is not mandatory, but if a backend does support it, it
must implement it in a manner consistent with the above definition.
\end{changebar}

\begin{changebar}
\subsection{Gamma Table Options}\index{gamma table options}
The \code{gamma-table} option defines a \code{SANE\_CONSTRAINT\_RANGE}
of the type \code{SANE\_TYPE\_INT} which represents the gamma correction
table for gray.  In color mode the \code{gamma-table} may be used to set
a common gamma correction for red, green and blue. The options
\code{red-gamma-table}, \code{green-gamma-table} and \code{blue-gamma-table}
are used in color mode to set a gamma correction for each color
separately. In color mode the backend is free to use only the
\code{gamma-table} option, only the \code{red-}, \code{green-} and
\code{blue-gamma-table} or all four options. When all four options
are used then the color tables should do a gamma correction with
the same input and output bit depth and the gray gamma table should
reduce (if necessary) the bit depth from the scanner internal
bit depth to the output bit depth. This should e.g. look like this:
\begin{verbatim}
red_value   = gamma-table(red-gamma-table(value))
green_value = gamma-table(green-gamma-table(value))
blue_value  = gamma-table(blue-gamma-table(value))
\end{verbatim}

The backend should not use the gamma tables to emulate other functions or options
like highlight, shadow, contrast, brightness, threshold, analog\_gamma.
These functions are common for all backends and should be added to the frontend
or a meta-backend.

It is also discouraged to emulate gamma tables in the backend. The backend
should disable (or not define) this option when the scanner does not support
gamma tables in hardware.
 
This option is not mandatory, but if a backend does support it, it
must implement it in a manner consistent with the above definition.
\end{changebar}

\begin{changebar}
\subsection{Analog Gamma}\index{analog gamma option}
The option \code{analog-gamma} is used to define the gamma value 
for an analog gamma function of the scanner in multi bit modes.
In 1 bit modes this option should be set inactive.
The type of this option is \code{SANE\_TYPE\_FIXED}.
The unit is \code{SANE\_UNIT\_NONE}. The value range
can be defined by the backend as supported. The values
have to be positive. A gamma value of 1.0 means that
the gamma correction has no effect. A value larger than
1.0 increases the brightness of the image.
In color mode there also can be options \code{analog-gamma-red},
\code{analog-gamma-green} and \code{analog-gamma-blue}.
It is not allowed to emulate an analog gamma function by
a digital gamma table. The backend has to disable (or not
define) this option when the scanner does not support an
analog (hardware) gamma function.

When analog gamma, highlight and shadow functions are available
at the same time then the backend author has to care about the order
in which the functions are implemented in the scanner hardware.
The SANE standard expects that changing the analog gamma value
has no effect on the shadow and highlight function. When the
analog gamma function is executed in the scanner hardware
before the shadow and highlight functions then the backend
should do a compensation. For this the shadow and highlight
values have to be gamma corrected with the relevant analog gamma value.

This option is not mandatory, but if a backend does support it, it
must implement it in a manner consistent with the above definition.
\end{changebar}


\begin{changebar}
\subsection{Shadow Option}\index{shadow options}
The option \code{shadow} is used to define the shadow level / black point level.
The type of this option is \code{SANE\_TYPE\_FIXED}.
The unit is \code{SANE\_UNIT\_PERCENT}. The value range
should be 0.0\ldots100.0 if possible.
It is used to define the maximum intensity level that creates an image data value
of 0 (black).

The backend has to scale the values in the following way:
A value of 0.0 means that the sensitivity range is not reduced, only the
minimum intensity produces an image data value of 0 (black).
A value of 50.0 means that that a medium intensity and everything that is darker
produces an image data value of 0 (black).
A value of 100.0 means the sensitivity range is reduced to 0, all image
data values are 0 (black).  If the scanner is not able to
cover the full range the backend has to define a reduced
value range (e.g. 30\ldots70 percent).
In color mode there can be options \code{shadow-red}, \code{shadow-green}
and \code{shadow-blue}, in this case the \code{shadow} function has to be disabled.
It is not allowed to emulate a shadow function by
a digital gamma table. The backend has to disable (or not
define) this option when the scanner does not support an
analog (hardware) shadow function.

This option is not mandatory, but if a backend does support it, it
must implement it in a manner consistent with the above definition.
\end{changebar}


\begin{changebar}
\subsection{Highlight Option}\index{hightlight options}
The option \code{highlight} is used to define the highlight level / white point level.
The type of this option is \code{SANE\_TYPE\_FIXED}.
The unit is \code{SANE\_UNIT\_PERCENT}. The value range
should be 0.0\ldots100.0 if possible.
It is used to define the minimum intensity level that creates the maximum possible
image data value (white/full intensity). 

The backend has to scale the values in the following way:
A value of 0.0 means the sensitivity range is reduced to 0, all image
data have maximum value (white / full intensity).
A value of 50.0 means that a medium intensity and everything that is brighter
produces the maximum possible image data value (white / full intensity).
A value of 100.0 means that the sensitivity range is not reduced, only the
maximum intensity produces an image data with maximum possible value (white / full intensity).
If  the scanner is not able to cover the full range the backend has to define a reduced
value range (e.g. 30\ldots70 percent).
In color mode there can be options \code{highlight-red}, \code{highlight-green}
and \code{highlight-blue}, in this case \code{highlight} has to be disabled.
It is not allowed to emulate a highlight function by
a digital gamma table. The backend has to disable (or not
define) this option when the scanner does not support an
analog (hardware) highlight function.

This option is not mandatory, but if a backend does support it, it
must implement it in a manner consistent with the above definition.
\end{changebar}


\begin{changebar}
\subsection{Lamp Options}\index{lamp-on option}\index{lamp-off option}

The options \code{lamp-on} and \code{lamp-off} are button options
(\code{SANE\_TYPE\_BUTTON}) and don't have a unit (\code{SANE\_UNIT\_NONE}).

Option \code{lamp-on} is used to turn on the lamp of the scanner while
\code{lamp-off} turns it off.

These options are not mandatory, but if a backend does support them, it
must implement them in a manner consistent with the above definition.
\end{changebar}


\begin{changebar}
\subsection{Scanner Button Options}\index{scanner button options}
Some scanners have buttons which state can be read by the scanner driver. 
It is necessary to implement a locking function for the buttons
because it is possible that several frontends try to connect to the
same backend/scanner at the same time. Imagine what could happen
when no locking would be implemented:
Five people have started a scanning application which is connected
via network to the scanner you want to use. You start a frontend,
put a paper to the scanner and press the scan-button on the scanner.
The scanner does scan three times (because three frontends asked the
button status when you pressed the button). For three people the
image is saved to the harddisk, but it is not sure that your
frontend did scan the image.

A backend that does make available the scanner-buttons has to
implement the following options:\\
\code{scanner-buttons-lock} is of type \code{SANE\_TYPE\_BOOL}, default = \code{SANE\_FALSE}\\
\code{scanner-buttons-status} is of type \code{SANE\_TYPE\_INT}, default = 0\\
\code{scanner-buttons-status-update} is of type \code{SANE\_TYPE\_BUTTON}\\
When setting these options the backend does not set \code{SANE\_INFO\_RELOAD\_OPTIONS}
or \code{SANE\_INFO\_RELOAD\_PARAMS} if not explictly defined.

A frontend has to disable the usage of the scanner-buttons by default. This is important
because other frontends will not be able to use the buttons when the button-functions are locked.
Another important thing is that some scanners do not turn off their lamp when the driver
does frequently talk to the scanner (what is done when reading the button status from the scanner).

\begin{itemize}

\item
A frontend that wants to read the button status has to lock the
button functions at first. For this it does set the option
\code{scanner-buttons-lock} to \code{SANE\_TRUE}.
While setting the value of option \code{scanner-buttons-lock} to \code{SANE\_TRUE}
the backend does check if a lockfile (e.g. ``backend''-buttons.lock) does exist.
The lockfile has to be placed in a directory where every user has read and write access to.

  \begin{itemize}
  \item
  If the lockfile does not exist then the backend creates the lockfile and writes the
  process ID (PID) of the backend to the file. Button access is allowed: the value
  of option \code{scanner-buttons-lock} is set to \code{SANE\_TRUE}
  \item
  If the file does exist and the backend PID is not the file PID then the
  backend has to check if the process with the PID stored in the lockfile
  still is running. If yes then the button access is not allowed: the value of
  option \code{scanner-buttons-lock} is set to \code{SANE\_FALSE}. If not then the lockfile
  is recreated and the PID of the backend is stored in the lockfile, button
  access is allowed: the value of option \code{scanner-buttons-lock} is set to \code{SANE\_TRUE}
  \end{itemize}

\item
The frontend does read the value of option \code{scanner-buttons-lock}. If
it is \code{SANE\_TRUE} then the frontend has access to the scanner buttons.
If it is \code{SANE\_FALSE} then access has been denied.

\item
If the button access is allowed the frontend has to do the following about
once per second (while not scanning):
  \begin{itemize}
  \item
  The frontend does set option \code{scanner-buttons-status-update}.
  The backend checks if access to the buttons is allowed by comparing
  the backend PID with the lockfile PID. If access is allowed it
  does read the button status from the scanner and stores it in
  the option \code{scanner-buttons-status}, each bit represents a button, a
  value of 0 means the button is not pressed, a value of 1 means that the button
  is pressed. When the scanner is busy the backend must not wait, it has to
  return immediately and the button state keeps unchanged.
  The backend has to implement a timeout function. When no button has been pressed
  within a predefined time (e.g. 15 minutes) then the access permission is lost.
  In this case the backend does set option \code{scanner-buttons-lock} to \code{SANE\_FALSE}
  and does set \code{SANE\_INFO\_RELOAD\_OPTIONS} to inform the frontend that it has
  lost permission to access the scanner-button functions.
  If access is not allowed it does set the \code{scanner-buttons-status} to 0.

  \item
  The frontend does read the value of option \code{scanner-buttons-status}
  \end{itemize}

\item
When the frontend does exit or it does not want to use the buttons
it does set option \code{scanner-buttons-lock} to \code{SANE\_FALSE}.
The backend does check if the backend PID and the lockfile PID are
the same. If this is true then it removes the lockfile and sets the value 
of \code{scanner-buttons-lock} to \code{SANE\_FALSE}.

\code{sane\_close()} should do the same as setting option
\code{scanner-buttons-lock} to \code{SANE\_FALSE}.

\end{itemize}
\end{changebar}

\begin{changebar}
\subsection{Batch Scan Options}\index{Batch Scan Options}

The batch scan options can be used by a frontend to indicate that more than
one image will be scanned in a batch. The backend can optimize for such scans
by e.g. avoiding calibration and not moving home the sensor array in this case.
The backend provides the options \code{batch-scan},
\code{batch-scan-next-tl-x}, \code{batch-scan-next-tl-y},
\code{batch-scan-next-br-x}, and \code{batch-scan-next-br-y}. The option
\code{batch-scan} provides a string list with the values of \code{No} if
batch scanning isn't used, \code{Start} for the first, \code{End} for the last
and \code{Loop} for any other (intermediate) image.  The batch-scan-next
options specify the coordinates of the next scan and follow the same rules as
the scan area options (see section \ref{subsec:scanarea}).

These options are not mandatory, but if a backend does support them, it must
implement them in a manner consistent with the above definition.  In this
case, all options must be implemented.
\end{changebar}




\input{net.tex}

\chapter{Contact Information}\label{chap:contact}

The SANE standard is discussed and evolved via a mailing list.
Anybody with email access to the Internet can automatically join and
leave the discussion group by sending mail to the following address.
\begin{quote}\index{mailing list}
\begin{verbatim}
sane-devel-request@mostang.com
\end{verbatim}
\end{quote}
To subscribe, send a mail with the body ``\verb|subscribe sane-devel|'' to the
above address.

A complete list of commands supported can be obtained by sending a
mail with a subject of ``\code{help}'' to the above address.  The
mailing list is archived and available through the SANE home page at
URL:
\begin{quote}
\url{http://www.mostang.com/sane/}
\end{quote}

\newpage
\input{sane.ind}

\end{document}


\end{document}


\end{document}


\end{document}
